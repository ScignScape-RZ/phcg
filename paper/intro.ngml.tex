{\EnglischeLinie}
\p{Computational and scientific data representation intrinsically 
balances two competing priorities: \q{semantic} expressiveness and 
computational tractability.  On the one hand, representations should not 
obscure important details: the formal requirements on 
representational validity should not force representations 
into structures that necessitate the elimination of meaningful 
information.  On the other hand, conversely, representations 
should have enough structural consistency that they are 
amenable to analysis and transformations driven by 
formal algorithms.
}
\p{I have just left a lot of loose ends: in particular, these comments need 
to give some meaningful definition to \q{representation}.  There 
are many media wherein \q{data} can be represented: via graphics 
(e.g., charts, \q{maps} in the cartographical sense, or \q{graphs} 
in the sense of plotted visualizations of 
mathematical functions), via printed documents (like the logarithmic 
tables or astronomic records of early modern science), or via mathematical 
equations and formulae (if a mathematical theory correctly 
predicts and quantifies empirical data \mdash{} e.g., fitting trajectories 
to elliptic or parabolic shapes \mdash{} then the numeric structures of 
the theory are a proxy representation of the corresponding data).  
Moreover, aside from these relatively formal modes of representation, 
we have the capacity to indirectly describe information via 
natural language. 
}
\p{The modern age has a further notion of representation: \i{digital} 
representations which leverage the capabilities of 
computer storage and processing to create persistent 
data repositories, with the possibility of manipulating 
data via computer programming and displaying data 
via computer software.  These digital representations incorporate 
features of the older representational media I alluded to: 
like Natural Language digital data often emanates from a 
textual representation, with data structures initialized 
from special \q{languages} like \XML{} or \JSON{}.  As with 
structured \q{printed} documents (of the astronomical-table 
or even grocery-shopping-list variety), digital representations 
often build off of a rigid structural template, like a two-dimensional 
table with rows and columns, or a hierarchical document 
whose elements can be nested documents.  And as with 
mathematical representations, digital representations can be 
analyzed as instances of spaces or structures with certain 
algebraic or syntactic properties and requirements.  
It is often possible to mathematically describe the 
full space of structures which conform to a given 
representational protocol, or the full space of transformations 
that can modify a given data structure while staying 
consistent with its enforced protocols.  Also, each data 
structure potentially has some notion of aggregateness: of 
having different \q{parts}, of being able to focus on one part 
at a time, and to \q{move} focus from one part to another.  
Given these qualities, digital representations extend and integrate  
the affordances of print, graphical, and linguistic media 
\mdash{} but they do so in an electronic environment that permits digital 
archives to be accessed via computer software.  With sotfware as their  
access point, digital representations can inherit the formal properties of 
the software that uses them, so we can think of digital representations as 
formal structures themselves, amenable to something like a mathematical analysis. 
}
\p{In short, digital representations have several general criteria: 
\i{validity} (a formal model of conformant vs. invalid 
structures\footnote{An invalid graph might be a case where an 
edge has no incident nodes; an invalid \XML{} structure is one without 
a single root element or (insofar as such a structure would be 
representable) with mismatched tags, and so forth)}); 
\i{traversability} (a notion of parthood and iterating
or refocusing among parts), and, let's say, \i{atomicity} 
(a notion of unitary parts that can be represented as 
integral wholes).  These criteria ensure that digitally 
represented data structures can work within software and 
networking representations: information is 
presented to software users by displaying atomic 
units (textually or graphically) and traversing through 
data structures to fill in, via application code, a 
visual tableau presenting compound data, with different 
unitary displays visually separated and often organized into 
coherent visual patterns, like the grid-pattern of a 
spreadsheet.  Meanwhile, atomic units can be textually 
encoded, and aggregate structure likewise notated through 
syntactic conventions that preserve atoms' boundaries, yielding 
textual serializations of data structures that can be 
shared between computing environments, allowing information to 
be copied and distributed.  Finally, digital representations 
of data structures can be marshalled into different binary forms \mdash{} 
encoded in byte- and bit-patterns \mdash{} to enable both 
persistent storage in a database and \q{runtime} presence 
as binary data that may be accessed by software applications.
}
\p{I take the time to lay out these basic principles because I want 
to emphasize the different contexts where digital representations can be 
found: in particular, textual encoding and serialization; 
graphical/interactive displays for software users; application runtimes; 
and long-term database storage.  A given representation will morph and 
mutate to accommodate these different contexts.  Moreover, these 
contexts correspond to distinct technical specializations: database engineers 
have a different perspective on data structures than network engineers designing 
protocols for encoding and exchanging data between network endpoints; and 
application designers focused on optimal human-computer-interaction 
understand digital representations as visual and interactive phenomena, 
whereas application \i{developers} need to focus on how to 
properly encode data structures for binary runtimes.  These various 
perspectives all influence the theory and technology 
behind digital data representation \mdash{} a successful 
representational paradigm needs to adapt to the operatoonal 
requirements of engineers in each of these disicplines.
}
\p{Additionally, insofar as the point of digital archives is to encode 
empirical, \q{real-world} data, a proper representational protocol 
needs to promote a synergy between the information as humans understand 
it and the data structures recognized by the technology.  For instance, 
if a published data set shares scientific data, it should be 
represented in ways that preserve scientifically significant 
details \mdash{} any derivations, descriptions, or observations 
which are intrinsic to the science's methodology, 
theory, assumptions, and experimental results.  The data needs to 
be structured according to the \q{semantics of the science}, so that 
the scientific background can be reconstructed along with future use 
of the data, even after a circuitous journey through different 
contexts, like through a database and over a network to 
a scientific-software application (maye years later).   
As formal models of data semantics have become more rigorous \mdash{} e.g. in our 
century with Ontologies and the \q{Semantic Web} \mdash{} this 
\q{semantic engineering} has become a further technical perspective 
needing consideration in the design and evaluation of digital representations.
}
\p{Over the decades, many digital representation strategies and 
\q{layouts} have been envisioned, from tabular structures 
in the manner of Relational Databases, to tree-like 
documents whose morphology is inspired by markup languages, 
to structurally looser variations on row/column 
arrangements like multi-dimensional key-value spaces or 
\q{Big Column} and other \q{\NoSQL{}}-database-inspired 
articulations.  These various paradigms are subject to 
selective pressures based on how well they meet the 
different engineering needs I have identified.  But the 
multiplicity of requirements complicates the \q{competition} between 
representational strategies: a paradigm optimized 
for one context (e.g., persistent databases) is not necessarily 
optimal for another (e.g., application and \GUI{} development).  
As a result, developers and computer scientists must 
continue to explore and collaborate on new paradigms which 
work better across contexts.
}
\p{In the plurality of representational paradigms, a significant 
evolution has been the emergent popularity of structures 
based on graphs.  The predominant influence on this 
line of development has been the Semantic Web and the specific graph 
architecture codified by \q{Web Ontologies}, although 
other graph-representation models (such as 
Conceptual Graph Semantics) were also candidates for 
potential technological \q{entrenchnment} before 
Semantic Web formats like \RDF{} and \OWL{} 
became standardized.\footnote{Why \RDF{}/\OWL{} and not, say, Conceptual 
Graphs, or the hybrid Object-Database/Graph-Database models 
studied in the 1990s, became canonized in the Semantic Web, is 
an interesting question but perhaps of mostly historical 
interest insofar as Hypergaphs can unify each of these paradigms.
}  More recently, scientists have proposed generalizations 
and/or alternatives to the Semantic Web based on 
hypergraphs in lieu of ordinary (directed, labeled) graphs.
}
\p{As might be ascertained from hypergraphs being the focus of this paper, 
I believe that representational paradigms based on hypergraphs can be superior to 
other formats and need to be studied with an integrated, generalized 
set of theories and computer tools.  I have introduced the 
topic of hypergraph data structures via general issues in 
digital representation in part to rest claims of hypergraphs' 
merit on explicit criteria.  In particular, I believe 
hypergraphs can adapt to the different technological 
contexts prerequisute to a decentralized digital ecosystem 
\mdash{} databases, networking, application implementation, visual/interactive 
software design, and semantic expressiveness \mdash{} more effeectively 
than other paradigms, like regular graphs or \SQL{}-style tables.
}
\p{I suspect other scientists and engineers have 
similar intuitions, because there has been a recent uptick in 
research on hypergraphs in various disciplines, such 
as Category Theory, Information Management, Artificial Intelligence, 
and Natural Language Processing.  Compared to 
the Semantic Web, however, there is a noticeable lack 
of standard tools or formats for expressing hypergraph 
data or sharing hypergraph structures across multiple 
applications and environments.  Whereas \RDF{} and \OWL{} are 
definitively associated with the Semantic Web, so that 
supporting these standards is a basic entry point for 
Semantic Web technologies, there is no comparable 
consensus on the underlying theory of hypergraph data 
in general.    
}
\p{This situation may be explained in part by subtle differences on 
what the word \q{hypergraph} is supposed to mean in different 
settings, so the overal terrain of hypergraphs is 
divided into distinct mathematical models, often without a 
rigorous theory of their interrelationships.  
Another problem is perhaps a failure to appreciate how hypergraphs 
are structurally different from ordinary graphs, so that 
hypergraph-shaped data may be imprecisely treated as 
a mere variation upon or a special structuration within 
ordinary graphs.  For eample, Semantic Web structures such as 
\RDF{} \q{bags} and \q{collections} introduce a kind of 
hierarchical organization to Semantic Web graphs \mdash{} qualifying as 
a form of hypergraph \mdash{} but these protocols are not described as 
enabling a transition from a networking framework based on 
directed labeled graphs to one based on hypergraphs.  Instead, 
hypergraphs become \i{de facto} embedded within ordinary graphs, 
exploiting the representational flexibility which also 
makes the Semantic Web suitable for spreadsheets, \XML{}, and 
other structures that are not graphs at a \i{prima facie} level.
The problem is that while hypergraphs can be \i{encoded} in 
ordinary graphs with a suitable labeling convention, 
their distinct structural advantages are lost as 
explicit architectural features once hypergraphs are 
\q{encoded} in conventional Semantic Web formats.
}
\p{As I will outine in the first section, there are at least some half-dozen 
different structures that may certainly be described as hypergraphs, generalizing 
various underlying graph models (or indeed, even if we restrict attention to 
labeled directed graphs).  Within this space of possibilities, 
recent computer-science-related) research into hypergraphs 
seems to emphasize two different topics: first, the 
representational potential for hypergraphs as a 
general morphology for encoding structured information-resources 
(so as a general tactic for data warehousing and modeling); and, second, the 
specific possibilities of using hypergraphs to model computer code and computational 
processes.  In the second subject-area, hypergraphs are studied 
as a medium for expressing details about computer algorithms as a 
way to reason about executable computer code as a structured system, and 
perhaps even to design execution engines (virtual machines to 
run computer code that is in a suitable representation).  This 
latter research sometimes appears to present a mathematical 
picture of hypergraphs as an abstract model 
of computing procedures and evaluations \mdash{} a kind of 
graph-based interpretation of the lambda calculus \mdash{} but 
sometimes is also marshaled into implemented execution 
environments, as in the OpenCog and lmnTal frameworks.
}
\p{My goal in this paper is to consider what a unified hypergraph 
paradigm can look like \mdash{} how the different strands which in their 
own way embody hypergraph structures can be woven together 
into a multi-purpose whole.  My emphasis is on computer implementations 
rather than mathematics \mdash{} that is, I will not present axioms 
or theorems formalizing different sorts of hypergraphs, though I 
ackowledge that such descriptions are possible.  Instead, my 
aim is to describe what might constitute a general software 
library or toolkit that can adapt to different hypergraph 
models and use-cases.  This paper is accompanied by a \q{data set} 
which involves a library (written in \Cpp{}) for creating and 
manipulating hypergraphs of different varieties, and also a 
\q{virtual machine} for modeling and realizing 
computatational procedures via hypergraph structures.  The 
point of the accompanying code is to 
demonstrate that a generalized hypergraph representation 
is possible, and that in addition to use-cases for modeling 
data structures such a representation can be used at the 
core of a virtual processor.  The code includes simple tests and 
\q{scripts} that can be executed via the provided virtual-machine 
code.
}
\p{Studying hypergraphs from a computational (and 
\q{implementational}) perspective \mdash{} not just as 
mathematical objects \mdash{} introduces useful details 
that can add depth to our overall understanding of 
hypergraphs.  For example, one question is how hypergraphs 
can be initialized \mdash{} how software systems can build up 
hypergraph structures, piece by piece.  This is related to the question of 
proper serialization formats for hypergraphs.  Given some 
specific hypergraph data aggregate \Hy{}, it is important to 
have a textual encoding where \Hy{} can be mapped to a 
character-string, shared as a document, and then reconstructed 
with the same hypergraph structure.  This raises the question of 
when and whether two hypergraphs are properly isomorphic 
\mdash{} such that serializing and then deserializing a 
hypergraph yields \q{the same} hypergraph \mdash{} and also the problem 
of validating and parsing textual representations of hypergraphs.  
The theory of parsing \i{serializations} of hypergraphs \mdash{} 
textual encodings constructed according to a protocol wherein their 
grammar and morphology are suitable to 
expressing and then rebuilding hypergraph structures \mdash{} 
then becomes an extension of hypergraph theory itself.
}
\p{In addition to creating hypergraphs via textual 
representations in the manner of markup languages 
(like \XML{} or \RDF{}), we can also consider the incremental 
accumulation of hypergraph structures via minimal 
units of transformation, providing a kind of 
\q{Intermediate Representation} embodying the form of a 
hypergraph via a sequence of effectively (at least on one scale 
of analysis) atomic operations.  For any variety of 
hypergraph, then, we can consider which is a suitable 
Intermediate Representation for conformant spaces \mdash{} in particular, 
which set of primitive options can, iteratively, construct 
any representative of the space of instantations of a 
given kind of hypergraph.  Insofar as a software framework 
seeks to work with several hypergraph varieties, we can consider 
how several different such operation-sets can be combined.  
}
\p{Moreover, we can consider both serialization and Intermediate Representation 
as parallel tactics for building hypergraphs.  
An Intermediate Representation language can be used to carry out 
transformations between hypergraphs \mdash{} producing \IR{} code based on 
the first hypergraph from which the second can be populated.  In conjunction 
with serialization and parsers, a hypergraph can then be realized 
via several stages, with one form of hypergraph used as an intermediary 
because it has a convenient grammar and works with a parsing engine; 
from that preliminary graph a new graph can be created via \IR{} code.  
Grammars, serializaion and deserialization, and \IR{} thereby become 
part of the formal architecture defining a hypergraph ecosystem and 
inter-graph transformations (analogous to the trio of \XML{} parsers, 
\XML{} transforms, and the \XML{} Document Object Model).
}
\p{Continuing the \XML{} analogy, note that \XML{} is not just a serialization specification: 
the full \XML{} technology defined requirements on a series of tools 
operating on \XML{} data, not just \XML{} files: tools 
for traversing \XML{} documents (including a programming interface 
for how traversal options are to be operationalized, e.g. 
via Object-Oriented methods like \parentNode{} and \nextSibling{}); 
for parsing \XML{} files into traversable data structures (including 
requirements on the traversals which the derived structurees 
\mdash{} the so-called \q{post-processing infoset} \mdash{} must 
support); and for inter-document mapping 
(via \XSLT{} or \XML{}-FO).  The Document Object Model and 
post-processing infoset is the structural core of the \XML{} 
format, even more than any surface-level syntax (the grammar 
for tags, attributes, and so forth).  Analogous specifictions 
would have to be developed for a generalized Hypergraph representation.
}
\p{The code discussed here does not purport to provide a mature or 
decisive implementation of a general-purpose hypergraph ecosystem, 
but rather to demonstrate by example what components may comprise 
such a framework and how they may interoperate.  The code includes 
hyprgraph models but also the preliminary and 
intermediary structures that can connct hypergraphs to 
a surrounding computational infrastructure \mdash{} parsers, 
Intermediate Representation, application-integration 
logic (such as mapping hypergraph nodes to 
application-specific data types), and a \q{virtual machine} 
for realizing hypergraphs as executable code.  
}
\p{The remainder of this paper will focus on three 
subjects: delimitating the proper range of hypergraph 
models; execution models and the \q{virtual machine}; and 
a review of formal structurs (like grammars and \IR{} 
formats) which are structurally different 
than hypergraphs but can help tie hypergraph 
representations togther into a unified ecosystem.  
I will also, in conclusion, sketch arguments to the effect that 
hypergraphs offer a plausible foundation for reasoning 
about structured/computation data and date types in general.  
I have spoken only informally about hypergraphs 
themselves, taking for granted that we have a general 
picture of what hypergraphs are, but this now 
demands more rigorous treatment.  There are actually several 
different research and engineering communities 
that talk about hypergraphs, in each case describing 
some sort of generalization of regular (often 
labeled and/or directed) graphs, but these generalizations 
fall into different kinds.  Hence there are several different 
varieties of hypergraph, and I will try 
to outline a general theory of hypergraphs in these 
various forms.
}
\p{}
