\section{Hypergraph Parsers and Intermediate Representation}
\p{Having pointed out the features desired for a general-purpose 
hypergraph framework, the next problem is to implement a 
hypergraph library supporting these various features.  
One dimension of this problem is the proper in-memory 
representation of hypergraph structures and hypergraph elements 
(hypernodes, hyperedges, and so forth) \mdash{} to implement 
the datatypes and procedures requisite for hypergraphs as software artifacts.  
I will not emphasize these internal details directly, except for making 
implementation-related observations when relevant to other contexts 
(the accompanying code demonstrates one approach to building an 
in-memory hypergraph library, though I make no claim that the 
implementation is optimized for speed or to \q{scale up} to large 
graphs; my priority was instead to demonstrate hypergraph semantics, 
representing multiple hypergraph varieties, as expressively as possible).  
}
\p{A second stage in the implementation process, which I will 
examine here in greater detail, involves transformations 
and initialization of hypergraphs \mdash{} creating hypergraphs from 
other sources (like text files) and using hypergraphs to 
populate other kinds of data structures, either directly 
or indirectly (such as, via code generators).  This dimension 
encompasses the overall capabilities that would permit software to 
use hypergraphs for their own data representations 
and/or as bridge structures for sharing data between applications 
and/or components.  In bridge cases, the strategies for building 
hypergraphs \i{from} other kinds of data, and then building 
other kinds of data from hypergraphs, are equally significant 
to techniques for modeling hypergraphs themselves.
}
\p{On the \i{input} side, hypergraphs may be initialized via several 
routes.  The most direct path is to program function calls that 
can modify (and incrementally build up) hypergraphs directly.  
If hypergraphs are expressed via a cohort of \Cpp{} classes 
(for graph, node, edge, etc.), that means using \Cpp{} method 
calls to perform operations like adding a node or 
adding an edge between nodes.  Less directly, hypergraphs can 
be initialized from text files, given a standard format for 
textually encoding hypergraphs.  For conventional (non hyper-) graphs, 
popular formats include \RDF{}, \GRAPHML{}, and Notation-3.\footnote{Some graph formats support certain hypergraph features, but 
not the full spectrum of features I outlined in the last section
}  Analogous 
standards for hypergraphs would be more complex because of the 
extra structures involved.  Nevertheless, the serialization can be 
expressly organized to facilitate hypergraph initialization; 
in essence, the grammar can be standardized to unambiguously map 
test encodings to low-level function-calls via which hypergraph 
are built directly.  A still less direct input strategy would 
involve more flexible input languages, designed for ease of 
use from the point of view of people composing or reading 
serializations.  Such encodings then require more complex grammars 
and parsers for the step of mapping input test to the intended 
hypergraph structures.   
} 
   
\p{Similar alternatives apply to \i{output}, in the sense of initializing 
other structures from hypergraph data.  Output structures can be 
compiled directly via function-calls, or indirectly via software-generated 
text files, which in turn might have a more low-level or a more high-level 
format.  Suppose hypergraphs are employed within publishing 
software to generate \XML{} and \LaTeX{} output.  This can happen 
mostly at the software level: there are many \q{\XML{} builder} libraries 
allowing \XML{} documents to be constructed via function calls, rather 
than by actually producing \XML{} code.  Alternatively, the 
software can act as a code generator, expressly composing \XML{} 
or \LaTeX{} code by creating the requisite character strings 
\mdash{} a process which would be complicated by the languages' syntactic 
requirements, such as the composition of \XML{} tags and \LaTeX{} commands 
with the requisite braces and brackets.  Compared to higher-level 
markup builders, code generation can be difficult because software has 
to take responsibility for every character in the output \mdash{} every 
closing brace, bracket, quote, and so on.
}
\p{In between these two 
alternatives \mdash{} using builder libraries and explicit 
code generation \mdash{} is the possibility of generating code in 
alternative formats that are optimized for machine processing, 
and consequently bypass some of the complications attending to 
generating normal (say) \XML{} or \LaTeX{}.  For instance, 
event-driven or \q{\SAX{}} parsers understand \XML{} as a sequence of 
\q{events}, like \i{open-tag}, \i{close-tag}, \i{annotation}, 
or \q{character data}.  Some projects adopt this principle to 
generate \XML{} from non-\XML{} sources (see for example 
\cite{BaltasarTrancon}, \cite{MartinWesthead}): any 
data source can be treated as an \XML{} front-end if there is a 
processor that can map the data to an \XML{} event-stream.  
On such a basis developers can design encoding languages that 
model \XML{} event-streams explicitly, as an alternative 
to normal \XML{} syntax.  I will propose similar techniques 
to derive Intermediate Representations for a hypergraph 
builder and Virtual Machine.
}
   
\p{Whatever the format, generating output from hypergraphs is closely 
tied to the issue of traversing or navigating within hypergraphs.  
In this context it is useful to consider a hypergraph as a kind 
of \q{space}, with a notion of \q{points} or \q{locations}.  
Suppose a particular graph element (hypernode, hyperedge, annotation, 
and values which may be contained in them) can be singled 
out as \q{foreground}.  Technology then needs some notation for 
moving from one foregrounded element to another \mdash{} a hypernode to 
an incident edge, or instance.  In conjunction with event-driven 
output models as mentioned last paragraph, hypergraph-traversal 
yields productive output algorithms so long as each successive 
foreground element can be mapped to one or more output events.     
}
   
\p{For this paper I will examine these and related concepts in the 
specific context of a hypergraph execution engine.  Specifically, consider 
the challenge of building a programming language based on hypergraphs 
as internal representations.  In this case the input format is not 
tailored to hypergraphs, but rather to a grammar for source code which 
reflects the language's high-level paradigms (Object Orientation, 
Functional Programming, Actor Models, Capability-Orientation etc.).  
Likewise, the output format 
is not a relatively high-level markup language like \XML{}, but rather 
something like a virtual machine or intermediate language which 
can translate to low-level (e.g., \C{} or \Cpp{}) function-calls.  
The pipeline demonstrated here also includes two different 
hypergraph formats, along with a textual serialization to marshal 
between them.  Accordingly, the system also demonstrates 
initialization of hypergraphs from a hypergraph-specific 
intermediate representation. 
}
    
\p{All told, the language engine reviewed here has a pipeline 
covering five or six steps, outlined in Figure~\ref{fig:figLanguageEngine}. 
\begin{figure}
\centering
\begin{tikzpicture}[shorten >=2pt,>=latex,shorten <=8pt,
triangle/.style = {draw=blue!20!black, inner sep=-9, 
	regular polygon, regular polygon sides=3,yscale=0.5,
	},
border rotated/.style = {shape border rotate=180},
hex/.style = {draw=red!20!black, inner sep=-1, 
	regular polygon, regular polygon sides=6,yscale=0.25,
},
framed,background rectangle/.style={thin,draw=darkRed!50, top color=blGreen!20, 
	bottom color=yellow!10, rounded corners}
]

    \node [circle, draw=black, fill=white, inner sep=0pt,minimum size=5mm, text width=2cm, align=center] (nSrcCode) at 
    (0,0) {Input Source Code}; 
 
    \trinode{below right = 2mm and 40mm of nSrcCode}{nRelaeGraph}{RelaeGraph}     
       
    \hexnode{below right = 10mm and -18mm of nSrcCode}{nPGBIR}{PGBIR} 
  
    \trinode{below right = 4mm and 30mm of nPGBIR}{nPhaonGraph}{PhaonGraph}     

    \hexnode{below right = 10mm and -2mm of nPGBIR}{nPVMIR}{PVMIR} 
  
    \node[draw, text width=4cm, align=center,below=1cm of nPVMIR]  (nPhaonRT)  {Phaon Runtime};   

   \draw (nRelaeGraph.north)+(-7mm,0mm) edge [yshift=1mm, bend right = 8, <-]  
     node [yshift=4mm,xshift=5.5mm] {Hypergraph Parser} (nSrcCode);
   \draw (nPGBIR.north)+(7mm,0mm) edge [above, bend left = 8, <-] 
     node [yshift=-1mm,xshift=-3.5mm] {Phaon Graph Build} (nRelaeGraph.west) ;
   \draw (nPGBIR) edge  [bend left = 8, ->, shorten >=1mm]  node {} (nPhaonGraph.north);
   \draw (nPhaonGraph) edge [->]  node {}  (nPVMIR)    ;
   \draw (nPVMIR) edge  [bend right, ->]  node {} (nPhaonRT)   ; 
    
    %end path 
\end{tikzpicture}
\caption{Compiler Pipeline for the Phaon Virtual Machine} \label{fig:figLanguageEngine}
\end{figure}
  The process begins by parsing source code (designed 
as a vaguely \C{}-like scripting language) yielding 
one style of hypergraph, with graph manipulation 
functions tied to the grammar's production rules.  
Those hypergraphs (after some preliminary analysis) are then 
traversed to produce an intermediate representation from which 
a second genre of hypergraphs is initialized, one better suited 
for generating executable instructions.  This second 
hypergraph is then traversed in turn, outputting code in 
a special-purpose \q{virtual machine} language.  As a final step, 
this last code is then evaluated by mapping instruction to 
relatively low-level (mostly \Cpp{}) function-calls.  
}
   
\p{The remainder of this section will examine some of the code 
and techniques associated with these different processing step, 
which hopefully can be proxies for investing techniques 
related to hypergraphs in general.
}  
\p{\parlead{General-Purpose Hypergraph Parsers}  The parser outlined here 
I will term \q{general-purpose} because, although it outputs hypergraphs, 
it assumes that the languages being processed are not consciously 
designed for hypergraphs.  Instead, hypergraphs are useful because 
they provide greater flexibility in capturing features of the 
input language, which helps for designing formal 
(e.g., scripting, query, or markup) languages to a desired level 
of expressiveness and contest-sensitivity.  In particular, the 
parsing capabilities and space of parsable grammars is expanded 
relative to mechanisms like \BNF{} grammars.  (As a practical matter, 
a further benefit of methods such as I will outline is 
that they can be realized in a normal, e.g. \Cpp{} code library 
rather than needing a separate code-generator as in \LEXX{}/\YACC{}.)
}
   
\p{I dub the actual \Cpp{} parsing library (and corresponding 
hypergraph representation) \q{RelaeGraph} (for \q{Regular Expression 
Labeling and Annotation Engine}).  Annotations 
in this context are understood to be details supplementing a basic 
vocabulary of labels.  RelaeGraph grammars are 
mostly constituted by rules which each pair a regular expression 
(a formula describing patterns in the input stream) with 
code that is executed in response to a pattern match.  This 
code, in turn, typically calls functions to modify 
the output hypergraph by inserting a new hypernode and/or 
hyperedge.  A common scenario is that the text matched by a 
rule's regular-expression pattern forms the basis of a hypernode, 
while the specific rule that matched (and, for context-sensitive 
languages, the contexts in effect for the match) determines 
how a new hyperedge, connecting the new hypernode to the 
existing hypergraph, is labeled and annotated.
}
   
\p{It is interesting to consider the connections between regular-expression 
oriented grammars and graphs in general: a single regular-expression 
pattern-match can be pictured as part of an edge, with the 
matched text becoming a node, and the successful rule 
(via a name or identifier) supplying an edge label.  The 
current parsing context then determines a site where 
the new node and edge are added to an output graph under 
construction.  This picture is very approximate; in 
practice, further processing is necessary to establish correct 
node content, edge labels, and parsing context.  But 
it shows an intuitive parallel between parsing theory and 
hypergraphs: parsers identify pieces of input text as anchored 
into grammatical contexts, with relations between textual 
elements and surrounding (or distant) text stipulated by 
syntactic forms.  Analogously, hypernodes and hyperedges are 
tied to the surrounding material in their hypergraphs according 
to various patterns and contexts.  The extra detail offered 
via hypergraphs permits contextual details to be 
explicated more precisely and compactly.  Generalizing from 
graphs to hypergraphs is then analogous to (and, in practice, an 
implementation for) generalizing from context-free grammars to 
a class of more nuanced, context-sensitive grammars. 
}
\vsftcl{h}{figStringContext}
\p{In a context-sensitive grammar, not every rule-match will 
directly modify an output graph; instead, the result of a 
match might be to alter the parsing context.  For one example, 
Listing~\ref{lst:figStringContext} 
shows code (sampled from the data set) that works with 
string literals in script code, via a distinct string-literal 
parsing context (the code provisioned directly to 
a rule-match, as at \OneOverlay{}, delegates most 
logic to a \q{graph builder} class that acts as an intermediary 
between the parsing-specific code and the class which 
directly modifies the output hypergraph; cf. \ThreeOverlay{}).  
The current parsing context is tracked by a specific 
\lstinlinestyle{parse\_context\_} object which is modified
by the graph builder (the parsing engine accesses that object 
as well and only attempts rule-matches for rules 
described as active for the current context).\footnote{To be precise, context is set both by the named contexts that 
may be dynamically declared in a grammar and by boolean flags 
which assert finer-grained contexts with conjunction, disjunction, 
and negation operations.  A full context declaration 
can look like at \raisebox{-1pt}{\TwoOverlay{}} in Listing~\ref{lst:figStringContext}: 
\lstinline[columns=fullflexible]
{flags\_all\_(parse\_context ,inside\_string\_literal), run\_context}.
}  Overall then the parsing response operations 
are split between three main classes: one which defines 
rules and match-callbacks themselves; one 
responsible for maintaining parse context and flags, 
plus bridging the other two layers; and one which conducts 
most of the graph-transformation operations. 
}
   
\vsftcl{h}{figAddToken}
\p{Of these three parsing layers, one layer \q{faces} the 
input text directly \mdash{} the layer that defines grammar rules 
and match-callbacks.  Another layer \q{faces} the hypergraph 
output directly \mdash{} it adds (and occasionally rearranges) 
hypernodes and hyperedges.  The third layer acts as an 
intermediary.  Separating these layers establishes 
a separation of concerns so that the grammar-specific 
layer can focus on recognizing parsing structures for an 
expressive, flexible input language, without itself 
attending to graph operations.  Likewise, the 
components which \i{do} modify hypergraphs are isolated 
from the input language's parsing logic 
and syntactic design choices.
}
\p{Concretely, listing~\ref{lst:figAddToken} samples code 
logically following Listing~\ref{lst:figStringContext} wherein 
a new hypernode (such as, a string literal token) is 
inserted.  The relevant \Cpp{} class offers a variety of 
insertion and related graph-modification routines, without 
paying attention to the rationale for choosing 
each particular kind of modification at each particular moment.  
The middle \q{Graph Build} layer then receives calls from the 
parser about rule-matches and has a range of graph-modification 
routines it can select to handle each match (often based 
on contextual data tracked at the middle layer).  
So the first layer is responsible for generating fine-grained notifications 
about input patterns to the second layer, while the third layer 
is responsible for presenting a flexible set of operations from 
which the second layer selects the proper graph-modification 
for each match and context.
} 
   
\p{While this kind of compiler architecture is not intrinsically 
dependent on hypergraph forms at the output level, hypergraphs 
augment the benefits of how such a pipeline is organized.   
Compared to other kinds of parser output (like Abstract Syntax Trees 
or Directed Acyclic Graphs), hypergraphs have fewer structural 
requirements and restrictions and therefore have greater leeway 
in how data about input texts is perpetuated in the output structures.  
This accordingly offers greater flexibility in systematizing the 
input language's syntax and semantics.
}
\p{To illustrate this point, consider the 
goal to support special built-in datatypes with particular language-level 
syntax, such as embedding \XML{} directly in source code, or \q{sugaring} 
compound expressions within concise and readable forms (along the 
lines of how \LINQ{} \mdash{} Language Integrated Query \mdash{} packages 
complex method-calls into a fluid 
syntax resembling \SQL{}).
In a context-sensitive hypergraph grammar, 
such language features can be realized by defining a parsing context 
for the non-standard syntax and defining special node-types, edge-labels, 
or nested graphs to hold data parsed from input when the special 
contexts are in effect.  The resulting data, which may have complex 
internal structure, can be inserted into the overall graph as a single 
node and/or edge, taking advantage of hypergraphs' capability to 
work with different scales of organization.  Languages designed 
against the backdrop of this kind of parsing engine can flexibly 
evolve to support new syntactic special cases and constructs, 
in part because a hypergraph output format is a basically free-form 
tableau for stashing details and contexts observed in input text.
}
   
\p{With that said, the hypergraphs built directly from 
rule-matches will still closely reflect the input text and 
may need substantial processing to identify the proper 
information and semantics embedded in the input.  Insofar as the 
input is script-like computer source code, this semantic 
interpretation includes details such as assigning data types to 
input tokens and mapping script-level function calls to binary 
structures that can be used at runtime to effectuate corresponding 
\C{} or \Cpp{} calls.  The requisite semantic analysis depends 
on traversing and drawing information from hypergraphs produced 
via the parsing stages, but those graphs may need to be restructured 
so as to make the interpretive analysis \mdash{} rather than 
the initial parsing \mdash{} most convenient.  Such inter-graph transforms 
and analytics lie outside the context of parsing itself, so 
I will discuss these next.
}
   
\p{\parlead{Hypergraph Intermediate Representation}  In the demo code, 
the graphs produced during the parsing phase I just outlined are 
subject to several rounds of modification.  The first stages 
of this follow-up are performed in the context of the initial hypergraphs, 
and involve changes that can be tacked on to those graphs without 
substantial alteration of their structure.  For instance, many source-code 
tokens can be assigned a provisional type just by examining 
the token's character content (e.g., numeric literals start with a decimal 
digit, and string literals are enclosed in quotes).  Also, certain 
code segments can be necessarily associated with function calls and 
can therefore be pre-emptively accommodated with calls to some internal 
functions even at this early stage.  For example, the demo language 
implemented in the data set has a grammar feature where a token preceded 
(without space) by a comma \mdash{} as in the (type-declaration) line 
\lstinlinestyle{,fnd ::: Fn\_Doc*;} \mdash{} is usually the introduction of a
new lexically-scoped symbol, so it can trigger a pre-emptive call to 
a method that keeps track of source-code scopes (the quoted line 
also implies that the symbol \lstinlinestyle{Fn\_Doc} designates a type,
so a function would be called registering this as a known type).    
}
   
\p{Once this provisional analysis has run its course, the system then 
undertakes a more extreme reworking where the original hypergraph 
is morphed into a new graph with a different vocabulary and 
structural conventions.  Where as the earlier graph was built to 
prioritize integration with a hypergraph parser, the new graph will 
be built to prioritize integration with a language runtime.  
In the demo, the first graph produces the second by emitting a 
special kind of graph-serialization file.  This indirection 
is useful for testing, development, and exposition because the 
intermediate file can be studied as a (relatively human-readable) 
artifact in its own right.  It also allows me to review the 
general subject of a hypergraph serialization (or initialization) 
format, which will be the focus of the next few paragraphs.
}
   
\vsftcl{h}{figIR}
\p{One way to conceptualize hypergraph serialization is to treat the 
graphs as built up from an empty graph via a finite series of steps.  
A serialization format can then be designed on the premise of 
expressing each construction-step as a distinct instruction, and then 
serializing a whole graph as the sequence of all its implied 
steps.  Listing~\ref{lst:figIR} samples a file in the 
demo's format (which is based on Common Lisp to leave open the 
possibility of these files being read by a Common Lisp evaluator 
for pre-processing).  Here though the intermediate language is 
parsed directly in \Cpp{} \mdash{} unlike the parsers discussed earlier, 
which aim to work with flexible scripting languages, an intermediate 
hypergraph representation can be narrowly targeted to machine-generated 
hypergraph stages and needs only simplistic parsers.
For this language, each form (adopting the term from Lisp's name for a 
single expression) generally corresponds to a single \Cpp{} method.  
For development, the steps described via intermediate code can 
be analogously implemented as \Cpp{} code directly (the demo includes 
project files for running an example of that direct construction in 
lieu of intermediate files).  Accordingly, designing a 
serialization format for hypergraphs can be closely tied 
to the design of a (say, \Cpp{}) library for building 
graphs via a set of primitive operations.
}
   
\vsftcl{h}{figPGBIR}
\p{The demo format employs a notion of \q{multigraph tokens}, which 
are string representations of nodes' values along with a set 
of token kinds.  An encoding system allows the kind to be 
notated via a signal embedded in the token string, yielding a 
node-representation which can work across different varieties 
of hypergraphs (facilitating graphs' inter-translation).  
Listing~\ref{lst:figPGBIR} shows some 
code involved with routing input text to graph-build operations.  
Unlike the earlier parse strategy, the simple input language 
does not call for complex contextual logic to determine the correct 
operation; instead, the requisite operation is named explicitly in the 
input, so the parsing callback mostly reduces to a long 
dispatch table.  Care does have to be taken, however, to 
correctly anticipate which kinds of tokens are expected on the 
input in light of the graph-manipulation method that will be 
called when the tokens are parsed (some tokens hold raw data; 
others refer to pre-existing and pre-allocated nodes that are 
given names in the intermediate representation). 
} 
    
\p{The intermediate language designed via these techniques is effective 
from the point of view of machine-generating code to construct a 
new hypergraph from some prior data (including other hypergraphs).  
Rather than having components emit correct 
intermediate code directly, 
the demo library includes a \Cpp{} class which builds the 
\IR{} via a method interface.  Listing~\ref{lst:figPGB}  
shows code using this interface, treating the \IR{} output as a 
vector of (Lisp-style) forms and inserting forms at the end 
(or occasionally in the middle) of the vector 
(the methods for adding a form mostly have the same names as the 
methods which the forms themselves cause to be called, so 
programming via the \IR{} intermediary feels like 
programming with the target graph builder directly). 
}
  
\p{In this section I discussed the first hypergraph \mdash{} built 
via input from a scripting language \mdash{} as an \q{output} 
graph; but more precisely the digital output of the 
first several processing stages is a text file in the 
intermediate language I have summarized.  This discussion therefore 
follows the pipeline up until the second \q{generation} of hypergraph 
is initialized, from the \IR{} built off of the first graph.  
Explaining this second graph further, and the remaining steps toward 
a working language runtime engine, requires a discussion of 
\i{channels} and so will be picked up in the next section.
}
\p{I'll remark to conclude the current section that I believe the 
kinds of components and workflow outlined here \mdash{} not only the 
graph libraries but the supporting tools and code-generators \mdash{} are 
indicative of the sorts of infrastructural elements that would have to 
be standardized for a common hypergraph ecosystem modeled on 
(for example) the Semantic Web.  Replacing \RDF{} with an explicit 
hypergraph paradigm is not only a project of writing code libraries 
that structure data in a standardized hypergraph format; it is also a 
matter of complementing such libraries with additional components 
that send data into and pull data from hypergraphs via standardized 
languages, as well as via a standardized interface to connect hypergraphs 
with parsers targeting more general-purpose languages.  While this is 
more code that has to be written (and agreed upon, if we're 
envisioning a communally-adopted platform), it also means that 
standardizing the supporting formats is a reasonable proxy for 
standardizing the hypergraph models themselves.  After all, graph 
markup languages and similar tools currently exist; the limitation 
is they do not properly enable and recognize 
the full spectrum of structures and 
data that would characterize a genuine hypergraph ecosystem.  But, consequently, 
the formal contrast between fully general hypergraph models and the 
limited features currently supported by (e.g.) graph serialization 
languages can be explicated by enumerating \mdash{} via a new hypergraph 
serialization protocol \mdash{} all the desired constructions in the 
more general paradigm.
}
\vsftcl{!h}{figPGB}   
%\p{%}
