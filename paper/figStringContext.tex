\tmphs\begin{lstlisting}[caption={\emblink{\#lst--StringContext--\thelstlisting.pgvm-pdf}{Contextual Parsing for String Literals}},
language = C++, numbers = none, label={lst:figStringContext},
    basicstyle = \ttfamily\bfseries\tiny, linewidth = .95\linewidth] 

 add_rule( flags_all_(parse_context ,inside_string_literal), 
   run_context,  "string-literal-character",
   " (?: [^\"\\\\]+ ) | (?: (?:\\\\++)(?!\") ) ",
  [&]
 {
  QString str = p.match_text();
  graph_build.add_to_string_literal(str);
 });

 add_rule( flags_all_(parse_context ,inside_string_literal), 
  run_context,  "string-literal-maybe-end",
  " \\\\*\" ",
  [&]
 {
  // depends on how many back-slahes ...
  QString str = p.match_text();
  if(str.size() % 2) // odd mans even backslashes ...
  {
   if(str.size() > 1)
   {
    str.chop(1);
    graph_build.add_to_string_literal(str);
   }
   graph_build.process_string_literal();
   parse_context.flags.inside_string_literal = false;
  }
  else
  {
   graph_build.add_to_string_literal(str);
  }
 });

 add_rule( run_context, "string-literal-start",
  " \" ",
   [&]
 {
  graph_build.string_literal_start();
 });
...
void RE_Graph_Build::process_string_literal()
{
 caon_ptr<RE_Token> token = new RE_Token(string_literal_acc_);
 string_literal_acc_.clear();
 token->flags.is_string_literal = true;
 add_run_token(*token);
}

\end{lstlisting}
\begin{tikzpicture}[remember picture, overlay, text width=.95\linewidth]

\lstovn{6.6,7}{1}
\lstovn{8.1,10.05}{2}
\lstovn{3.25,2.6}{3}

%\node [circle, draw=blue,text width=3mm,text centered,
%fill=DarkRed,radius=3mm,inner sep=0pt] at (3.25,2.6){\ovn{1}};



%\draw[draw=blue,fill=DarkRed] (4,2.6) circle[radius=3mm] ;
%\node at (8.25,2.6){\ovn{1}};

%\draw[draw=blue,fill=DarkRed] (4,1.4) circle[radius=3mm];
%\node at (4,1.4){\ovn{2}};
\end{tikzpicture} 
