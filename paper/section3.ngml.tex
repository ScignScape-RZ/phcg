\section{Channels and Edge Diversification}
\p{Graph representations have obvious applications 
to modeling computer code and computations.  
Intuitively, consider procedures as nodes; 
edges incoming to or outgoing from nodes 
then may represent the value passed into or 
returned from procedures.  In order to be a 
reasonable code representation, however, this 
model has to capture further details via 
edge-annotations, grouping nodes and/or edges 
into higher-level units, or similar hypergraph 
constructions.
}
\p{As I argued last section, hypergraphs are useful target 
for general-purpose programming language parsers because 
these general languages tend to have relatively complex 
grammars and special syntactic contexts.  While 
programming languages in a theoretical sense may 
have an expected translation to Abstract Syntax Trees, 
real-world languages embrace syntactic forms 
that complicate this abstract picture.  Earier I cited 
\Csharp{}'s \LINQ{} as an example of an intra-linguistic 
special syntax; other examples include embedded \XML{} in 
Scala; attribute markings in \Csharp{} and \Java{}; 
template metaprogramming in \Cpp{}; regular-expression 
syntax in languages like \Perl{} and \Ruby{}; and so 
forth.  I have argued that a hypergraph (rather than 
graph or tree) paradigm is a more natural foundation 
for representing code in these practically evolving 
languages, which are designed around semantic expressiveness 
rather than an \i{a priori} commitment to theoretical 
compilation models. 
}
\p{As I will focus on in this section, similar comments apply to the 
semantics of procedures and function-calls.  The \q{mathematical} 
picture of procedures as functions mapping inputs to outputs is 
only approximately accurate in the context of modern programming 
languages.  One issue is that the distinction between input and output 
parameters is imprecise, given that input parameters can be 
modifiable references.  A common \Cpp{} idiom, for example, 
is to pass a non-constant reference instead of returning 
relatively large data structures \mdash{} mutable references 
are technically an input rather than output value, even 
if the intent is for a called procedure to supply data for 
the mutable input (this technique avoids extra copying of 
data).  Another complication is that procedures can have multiple 
sorts of input and output: e.g., throwing an exception rather than 
returning a value.  Likewise, Object-Oriented languages' 
\q{message receivers} are input parameters with distinguished 
properties as conmpared to normal inputs.
}
\p{Since programming notation visually distinguishes procedure inputs 
from outputs, it is easy to think of the input/output distinction 
as mostly syntactic.  The significant differences, however, 
are more semantic.  Consider a \Cpp{} function which returns an 
integer: it is a compiler error to return from such a function 
without suppying a value; also, the called procedure has 
no access to the memory which would hold the return value.  
By contrast, a function which takes a non-constant 
integer reference is not forced by the compiler to 
modify that value, nor prohibited from accessing the 
input's value.  This points to the practical differences 
between returning values and altering input references, but it 
also indicates that inputs can be treated as \i{de facto} 
outputs by adhering to usage conventions \mdash{} a reference 
mimics the behavior of a return value to the extent that 
procedures initialize the reference on all execution paths, and 
refrain from using any value which the reference might 
have before the procedure starts.  Generalizing this point, 
the badic contrast between input and output can be 
seen as a semantic matter of usage protocol, rather 
than a syntactic contrast between passing valued to a 
function vs. accessing a function's returned result.
}
\p{Of course, input and output are usually both distinguished 
and interconnected by how outputs are used as inputs to 
other functions.  With respect to procedures which take 
one parameter, the general function-composition operator 
\fOfG{} takes the output of \gFun{} as the input to 
\fFun{}.  We could of course define a different operator 
where \gFun{} and then \fFun{} both run, taking the same 
value (potentially a reference-type value that \gFun{} 
could potentially modify before \fFun{} runs).  The 
former (composition) operator establishes an inevitable 
dependence of \fFun{} on \gFun{}: \fFun{} needs \gFun{}'s 
output for an input value.  The second case is less obvious 
\mdash{} in a sequential call like \fAfterG{} (\gFun{} runs then 
\fFun{} runs \mdash{} I notate the relation with an inverted 
semi-colon to highlight the parallels with \fOfG{}, but 
the analogous case in practice would be \gThenF{}), 
there may or may not be a \gFun{} side-effect that 
infuences \fFun{} (even if they are passed the same 
inputs).  So inputs and outputs are distinguished 
by how output-to-input links \mdash{} outputs from one procedure 
becoming inputs to another \mdash{} have certain semantic 
properties in the context of functional composition, and by 
extension the relations between precedures which 
run in sequence.  Again, though, this call for a semantic 
analysis which looks beyond languages' surface syntax for 
function calls, and how usages of return valued are notated.
}
\p{Mutable references and procedures-with-side-effects expose limits 
in the representational expressivity of graphs (even hypergraphs) 
for modelling inter-procedure relationships.  Consider a case 
such as represented in Figure~\ref{fig:figGF} 
\begin{figure}
\centering
\begin{tikzgrid} 
    \node[]  (nn)                   {$n$}; 
    \node[]  (fng)   [right=of nn]  {\gFunB};
    \node[]  (fnf)   [right=of fng]  {\fFunB};

    \path[->]
    (nn) edge   node {\textit{input}}           (fng)
    (fng) edge                  node {\textit{modifies}}    (fnf);

    \node[]  (code)  [below right = 5.5mm and 1mm of nn]   {\lstinlinebstyle{g(n); f(n)}}; 

    \node[]  (nn1)   [below = 21mm of nn]                {$n$}; 
    \node[]  (fng1)   [right=of nn1]  {\gFunB};
    \node[]  (fnf1)   [right=of fng1]  {\fFunB};
    
    \path[->]
    (nn1) edge   node {\textit{input}}           (fng1)
    (fng1) edge                  node {\textit{output}}    (fnf1);
    
    \node[]  (code1)  [below right = 5mm and 1mm of nn1]   {\lstinlinebstyle{f(g(n))}}; 
        

\end{tikzgrid}
\tmphscol\caption{\emblink{\#fig--GF--\thefigure.pgvm-pdf}{Parameter Modification vs. Output}} \label{fig:figGF}
\end{figure}
, where \gFun{} modifies 
input $n$ that is then also input to \fFun{}.  Here \gFun{}'s 
effect on \fFun{}'s input is marked by a \gFun{}-to-\fFun{} 
arrow, but this directed edge implies a different semantic 
relation than \gFun{}'s output being \fFun{}'s input 
(as in \fOfG{}).  The specific difference can be visualized 
in a scenario like Figure~\ref{fig:figGFcomplex},\begin{figure}
\centering
\begin{tikzgrid}
    \node[]  (xy)                   {$x$,$y$}; 
    \node[]  (fng)   [right=of xy]  {\gFunB{$x$}{$y$}}; 
    \node[]  (zz)    [right=of fng]              {$z$}; 
    \node[]  (fnf)   [right=of zz]  {\fFunB{$x$}{$y$}{$z$}}; 
    
    
     \node[draw=darkRed!50,
     fill=red!20,line width=1mm,minimum width=16mm,minimum height=5mm]  (exp)  
     [above right=6.5mm and -5mm of fng]  {}; 
     
     \node[draw=darkRed!50,
     fill=red!20,thin,minimum width=42mm,minimum height=13mm]  (channel)  
     [above right=9mm and 18.5mm of fng]  {}; 
     
     \node[
     fill=red!20,thin,minimum width=16mm,minimum height=5mm]  (exp)  
     [above right=6.5mm and -5mm of fng]  {{\footnotesize Channel:}};    

 %   \path[->]
    \draw   (xy) edge [below, bend right=10, ->,-to]  node {\textit{input}} (fng);
    \draw   (xy) edge [bend left=10, ->,-to]  node {} (fng);
    \draw   (fng) edge [below, ->]  node [xshift=-8pt]{\textit{output}} (zz);
    \draw   (zz) edge [below, ->]  node [xshift=-2pt]{\textit{input}} (fnf);
     %end path

   \draw (fnf.north)+(-19mm,2mm) edge [yshift=5mm,
     shorten <=-19mm, shorten >= 2mm, bend right = 41, <-,>=to]  
     node [yshift=5mm,xshift=-12mm] {modified input} (fng);

   \draw (fnf.north)+(-14mm,2mm) edge [yshift=1mm,
     shorten <=-11mm, shorten >= 2mm, bend right = 24, <-,>=to]  
     node [yshift=5mm,xshift=-12mm] {} (fng);

    \node[]  (code)  [below right = 15mm and 15mm of xy]   {\lstinlinebstyle{z = g(x, y); f(x, y, z)}}; 

 
\end{tikzgrid}
\caption{Edge Aggregation/Diversification via Channels} \label{fig:figGFcomplex}
\end{figure}
 
based on code such as written below the 
graph.  The situation here is that \fFun{} both uses 
\gFun{}'s output return (via the symbol $z$ the value is assigned to) 
and also the values held by $x$ and $y$ which are altered by 
\gFun{}.  I notate these dependencies by grouping the \gFun{}-to-\fFun{} 
edges through $x$ and $y$ separately from the edge 
with $z$ \mdash{} the latter edge conforms to the relation of 
a procedure taking another procedure's output as input 
(although here the relation is indirect through a 
lexically-scoped symbol), but the former edges reflect a 
different inter-procedure pattern.  The point of this 
example is that separating input and output edges is not 
enough to express program-flow semantics in anything but 
pure functional languages \mdash{} even given the \q{hypergraph categories} 
strategy of grouping input an output values into hypernodes.
}
\p{For situations like these I propose a notion of \i{channels}, which are 
higher-scale groupings of edges analogous to how hyperedges 
involve higher-scale groupings of nodes.  In Figure~\ref{fig:figGFcomplex}, 
for instance, the $x$ and $y$ edges can be considered a separate 
channel than the $z$ edge, because $x$ and $y$ bridge \fFun{} and \gFun{} 
according to one sort of program-flow semantics (with side effects and 
mutable inputs passed to sequential procedures) while $z$ follows a 
different pattern (with output values and lexical symbols).  
The shaded box in the figure shows the core role of channels, 
grouping together edges that have some particular connection 
which could not be expressly identified by other means (like 
shared annotations) and also isolating edges and edge-groups 
from other elements (here, isolating the $x$-and-$y$ pairing 
from the $z$ edges).   
}
\p{This case illustrates the basic premise of channels as 
distinguished and/or edge-groupings, and also one modelling 
domain where I contend channels can be applied; specifically, 
representing details in computer code (in a general hypergraph 
representation of source code) such as program flow.  
To discuss channels as graph elements further, I will 
start by examining channels in the specific computer-code 
context, which calls for a detour review of 
certain semantic structures central to the semantic of computer programs. 
}
\p{\parlead{Typed Values and \q{Careers}}  It is commonly accepted that a 
theory of types \mdash{} with at least some details adopted from mathematical 
type theory \mdash{} is a necessary component of a rigorous treatment of 
computer software and programming languages.  However, we should not 
neglect to keep in mind the specific realities of programming 
environments that complicate such basic notions as types' 
instances, the nature of entities that we should deem to be 
bearers of types, the status of special (non-)values like 
\lstinline{null} or \q{Not A Number}, and related details that 
contrast programming \q{types} from mathematical sets or \q{spaces} 
in various incarnations (sheaves, topoi, etc.).  
}
\p{}
