\begin{lstlisting}[caption={Identifying Function 
 Equivalence Classes},
language = C++, numbers = none, label={lst:figPVMrt},
    basicstyle = \ttfamily\bfseries\tiny, linewidth = .95\linewidth] 

void PHR_Command_Runtime_Router::proceed_s0_2(
  PHR_Function_Vector* pfv, void** pResult)
{
 s0_fn1_p_p_type fn = nullptr;
 bool sr; int mc = 0, bc = 0, byte_code = 0;
 if(pfv)
 {
  if(void* fnp = pfv->match_against_codes({6002, 7002}, 
    mc, bc, &result_type_object_))
  {
   fn = (s0_fn1_p_p_type) fnp;
   byte_code = bc;
   sr = mc == 6002;
  }
  else byte_code = 0;
 } 
... 
 if(fn)
 {
  proceed_s0<2, s0_fn1_p_p_type>(pResult, fn, 
    byte_code, sr, false);
 } 
...
}
...
template<>
void PHR_Command_Runtime_Router::proceed_s0_r<2, 
  s0_fn1_p_p_type>(QVector<quint64>& args, void*& result,
  s0_fn1_p_p_type fn, int byte_code)
{
 switch(byte_code)
 {
 case 944: tybc_<2>::run<944, s0_fn1_p_p_type>(args, result, 
   fn); break;
 case 984: tybc_<2>::run<984, s0_fn1_p_p_type>(args, result, 
   fn); break; 
 ...
 }
}
...
struct tybc_<2>
{
 template<int byc, typename fn_type>
 static void run(QVector<quint64>& args, void*& result,
   fn_type fn)
 {
  result = ( (typename tybc<byc>::_r) fn)
    ( (typename tybc<byc>::a0) *((quint64*) args[0]),  
    (typename tybc<byc>::a1) *((quint64*) args[1]));
 }
}
\end{lstlisting}
\begin{tikzpicture}[remember picture, overlay, text width=.95\linewidth]

\lstovn{4.6,5.6}{1}
\lstovn{6.05,1.95}{2}

%\lstovn{3.25,2.6}{3}

\end{tikzpicture} 
