\section{Procedures and Integration}
\p{So far I have criticized paradigms which try to account for linguistic 
meaning via concordance between linguistic and proppositional 
structur; the shape of predicate complexes.   This critique has two dimnsions: 
first, although a predicate structure, a predicative specificity, does indeed 
permeate  states of affairs insofar as we engage thm rationally, such 
logical order is not modeled by language itself so much as by cognitive 
pictures we develop via interprtive processes \i{triggred} by language dtails 
but, I believe, to some not insubstantial degree pre- or xtra-linguistic.  
Moreover, second, insofar as we \i{can} decelop formal models of 
language, these are not going to be modls of prdicate structure 
in any conventional sense.  Cognitive-interpretive processes may 
have formal structure \mdash{} structure which may even show a lot of overlap 
with propositional forms \mdash{} but these are not \i{linguistic} structures.   
Insofar as language triggers but does not constitute interpretive 
\q{scripts}, the scripts themselves (i.e., conceptual prototypes and 
perceptual schema we keep intelectually at hand, to interpret and 
act constructively in all situations, linguistic and otherwise) are 
not linguiatic as such \mdash{} and neither is any propositional order they 
may simulate.  Language \i{does}, however, structure the \i{integration} 
of \i{multiple} interpretive scripts, so the structure of this 
integration \i{is} linguistic structure per se \mdash{} and formally 
modeling such integration can be an interesting tactic for 
formally modeling linguistic phenomena.  However, 
we should not assume that such a formal model will reemble or 
be reducible to formal logic in any useful way \mdash{} formalization 
does not automatically entail some kind of de facto isomorphism 
to a system of logic (if not first-order then second-order, modal, etc.).  
}
\p{Instead, I want to focus in on branches of computer science and 
mathematics (such a process algebra, which I have already referenced) 
as part of our scientific background insofar as the \i{structural 
integration} of diverse \q{processes} (computational processes 
in a formal sense, but perhaps analogously 
cognitive processes in a linguistic sense) can be technically represented.  
}
\p{}
