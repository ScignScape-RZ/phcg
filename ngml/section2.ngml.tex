\section{Procedures and Integration}
\p{My central thesis in this paper is that language understanding 
involves integrating diverse \q{cognitive procedures}, 
each associated with specific words, word morphologies (plural 
forms, verb tense, etc) and sometimes phrases.  
This perspective contrasts with and adds nuance to 
a more \q{logical} or \q{truth-theoretic} paradigm which 
tends to interpret semantic phenomena via formal logic \mdash{} 
for example, singular/plural in Natural Language 
as a basically straightforward translation of the 
individual/set distinction in formal logic.  Such 
formal intuitions are limited in the sense that (to 
continue this example) the conceptual mapping from 
single to plural can reflect a wide range of 
residual details beyond just quantity and multitudes.  
Compare \i{I sampled some chocolates} (where the count-plural 
suggests \i{pieces} of chocolate) and 
\i{I sampled some coffees} (where the count-plural implies 
distinguishing coffees by virte of grind, roast, and other 
differences in preparation) (note that both are contrasted 
to mass-plural forms like \i{I sampled some coffee} where 
plural agreement points toward material continuity; there 
is no discrete unit of coffee qua liquid).  Or compare 
\i{People love rescued dogs} with \i{People fed the rescued dogs} 
\mdash{} the second, but not the first, points toward 
an interpretation that certain \i{specific} people 
fed the dogs (and they did so \i{before} the dogs 
were rescued).
}
\p{The assumption that logical modeling can capture all the 
pertinent facets of Natural-Language meaning can 
lead us to miss the amount of situational reasoning 
requisite for commonplace understanding.  In 
\i{People fed the rescued dogs} thre is an exception to the 
usual pattern of how tense and adjectival modification 
interact: in \i{He has dated several divorced millionaires} 
it is implied that the ladies or gentlemen in question 
were divorced and millionaires \i{when} he dated them: that the 
events which gave them these properties occurred \i{before} 
the time frame implied (by tense) as the time of reference 
for the states of affairs discussed in the sentence 
(jokes or rhetorical flourishes can toy with thse expectations, 
but that's why they \i{are}, say, jokes; consider the dialog: 
\q{He likes to date divorced women \mdash{} I thought they were all married? 
\mdash{} Not after he dated them!}). But we read \q{people fed} in 
\i{People fed the rescued dogs} as occurring before the rescue; 
because we assume that \i{after} being rescued the dogs would be 
fed by veterinarians and other professionals (who would 
probably not be designated with the generic \q{people}), and 
also we assume the feeding helped the dogs survive.  We also 
hear the verb as describing a recurring event; compare 
with \i{I fed the dog a cheeseburger}.   
}
\p{To be sure, there are patterns and templates governing 
scope/quantity/tense interactions that help us build logical models 
of situations described in language.  Thus 
\i{I fed the dogs a cheeseburger} can be read such that there 
are multiple cheeseburgers \mdash{} each dog gets one \mdash{} 
notwithstanding the singular form on \q{a cheeseburger}: 
the plural \q{dogs} creates a scope that can elevate 
the singular \q{cheeseburger} to an implied plural: 
the discourse creates multiple reference frames each 
with one cheeseburger.  Likewise this morphosyntax is 
quite correct: \i{All the rescued dogs are taken to an 
experienced vet; in fact, they all came from the same 
veterinary college}: the singular \q{vet} is properly 
alligned with the plural \q{they} because of the scope-binding 
(from a syntactic perspective) and space-building 
(from a semantic perspective) effects of the \q{dogs} plural.  
Or, in the case of \i{I fed the dog a cheeseburger every day} 
there is an implicit plural because \q{every day} builds 
multiple spaces: we can refer via the spaces collectively 
using a plural (\i{I fed the dog a cheeseburger every day \mdash{}  
I made them at home with vegan cheese}) or refer within 
one space more narrowly, switching to the singular 
(\i{Except Tueday, when I made it out of ground turkey and swiss}).  
}
\p{Layers of scope, tense, and adjectives interact in comples ways that 
leave room for common ambiguities: \i{All the rescued dogs are [/were] 
taken to an experienced [/specialist] vet} is consistent with a reading 
wherein there is exactly one vet, and she has or had treated every dog, as 
well as where there are multiple vets and each dog is or was treated 
by one or another.  Resolving such ambiguities 
tends to call for situational reasoning and a \q{feel} for situations, 
rather than brute-force logic.  If a large dog shelter describes 
their operational procedures over many years, we might assume 
there are multiple vets they work or worked with.  If instead the 
conversation centers on one specific rescue we'd be 
inclined to imagine just one vet.  Lexical and tense 
variation also guides these impressions: the past-tense 
form (\q{...the rescued dogs were taken...}) nudges us 
toward assuming the discourse references one rescue (though it 
could also be a past-tense retrospective of general operations).  
Qualifying the vet as \i{specialist} rather than the vaguer 
\i{experienced} also nudges us toward a singular interpretation. 
}
\p{What I am calling a \q{nudge}, however, is based on situational 
models and arguably flows from a conceptual stratum outside 
of both semantics and grammar proper, maybe even prelinguistic.  
There appears to be no explicit principle either in the semantics 
of the lexeme \q{to feed}, or in the relevance tense agreements,  
stipulating that the feeding in \i{People fed the rescued dogs} 
was prior to the rescue (or conversely that 
\i{Vets examined the rescued dogs} describes events 
\i{after} the rescue).  Instead, we interpret the discourse through 
a narrative framework that fills in details not provided by 
the language artifacts explicitly (that abandoned dogs are 
likely to be hungry; that veterinarians treat dogs in clinics, which 
dogs have to be physically brought to).  For a similar case-study, 
consider the sentences:   
\begin{sentenceList}\sentenceItem{Every singer performed two songs.}
\sentenceItem{Everyone performed two songs.}  
\sentenceItem{Everyone sang along to two songs.} 
\sentenceItem{Everyone in the audience sang along to two songs.} 
\end{sentenceList}
The last of these examples strongly suggests that of potentially 
many songs in a concert, exactly two of them were popular and singalbe 
for the audience.  The first sentence, contrariwise, fairly strongly 
implies that there were multiple pairs of songs, each pair performed 
by a different singer.  The middle two sentences imply either 
the first or last reading, respectively (depending on how we 
interpret \q{everyone}).  Technically, the 
first two sentences imply a multi-space reading and the latter two 
a single-space reading.  But the driving force 
behind these implications are the pragmatics of \q{perform} versus 
\q{sing along}: the latter verb is bound more tightly to 
its subject, so we hear it less likely that 
\i{many} singers are performing \i{one} song pair, or conversely that 
every audience member \i{sings along} to one song pair, but 
each chooses a \i{different} song pair.
}
\p{The competing interpretations for \i{perform} compared to  
\i{sing along}, and \i{feed} compared to \i{treat}, are grounded 
in lexical differences between the verbs, but I contend 
the contrasts are not laid out in lexical specifications 
for any of the words, at least so that the implied readings 
follow just mechanically, or on logical considerations 
alone.  After all, in more exotic but not implausible 
scenarios the readings would be reversed:
\begin{sentenceList}\sentenceItem{The rescued dogs had been treated by vets before 
(but were subsequently abandoned).}
\sentenceItem{Every singer performed (the last) two songs 
(for the grand finale).}  
\sentenceItem{Everyone in the audience sang along to two songs 
(they were randomnly handed lyrics to different songs when 
they came in, and we asked them to join in when the song being 
performed onstage matched the lyrics they had in hand).}
\end{sentenceList}
In short, it's not as if dictionary entries would specify that 
\q{to feed} applies to rescued dogs before they are rescued, 
and so forth; these interpretations are driven by narrative 
construals narrowly specific to given expressions.  The 
appraisals would be very different for other uses of the verbs in 
(lexically) similar (but situationally different) cases: 
to \q{treat} a wound or a sickness, to \q{perform} a gesture or a 
play.  One can assert \i{The city's largest theater company 
will perform \q{The Flies}} without implying that the 
Board of Directors will actually take the stage (the President 
as Zeus, say).  We construct an interpretive scaffolding 
for resolving issues like scope-binding and space-building based 
on fine-tuned narrative construals that can vary alot 
even across small word-sense variance: 
\begin{sentenceList}\sentenceItem{The city's largest theater company 
performed \q{The Flies} in French, but everyone's accent 
sounded Quebecois.} 
\sentenceItem{The city's largest theater company 
performed \q{The Flies}; then they invited a professor 
to discuss Sartre's philosophy when the play was over.}
\end{sentenceList}
In the first sentence, the \q{space} built by the sentence is wider 
initially but narrows to encompass only the actual actors on stage.  
In the second, the \q{space} narrows in a different direction, since 
we hear a programming decision like pairing a performance with a 
lecture as made by a theater's board rather than its actors.  
As we follow along with these sentences, we have to build a narrative 
and situational picture which matches the speaker's intent, 
sufficiently well.  And that requires prelinguistic background 
knowledge which is leveraged and activated (but not mechanically 
or logically constructed) by lexical, semantic, or grammatical 
rules and forms: \i{rescued dogs} all alone constructs a fairly 
detailed mental picture where we can fill in many details by 
default, unless something in the discourse tells otherwise 
(we can assume such dogs are in need of food, medical care, 
shelter, etc., or they would not be describd as 
\q{rescued}).  Likewise \q{sing along} carries a rich mental 
picture of a performer and an audience and how they interact, one 
which we understand based on having attended concerts rather than 
by any rule governing \q{along} as a modifier to \q{sing} 
\mdash{} compare the effects of \q{along} in \q{walk along}, 
\q{ride along}, \q{play along}, \q{go along}.  Merely 
by understanding how \q{along} modifies \q{walk}, say 
(which is basically straightforward; to 
\q{walk along} is basically to \q{walk alongside}) we 
would not automatically generalize to more idiomatic 
and metaphorical uses like \q{sing along} or \q{play along} 
(as in \i{I was skeptical but I played along (so as not to 
start an argument)}).  
}
\p{We have access to a robust collection of \q{mental scripts} which 
represent hypothetical scenarios and social milieus where 
language plays out.  Language can activate various such 
\q{scripts} (and semantic as well as grammatical formations 
try to ensure that the \q{right} scripts are selected).  
Nonetheless, we can argue that the conceptual and cognitive 
substance of the scripts comes not from language per se 
but from our overall social and cultural lives.  
We are disposed to make linguistic inferences \mdash{} like 
the timeframes implied by \i{fed the rescued dogs} or the scopes 
implied by \i{sang along to two songs} \mdash{} because of 
our enculturated familiarity with events like dog rescues 
(and dog rescue organizations) and concerts 
(plus places like concert halls).  These concepts are not 
produced by the English language, or even by any dialect 
thereof (a fluent English speaker from a different 
cultural background would not necessarily make the 
same inferences \mdash{} and even if we restrict attention to, 
say, American speakers, the commonality of disposition 
reflects a commonality of the relevant cultural 
anchors \mdash{} like dog rescues, and concerts \mdash{} rather than 
any homogenizing effects of an \q{American} dialect).  
For these reasons, I believe that trying to account for 
situational particulars via formal language models alone 
is a dead end; which does not mean formal language 
models are unimportant, only that we need to picture them 
resting on a fairly detailed prelinguistic 
world-disclosure.
}
\p{There are interesting parallels in this thesis to the role 
of phenomenological analysis, and the direct thematization 
of issues like attention and intentionality: analyses 
which are truly \q{to the things themselves} should take for 
granted the extensive subconscious reasoning that undergirds 
what we consciously thematize and would be aware of, in terms of 
what we deliberately focus on and are conscious of 
believing (or not knowing), for a first-personal expose.  
Phenomenological analysis should not consider itself as 
thematizing every small quale, every little patch of 
color or haptic/kinasthetic sensation which by some subconscious 
process feeds into the logical picture of our surroundings that 
props up our conscious perception.  Analogously, linguistic 
analysis should not thematize every conceptual and inferential 
judgment that guides us when forming the mental, situational 
pictures we then consult to set the groundwork for linguistic 
understanding proper.
}
\p{Thse comments apply to both concptual \q{background knowledge} and 
to situational particulars of which we are cognizant in 
rference to our immediate surroundings and actions.  This 
is the perceptual and operational surrounding that gets 
linguistically embodied in deictic reference and other 
contextual \q{groundings}.  Our situational awareness therefore 
has both a conceptual aspect \mdash{} while attending a concert, 
or dining at a restaurant, say, we exercise cultural background 
knowledge to interpret and participate in social events 
 \mdash{} and also our phenomenological construal of our locales, 
our immediate spatial and physical surroundings.  
Phenomenological philosophers have explored in detail how these 
two facets of situationality interconnect (David Woodruff Smith and 
Ronald McIntyre in \i{Husserl and Intentionality: 
A Study of Mind, Meaning, and Language}, for instance).  
Cognitive Linguistics covers similar territory; the \q{cognitive} 
in Cognitive Semantics and Cognitive Grammar generally tends 
to thematize the conception/perception interface and 
how both aspects are merged in situational understanding 
and situationally grounded linguistic activity (certainly 
more than anything involving Artifcial Intelligence or 
Computational Models of Mind as are connoted by terms like 
\q{Cognitive Computing}).  Phenomenological and Cognitive 
Linguistic analyses of situationality and perceptual/conceptual 
cognition (cognition as the mental synthesis of 
preception an conceptualization) can certainly enhance and 
reinforce each other.  
}
\p{But in addition, both point to a cognitive and situational 
substratum underpinning both first-person 
awareness and linguistic formalization proper \mdash{} in other words, 
they point to the thematic limits of 
Phenomenology and the analytic boundary where they give way to 
an overarching Cognitive Science.  In the case of 
Phnomenology, there are cognitive structures that suffuse  
conscipusnss without being directly objects of attention 
or intention(ality), just as sensate hyletic experience is 
part our consciousness but not, as explicit content, 
something we in the general case are conscious \i{of}.  
Analogously, conceptual and situational models 
permate our interpretations of linguistic forms, but 
are not presented explicitly \i{through} these 
forms: instead, they are solicited obliquely and 
particularly.
}
\p{}
