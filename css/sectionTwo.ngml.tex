\section{Link Grammar and Type Theoretic Semantics}
\p{From one perspective, grammar is just a 
most top-level semantics, the primordial Ontological division of language into designations of 
things or substances (nouns), events or processes (verbs), qualities and attributes (adjectives), 
and so forth.  Further distinctions like count, mass, and plural nouns add 
semantic precision but arguably remain in the orbit of grammar (singular/plural 
agreement rules, for example); the question is whether semantic detail gets 
increasingly fine-grained and somewhere therein lies a \q{boundary} between syntax and 
semantics.  The mass/count distinction is perhaps a topic in grammar more so than 
semantics, because its primary manifestation in language is via agreement 
(\i{some} wine in a glass; \i{a} wine that won a prize; \i{many} wines 
from Bordeaux).  But are the distinctions between natural and constructed objects, 
or animate and inanimate kinds, or social institutions and natural 
systems, matters more of grammar or of lexicon?  Certainly they engender 
agreements and propriety which appear similar to 
grammatic rules.  \q{The tree wants to run away from the dog} sounds wrong \mdash{} because 
the verb \q{want}, suggestive of propositional attitudes, seems incompatible with 
the nonsentient \q{tree}.  Structurally, the problem with this sentence seems analogous 
to the flawed \q{The trees wants to run away}: the latter has incorrect singular/plural linkage, 
the former has incorrect sentient/nonsentient linkage, so to speak.  But does this 
structural resemblance imply that singular/plural is as much part of semantics as grammar, or 
sentient/nonsentient as much part of grammar as semantics?  It is true that there are no 
morphological markers for \q{sentience} or its absence, at least in English \mdash{} except 
perhaps for \q{it} vs. \q{him/her} \mdash{} but is this an accident of English or revealing 
something deeper?
}
\p{To explore these questions it is first necessary to consider how a grammar theory can be 
extended to and/or connected with a formal or, to some measure, informal semantics.  
Here I will present one approach to make this extension \visavis{} Link Grammar.
}
\p{Insofar as grammatic categories do provide a very basic \q{Ontological} viewpoint, 
it is reasonable to build semantic formalization on top of grammar theories.  
Link Grammar, for example, explicitly derives \q{link types} \mdash{} species of word-to-word 
relations \mdash{} by appeal to \q{Categorial} grammars which define parts of 
speech in terms of their manner of composition with other, more \q{fundamental} parts 
of speech \cite{Kiselyov}, \cite{Rossi}, \cite{MartinPollard}.  The most primordial 
grammatic categories are generally seen to be nouns and 
\q{propositions} (self-contained sentences or sentence-parts which assert individual states of 
affairs), and categories like verbs and adjectives are derived on their basis.  For example, a 
verb \q{combines} with a noun to produce a proposition.  \i{Students} is an abstract 
concept; \q{Students complained}, tieing the noun to a verb, tethers the concept to an 
assertorial flesh, yielding something that expresses a belief or observation.  
Meanwhile, Categorial Grammar models not only the semantic transition from abstract to concrete, but 
surface-level composition: in English and other \SVO{} language for example the verb should 
immediately follow the noun; in German and all \SVO{} languages the verb tends the come 
last in a sentence, and can be well apart from its subject.  The semantic pattern in the 
link is how the verb/noun pair yields a new semantic category (propositional) whereas the 
grammatic component lies in how the link is established relative to other words 
(to the left and not the right, for example, and whether or not the words are adjacent).  
}
\p{Assuming that surface-level details can be treated as grammar rules and abstracted from 
the semantics, we can set aside Categorial Grammar notions like connecting 
\q{left} vs. \q{right} or \q{adjacent} (near) vs. \q{nonadjacent} (far).  
With this abstracting, Categorial Grammar becomes similar 
to a Type-Theoretic Semantics which recognizes, in Natural Language, operational 
patterns that are formally studied in mathematics and computer science
\cite{ZhaohuiLuo}, \cite{MaillardClarkGrefenstette}, 
\cite{ZhaohuiLuo}, \cite{Pollard}, \cite{MeryMootRetore}.
A verb, for example, 
\i{transforms} a noun into a sentence or proposition (at least an intransitive verb; 
other kinds of verbs may require two, or even three nouns).  In some schematic sense a 
verb is analogous to a mathematical \q{function}, which \q{takes} one or more nouns 
and \q{yields} propositions, much like the \q{square} function takes a real number and 
yields a non-negative real number.  To make this analogy useful, however, 
it is necessary to clarify how \q{types} in a mathematical or computational 
context may serve as appropriate metaphors for syntactic and/or semantic groupings in language.
}
\subsection{Types, Sets, and Concepts}
\p{Most Computer Science rests on types rather than (for example) sets, 
because abstract reasoning about data types 
requires some abstraction from practical limitations about how particular values may be 
digitally encoded.  
Types can be defined as sets of both values and 
\q{expectations} \cite{MathieuBouchard} (meaning assumptions which may be made about 
all values covered by the type); alternatively, we can (perhaps better) consider types as 
\i{spaces} of values.  Types' extensions have internal structure; there 
can be \q{null} or \q{invalid} values, default-constructed values, and 
so forth, which are \q{regions} of type-space and can 
be the basis of topological or Category-Theoretic rather than 
set-based analyses of type-extension.  Also, expectations intrinsically include 
functions which may be \q{called on} types.  There is definitional interdependence 
between types and functions: a function is defined in terms of the types it accepts as parameters and 
returns \mdash{} rather than its entire set of possible inputs and outputs, which can 
vary across computing environments.  These are some reasons why in theoretical 
Computer Science types are not \q{reduced} to underlying sets; instead, extensions 
are sometimes complex spaces that model states of, or internal organization of comparisons 
among, type instances.
}
\p{An obvious paradigm is organizing type-extensions around prototype/borderline
cases \mdash{} there are instances which are clear examples of types and
ones whose classification is dubious.  I will 
briefly argue later, however, that common resemblance is not always a good marker
for types being well-conceived \mdash{} many useful concepts are common
precisely because they cover many cases, which makes defining
\q{prototypes} or \q{common properties} misleading; this reasoning
arguably carries over to types as well.  Also, sometimes the clearest
\q{representative} example of a type or concept is actually not a
\i{typical} example: a sample latter or model home is actually not (in
many cases) a real letter or home.  So resemblance-to-prototype is at
best one kind of \q{inner organization} of concepts' and types' spaces
of extension.  Computer Science develops other 
pictures of types' \q{state space}, reflecting the trajectory of symbols 
or channels which hold type instances, which at different moments in time 
become initialized \mdash{} acquiring a value obtained from a \i{constructor} 
function (one \q{type space region} is then demarcated by which values 
can be direct results of constructors) \mdash{} then possibly subject to change in the value they hold, and finally 
(often) transitioning to a state where the held value is no longer \q{valid}.\footnote{Managing the \q{lifetime} of values from many types, especially \q{pointer} types 
(that hold a numeric value representing the current memory address of some other value),
has been a notorious source of programming errors, especially in older computer 
languages.  Of late, also, data types often need to be designed to minimize the 
risk of data corruption, theft, and malicious code.  For these reasons, 
Cybersecurity takes particularly interest in studying types' extensions and transitions 
between different values (morphisms within a type space) to formally describe 
states or state-transitions which are security vulnerable.
}  Type \i{spaces} have potentially complex patterns of regions and equivalence classes of 
inter-value mappings (in the sense of behavioral equivalence relative to 
code analysis, testing, or security) 
\mdash{} the \i{conceptual} properties of types are expressed in the 
\i{internal structuration} of their associated state-space.  Putting this in mathematical 
language, an in-depth treatment of types cannot work \q{in the Category} of 
sets, even for basic type-extension, but rather (for instance) the 
Category of Topological Spaces.
}
\p{Moreover, expectations in a particular case 
may be more precise than what is implied by the type itself \mdash{} it is erroneous 
to assume that a proper type system will allow a correct \q{set of values} to 
be stipulated for each point in a computation (the kind of contract enforced via 
by documentation and unit testing).  So state-space in a given context may include many
\q{unreasonable} values, implying that within the overall space there is a \q{reasonable}
subspace, except that this subspace may not be crisply defined.  
A value representing someone's age may be assigned 
a type for which a legal value is, say, 1000 years, which is obviously unreasonable 
\mdash{} the conceptual role served by the \i{particular} use of a type in some context can 
be distinct from the entire space of values exhibited by the type.  It is possible to 
construct types which are narrowed down to more precise ranges, but in many cases this is 
unnecessary or poorly motivated: while 1000 years is clearly too large for an age, 
it would be arbitrary to specify a \q{maximum allowed} age (recall that assuming a 
\q{maximum allowed} year of 1999 \mdash{} so that the year in decimal only required two digits 
\mdash{} led to costly reprogramming of archaic legacy code during \YtwoK{}).  
In this kind of situation programmers 
usually assign types based on properties of binary representation \mdash{} what number of 
binary digits is optimal for memory and/or speed, even if this allows \q{absurd} values 
like 1000 years old.  Run-time checks, rather than type restrictions, may be used to 
flag nonsensical data and prevent data corruption.  In these scenarios, types represent a 
compromise between \i{concepts}, which can be fuzzy and open-ended, and 
\i{sets}, which conceptually are nothing more than the totality of their extension.\footnote{Nevertheless, there is interesting (and potentially practically useful) research in 
how formal type-constructions model conceptual organization: for example, 
\Gardenfors{} Conceptual Space Theory has seen formal implementations 
\cite{RaubalAdams}, and it is very interesting to juztapose 
scientific and mathematical treatments of Conceptual Spaces (as in \cite{Strle} or 
\cite{Zenker}) with mathematical (e.g., topological) thoeries of 
data types \cite{Escardo}, \cite{StellConvexivity}.
}
}
\p{Sets, concepts, and types represent three different primordial thought-vehicles for 
grounding notions of logic and meaning.  To organize systems around \i{sets} is 
to forefront notions of inclusion, exclusion, extension, and intersection, 
which are also formally essential to mathematical logic and undergird the 
classical interdependence of sets, logic, and mathematics.\footnote{Recent work in mathematics, however (partly 
under the influence of computational proof engines and foundations research 
like Homotopy Type Theory) shows that type and/or Category theory 
may replace sets as a groundlevel for logico-mathematical reasoning (if 
not notation) in the future \cite{HOTT} (It is worth pointing out that 
despite their similar ordinary meanings, mathematically \i{type} is much different 
from \i{Category} even though these respective theories can be usefully integrated).
}
To organize systems around \i{concepts} is to forefront practical engagement 
and how we mold conceptual profiles, as collections of ideas and pragmas, 
to empirical situations.  To organize systems around \i{types} is to forefront 
\q{functions} or transformations which operate on typed values, the interrelationships 
between different types (like subtypes and inclusion \mdash{} a type can itself 
encompass multiple values of other types), and the conceptual abstraction 
of types themselves from the actual sets of values they may exhibit 
in different environments.  Sets and types are 
formal, abstract phenomena; whereas concepts are characterized by 
gradations of applicability, and play flexible roles in thought and language.  
The cognitive role of concepts can be discussed with some rigor, but there is a 
complex interplay of cognitive schema and practical engagements which 
would have to be meticulously sketched in many real-world scenarios, if 
our goal were to translate conceptual reasoning to formal structures 
on a case-by-case basis.  We can, however, consider in general 
terms how type-theoretic semantics can capture conceptual structures 
as part of the overall transitioning of thoughts to langauge.
}
\p{A concept does not merely package up a definition, like \q{restaurant} as 
\q{a place to order food}; instead concepts link up with other concepts 
as tools for describing and participating in situations.  Concepts are 
associated with \q{scripts} of discourse and action, and find their 
range of application through a variegated pragmatic scope.  
We should be careful not to overlook these pragmatics, and 
assume that conceptual structures can be simplistically 
translated to formal models.  
Cognitive Linguistics critiques 
Set-Theoretic or Modal Logic reductionism (where a concept is just a set 
of instances, or an extension across different possible worlds) \mdash{} George Lakoff and Mark Johnson, 
prominently, argue for concepts' organization around 
prototypes (\cite[p. 18]{LakoffJohnson}; \cite[p. 171, or p. \textit{xi}]{Johnson}) 
and embodied/enactive patterns of interaction (\cite[p. 90]{LakoffJohnson}; 
\cite[p. 208]{Johnson}).  
Types, by contrast, at least in linguistic applications of type theory, are abstractions 
defined in large part by quasi-functional notions of phrase struture.  
Nevertheless, the \i{patterns} of how types may inter-relate 
(mass-noun or count-noun, sentient or non-sentient, and so forth) 
provide an infrastructure for conceptual understandings to be 
encoded in language \mdash{} specifically, to be signaled by which typed 
articulations conversants choose to use.  A concept like 
\i{restaurant} enters language with a collection of understood 
qualities (social phenomena, with some notion of spatial location and 
being a \q{place}, etc.) that in turn can be marshaled by sets of 
allowed or disallowed phrasal combinations, whose parameters 
can be given type-like descriptions.  Types, in this sense, 
are not direct expressions of concepts but vehicles for 
introducing concepts into language.
}
\p{Concepts (and types also) are not cognitively the same as their 
extension \mdash{} the concept \i{restaurant}, I believe, is distinct from  
concepts like \i{all restaurants} or \i{the set of all restaurants}.
This is for several reasons.  First, Concepts can be pairwise different 
not only through their instances, but because they highlight different 
sets of attributes or indicators.  The concepts \q{American President} and \q{Commander in Chief} 
refer to the same person, but the latter foregrounds a military role.  
Formal Concept Analysis considers \i{extensions} and \q{properties} 
\mdash{} suggestive indicators that inhere in each 
instance \mdash{} as jointly (and co-dependently) determinate: concepts 
are formally a synthesis of instance-sets and property-sets \cite{YiyuYao}, 
\cite{Belohlavek}, \cite{Wille}.  Second, 
in language, clear evidence for the contrast between \i{intension} and 
\i{extension} comes from phrase structure: certain constructions specifically 
refer to concept-extension, triggering a mental shift from thinking of the 
concept as a schema or prototype to thinking of its extension (maybe in some context).  
Compare these sentences: 
\sentenceexamples{\sentenceexample{Tigers in that park are threatened by poachers.}
\sentenceexample{Young tigers are threatened by poachers.}
}
Both sentences focus a conceptual lens in greater detail than \i{tiger} in general, but 
the second does so more intensionally, by adding an extra indicative criterion; while 
the former does so extensionally, using a phrase-structure designed to operate on 
and narrow our mental construal of \q{the set of all tigers}, in the sense of 
\i{existing} tigers, their physical place and habitat, as opposed to 
the \q{abstract} (or \q{universal}) type.  So there is a familiar semantic 
pattern which mentally transitions from a lexical type to its extension and 
then extension-narrowing \mdash{} an interpretation that, if accepted, clearly 
shows a different mental role for concepts of concepts' \i{extension} than the 
concepts themselves.
}
\p{There is a type-theoretic correspondence between intension and 
extension \mdash{} for a type \Tnoindex{} there is a corresponding \q{higher-order} type 
of \i{sets} whose members are \Tnoindex{}.\footnote{Related constructions are the type of \i{ordered sequences} of \Tnoindex{}; 
unordered collections of \Tnoindex{} allowing repetition; and stacks, queus, and 
deques (double-ended queues) as \Tnoindex{}-lists that can grow or shrink 
at their beginning and/or end.
}  If we take this (higher-order) 
type gloss seriously, the extension of a concept is not its \i{meaning}, but a  
different, albeit interrelated concept.  Extension is not definition.  
\q{Tiger} does not mean \i{all tigers} (or \i{all possible tigers}) \mdash{} though arguably 
there are concepts \i{all tigers} and \i{all restaurants} (etc.) along with the concepts 
\i{tiger} and \i{restaurant}.  Concepts, in short, do not mentally signify sets, or 
extensions, or sets-of-shared-properties.  Concepts, rather, are cognitive/dialogic tools.  
Each concept-choice, as presentation device, 
invites its own follow-up.  \i{Restaurant} or \i{house} have meaning not via  
idealized mental pictures, or proto-schema, but via kinds of things 
we do (eat, live), of conversations we have, of qualities we deem relevant.  Concepts do not 
have to paint a complete picture, because we use them as part of ongoing situations 
\mdash{} in language, ongoing conversations.  Narrow concepts \mdash{} which may best exemplify 
\q{logical} models of concepts as resemblance-spaces or as rigid designators to 
natural kinds \mdash{} have, in practice, fewer use-cases \i{because} there 
are fewer chances for elaboration.  Very broad concepts, on the other hand, can have, 
in context, too \i{little} built-in \i{a priori} detail.  
(We say \q{restaurant} more often than \i{eatery}, and 
more often than \i{diner}, \i{steakhouse}, or \i{taqueria}).  Concepts dynamically play 
against each other, making \q{spaces} where different niches of meaning, including 
levels of precision, converge as site for one or another.  Speakers need freedom to choose 
finer or coarser grain, so concepts are profligate, but the most oft-used trend toward middle 
ground, neither too narrow nor too broad.  \i{Restaurant} or \i{house} are useful because they are noncommittal, inviting more detail.  
These dynamics govern the flow of inter-concept relations (disjointness, subtypes, partonymy, etc.). 
}
\p{Concepts are not rigid formulae (like instance-sets or even attributes fixing when 
they apply); they are mental gadgets to initiate and 
guide dialog.  Importantly, this 
contradicts the idea that concepts are unified around instances' similarity (to each other or 
to some hypothetical prototype): concepts have avenues for contrasting 
different examples, invoking a \q{script} for further elaboration, or for building temporary filters 
(\q{Let's find a restaurant that's family-friendly}; allowing such one-off narrowing is a feature 
of the concept's flexibility).
No less important, than acknowledged similarities across all instances, are well-rehearsed ways 
\visavis{} each concept to narrow scope by marshaling lines of \i{contrast}, of \i{dissimilarity}.  
A \i{house} is obviously different from a \i{skyscraper} 
or a \i{tent}, and better resembles other houses; but there are also more nontrivial \i{comparisons} 
between houses, than between a house and a skyscraper 
or a tent.  Concepts are not only spaces of similarity, but of \i{meaningful kinds of differences}.
}
\p{To this account of conceptual spaces we can add the conceptual matrix spanned by 
various (maybe overlapping) word-senses: to \i{fly}, for example, names 
not a single concept, but a family of concepts all related to airborn 
travel.  Variations highlight different features: the path of flight (\i{fly to Korea}, \i{fly over the mountain}); 
the means (\i{fly Korean air}, \i{that model flew during World War II}); 
the cause (\i{sent flying (by an explosion)}, \i{the bird flew away (after a loud noise)}, 
\i{leaves flying in the wind}).  Words allow different use-contexts 
to the degree that their various \i{senses} offer an inventory of aspects for 
highlighting by \i{morphosyntactic} convention.  Someone who says \i{I hate to fly} is not 
heard to dislike hand-gliding or jumping off mountains.\footnote{People, unlike birds, do not fly \mdash{} so the verb, used intransitively 
(not flying \i{to} somewhere in particular or \i{in} something in particular), 
is understood to refer less to the physical motion and more to the socially 
sanctioned phenomenon of buying a seat on a scheduled flight on an airplane. The construction 
highlights the procedural and commercial dimension, not the physical mechanism and 
spatial path.  But it does so \i{because} we know human flight is 
unnatural: we can poetically describe how the sky is filled with flying leaves or birds, 
but not \q{flying people}, even if we are nearby an airport.  
Were \q{flying people} used jokingly, it would be in bad taste, like
\q{cat all over all over the driveway} from Pinker 
\cite{Pinker} on page 119 and Langacker's \q{Nouns and Verbs} \cite{Langacker87} on page 67.
}  Accordant variations 
of cognitive construal (attending more to mode of action, or path, or motives, etc.), 
which are elsewhere signaled by grammatic choices, are also spanned by a conceptual 
space innate to a given word: senses are finer-grained meanings availing themselves to one construal or another.    
}
\p{So situational construals can be signaled by word- and/or 
syntactic form choice (locative, benefactive, direct and indirect 
object constructions, and so forth).  Whereas conceptual organization 
often functions by establishing classifications, and/or invoking 
\q{scripts} of dialogic elaboration, cognitive structure tends to apply more 
to our attention focusing on particular objects, sets of objects, events, or 
aspects of events or situations.  \i{Conceptual} is more abstract and belief-oriented; 
\i{Cognitive} is more concrete and phenomenological.  Concepts organize our 
\q{background knowledge} \cite{SmithMcIntyre}; cognitions allow it to be latent 
against the disclosures of material consciousness 
\cite{DavidWoodruffSmith}, \cite{DavidWoodruffSmithChapter}, 
\cite{JordanZlatev}, \cite{SeanDorranceKelley}. 
So the contrast between singular, mass-multiples, and count-multiples, 
among nouns, depends on cognitive 
construal of the behavior of the referent in question (if singular, its 
propensity to act or be conceived as an integral whole; if multiple, its 
disposition to either be divisible into discrete units, or not).  
Or, events can be construed in terms of their causes 
(their conditions at the outset), or their goals (their conditions at 
the conclusion), or their means (their conditions in the interim).  
Compare \i{attaching} something to a wall (means-focused) to 
\i{hanging} something on a wall (ends-focused); \i{baking} a cake 
(cause-focus: putting a cake in the oven with deliberate intent to cook it) 
to \i{burning} a cake (accidentally overcooking it).\footnote{We can express 
an intent to bake someone a cake, but not (well, maybe comedically) to 
\i{burn} someone a cake (\q{burn}, at least in this context, implies 
something not intended); however, we \i{can} say 
\q{I burnt your cake}, while it is a little jarring to say 
\q{I baked your cake} \mdash{} the possessive implies that some 
specific cake is being talked about, and there is less apparent reason 
to focus on one particular stage of its preparation (the baking) once 
it is done.  I \i{will} bake a cake, in the future, uses 
\q{bake} to mean also other steps in preparation (like \q{make}), while, 
in the present, \q{the cake \i{is} baking} emphasizes more its 
actual time in the oven.  I \i{baked your cake} seems to focus 
(rather unexpectedly) on this specific stage even after it is completed, 
whereas \i{I baked you a cake}, which is worded as if the recipient 
did not know about the cake ahead of time, apparently uses \q{bake} in 
the broader sense of \q{made}, not just \q{cooked in an oven}.  
Words' senses mutate in relation to the kinds of situations where they are used 
\mdash{} why else would \i{bake} mean \q{make}/\q{prepare} in the past or future tense but 
\q{cook}/\q{heat} in the present?  
}
These variations are not random assortments of polysemous words' senses: 
they are, instead, rather predictably distributed according 
to speakers' context-specific knowledge and motives.  
}
\p{I claim therefore that \i{concepts} enter language complexly, influenced by 
conceptual \i{spaces} and multi-dimensional semantic and syntactic selection-spaces.  
Concepts are not simplistically \q{encoded} by types, as if for 
each concept there is a linguistic or lexical type that just 
disquotationally references it \mdash{} that the type \q{tiger} means the concept 
\i{tiger} (\q{type} in the sense that type-theoretic semantics would model lexical 
data according to type-theoretic rules, such as \i{tiger} as subtype of \i{animal} or 
\i{living thing}).  
Cognitive schema, at least in the terms I just laid out, select particularly 
important gestalt principles (force dynamics, spatial frames, action-intention) 
and isolate these from a conceptual matrix.  On this basis, we can argue that 
these schema form a precondition for concept-to-type association; or, 
in the opposite logical direction, that language users' choices to employ 
particular type articulations follow forth from their prelinguistic 
cognizing of practical scenarios as this emerges out of collections 
of concepts used to form a basic understanding of and self-positioning within them.
}
\p{In this sense I called types \q{vehicles} for concepts: not that types \i{denote} 
concepts but that they (metaphorically) \q{carry} concepts into language, as a bus 
carries people into a city.  \q{Carrying} is enabled by types' semi-formal rule-bound 
interactions with other types, which are positioned to capture concepts' variations and 
relations with other concepts.  
To express a noun in the benefactive case, for example, which can be seen as attributing to 
it a linguistic type consistent with being the target of a benefactive, 
is to capture the concept in a type-theoretic gloss.  
It tells us, I'm thinking about this thing in such a way that it
\i{can} take a benefactive (the type formalism attempting to capture
that \q{such a way}).
A concept-to-type \q{map}, as I just
suggested, is mediated (in experience and practical reasoning) by
cognitive organizations; when (social, embodied) enactions take
linguistic form, these organizing principles can be encoded in how
speakers apply morphosyntactic rules.  So the linguistic structures,
which I propose can be formally modeled by a kind of type theory, work
communicatively as carriers and thereby signifiers of cognitive
attitudes. The type is a vehicle for the concept because it takes part in constructions 
which express conceptual details \mdash{} the details 
don't emerge merely by virtue of the type itself.
I am not arguing for a neat concept-to-type correspondence; instead, a type system provides a 
\q{formal substrate} that models (with some abstraction and simplification) how 
properties of individual concepts translate 
(via cognitive-schematic intermediaries) to their 
manifestation in both semantics and syntax.
}
\p{Continuing with benefactive case as a case study (no pun intended), 
consider how an ontology of word senses (which could plausibly be
expressed by types and subtypes) can interrelate with the benefactive.  
A noun as a benefactive target most often is a person or some other
sentient/animate being; an inanimate benefactive is most likely
something artificial and constructed (cf., \i{I got the car new tires}).  
How readily hearers accept a sentence -- and the path they
take to construing its meaning so as to make it grammatically acceptable
-- involves interlocking morphological and type-related considerations;
in the current example, the mixture of benefactive case and which noun
\q{type} (assuming a basic division of nouns into e.g.
animate/constructed/natural) forces a broader or narrower
interpretation.  A benefactive with an \q{artifact} noun, for example, 
almost forces the thing to be heard as somehow disrepaired:
\sentenceexamples{\sentenceexample{I got glue for your daughter.}
\sentenceexample{I got glue for your coffee mug.}
}
We gather (in the second case) that the mug is broken \mdash{} but this is never spelled out 
by any lexical choice.  It is implied indirectly by benefactive case along with 
notions of classification, on the grammar/semantic border, that have a potential 
type-theoretic treatment.  It is easy to design similar examples with other cases: 
a locative construction rarely targets \q{sentient} nouns, so in 
\sentenceexamples{\sentenceexample{We're going to Grandma!}
\sentenceexample{Let's go to him right now.}
\sentenceexample{Let's go to the lawyers.}
\sentenceexample{Let's go to the press.}
}
we mentally substitute the person with the place where they live or work.  Morphosyntactic 
considerations are also at play: \i{to the lawyers} makes \q{go} sound more like \q{consult with}, 
partly because of the definite article (\i{the} lawyers implies conversants have some prior involvement 
with specific lawyers or else are using the phrase metonymically, as in \q{go to court} or 
\q{to the courts}, for 
legal institutions generally; either reading draws attention away from literal spatial implications of 
\q{go}). \q{Go to him} implies that \q{he} needs 
some kind of help, because if the speaker just meant going to wherever he's at, she probably would 
have said that instead.  Similarly, the locative in \i{to the press} forces the mind to 
reconfigure the landmark/trajector structure, where \q{going} is thought not as a literal 
spatial path and \q{press} not a literal destination \mdash{} in other words, the phrase must be 
read as a metaphor.  But the \q{metaphor} here is not \q{idiomatic} or removed from linguistic rules 
(based on mental resemblance, not language structure); here it 
clearly works off of formal language patterns: the landmark/trajector 
relation is read abstracted from literal spatial movement because the locative is applied 
to an expression (\i{the press}) which does not (simplistically) meet 
the expected interpretation as \q{designation of place}.  
We need to analyze syntactic details like noun case and 
forms of articles, but also finer-grained (though not purely lexicosemantic) classifications 
like sentient/nonsentient or spatial/institutional.  
}
\p{One way to engage in classification in this kind of example is just to consider subtyping: 
divide nouns into sentient and non-sentient, the former into human and animal and the latter 
into artifacts and natural things, and so forth.  But other options are less blunt.  
For example, notions like sentient/nonsentient can be construed as \q{higher-order types}, 
meaning that for broadly-hewed types like nouns or verbs, there are sentient (and non-sentient) variants, 
just as for a type \TypeCat{} there are mass-plural and count-plural collections of \TypeCat{}, 
ordered and unordered \TypeCat{} collections, and so on.  Subtyping, higher-order types, inter-type 
associations and various other formal combinations are options for encoding grammatic and 
semantic classification in something like a formal type theory.  The key properties of 
type systems are not only meanings atttached to individual types but notions of functionality 
(according to the central notion that a type system includes \q{function} 
types which are mappings between other types; in Category Theory, any 
formal type system is \q{Cartesian Closed}, meaning that if \T1{} and \T2{} are 
types, there is necessarily a type \TSupT{} of functions between them).  
So if adjectives, say, are most basically \NtoN{} (they modify nouns and yield noun-role phrases), 
we can then consider how adjectives should be modeled when their modified nouns 
are associated with or attributed sentience, mass-plural, or any other variation (whether 
via subtyping or some other association).  How these \q{variations} are modeled in accord 
with one single type is less important than how they \q{propagate} via applicative 
structures, where \q{function-like} types apply transformations and produce phrases.
}
\p{To build up a linguistic type theory, I assume, then, a framework of
types and type associations with a few underlying properties, such as
these:
\renewcommand{\labelitemi}{$\bullet$}
\begin{itemize}\item{} Types have a spectrum of granularity, from the very broad (Parts
of Speech) to the much narrower, 
including (at the fine end of the scale) where
they incorporate lexical data
(types can potentially include \i{tiger}, \i{house}, and so on).  
In between are constructions related to \q{Ontology}, 
like sentient/nonsentient, pointwise/extended, artifact/institution, among many others.
\item{} Types are neither strictly grammatic nor strictly semantic, but
their gradations of precision cross between grammar and semantics.
\item{}  Returning to \q{Ontology}: types have associated qualities like sentient/nonsentient;
spatially (and/or temporally) extended, pointwise, or non-spatial
(/non-temporal); caused, self-causing, self-determining, affected by
other things, affecting other things; objects, events, processes, or 
institutions; abstracta or spatetime present things; observables 
or subjectives like emotions or sensations, which are temporally present 
for someone but not (directly) encountered by others.  These are qualities 
pertaining to the manner of referents' appearing, causing, and extending 
in the world and in consciousness, and to a \q{classification} of kinds 
of entities (like a metaphysical Ontology, though the point is not 
to reproduce Medieval philosophy but, more modestly, to catalog word senses).  
I will refer to these qualities generically as \q{associations}.  They may be 
introduced via subtyping or more complex type operators.
\item{}  Some types are \q{function like}: this means that they are
\i{applied} to senses which have their own types.  This introduces one
form of head/dependent relation, where a head word instances a
function-like type and is applied to one or more \q{dependents}.
\item{}  Type information \q{distributes over} Link Grammar pairs.  For
any pair of words which have a meaningful inter-word relation, we can
consider types which may be applicable to both words, and how these
types affect and are affected by the significance of the particular
kind of link.  Some kinds of links mandate particular type
interpretations of the links elements: \TS{} links,\footnote{http://www.link.cs.cmu.edu/link/dict/section-TS.html
} to cite a narrow
example, would only be formed between verb and \Prop{} types (at least
this is a plausible interpretation of the relevant Link Grammar rules.  
Other type/link combinations are more
open-ended.
\item{}  Type information similarly \q{distributes} over clusters of
link-pairs, where the presence of one such link influences how a
connected link is understood (or whether it is allowed).  Type-related
qualifications can propagate from one link-pair to connected
link-pairs.\footnote{For example, we can say that the linkage structure in
\q{Three times students asked an interesting question} alters the
normal type-attribution of \q{students} as just a plural noun;
relative to the connected structure linking \q{three times} through
\q{students} to \q{a question}, we can say that \i{three times}
modifies \q{students} so that it may function, as subject of
\q{asked}, as if typed as singular, because \i{three times} acts as a
\q{space builder} and creates a mental frame wherein the students are
singular, even if the word is plural.  Because of this frame
phenomenon, the singular/plural status of students does not propagate
to \q{a question}; collectively they presumably did  not all ask just
one question.  Type annotation for \q{students} has to be defined, 
in this case, relative to multiple \q{cognitive frames}.
}
\item{}  Type information also \q{distributes over} applicative
structures.  Given a function-like type we can consider how
associations for the head and dependent elements propagate to
associations on the resulting phrase \mdash{} again, via subtyping or some
other mechanism.
\end{itemize}
Such a \q{linguistic type theory} needs to model (at the least) these
aforementioned associations, the \q{distribution} of type details over
link and applicative structures, and the \q{propagation} of
associations and other type details.  While informal analyses in any 
single case may be clear, integrating many case-studies into a unified 
theory can be advanced by drawing ideas from rigorous, quasi-mathematical 
type theories \mdash{} relevant research has adopted technical
formations like \q{dot-types}, higher-order types, dependent types,
Monoidal Categories, Tensors, Continuations, \q{Linguistic Side Effects}, 
Monads, Combinatory Logic, and (Mereo)Topology/Geometry.\footnote{Monoids: \cite{DelpeuchPreller}; 
Tensors: \cite{MaillardClarkGrefenstette};  
Continuations: \cite{BarkerShan}; 
Combinators: \cite{Villadsen}; 
Side Effects: \cite{ShanThesis}; 
Monads: \cite{GiorgoloAsudeh}, \cite{ShanMonads}, \cite{Kiselyov}; 
Topology: \cite{Petitot}, \cite{CasatiVarzi}.
}  Such techniques can marshal type-theoretic ideas without 
falling back on simplistic type notions that can end up collapsing a type-system into a 
one-dimensional \q{Ontological} classification, rather than exploring more advanced formulations 
like higher-order types and (what I am callling) \q{associations}.
}
\p{With respect to Type Theory related to Link Grammar, consider again the \TS{} links 
(there are dozens of potential link-grammar pairs, of which \TS{} are among the 
less common, but they provide a useful example).  First, note that \Prop{} provides a 
type attribution for sentences, but also for sentence parts: \i{he is at school}, 
for example, presents a complete idea, either as its own sentence or part of a 
larger one.  In the latter case, a \Prop{} phrase would typically be preceded with a 
word like \i{that}; in the case of Link Grammar, we can define words relative 
to their semantic and/or syntactic role, which often lies primarily in linking 
with other parts of a sentence or helping those parts link with each other.  
Type-theoretically, however, we may want to assign types to every word, even those 
which seem auxiliary and lacking much or any semantic content of their own.
Arguably, \i{that} serves to \q{package} an assertion, encapsulating 
a proposition as a presumed fact designated as 
one idea, for the sake of making further comments, as if \q{making a noun} out 
of it: \PropToN{}.  Perhaps our intuitions are more as if \i{that he is at school} 
is also a proposition, maybe a subtly different kind, by analogy to how 
questions and commands are also potentially \Prop{} variants.  Since \thatPhrases{} are \q{arguments} for verbs, 
the choice then becomes whether it is useful to expand our type picture of verbs 
so that they may act on propositions as well as nouns, 
or rather type \q{encapsulated} propositions as just nouns 
(maybe special kinds of nouns).
}
\p{In either case, \i{I know that ...} clearly involves a verb with subject and direct 
object: so either \VisNNtoProp{} or \VisNProptoProp{}.  Consider the role of a \TS{}-link here: 
specifically, \TS{} connects the verb to the assertorial direct object (most 
directly, to \i{that}).  The purely formal consideration is ensuring that 
types are consistent: either the \TS{} target is \Prop{}, as I suggested 
above, with the verb type modified accordingly; or the \TS{} target is a noun, 
though here it is fair to narrow scope.  For this particular kind of 
link, the target must express a proposition: either typed directly as 
such or typed as, say, a noun \q{packaging} a proposition, which would then 
be a higher-order type relation (just as \q{redness} is a noun \q{packaging} 
an adjective, or \q{running} is an adjective packaging a verb).  In other words, 
it is difficult to state the type restrictions on the link-pair without employing 
more complex or higher-order type formations.
}
\p{On the other hand, this is 
another example of the fuzzy boundary between syntax and semantics: given a sentence 
which seems to link a verb calling for a belief or assertion (like \q{know}, 
\q{think}, \q{suggest}, \q{to be glad}) to something that is not proposition-like, is such a 
configuration ungrammatical, or just hard to understand?  Clearly, the 
\i{semantic} norms around verbs like \q{know} is that their \i{subject} 
has some quality of sentience (or can be meaningfully attributed 
belief-states, even if speakers know not to take it literally: \q{The function 
doesn't know that this number will never be zero}); and their \i{object} should 
be somehow propositional.  But applying type theory (or type theory in conjunction 
with Dependency Grammar) leaves open various analytic preferences: these 
requirements can be presented as rigid grammatic rules or as \q{post-parsing} 
semantic regulations.  How to model the qualities of sentience (or at least of having 
propositional attitudes broadly conceived), for the noun, and of propositionality, 
for the direct object, are again at the discretion 
of the analysis (subtypes, quality-associations, or etc.) \mdash{} Figure ~\ref{fig:Iknow} shows one potential, 
rather simplified unpacking of the sentence; from this structure details 
can be added perhaps as extra syntax constraints or perhaps more as cues 
to interpretation.\begin{figure*}
\caption{Dependency-style graph with type annotations}	
\label{fig:Iknow}
\vspace{1em}
\hspace{0.15\textwidth}	
\begin{minipage}{0.7\textwidth}	
\begin{tikzpicture}

%\draw

%\node [s1] at (0,0) {Student};

\node (I) at (1,1) {\textbf{I}};
\node (know) [right=9mm of I] {\textbf{know}};
\node (that) [right=9mm of know] {\textbf{that}};
\node (he) [right=12mm of that] {\textbf{he}};
\node (is) [right=10mm of he] {\textbf{is}};
\node (at) [right=10mm of is] {\textbf{at}};
\node (school) [right=5mm of at] {\textbf{school}};


\node (IRep) [double,draw=black,shape=circle,thick,fill=gray!50,inner sep=.5em,below=2cm of I] {};
\node (knowRep) [double,draw=black,shape=circle,thick,fill=gray!50,inner sep=.5em,below=1.5cm of know] {};
\node (thatRep) [double,draw=black,shape=circle,thick,fill=gray!50,inner sep=.5em,below=2cm of that] {};
\node (heRep) [double,draw=black,shape=circle,thick,fill=gray!50,inner sep=.5em,below=2.6cm of he] {};
\node (isRep) [double,draw=black,shape=circle,thick,fill=gray!50,inner sep=.5em,below=2.1cm of is] {};
\node (atRep) [double,draw=black,shape=circle,thick,fill=gray!50,inner sep=.5em,below=2.6cm of at] {};
\node (schoolRep) [double,draw=black,shape=circle,thick,fill=gray!50,inner sep=.5em,below=3cm of school] {};

\node (knowRepType) [below right = .2cm and -1.2cm of knowRep] 
 {\colorbox{yellow!20!red!30}{\scalebox{.7}{\NNtoProp}}}; 

\node (thatType) [below right = .2cm and -.8cm of thatRep] 
{\colorbox{yellow!20!red!30}{\scalebox{.7}{\PropToN}}}; 

\node (isRepType) [below right = .1cm and -1.4cm of isRep] 
{\colorbox{yellow!20!red!30}{\scalebox{.7}{\NNtoProp}}}; 

\node (atRepType) [below right = .15cm and -.9cm of atRep] 
{\colorbox{yellow!20!red!30}{\scalebox{.7}{\NtoN}}
}; 

\node (atRepTypeNote) [below right = .5cm and .1cm of atRep] {
	\footnotesize{(location)}
}; 


\draw [ |-,-|, <->, line width = .8mm, draw=gray!70, 
 dashed, double equal sign distance, >= stealth, shorten <= .25cm, shorten >= .25cm ]
 (I) to (IRep);

\draw [ |-,-|, <->, line width = .8mm, draw=gray!70, 
 dashed, double equal sign distance, >= stealth, shorten <= .25cm, shorten >= .25cm ]
(know) to (knowRep);
 
\draw [ |-,-|, <->, line width = .8mm, draw=gray!70,  
 dashed, double equal sign distance, >= stealth, shorten <= .25cm, shorten >= .25cm ]
(that) to (thatRep);

\draw [ |-,-|, <->, line width = .8mm, draw=gray!70, 
 dashed, double equal sign distance, >= stealth, shorten <= .25cm, shorten >= .25cm ]
(he) to (heRep);
 
\draw [ |-,-|, <->, line width = .8mm, draw=gray!70, 
dashed, double equal sign distance, >= stealth, shorten <= .25cm, shorten >= .25cm ]
(is) to (isRep);

\draw [ |-,-|, <->, line width = .8mm, draw=gray!70,  
dashed, double equal sign distance, >= stealth, shorten <= .25cm, shorten >= .25cm ]
(at) to (atRep);

\draw [ |-,-|, <->, line width = .8mm, draw=gray!70, 
dashed, double equal sign distance, >= stealth, shorten <= .25cm, shorten >= .25cm ]
(school) to (schoolRep);
 
 
\draw [shorten <= .25cm, shorten >= .25cm ] 
(knowRep) to node [draw=black,shape = star,star points=4,thick,inner sep = 0mm, above] {1} (IRep);

\draw [shorten <= .25cm, shorten >= .25cm ] 
(knowRep) to node [draw=black,shape = star,star points=4,thick,inner sep = 0mm,above] {2} (thatRep);

\draw [shorten <= .05cm, shorten >= .05cm, bend left=60] 
(thatRep) to node [draw=black,shape = star,star points=4,thick,inner sep = 0mm,above, 
near start] {3} (isRep);


\draw [shorten <= .15cm, shorten >= .15cm ] 
(isRep) to node [draw=black,shape = star,star points=4,thick,inner sep = 0mm,
 above, pos=0.4] {1} (heRep);

\draw [shorten <= .15cm, shorten >= .15cm ] 
(isRep) to node [draw=black,shape = star,star points=4,thick,inner sep = 0mm,above, 
 pos=0.4] {2} (atRep);

\draw [shorten <= .15cm, shorten >= .15cm ] 
(atRep) to node [draw=black,shape = star,star points=4,thick,inner sep = 0mm,above] {4} (schoolRep);


%\draw [shorten <= .25cm, shorten >= .25cm ] 
%(s1Rep) to node [draw=black,shape = star,star points=4,thick,inner sep = 0mm, below] {2} (s2Rep);

%\draw [shorten <= .5cm, shorten >= .5cm ] 
%(afterRep) edge [bend left=20,looseness=1] node [draw=black,shape = star,star points=4,thick,inner sep = 0mm, 
% above, near end ] {3} (complainedRep);

%\node (frameTopLeft) [below left = 1.5cm and -.65 cm of s1] {};
%\node (frameBottomLeft) [below = 2.5cm of frameTopLeft] {};
%\node (frameBottomRight) [right = 3.95cm of frameBottomLeft] {};
%\node (frameTopRight) [above = 2.5cm of frameBottomRight] {};

%\draw [shorten <= 0.15cm, shorten >= 0.15cm ] 
%(frameTopLeft) edge [bend right=30,looseness=1] (frameBottomLeft);

%\draw [shorten <= 0.15cm, shorten >= 0.15cm ] 
%(frameBottomLeft) edge [bend right=30,looseness=1] (frameBottomRight);

%\draw [shorten <= 0.15cm, shorten >= 0.15cm ] 
%(frameBottomRight) edge [bend right=30,looseness=1] (frameTopRight);

%\draw [shorten <= 0.15cm, shorten >= 0.15cm ] 
%(frameTopRight) edge [bend right=30,looseness=1]
%node [draw=black,shape = regular polygon,regular polygon sides=3,thick,inner sep = .2mm, 
%above, near start, shape border rotate = 180] {4} (frameTopLeft);


%node [draw=black,shape = star,star points=4,thick,inner sep = 0mm, above, 
%bend left=100,looseness=3] {3}

%;


\end{tikzpicture}
\end{minipage}

\hspace{0.1\textwidth}
\begin{minipage}{0.8\textwidth}
		\renewcommand{\labelitemi}{$\blacklozenge$}
	
\begin{itemize}\setlength\itemsep{-.3em}
\item 1 \hspace{12pt}  Verb's subject argument
\item 2 \hspace{12pt}  Verb's direct object argument
\item 3 \hspace{12pt}  Propositional \q{packaging} (\q{typed} as {}\PropToN{})
\item 4 \hspace{12pt}  Locative auxiliary link \\
(may be typed as converting nouns to place-designations) 
\end{itemize}
\end{minipage}
\end{figure*}  If these 
requirements are seen as more syntactic, so qualities are incorporated into 
data like Part of Speech (say, a noun designating something with propositional attitudes 
being a subtype of of a generic \N{} type), then we are more likely to analyze 
violations as simply incorrect (recall \q{The tree wants to run away from the dog}
\mdash{} ungrammatical or just somehow \q{exotic}?).  
Some examples suggest less incorrectness as clever or poetic usage \mdash{} so a 
richer analysis may recognize expressions as type- and link-wise acceptable, but 
showing incongruities (which is not the same as impropriety) 
at a more fine-grained type level.  That \i{to want} takes a subject 
\i{associated} with sentience does not force type annotations to inscribe this 
in grammatic or lexical laws; instead, these associations can be 
introduced as potential \q{side effects}, \i{triggering} re-associations 
such as forcing hearers to ascribe sentience to something (like a tree) where 
such ascription is not instinctive.  The type effect in this case lies more 
at the conceptual level, the language-user sifting conceptual 
backgrounds to find a configuration proper to the type requirements (in what 
sense can a tree \q{want} something?).  In this \q{tree} case we probably 
appeal to concepts of \q{as if}: if the tree \i{were} sentient, it 
would be nervous of the dog sniffing around \mdash{} a humorous way of calling 
attention to the dog's actions (obliquely maybe alluding to people's background 
knowledge that dogs sometimes do things, like pee, in inconvenient 
places, from humans' perspectives).  
}
\p{In brief, it is certainly possible \mdash{} though by no means mandatory \mdash{} to model 
type requirements with greater flexibility at a provisional grammatical layer, 
and then narrow in on subtypes or extra accumulations of qualifications on 
type-instances in a transition from grammar to semantics.  Perhaps cognitive 
schema occupy an intermediary role: progressing from basic recognition of 
grammaticality, through cognitive schema, to conceptual framing, with type 
machinery capturing some of the thought-processes at each \q{step} 
(not that such \q{steps} are necessarily in a temporal sequence).  The basic 
verb-subject-direct object articulation sets up an underlying cognitive 
attitude (represented by a basic type-framing of verb, noun, and proposition, 
like the \VisNNtoProp{} signature).  Cognitive ascriptions fill this out 
by adding detail to the broader-hewed typing, associating sentience with the 
subject and propositionality with the object (sub- or higher-order typing 
modeling this stage).  And how the actual lexical choices fit these cognitive 
expectations \mdash{} I call them cognitive because they are intrinsically tied 
to structurational schema in the type, morphology, and word-order givens 
in the encountered language \mdash{} compels conversants to dip into background 
beliefs, finding concepts for the signified meanings that hew to the 
intermediary cognitive manipulations (finding ways to conceptualize the 
subject as sentient, for example).  This also has a potential type model, 
perhaps as forcing a type conversion from a lexical element which does 
not ordinarily fit the required framing (such as giving inanimate things 
some fashion of sentience).  Type theory 
can give a window onto unfolding intellection at these multiple stages, 
although we need not conclude that the mind subconsciously doing this thinking 
mimics a computer that churns through type transformations mechanically and exactly.
}
\p{I envision the unfolding that I have just sketched out as something Phenomenological 
\mdash{} it arises from a unified and subjective consciousness, one marked by 
embodied personal identity and social situation.  If there are structural stases 
that can be found in this temporality of experience, these are not constitutive 
of conscious reality but a mesh of rationality that supports it, like the veins in 
a leaf.  Stuctural configurations can be lifted from language insofar as it is a 
conscious, formally governed activity, and lifted from the ambient situations which 
lend language context and meaning intents.  So any analytic emphasis on  
structural fixpoints threaded through the lived temporality of consciousness is an 
abstraction, but one that is deliberate and necessary if we want to make scientific 
or in any other manner disputatable claims about how language and congition works.  
In that spirit, then, I will try to condense the three \q{layers} of unfolding 
understanding, which as I have sketched them are posited in the metaphysical order 
of temporal experience \mdash{} \q{unfolding} in likely overlapping, blending ways 
\mdash{} I will \q{read into} them a more static and logically stacked meta-structure.  
Where I have sketched three layers or stages of unfolding language understanding, 
I will transition to proposing three \q{tiers} of language organization, in particular 
three levels where type-theoretic models can be applied.
}
