
 \saying{
 	On conna\^{\OldI}t la c\'el\`ebre affirmation de Claude L\'evi-Strauss: 
 	\q{les sciences humaines seront structurales ou ne seront pas}.  Nous aimerions lui en
 	adjoindre une autre: \q{les sciences humaines seront des sciences naturelles ou ne seront pas}. 
 	Evidemment, sauf \`a en revenir \`a un r\'eductionnisme dogmatique, une telle
 	affirmation n'est soutenable que si l'on peut suffisamment g\'en\'eraliser le concept
 	classique de \q{naturalit\'e}, le g\'en\'eraliser jusqu'\`a pouvoir y faire droit, 
 	comme \`a des ph\'enom\`enes naturels, aux ph\'enom\`enes d'organisation structurale.
 }
 \sayingsrc{Jean Petitot, \textit{Syntaxe Topologique et Grammaire Cognitive}}
 
{\vspace{-.25em}}
\saying{The nature of any entity, I propose, divides into three aspects or facets, which we may call its
	form, appearance, and substrate.  In an act of consciousness, accordingly, we must distinguish
	three fundamentally different aspects: its form or intentional structure, its appearance or
	subjective \q{feel}, and its substrate or origin.  In terms of this three-facet distinction, 
	we can define the place of consciousness in the world.}
\sayingsrc{David Woodruff Smith, \i{Mind World}}
 
{\vspace{-.25em}}
