\section{Conclusion}
\p{Without reducing linguistic \i{performance} to language qua 
field of propositional expression, and without collapsing linguistic meaning 
to a computable/propositional fragment, we can still allow 
interpretive-Phenomenological and formal/mathematical perspectives to co-exist.  
In the theory I have sketched, Cognitive Schema summarize lived, situated 
judgments and intentions that (in concrete form) are not \q{computable} 
(again with the caveat that our mostly science-driven worldview may imply 
that all reality is \q{computable} in some infinitely-powerful computation; 
I understand \q{computability} to terminologically exclude 
such a purely speculative level of capacity).  
However, our propensity to call up certain construals rather than others is triggered by 
linguistic formations, and in broad outline the catalog of these triggers, 
and their compositional structure, can be formalized (and even 
used to improve formal systems, like programming languages).  The 
challenge is to advocate for this co-existence without implying 
that formal systems, and mathematically provable system-properties, 
are the only kind of research tools which have scientific merit.   
}
\p{Subjective assessments are intrinsic to most linguists' argumentation \mdash{} warranting 
claims not with empirical data or logico-mathematical proof but by appealing to 
speakers' intuitions, so that reading linguistic texts is also collaborating 
on an ongoing research project (partly because language evolves, so 
word-meanings change, and formations which are ungrammatical for one generation 
may be experienced differently by others).  Nevertheless, linguistics, like 
economics, seems broadly accepted as a human \i{science}, not just an 
interpretive discipline.    
The claim that an economist's equation or a linguist's meta-grammar  
are accurate explanations, useful explanatory frameworks, seems generally 
evaluated in terms of whether their framework captures emergent higher-order 
structure, and offers an explanatory potential that does not merely reiterate lower-scale 
paradigms.  A theory expressed in the language of linguistics (not, say, 
neural networks), if it meets general criteria of testability and refutability 
(not necessarily empiricist/quantitative), 
arguably carries even more weight than lower-level neurophysical 
explanation \mdash{} precisely because the higher-scale \q{theory language} carries the burden of 
explaining emergent properties, which as \i{emergent} bear some 
descriptive/behavioral (if not causal) autonomy.  Likewise, a subjectively plausible 
and theoretically motivated equation which fits economic data 
probably carries more weight than a mere statistical analysis.  An explanatory focus 
on the higher-scale in terms of its own distinct (emergent) structures and theorized 
entities (like words and morphemes, in the case of linguistics, or markets and commodities, 
in the case of economics), reflects the linguist's or economist's charge to connect 
human phenomena with mental (and therefore, ultimately physical) law.  Nonetheless, 
even with liberal use of subjective judgments, economics and linguistics (and some 
other human sciences as well, potentially) are attached to the overall sphere of 
natural science, by virtue of causal links in principle even if not in practice.  
Scientific rigor in this humanistic setting is neither reducible to the techniques 
of natural science, nor dualistically separate from them.  Natural 
science and humanities are certainly not mutually irrelevant, but nor is 
the proper vehicle for scientific literacy to find a forum in the humanities 
merely to emulate numeric methods, as with statistics in sociology, or 
a retreat to narrow and behavioristic reductionism, in 
place of localized interpretation and situational particularism.   
}
\p{Subjective impressions (conscious experiences, emotions, intuitions, qualia, qualitative universals and 
particulars \mdash{} the qualitative characteristic in itself, and the hyletic-spatial trace, 
the site in experiential space as the quale becomes a moment of consciousness) \mdash{} these are not scientifically 
tractable and do not have obvious physical location or measurability, which makes them controversial as 
objects of scientific method.  Yet, even so, we do have conscious experiences, we 
do subconsciously (and when needed consciously, or with deliberate conscious attention) 
make judgments about classifications, or how parts aggregate into wholes, or are individuated 
apart from a larger whole in context; we can reflect on patterns in these judgments, 
not \i{introspectively} examining thoughts as they occur, but marshalling an overall 
familiarity with mental processes.  Consciousness is not only a kind of mentality, 
shared by humans and some animals; it is also a metacognitive tool, something we 
deploy to focus attention on a certain object or topic.  We \q{practice} how to \i{be} 
conscious, how best to distribute attention, in each setting 
(like an athlete maintaining a meditative state of ambient awareness, 
poised to latch conscious attention onto playing technique which is optimally instinctive, 
but \q{feels} different when degraded by fatigue or distraction).  
Our faculty for these modulations, switching among sub- and passive consciousness, 
attentive consciousness, \q{ambient} awareness, and back again, 
reveals that consciousness is not only an aspect of mind 
but a tool; it has a meta-cognitive and epistemic dimension, 
an awareness of what is known or not-yet-known and a technique of directing attention to the 
latter.
}
\p{A case-study: in a motel I unexpectedly find a newspaper outside the 
door.  Next morning I look outside curious whether a paper is there; after several days 
I come to expect the paper.  So I open the door not preoccupied with confirming this, but 
with (maybe rather distractedly) fetching it.  Initially I do not expect the paper, but, generally poised 
to notice both expected and unexpected circumstances, I make a mental adjustment and 
interpret the situation quickly; by the third day the paper has become expected, 
like other things I anticipate finding in a motel hallway, and the thrust of my attention, 
during the brief episode of my picking it up, is kinaesthetic and motor-intentional 
more than visual and inquisitive.  
Only on the second morning is the question of a paper's presence intended  
in an epistemic mode; but, while it is so thematized, I direct attention 
to optimize my ability to resolve the question.  How we engage attention is a 
deliberate choice, reflecting and responding to our metacognitive attitudes, 
what we think we know and do not know.  
}
\p{Because consciousness is in some ways a mental tool, we have an intimate familiarity with it, a 
familiarity which extends beyond our own minds: we can make reasonable guesses about what 
others do or do not know and perceive.  Our ability to anticipate others' epistemic states 
is an intrinsic feature of social interaction, of intersubjectivity; 
we therefore understand consciousness not only via our 
own use and possession/experience of it, but as a general feature of the human mind.  
We can accordingly make structured claims about conscious processes, not 
in the sense of introspective reports but of retrospective suggestions \mdash{} by analogy, a 
pianist on reflection may have alot to say about playing technique, but she does not 
acquire this wisdom from introspective study of her own playing while it happens;  
rather with accrued wisdom and reflection.  In terms of phenomenological method, our study of thought 
and consciousness is analogous: it is reflective examination of what it means to be 
consciously intelligent beings, not introspective psychology, or meditative meta-experience. 
}
\p{The methodological implications of this retrospection 
(as opposed to \i{intro}spection), how phenomenological writing seeks 
reflective consensus on claims about consciousness \mdash{} 
this fashion of constructing a research community, a discursive-methodological 
field, does not conform to empirical scientific method, but is arguably a quite 
valid and defensible means of meeting the criteriological goals \mdash{} the discourse 
ethics, the democratization of scientific participation \mdash{} which 
physical science achieves via empiricist Ontology.  
For all its limitations, Positivism 
has the one virtue of disputational inclusiveness, demanding potential 
observability (not some special revelation or insight) 
for theoretic ur-entities.  The civic norms of Phenomenology are more complex, because both 
\q{transcendental} analysis of consciousness \mdash{} as a kind of philosophical ground zero, a neo-Cartesian 
fortress against skepticism and empiricism \mdash{} and also a more pluralistic, enculturated, 
embodied, social Phenomenology, are well-represented (and interpenetrate in complex ways) 
in the continuing post-Husserl tradition.  That being said, even in its most neo-Idealist, reifying consciousness 
as a primordial frame on any cognitive-scientific reasoning, as human sciences' condition of 
possibility, Phenomenology cannot help but textually acknowledge pluralism, and philosophical 
collaboration \mdash{} precisely because its claims are not descriptive of empirically locatable/observable objects.
}
\p{Interestingly, the phenomenological tradition reveals substantial interest in both 
the socio-political and the formal-mathematical: this is not so noteworthy 
in itself, because Analytic philosophy also connects (say) language with 
(say) logic, but Phenomenology is distinct in that 
it joins the humanistic and the formal/mathematical without the same 
tendency to hone in on a overlapping, logico-semantic core.  In writings where 
Analytic philosophers appear to address both social and mathematical concerns, usually their 
underlying motivation, or so it seems to me, is to find some logical 
underpinnings to linguistic or cognitive structure (say, \i{implicatures}) \mdash{} logic, subject 
to formal treatment, also manifesting itself in the organization of thoughts and 
expressions.  Amongst phenomenologists, however, for example Husserl, 
Merleau-Ponty (in his science-oriented writings; \cite{DavidMorris}), and Anglo-American 
writers in the \q{Naturalizing Phenomenology} tradition, 
there is evident interest in mathematics \i{apart from} logic: topology, 
differential geometry, mereotopology, multi-granularity.\footnote{Not that logic is wholly unrelated to these subjects: consider 
topological and type/Category-theoretic embeddings of logical systems 
within certain categories, or technical domains, like toposes, sheaves, granules; 
but logic in this sense, mathematically founded within spaces otherwise discussed 
at least as metaphoric guides within Phenomenology, does not 
appear to be the dominant understanding of logic in the Analytic 
philosophical tradition.  To be fair, style may dictate that 
argumentation should be trimmed to its essential elements, and 
mathematical deductions are rarely if ever essential for 
defending phenomenological claims.  In Jean Petitot, for example, 
mathematics is sometimes intrinsic to empirical backing for 
phenomenological ideas, but other times (say, sheaf mereology),  
the formal theories, while useful analogies, do not clearly pair up with 
logico-deductive justifications.  But, I would reply, there 
is so much unexplained about consciousness, and cognition 
as it occurs in conscious minds --- the controversial \q{Explanatory Gap} 
between mind and matter --- that much of the important 
argumentation does not yet have deductive signposts; we need an 
effective methodology which is not so linear.  
As we approach beyond a simplifying, logico-functionalist vantage, 
which we eventually must transcend, both functionalization and empiricism 
fall by the wayside as reasonable methods for \q{Naturalizing} consciousness.  
We have to accept when the formal/mathematical stands as more intuitive 
than rhetorical, on pain of \q{Naturalization} being quarantined 
from a humanistic core entirely.
}  
Phenomenology therefore uncovers 
an arguably deeper and truer bridge between human and \q{eidetic} sciences, 
in Petitot's phrase, one which is not pre-loaded with logico-reductive presuppositions.  
If this is accurate, Phenomenology can provide a deeper methodology 
for the humanities in their interactions with natural science.  
Even insofar as we stay committed to the idea 
that social/cultural/mental phenomena emerge from (neuro-)physical ones, we 
need to curate methods for these \q{emergent} sciences which have the requisite 
theoretical autonomy to actually extend the explanatory reach of the natural 
sciences on which they causally rest.  Cognitive Linguistics, I would argue, 
is a good example of this notion of autonomy, and its methodology, I would 
also argue, bears an important resemblance to phenomenological research. 
}
\p{Another brief case-study (revisiting footnote \ref{footnoteVision}):  
our environing world mostly discloses itself through objects' visible exterior: 
as much as we have on occasion a palpable sense of volume as well (as when looking through a fog)  
\mdash{} and as much as what we see is inextricable from our embodied interactions with objects, 
adding tactile and kinaesthetic dimensions, a canonical sense of perception 
is still the vision of distant objects, usually through their 
surface geometry.  A canonical example of perceptual 
cognition is therefore reconstructing geometry from visual appearances, 
especially color gradations \mdash{} mathematically, converting \q{color} vector 
fields to curvature vector fields (it's worth noting that color is an 
almost primordial example of a Conceptual Space Theory as developed 
by \Gardenfors{} and others \cite{Strle}).  This kind of transformation,  
described (say) via differential geometry, is \i{qua} theoretical device 
an example of semiotic morphism, a mapping between 
representation disciplines \cite{GoguenSemioticMorphism}, \cite{GoguenWhatIs}.  
The point is not, however, that there are precise correlates in the brain which 
\q{implement} this procedure; that the semiotic morphism takes 
a domain and codomain that quantify over empirically locatable, 
neurophysical entities.  We can study how software reconstructs 
geometry from color data as an approximation to a \i{process}, 
a model-building whose semiotics of approximation is coarse-grained and 
holisitc.\footnote{The experiential verisimilitude of computer graphics is a 
phenomenological data point, but so is their obvious unreality \mdash{} 
the mathematics reveals something about, but is not an all-encompassing 
model for, shape and color \i{qua} material phenomenon, still less 
the neuroscience of color experience.  Morphism between structures 
may model \i{processes} more correctly than the structures themselves approximate their substrata 
\mdash{} but this is no longer a semiotics of causal/physical reductionism, 
a use of mathematics (like differential geometry) to iconify empirical givens, the way 
that (say) the Navier-Stokes equations are understood to refer explicitly to (even while idealizing 
and abstracting from) fluid-mechanical dynamics.   
Our theory-semiotics has to locate the site of designation 
at a more oblique scale, a different Ontological register, of processes and 
transformations \mdash{} seeing in phenomena the image of a theoretical 
model because of its global structure, as a sign in its own right, 
rather than a collage of symbols and numbers to which are 
reduced spatializations and trajectories of causation and physical influence.
}  Formal 
devices like vectors or vector fields need not mold symbolic systems 
by mapping individual symbols to spacetime objects, or processes, but rather 
afford representation-mappings that capture cognition indirectly and patternwise.   
}
\p{I make this point using visual consciousness as an example, but it applies 
also to cognitive grammar, where the color -to- curvature-vector 
morphism has an analogue in the mapping of word-sequences to tree- or graph-algebras.  
I do not intend to claim that there are specific, individuated neurophysical 
analogues to theoretical posits in the symbolic regime I sketched earlier, 
in terms of \POS{} and lexical annotations, inter-word and inter-phrase connections, 
applicative structures, and the rest.  There are not, necessarily, for example, 
little brain regions whose role is to represent different types of 
phrase structures (e.g., different flavors of pluralization).  
Our explanatory ambitions, instead, should be cognitive-linguistic models of a
global process-structure, agnostic about one-to-one correspondence between the posits of the theory and the
empirical stuff whose behaviors it wants to explain.  Cognitive triggers bridge formal/empirical sciences with
the phenomenological/humanistic: their causal engenderings are physical and structural phenomena, but their
manifestation in the world is not fully tractable without an interpersonal deliberation accounting for both the
privateness of consciousness and the sociality of mind, and, so, something akin to Phenomenology.  
}
\p{It may appear that I am describing a weak-functional theory (or metatheory) 
which uses functional description in lieu of precise micro-physical 
explanation \mdash{} in other words, that in lieu of explaining precisely how 
the brain achieves vision or language, we describe functional capabilities 
that are prerequisite for these competences, and refactor the goal 
of scientific explanation as to describe the system of intermediate functionality 
as correctly as possible, rather than describe how this functionality 
is physically realized.  In a strong form, this re-orientation yields 
functionalism in theories/philosophies of Mind, that try to refrain from 
Ontological commitments to mental states or properties \i{apart from} descriptions of 
their functional roles.  In other words, according to the 
parameters of the field of study and its institutions, even if not deep 
metaphysical beliefs, mental states are reducible to functional states, and cognitive 
systems are scientifically equivalent if they reveal similar functional organization, whether 
they belong to human or animal minds or computers or extra-terrestrials.  A 
more modest functionalism would reject the implied reductionistic 
(maybe eliminative) Ontological 
stance, and maintain that mental things are not wholly, metaphysically subsumed 
by their functional organization, while still practicing a kind of 
theory whereby this functional organization is the proper object of 
study; the specific aspect of the mental realm 
which is scientifically tractable.  
}
\p{I do not believe I am making even such weak-functionalist claims:  
either branch of functionalism can misattribute the methodological 
association between theoretical structures and explanatory goals.  
We may be led toward the stronger or weaker functionalist viewpoints 
if we understand that a cognitive theory should task itself with making 
symbolic icons for scientifically grounded referents, grounded in an abstract 
space of functional organization if not in empirical space-time.  
Of course, most scientific explanation does construct a specialized, 
technical semiotics whose signs refer into either formal spaces or 
accounts of empirical space-bound things, however abstracted or idealized.  
But, conversely, insofar as I propose to focus on functional structures, and particularly 
cross-representation-framework transformations, my intent is to 
\q{functionalize} the discursive norms of the theory, not the 
phenomena it investigates.  In order to negotiate between the competing 
demands of scientific rigor and formalization \mdash{} on the one hand \mdash{} 
with the immediacy and etheriality and subjectivity of consciousness, on the other, 
we need to \q{attach} theoretical structures to mental phenomena without 
getting bogged down in questions of the scientific or Ontological 
status of mental things, how they are \q{scientific} individually and collectively 
(collectively as in the Ontology of \q{Mind} overall).
}
\p{This suggests adopting functional attitudes not 
in the theory but the metatheory: to use functionalism as an organizing 
principle on the theoretical \i{discourse}, on the attitudes of the 
scientists and scholars who want to straddle the divide between natural 
and mathematical sciences and humanism and consciousness.  The 
\q{semiotic morphism} of color-to-curvature vector fields, or word-sequences 
to typed semantic graphs, are recommendations for guidelines on how 
researchers should write and communicate about cognitive processes in their 
global structure.  I have tried to outline a metadiscourse more than 
a metalanguage \mdash{} not a template for building theory-languages whose 
signs refer into a realm of posited empirical or abstract entities, but a 
template for using certain formal-mathematical constructions (in 
domains like typed lambda calculus, type theory, or differential geometry) as a 
textual prelude, a way to position the norms of writing to be receptive 
to both scientific-mathematical and phenomenological concerns.  If 
semiotic morphisms like color-to-curvature or word-sequence-to-semantic-graph 
have explanatory merit as ways to picture cognitive processes, this 
merit is intended to be judged according to how it affects 
discursive norms on this scientific borderlands between mathematics 
and humanities, rather than how it reduces empirical phenomena 
to mathematizable abstractions.  If there is \i{something} in cognition 
analogous to these morphisms, even if \q{analogous} means merely that 
holding the morphisms as formally defined in our minds while 
thinking about cognition can show us philosophical ways forward, 
then we should be interested in refining these 
formalizations as part of the overall Cognitive-Phenomenological project.
}  
