Suppose 
I am writing a function which counts the 
number of non-blank, non-comment 
lines in a file.  My implementation might start like this 
(in pseudo-code):
\begin{sentenceList}\sentenceItem{} File f = File.open(path, File.READ\_ONLY);
\sentenceItem{} if f.isEmpty() return 0;
\end{sentenceList}
So the first line tries to open a file with the 
specified path, and the second line checks if the file 
is indeed open and non-empty.  If not, it returns the 
number zero, meaning that there are no non-blank non-comment 
lines in the file.
`p`
\p{In a typical application framework a file will be 
opened if it exists and if the current user has permission 
to read and/or write to/from the file (here the necessary 
permission is read-only).  For sake of presentation 
I assume that a file is considered non-empty 
only if it is open and has some content (i.e., you 
can read something from the file).  So if 
the file \i{is} empty, that can mean several things: 
either it does not exist; or it cannot be opened because 
of inadequate permissions; or it \i{can} be 
opened but has no content.  My function is 
noncommital and returns zero in each of these cases.  
In particular, I gloss over some information: 
the zero return value is analogous to Mr. Jones 
being \q{unavailable} for an interview.  
}
\p{In the formal computer setting, we are not allowed to 
logicaly infer what expressions \q{really} mean: we 
should not, for instance, if some file does not 
exist, instead try to open a different file 
with a similar name.  Therefore the \q{meaning} of 
code expressions should not be tied explicitly to 
individual logical conditions.  For example, the 
\q{meaning} of line () should be tied to the proposition 
that the file (whose location is specified by \i{path}) 
is open.  However, the code never actually enagages with 
that proposition.  I do not necessarily determine whether 
the file is open, can be opened, or even whether it 
exists. 
}
\p}
On this basis, it seems as if the \q{meaning} of my 
call to open the function is not a matter of 
ascertaining or bringing about certain truth 
conditions.  It is true that my instruction 
may bring about certain truths (specifically, make it true 
that the file is open).  But we should 
not conclude that this potential state of affairs is 
what the instruction \i{means}.  As the programmer, I 
do not \q{want} the file to be opened; I have no 
vested interest (this is not like my wanting milk 
for coffee or to open a beer bottle).  Instead, 
I am an intermediary between application-users and 
the system kernel: my role is translate what \i{users} 
want into system instructions.  Granted, presumably the 
\i{user} wants the file she's interested in to be opened 
(as an intermediate step toward getting information 
from that file).  But what I contribite as 
code are \i{instructions}, and instructions have 
an effect on the state of the overall computational 
environment where my application is hosted.  
In the general case I am not aware of which 
new state obtains.  Attempting to open a file 
may cause it to be opened, but it may also 
cause the software reprsentation of that 
file \mdash{} a so-called \q{handle} \mdash{} to acquire a flag 
indicating that the referenced file does not exist 
(i.e., its path descrribes a nonexistent location), 
or that it does exist but cannot be opened due to 
insufficient permission, or that the permissions 
are satisfactory but there is a temporary lock 
from another application writing to the file.  
If needed I could attempt to ascertain the file's state 
at this level of detail, but it turns out to be 
irrelevant to my own algorithm.  In short, 
insofar as the \q{meaning} of computer code are the 
instructions it emits, these meanings correspond 
to state-changes only in coarse-grained ways.  
There may be propositional content associated to each possible 
state \mdash{} there are propositions that 
the file is open, or nonxistent, or inaccessible, 
or temporarily locked \mdash{} but the code does not engage 
the question of \i{which} proposition is 
(or becomes) true.     
A hammer is a tool to pound nails, and given some hammer, nail,  
and board we can have the eventuality that the nail is 
now in the board: but we have not thereby established that 
such proposition is the \i{meaning} of the hamer. 
\p{At this point I will focus in on one point of analysis which, 
I believe, is canonical relative to others: changes to 
our respective framing of situations by adding new conceptual 
details. If I were to choose one primordial 
account of linguistic meaning, this genre of 
morphism is probably what I would nominate.
}
\sentenceItem{}