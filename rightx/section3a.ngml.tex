\spsubsectiontwoline{The co-framing system and the doxa system}
\p{It may be argued that a \q{change-initiation} semantic 
theory such as I have laid out is, in fact, circular.  The 
meaning of my assertion that this wine is Cabernet Franc 
is, by that theory, the change that occurs in your 
cognitive frame as and if you trust my claim.  But presumably the 
only reason you do this is because you understand my 
enunciation as presenting some proposition, which you can 
accept to be true.  In order for my language to update 
your beliefs (or indeed to fail to do so, 
if you doubt me), you have to entertain 
propositional attitudes to the content of my 
language.  Indeed, your propositional attitudes have to 
structurally integrate with mine: I am evoking that 
wine's being Cabernet Franc as an assertion and 
not a question or request.  So you have to identify 
my attitude to some proposition and on that basis 
formulate your own attitude to the same proposition.  
This can only work if my language signifies some 
proposition \mdash{} so why can't we call 
that proposition the \q{meaning} of my 
utterance, rather than the effect which my declaring 
said proposition in some attitudinal package has on you?  
}
\p{The rationale for this challenge would seem to be 
that propositional content does not belong exclusively 
to one or another person, or even to the participants 
in a conversation.  Sure, many referential and conceptual 
details \i{are} context-specific.  But our joint 
process of cognitive framing seems intended to 
align our respective frames so that a genuine propositional 
content can emerge \mdash{} as we resolve all pronouns, follow all 
anaphora, and agree on all conceptual roles.  Hence 
from \q{that wine is Cabernat Franc} we can arrive at a 
content that thematizes the viniferous 
properties of some particular liquid.  Our interpersonal 
negotiations may be required to converge our attention on 
\i{that particular} liquid, but \mdash{} once we are there \mdash{} the fact of 
its being Cabernat Franc (and any other culinary, chemical, physical) 
properties is independent of our collective and individaul 
framing.
}
\p{That is, there is a nugget of propositional content that can be 
designated in a context-neutral way.  That content is 
expressed \i{in conversation} using \q{locally significant} 
terms, for convenience, but those details of \i{naming} 
the proposition involved are arguably tangential to 
the proposition itself.  We can, for sake of discussion, 
imagine a more neutral naming: imagine we could give 
GPS coordinates for one glass at one stretch of time 
and thereby refer to the wine in the glass thereby located 
(call it W) and declare that W is Cabernet Franc.  
(Meanwhile, let's agree that the concept \q{Cabernet Franc} refers 
to a wine with a specific genetic profile \mdash{} some property 
\q{CF} \mdash{} i.e., \i{being CF} is an unambiguous biological 
property that exists outside of any branding or vinological 
contingencies).  It would 
seem as if my linguistic performance in terms like 
\q{that wine is Cabernet Franc} works because you 
recognize me to be claiming \q{W is CF}.  
And the only obvious way that can happen is if what 
I say somehow \i{means} \q{W is CF}.
}
\p{Here I am recognizing the intuition that the effect 
which an asserting act has on addrssees is (skepticism 
aside) to accept the asserted content as true.  The 
intuition seems to be that the assertion is posed in the guise 
of something whose truth is independent of the effect it 
has on the addressee \mdash{} it's not as if you \i{make} it true by 
aggreeing to it.  An implicit assumpion is that any competent 
person would also deem it true \mdash{} a sommelier and a chemist 
would confirm that W is, yes, CF.  So the idea 
that meanings are propositional content \mdash{} motivated by the 
intuition that assertorial effects depend on all parties' 
grasping a propositional content that can be lifted outside 
the immediate context \mdash{} seems driven by 
the idea that parties \i{outside} the conversation would 
be equally disposed to view the assertions as truthful.
}
\p{There is, of course, vagueness and context-sensitivity in 
language.  But that does not preclude massaging linguistic 
content to reduce or eliminate those contingencies \mdash{} as if 
there is some subset of linguistic expression that has a basically 
pristine referential structure, one which allows a 
certain mathematical precision at least in the areas of 
designating natural kinds (and designating physical objects via 
spacetime regions).  So, \q{W is CF}, involving only 
globally meaningful spatiotemporal and genetic designations, 
would be an example of such \q{pristine} language.  
And while people do not actually \i{talk} in that kind 
of language, we can argue that when our cognitive 
frames are correctly aligned, we communicate \i{as if} 
we were using pristine language.
}
\p{The implication 
of this possibility is that semantics may indeed 
be logically transparent: the contextual complexities 
evident in surface-level language are biproducts of 
our cognitive autonomy, instead of intrinsic to language.  
They are facets of the minds which are the \i{vehicles} 
for language, and so from the perspective of linguistics 
proper they are \q{implementation details} rather 
than theoretical problems.  That is, we need to exert conscious 
effort to synchronize our attentional foci and 
conceptual mappings with others', given the private nature 
of our perceptual observations and \q{inner thoughts}:   
this is why we have both linguistic and extralinguistic 
signifiers of perceptual frame (\q{this}, \q{that}, 
\q{there}, \q{last summer}, pointing), and a social 
infrastructure to conventionalize lexical and natural-kind 
meanings (why, for instance, usage like \q{corn sugar} is 
regulated, not only even by convention, but 
sometimes by law).  But \i{above and beyond that} we have semantic 
faculties that trade in propositional contents once 
we have achieved a proper alignment with our conversational 
peers.  The \q{alignmnt process} may itself involves language, 
but language in a different register, meta-discursive 
more than semantic.  Linguistics proper, some can argue, 
prioritizes the study of communication \i{after} alignment.
}
\p{One way to describe this is to posit that what we 
call \q{language} is really two different 
systems: one that effectuates frame-alignment to 
compensate for the \q{centrifugal} force of cognitive 
autonomy, and a different architecture for 
signifying activity in the context 
of neatly aligned cognitive frames.  For convenience 
\mdash{} to avoid debating whether this distinction merely 
reciprocates, say, prgamatics vs. semantics \mdash{}  
let's call this the \q{coframing} system and the 
\q{doxa} system.  We could 
also guess that the hard part of AI-driven 
Natural Language Processing is the co-framing 
system; the \q{doxa} system has enough logical 
polish that computers can play the game as 
well as people.  A robot in a testing room could geolocate some glass, 
take a sample to a DNA analyzer, test its profile against 
a database of cultivars, and conclude that \q{W is CF}.  
Sure, we need human ingenuity to communicate effectively 
in the \i{absence} of \q{context-stripping} possibilities: 
we do not talk in terms of GPS coordinates and 
laboratory-testable property-ascriptions.  
But how do we deny that our context-dense language is 
possible only because there is a logical kernel that 
\i{could}, in principle, be solicited in 
context-neutral terms?  
}
\p{If I say that \q{the meaning of \i{this wine is Cabernet Franc} is 
its side-effect} \mdash{} how it initiates a procss whose telos 
is your believing \i{W is CF} \mdash{} I can be accused of circularity 
because I seem to presume what needs to be explained: that 
my language contains within it a signification 
of \i{W is CF} as propositional content.  On that objection, 
if my language did not carry that 
content, it would not cause the desired effect. 
And if it \i{does} carry that content, this given would 
seem logically prior to the side-effect, since the side-effect 
can happen only because of the carried content.  Ergo, apparently, 
the \i{real} meaning is the content, not the side effect 
(or the process or initiation of the process 
that has the side-effect).
}
\p{My rejoinder to this objection is an line of argument that 
starts by observing the conceptualization whereby the objection can 
be articulated.  Specifically, to formulate the objection, 
I have tried to imagine a competent 
language-understander who responds to \q{pure} or de-contextualized 
propositional content.  My specific example was a robot who tests 
a wine sample to confirm \i{W is CF}.  In the robot's computational 
capabilities, language only exists as logical structures: spacetime 
references are defined as geolocations and timestamps; adjectival 
qualities are defined as scientific properties computable 
in the relevant metrics (a genetic profile, a 
chemical signature, etc.).  We can imagine a cohort 
of intelligent robots listening in on our conversations and 
translating from our human context-sensitive language to 
their computable context-stripped representations.  By this 
thought experiment we can \mdash{} or we can contemplate 
that we can \mdash{} imagine robots for whom language communicates 
propositional content directly.  We can imagine sentences 
\q{naming} propositions the same way that first and last names 
identify people.
}
\p{But is that what is happening?  If the robot wants to 
confirm \i{W is CF} it has to effectuate certain actions: roll 
to the right place, take the wine sample, 
test it, match the rsults to a database.  And even if 
the robot taks our ascriptions on faith \mdash{} maybe it has a database 
that matches glasses to both GPS locations and wine styles, 
to record facts like \q{this glass has Cabernet Franc} \mdash{} 
responding to my assertion still involves some 
activity (updating the database).  So even though 
we have attributed power to the robot to traffic 
in logically pure expressions of propositional content, 
we have not shown that the robot lies outside the side-effect 
cycles of language. 
}
\p{Let's suppose that there is indeed a \q{doxa} system within 
language, so there is a space of logically pristine meanings 
conveyed via language.  As I proposed earlier, a \q{doxa} inventory 
\mdash{} a set of provisional beliefs \mdash{} forms part of each 
cognitive frame.  When our language-processing faculties 
encounter linguistic artifacts which express \mdash{} or within the 
context of suitably constructed cognitive 
frames can be translated to \mdash{} the \q{doxa system}, 
we respond to those stimuli by (evaluating and then, 
often) adding the \q{signified doxa} to the \q{doxa inventory}.  
But this is still a side-effect: the logical structure 
of the doxa has a role to play in this overall process, 
but this is far from authorizing us to 
reify the doxa as the philosophical core 
of linguistic meaning overall. 
}
\p{To put it differently, th claim of circularity that I acknowledged 
is itself circular.  Yes, a side-effect due to newly 
believing \Pprop{} would seem to depend on  \Pprop{} being expressed 
as propositional content by any act initiating the side-effect.  
And \Pprop{} is a propositional content that can inspire 
belief-change side-effects because it has the form of a 
trans-personal articulation: any reasonable person 
(even a robot) should accept it.  The circularity 
here is that the \q{work} is done by \Pprop{}, not by the 
side-effect per se.  But in order to theoretically 
posit \Pprop{} outside the side-effect, we have to posit a kind 
of decontextualized rational community: \Pprop{} is 
logically distinct from the side effect because other people 
and robots should engage it too.  But their getting thus 
engaged is also \i{for them} a side-effect: to believe or 
test \Pprop{} the robot has to perform certain acts \mdash{} i.e., 
whatever software runs its language-comprehension modules 
has to call some function than run its database and/or 
motor-location modules.  The enunication of \q{that wine is 
Cabernet Franc} is still \i{initiating a process}.
}
\p{Insofar as my assertive speech-acts are rationally performed, 
their side-effects on \i{one} addressee should resemble 
their side-effects on others, including other hypothetical or potential 
addressees (even robots).  There is clearly then a kind 
of \q{publicness} or \q{communalization} of side-effects, 
and language seems logical if we get the impression that 
its effects on different listeners will be mostly the same.  
If there is circularity here, it seems to go two ways: 
arguably, side-effects can be similar because there is 
a logical nexus in language that fixes content
across minds.  Surely \q{Sanders is a presidential candidate} 
(stated as a simple fact, without polarity) evokes 
similar effects because it is objectively true (he has 
formally declared he's running).  So language can guarantee 
effect-similarity because it has the resources 
\mdash{} sometimes albeit not always utilized \mdash{} to formulate 
assertions that are relatively transparent, logically 
(of course, it can also produce provocations like 
\q{Sanders is a terrible presidential candidate} or 
\q{Sanders is an unelectable presidential candidate}).
}
\p{So communality of side-effects depends on 
(sometimes, potential) logicality of language.  But 
conversely, it is hard to define the logicality of 
language without pointing out that logically 
transparent language (like \q{Sanders is running}) 
evokes different kides of side effects than polarizing 
language (like \q{Sanders is unelectable}).  
After all, the effect of some logically transparent 
enunciation is to introduce some propositional 
content into a public arena.  But communication 
only happens when the content thereby publicized 
is considered and maybe deemed true, which 
requires certain cognitive processes in a community of addressees.  
The logical content of language only \q{exists} insofar 
as logically reasonable utterances trigger 
logically guided cognitive operations.
}
\p{Even if we accept that linguistic expressions can 
signify propositional content, this does not mean 
that a sentence is like a djinn which conjures 
propositions into material form.  Logical structures 
do not float around like snowflakes: if they exist, they 
do so as regulatory structures or specifications 
guiding the behavior or implementation of 
physically realized, dynamically changable systems.  A 
computing platform can exemplify a Typed Lambda Calculus 
or Adjoint Tensor Logic or Modal Process Algebra by 
\i{implementing} such a system, but this 
does not mean a software artifact can \i{be} a 
logical system (or even can be a \i{token} of a 
logical system).  But the implementation of the 
system establishes an Ontological gap between the 
system as abstract Category and its physical realization.
}
\p{Let's say, for sake of argument, that someone develops a \Cpp{} 
Functional-Reactive Programming 
library (a not-too-ambitious enhancement of existing software) 
which fully realizes Jennifer Paykin's version of Tensor Algebra.  
It would be entirely possible for most (even expert) 
\Cpp{} programmers to use that library without understanding or 
even being aware that their code was embodying some logical 
structure, separate and apart from the system 
of side-effects and function-calls that they orchestrate.  
Similarly, developers can create \Cpp{} types that are functionally 
identical to Haskell monads, without being aware of 
the monadic logic thereby exmplified.  To say that 
logical systems are implemented in software is to say that 
the totality of all function-calls \mdash{} both actually 
observed at runtime and theoretically possible for 
any run of a program \mdash{} span an 
abstract space that is fully and adequately specified by 
the logic.  So we can say that a signal-slot connection 
causing some function-pointer to be followed represents the 
concrete manifestation of an abstract \q{temporal-monadic 
modality}.  This means that the pattern of signal-slot 
connections does and will always conform to regulations 
that can be modeled via Adjoint Tensor Logic.  It also means 
that this coformance is a result of deliberate design 
\mdash{} the logic exerts a normative effect on the 
software; it is not just a pattern retroactively discoverable 
in observed function-calls.  But 
what actually exists are the function-calls themselves, and 
there are many ways to comport to them without 
considering or being aware of the logic (we can enumerate 
function calls as a debugger trace, or study them in 
conventional \Cpp{} terms without the added logical 
details).  The logic is manifest as a regulatory and emergent 
pattern and influence, but is also only one facet of 
the full ontological status of the vehicles (e.g., function-calls) 
wherein the logic is realized.
}
\p{Insofar as Natural Language is logical, I would argue that its logic 
is manifest analogously: it is realized in the pattern 
of whatever cognitively corresponds to \q{function calls}; 
e.g., the tendency of external (linguistic 
and otherwise) stimuli to trigger cognitive processes.  
}
\p{My point in this argument is not that linguistic abstraction is 
wrong: after all, linguistics is not neuroscience, 
and the theoretical arsenal of linguistics can rightly 
neglect to target such topics as the neurophysical 
encoding of linguistic processes.  I can accept some 
form of truth-theoretical semantics if it provides 
a broad abstract description of linguistic processes 
that can be \q{handed off} to other disciplines, 
like psycholinguistics and language-acquisition studies.  
This would be a reasonable \q{division of labor} 
if we believed that at the \i{abstract} level 
language is really about propositional content, 
and that notions like \q{message passing between 
processing units} are attempts to introduce 
theoretical concepts at the \q{realization} 
level.  That is, something like truth-theoretical semantics 
would be apropos if the \i{abstract} formulation was mostly 
logical, even if some formally rigorous (but not in a 
manner amenable to symbolic logic) model was a 
better paradigm for studying the \i{concrete} implementation 
of the logical architecture. 
}
\p{However, the de facto assumption that \q{everything abstract is logic},  
and that any sub-logical details are the tangential impurities 
of concreteness itself, is a prejudice that isn't borne out even 
in highly formal milieus.  For example, the digital 
encoding of typed values are a concrete detail counterposed 
to the mathematical abstractions of type thory, but 
it's not as if there is a single line between \q{abstract} types 
and \q{concrete} binary-electrical codes.  Programming language
theory recognizes digital encoding as byte-sequences, and so 
for any typed value there is a mapping of 
that value to a string of base-256 integers (the 
value in runtime computer memory).  Moreover, any string 
of base-256 integers can potentially be interpreted as a 
typed value (for example, by derefencing a 
non-correctly-initialized pointer).  These are still 
abstract posits: it requires some abstraction to model electrical 
signals in disk drives or CPU  registers as \q{base-256 integers}.  
However, this represents a layer of abstraction which 
stands between the more \q{mathematical} abstractions of 
formal type theory and the bare metal of computer hardware.
}
\p{One feature of this \q{intermediary} abstraction is that 
the abstract posits are more likely to be mathematically 
opaque.  In functional programming, for example, we can 
associate each type 
with an algorithm that can construct every element 
of that type, and also often run that construction 
\q{backward} to analyze properties of type-instances 
(which is called \q{pattern matching}).  
The canonical example is a list: any list of size 
$n$ can be derived from a list of size $n-1$ by 
adding one element to the end of the list.  Starting 
from an empty list we can therefore build any 
arbitrary list by a sequence of these \q{constructors}.  
Working in the reverse direction, we can then calculate 
values \mdash{} such as, the largest element in the 
list \mdash{} by calling a function on every 
smaller list in the \q{chain} of constructors: 
the largest value is the maximum of 
the \i{last} value and the largest value of 
the \q{predecessor} list wherein that last 
value is not appended; the largest value of 
the predecessor is the maximum of \i{its} last 
value and the largest value of \i{its} 
predecessor, and so on.  The recursive structure here 
is directly tied to the arithmetic encoding (influential 
in early analytic philosophy) wherein the number 
$1$ is the successor to $0$, $2$ is the successor to $1$, etc.  
This gives numbers a logical form rather than 
leaving them as a kind of prelogical Platonic given.  
The analogous formulation in type theory is that any type 
is isomorphic to the set of algorithms which generate 
each of its values \mdash{} for instance, every list 
can be associated with the algorithm which builds 
the list iteratively, starting from 
the empty list.  This introduces a logical structure 
on types amenable to logical analysis \mdash{} we can 
prove properties about functions on types 
by analyzing how those functions operate on 
values given the specific construction-chain that 
produce them.  Continuing the example, I can 
prove something about my implementation of a 
function on lists if I prove it for 
the empty list and then prove that, 
if I know my function works for a list of 
length $n-1$, and I then append a value, 
it will still work for the length-$n$ list. 
}
\p{The problem with these functional-programming 
techniques in the context of programming language theory 
in general is that many applied type systems 
do not have this kind of isomorphism 
between types and construction-chains.  In \Cpp{}, 
say, I can get a list by dereferencing a pointer, 
and I have no way of knowing the provenance of the 
pointed-to memory.  There are many ways to 
construct \Cpp{} values \i{other} than by going 
through construction-chains.  It \i{may} be 
that values have properties consistent with their 
being built up by an incremental, logically 
regulated proess.  However, a \Cpp{} programmer 
often cannot \i{assume} that types have this logical 
orderliness.  In short, \Cpp{} types are more logicallly 
opaque than, say, Haskell types.  This does not make 
the \Cpp{} type system less \q{abstract} than Haskell's; 
it just means that there is less information 
embedded in \Cpp{} types which would make them amenable to 
analysis from a mathematical perspective.
}
\p{An interesting question is then which 
language is a better metaphor for \i{human} 
language \mdash{} a functional language like Haskell, which 
enforces logical rigor by design?   Or a procedural 
language like \Cpp{}, whose operational 
dynamics is essentially concerned with properly 
orchestrating function calls, even in the absence of 
logical guarantees?  The theory I intend to dvelop here, 
some variation on an \q{interface theory of meaning}, 
is probably closer in spirit to \Cpp{} than Haskell.
}
\p{}
