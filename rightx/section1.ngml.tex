\section{Conceptual Role Semantics and Externalism}
\p{Conceptual Role Semantics is often discussed together 
with a particular internalism/externalism debate 
which it tends to engender.  Here I want to 
defend a kind of Conceptual Role Semantics (hereafter 
CRS) but I will first outline an account 
of compromise between externalism and internalism.  
I will suggest a compromise different, I believe, 
than Ned Block's \q{two factor} model that seems 
considered the leading example of an 
externalist/internalist hybrid.
}
\p{The basic CRS picture is that linguistic meanings 
should be associatd with conceptual rols in pur 
understanding situations more than in terms 
of their reference to external objects.  Given 
sentences like
\begin{sentenceList}\sentenceItem{} \label{itm:ornate} He opened the wine bottle with an ornate corkscrew.
\sentenceItem{} \label{itm:butterfly} He opened the beer bottle with a butterfly corkscrew.
\sentenceItem{} He collects antique corkscrews and just bid on one online. 
\sentenceItem{} \label{itm:screwtop} I thought this was a screw-top but it turns out I need a corkscrew.
\sentenceItem{} \label{itm:x3d} This X3D file shows a very realistic corkscrew created with NURBS surfaces.
\sentenceItem{} \label{itm:sendx3d} Could you send me the corkscrew (the X3D file you just mentioned)?
\end{sentenceList}
we should interpret \q{corkscrew}, first, as a concept 
in a kind of functional organization.  In some of these sentences 
there is also a specific corkscrew (qua phsyical object) on hand 
as a referent, but its actual physical properties \mdash{} or 
even identity \mdash{} is not decisive for the meaning of 
the sentence.  After all, in (\ref{itm:screwtop}) the speaker is 
not thinking of any corkscrew in particular (probably 
\mdash{} more on that later) and in (\ref{itm:x3d}) and 
(\ref{itm:sendx3d}) the corkscrew is 
not real (at least not real qua corkscrew).  
But the conceptualization associated with \q{corkscrew} does 
not seem markedly different in (\ref{itm:ornate}) or 
(\ref{itm:butterfly}) versus (\ref{itm:screwtop}), 
at least (more on the other three later).
}
\p{Not only physical details but even lexical identity 
seems tangential to the important conceptual 
meanings.  Suppose I am hosting two guests, 
one has a magnum of ale and one a bottle of Malbec.  
They ask, respectively: 
\begin{sentenceList}\sentenceItem{} Do you have a bottle opener?
\sentenceItem{} Could you get me a corkscrew?
\end{sentenceList}
and I give the first guest a butterfly corkscrew and the 
second a folding multi-knife.  What I gave them is 
different from their request, but they should think 
nothing of it insofar as the winged corkscrew has a 
gap on its handle suitable for beer bottles and the 
multi-knife has a fold-out corkscrew helix.  I have 
not violated any conversational maxims, because I 
reasonably assume that the instruments I gave them 
are suitable for the desired goals, of opening their 
bottles.  Semantically \q{corkscrew} really means 
\q{something that can be used to open a wine bottle}, 
and in that sense the lexeme gets it principle content 
from this operational role, not some list of attributes 
(like spiraly and graspable) or prototypes.
}
\p{Granted, a suitably designed winged corscrew can be 
construed as a kind of bottle opener, and a multi-knife 
a kind of corkscrew respectively.  We are prepared to 
accept these tools as examples of the respctive 
concepts if they are functionally designed to support 
those tasks, even if they are not the primary function.  
But our inclination allowing concepts to dilate 
modulo functional criteria suggest that our grasp of 
concepts is first and foremost functional-pragmatic: 
we tend to internalize concepts in reference to 
(extralinguistic) functional roles and expand concpts 
to accommodate varigated implementers of those roles.
}
\p{We can indeed accept sentences like: 
\begin{sentenceList}\sentenceItem{} \label{itm:hammer} He opened the bottle of beer with a hammer.
\sentenceItem{} He pounded the nail with a lever corkscrew.
\end{sentenceList}
Of course here we are inserting objects into a conceptual 
nexus where they are not usually found.  Winged corkscrews 
are often \i{designed} to double as bottle-openers, but 
lever corkscrews are not designed to double 
as hammers.  Nevertheless we have no trouble imagining 
the scenarios being described, where someone uses 
the thick part of a corkscrew to pound a nail, or a 
hammer's handle-claw gap to pry off a bottle cap.  We have 
schemata for \q{a tool to open a capped bottle} and 
\q{a tool to pound a nail}, and the concepts 
of bottle-opener and hammer occupy that conceptual niche 
insofar as they are artifacts designed for those purposes.  
But the conceptual \q{slot} for, say, \q{a tool to 
open a capped bottle} is more general than 
the specific tools designed for those purposes.
}
\p{We nonetheless \i{would} be presumably violating 
conversational maxims if we handed our friend who 
wanted to open a beer bottle a hammer.  Even if 
there's a way to make the hammer work for that 
purpose, it's further outside the norm than, referring 
back to (\ref{itm:butterfly}), proposing to use a winged corkscrew.  
So the implicature in (\ref{itm:butterfly}) is satisfied, let's say, 
by bringing my guest a winged corkscrew, but not a 
hammer.  But we can entertain the \i{thought} of using 
a hammer as a bottle-opener, and even this 
possibility presents problems for simplistic 
thories of language acquistion as essentially learning 
a static set of word correspondances, like 
\q{a hammer is used to pound nails} or \q{a corkscrew 
is used to open wine} \mdash{} after all, you cannot conclude 
from 
\begin{sentenceList}\sentenceItem{} A hammer is somthing used to pound nails, \i{and}
\sentenceItem{} A lever corkscrew is something used to open wine, \i{and}
\sentenceItem{} A lever corkscrew can be used to pound nails 
\end{sentenceList}
that a hammer is a kind of lever corkscrew and can 
therefore open wine.  What we 
\i{do} have are conceptual slots available encapsulating 
ideas like \q{that which can open bottles} or 
\q{that which can pound nails}, and we \q{fill} these 
conceptual slots with different lexical content 
in different situations.  The 
\q{that which can open capped bottles} slot can be filled 
descriptively \mdash{} i.e., in declarative speech, 
like in (\ref{itm:hammer}) \mdash{} by a hammer, but not in 
other kinds of speech acts (we cannot read the concept 
\q{bottle opener} as satisfied by \q{hammer} in 
the context of a request for a bottle opner).  Note that 
the scope of conceptual roles can change merely 
by switching between locutionary modalities.
}
\p{The takeaway from this discussion in the 
internalism/externalism setting is that conceptual 
roles have a linguistic priority over and against 
both lexical and physical realizers, and the 
scope for things inside and outside of language 
to play (or not play) such roles varies with 
context.  I have introduced these issues via 
tool artifacts (like corkscrews) but would be 
closer to the spirit of the CRS internalism/externalism 
debate by discussing natural-kind concepts.  Suppose 
I am building a sand castle on a beach and ask someone 
one of:
\begin{sentenceList}\sentenceItem{} \label{itm:bucket} Can you bring me a bucket of water?
\sentenceItem{} \label{itm:glass} Can you bring me a glass of water?
\end{sentenceList}
For (\ref{itm:bucket}), a reasonable reaction would be a bucket filld 
with ocean water; but for (\ref{itm:glass}) my addressee would probably 
infer that I was thirsty, and \mdash{} since salt ater is 
non-potable \mdash{} was requesting water I could drink.  
But \q{\i{glass of} water} probably figures here just 
to establish my intention to drink it: you are 
entitled to bring me bottle of water instead.  In 
other words, my request has implied content which 
in some aspects loosens and in some aspects restricts 
the conceptual scope of semantic entries in my utterance.  
Thus oceans are composed of water, and near a beach I can 
say: 
\begin{sentenceList}\sentenceItem{} The ocean is over there.
\sentenceItem{} The water is over there.
\sentenceItem{} You can see the ocean from here.
\sentenceItem{} You can see the water from here.
\end{sentenceList}
Each pair is almost identical.  But ocean-water ceases to 
fall under the conceptual role of \q{water} when we are in 
the context of drinking things instead of the context 
of geography.  This suggests that water does not \q{mean} 
\htwoo{} or other saline or non-saline water: the meaning 
is not fixed to any particular chmical composition but 
adapts to the situational context, including what the 
water is used for \mdash{} e.g. as a drink or as a binder 
for a sand castle.
}
\p{The most-discussed \q{water} analysis in the literature is 
less earthly than this: Putnam's \q{twin earth} argument 
about a planet whose substance (with chemical 
makeup) XYZ functionally indisinguishable 
from our (\htwoo{}) water.  Externalists and internalists 
use this thought-experiment to express their 
differences as disagreements over 
whther twin-earthers' XYZ concept is the same as our 
\htwoo{} concept.  For the latter, as the basic account 
goes, XYZ plays the same conceptual role in their lifeworld 
as \htwoo{} plays in ours, so it is the same concept; 
for the former, the concepts designate different 
material substances (even if twin-earthers don't know 
this) so they can't mean the same thing, even if 
there is some sort of analogy or resemblance between 
them (concepts can be analogous or similar while still 
being different concepts).
}
\p{Before making a case for one alternative here over the 
other, let me note the following: it is unfortunate 
that the case-study is formulated in terms of XYZ vs. 
\htwoo{}, because at the level of molecular composition 
it is hard for us to conceive that XYZ is 
\i{really} indistinguishable from water.  After all, 
our concptual understanding of water includes things 
like electrolysis \mdash{} if XYZ does not emit hydrogn 
and oxygen when electricallly charged under certain 
controlled conditions, it is not behaving like water 
and can not be (even internalistically) construed as 
conforming to our concpt of water.  Of course, we are 
free to expand our water-concept, just as we contract 
it when switching from geology/geography to drinking.  
But here we expand it with full recognition that 
finer-grained concptual distinctions are possible, 
just that there are many contexts where they are 
unnecessary.
}
\p{We do not need to contemplate far-fetched twin-earth 
scenarios to see this in practice: here on earth 
we have deuterium water which is chemically different 
from normal water (but both have the \htwoo{} signature, 
although heavy water is also described as \dtwoo{}).  
We are free to let \q{X} mean normal hydrogen, 
\q{Y} mean deuterium ions, and \q{Z} mean oxygn, so 
XYZ becomes what chemists call HDO \mdash{} semi-heavy water.  
Most people would probably say that HDO is just 
a kind of water, and so can be subsumed under the 
concept \q{water}, but this is not conclusive.  
In reality, I don't think the English community has 
needed to establish whether  \q{water} should mean 
ordinary \htwoo{} or should include variations 
containing different hydrogen isotopes 
\mdash{} whether heavy and semi-heavy and other variants 
of water should be considered \q{water} or some other 
concepts.
}
\p{In practice, a fraction of ocean water has deuterium, 
which might argue for \q{water} subsuming heavy water 
\mdash{} we don't point to the ocean and say
\begin{sentenceList}\sentenceItem{} The water and the Deuterium Dioxide is over there.
\end{sentenceList}
But this can alternatively be explained by the principle 
that referring to an impure sample of a substance is still 
a valid use of the concept: 
\begin{sentenceList}\sentenceItem{} Here's a glass of water (even though 
tap water is mixed with flouride).
\sentenceItem{} Bing cherries are dark red (even though 
the stem is brown).
\end{sentenceList}
In the second case, we can validly call something red 
even if something less than its whole surface shows a red 
color.  Applying a similar rule, we can call a solution 
\q{water} if there are only \q{sufficintly small} 
amounts of solutes.  Clearly we use \q{water} to dsignate 
many substances other than pure \htwoo{}.  I can 
think of two options for explaining that semantically: 
(1) Salt water, tap water, distilled water, (semi) 
heavy water, etc., are all different kinds of water, but 
our coarser \q{water} concept subsumes them all (in 
most contexts).
(2)  There is only one water concept, pure \htwoo{}, 
but impure samples of liquid that are mostly water 
can be called \q{water} by the same principle that a 
mostly red-colored object can be called just \q{red}. 
}
\p{The second option has a common-sensical appeal because 
it fits a succinct \q{concepts as natural kinds} 
paradigm but does not venture too far from 
normal language use \mdash{} that \q{red} 
actually means \q{mostly red} is a pattern common 
with many nouns and adjectives (someone can be 
\i{bald} with a bit of hair; I can point 
to a turkey burger made with bread crumbs and 
spices and say \q{that's turkey}; I can 
tell someone listening to Keny Arkana's song 
\q{Indignados} \q{that's French}, although some of 
the lyrics are Spanish).   However, the \q{mostly 
water} reading has a couple of problems: first, 
what about cases like a \q{glass of water} where 
\q{mostly water} is not \q{mostly} enough to drink?  
And, second, why can't we refer to plasma, say \mdash{} which 
is 92\% water \mdash{} as water?  This is not just a 
matter of numbers: the dead sea water is much 
less pure than plasma in the hospital (in terms of 
percentage \htwoo{} in solution) yet we are authorized 
to call the former \q{water} but not the latter.  
This certainly seems to be a matter of conceptual roles 
\mdash{} plasma occupies a certain place in our 
conceptual systems about blood and medicine (largely 
because it plays a specific role in biology 
and medicine) which does not fit the profike 
of \q{water}, while the stuff in lakes 
\i{does} fit that profile, even if the lakes are 
hypersaline.  Blood fits a conceptual ecosystem where 
we are not tempted to subsume it uner the concept 
\i{water}, whereas our conceptualization of 
lakes pulls in the opposite direction \mdash{} even 
though by purity the water in Gaet'ale Pond in 
Ethiopia is apparently not much more 
watery than blood.  Our disposition to 
either contract or dilate the sense \q{water} 
seems to be determined by context \mdash{} by the conceptual 
role water plays in different context \mdash{} rather than 
by actual hydrological properties.  
}
\p{What about the hypothetical twin-earth XYZ that 
Putnam imagines is indistinguishable from 
our \htwoo{}?  Well, for this hypoyhsis to 
even make sense we have to assume that XYZ is 
scientifically indistinguishable from water, 
which is a matter not just of pure \htwoo{} but 
of all solutions and deuterium- or triterium-related 
variants of water, and so forth.  As a thought experiment, 
where we are free to conceive almost anything, this is not 
impossible.  Let's imagine that there is an undiscovered 
subatomic particle that on some planets clings to 
atomic nuclei without affecting them in almost any 
way.  We can call nuclei harboring these particles 
\q{twin nuclei}, so hydrogen becomes \q{twin hydrogen}, 
oxygen becomes \q{twin oxygn}, and presumably 
water becomes \q{twin water}.  This twin water would 
essentilly retrace the the compositional structure of 
water \mdash{} since it would have to form (and unform, under 
electrolysis) just like \q{our} water.  If we plug this 
\q{twin water} into Putnam's scnario, I can't see why 
we don't just call this a variant kind of water, 
water with some extra (but observationally negligible) 
particles, just like heavy water is water with 
extra neutrons.
}
\p{This does not do perfect justice to \q{twin earth} 
discussions, because I am describing \q{twin} water 
as something whose composition is almost identical to 
\q{our} water.  In the original story, \q{twater} is 
XYZ, which as written suggests something whose 
physical constituents are much different than water, even 
if all propensities that influence our \q{water} 
conceptualizations are exactly the same as our water.  
But something compositionally different than 
water \i{can't} be functionally identical to water, 
at least if any of the actions we can take that reveal 
water's composition come out different.  In short, 
whatever XYZ are, they must have a capability to 
\i{become} hydrogen and oxygen, because XYZ's emulating 
water means it emits hydrogen and oxygen under 
electrolysis.  Meanwhile there is no action that could 
\q{release} the \q{X} (or whatever) because 
that would also behaviorally differ from water.  
So XYZ would differ from water only insofar as in its 
\q{unobserved} states it can float around as something 
without hydrogen or oxygen but, whenever subject to 
actions that cause water proper to emit these gasses, 
it would somehow conjure them up in exactly the same 
patterns as water (which actually \i{is} composed of 
hydrogen and oxygen) does.  
}
\p{By dictum, then, XYZ is not actually composed of 
hydrogen and oxygen, but whatever it \i{is} composed 
of can act as \i{as if} it \i{does} contain these gasses 
so as to emit them.  In that case I'd question the 
argumentative force of claiming that XYZ does not 
contain hydrogen and oxygen to begin with.  We are asked 
to believe that XYZ is made up of some 
ethereal non-hydrogen and non-oxygen that can 
nevertheless become hydrogen and oxygen whenever it 
is in the physical states wherein water that \i{is} made 
of hydrogen and oxygen will release them.  I am inclined 
to say that this is just another way of being made of 
hydrogen and oxygen.  After all, atoms are not little 
ping-pong balls: what we picture as a water molecule 
is actually apparently much more ethereal, suspended in 
quantum indeterminacy.  I take it there is some 
Shrodinger equation for a water molecule, and only 
when the \q{wave function} collapses \mdash{} say, by 
our observing the water subject to electrolysis \mdash{} 
do we actually get hydrogen or oxygen atoms.  So 
\q{our} water isn't really \q{composed} of hydrogen or 
oxygen in its pure quantum state.  Maybe XYZ 
\q{collapses} to hydrogen or oxygen in different 
ways than earthly water (but with no way to 
measure the difference), but this is still not 
divergent enough that for me to feel compelled 
to call XYZ anything other than some variant 
form of water.
}
\p{Of course, I am assuming that twin earthers have 
\i{the same} water-concept that we do, 
\i{in all respects}.  Maybe a more faithful 
review would consider that twin earthers might have a 
related but more primitive water-concept than ours 
\mdash{} maybe some subset of our concept in terms of 
the scientific knowledge embedded in our concept.  
Before we earthers knew about hydrogen, oxygen, 
or electrolysis, the behavior of water under 
electrolysis was not a factor in our concept of water.  
So imagine if twin earthers' level of scientific knowledge 
was akin to that on earth centuries ago \mdash{} their 
XYZ is measurably different from our water, but 
they have no experimental our scientific apparatus to 
notice the difference.  But this is \i{contingent}: 
the twin earthers \i{could} some day discover 
hydrogen and oxygen.  Then, if 
XYZ really is not composed of hydrogen and oxygen 
(or acts as if composed of them when not in a 
nonobservable ethereal state) their scientific theory 
of water, and accordingly their conceptualization, 
would diverge from ours. 
}
\p{We can imagine a non-water XYZ that is water-like enough 
to play an identical rooe to (our) water, but this story 
can go in two directions: either XYZ is \i{absolutely} 
identical to water, its differences from water so obscure 
as to be observationally and causally maningless; or it 
has legitimate differences from water that \i{could} be 
conceptually significant but in some context are not 
(at least not yet).  These are two different thought 
experiments.  If some substance is in alll respects and 
under any conceivable science identical to water, yet 
somehow compositionally different from it, I 
think the plausible respone among normal language communities 
would be to extend the concept of water \mdash{} subsuming 
XYZ under the concept, analogous to heavy water when it 
was discovered.  We are generally prepared to expand the 
reach of concepts when there is no compelling reason 
not to do so.  Whether a potential expansion takes 
hold probably varies by context.  We are \mdash{} a point 
that generally fits on the externalist side of the ledger 
\mdash{} more willing to accept expansion when the revised 
conceptualization would not deviate too far from 
a basic alignment of natural kind concepts to 
scientifically reasonable classifications.  We can readily 
extend \q{water} to \dtwoo{} because the two substances are 
compositionally very similar.  We are less likely 
to accept conceptual mergers when they seem to violate 
our natural-kind pictures, even if they 
are functionally plausible: we do not accept \q{agave} 
as a subconcept of \q{honey}, even though the two are 
physically rather similar and functionally very similar.  
Nor does physical form alone drive conceptual 
boundaries: we know full well that water vapor 
and ice are the same stuff as liquid water, but we 
recognize a conceptual distinction between them. 
}
\p{But these are not hard and fast rules: we may be 
inclined in many contexts to treat frozen-concntate 
juice as conceptually subsumed under \q{juice} 
(as in \q{juice on sale}), and we will often 
accept almond milk or cashew milk as \q{milk}, 
despite physical differences which we certainly 
acknowledge.  In short, concptual boundaries 
tend to be drawn to honor, albeit without 
excess granularity, both physical and functional 
factors \mdash{} neither physical/compositional similitude 
alone, in the absent of functionalv resemblance 
(see water/ice) tends to earn concept dilation, nor 
vice-versa, but a mixture of functional and physical 
similarity even with \i{some} differences in both 
aspects tend to be likelier drivers of concept-expansion 
(see water vs. chlorinated water, or red 
wine vs. white wine).  By these rules, expanding 
\q{water} to include XYZ \mdash{} if XYZ is functionaly 
identical to \i{but} compositionally different from 
water \mdash{} would be abnormal, like 
expanding \q{milk} to \mdash{} without any qualification \mdash{} 
include almond milk.  But 
these rules are approximate, and on the idiosyncratic 
case where XYZ is \i{completely} functionally like 
water but (stipulated to be) physically different 
(though by functional identity we could not detect 
as much), I think the normal \q{conceptual dilation} 
rules would side with the functional identity and 
ignore the physical differences.  
}
\p{On the other hand, if XYZ has real discoverable 
differences from water, then the potential exists 
for twin earthers' concept of water to 
divrge from our own, even if at any point 
in time the concepts are identical.  The time 
\q{points} don't need to be simultaneous: we can 
compare one country's concept of water in the year 
1800 with a diferent county's in the 16th century.  
It is plausible that different people at different 
times have effectively the same conceptual 
attitudes toward concepts that, with the benefit of 
hindsight and more science, we know have potential 
for differentiation.  I think the mere potential for 
differentiation warrants our identifying 
conceptual differences even if the parties involved are 
not aware of this potential.  I am prepared, 
for example, to accept that a child's water-concept 
in our time can be different from a medieval 
child's water-concept merely by virtue of the modern 
child potentially learning about deuterium, 
hypersalinity, and other scientific nuances that 
complicate the modern conception of water relative 
to our forebearers.
}
\p{We certainly accept that people may have different 
understandings of a concept and, on that basis, 
may judge that what two people are entertaining are 
two different concepts \mdash{} though we may also feel 
that they entertain two variations of \i{the same} 
concepts.  There's room for most concepts to 
\q{diversify}, subsuming subconcepts and 
variations; hence there's roon for a concept to 
expand (see water to heavy water) without fragmenting.  
But sometims we \i{do} insist on splitting concepts 
\mdash{} or, eqivalently, refuse to accept a concept-enlargement 
\mdash{} and \i{the reasons for this refusal may be external 
to some peoples' use of the concepts}.  Current 
political discourse in the United Stats, for example, is 
driven by turns of phrase that are rather haphazardly 
defined: \q{Climate Change}, \q{Border Wall}, \q{Green New Deal}, 
\q{Free Tuition}, etc.  Suppose a health policy 
expert observes that Bernie Sanders's use of the term 
\q{Medicare for All} is different from Kamala Harris's.  
She may conclude that Sanders's concept 
\q{Medicare for All} is different from Harris's 
concept \mdash{} and the rationale for this conclusion ned 
not take into account whether the two candidates are 
aware of the differences.  Suppose, as an expert, 
she has to mentally track the differences \mdash{} she 
has a well-informed judgment that each of the 
\q{Medicare for All} plans have different 
ramifications due to policy differences; 
as a result when discussing \q{Medicare for All} she 
needs to note in her own mind which version of that 
idea is under discussion at any moment in a discourse.  
That is to say, she needs to subsume them 
under different concepts.  Moreover, we endorse that she 
\i{should} do so, even if she thereby makes a distinction 
that the politicans or their supporters themselves 
do not realize.  In this kind of case we may defer 
to expert opinion when adjudicating a potential 
conceptual divorce, even if there is only 
minimal differences in the role of the concepts 
\visavis{} the conceptual systems 
of many relevant parties. 
}
\p{The possibility that \q{Medicare for All} 
may play the same \i{role} in a Sanders supporter's and 
a Harris supporter's conceptualizations does 
not preclude our judging that they are nonetheless 
different concepts \mdash{} if by virtue of 
more information and more access to expert 
counsel we can understand that there are potential 
differences in their conceptualizations that \i{could} 
drive the conceptual roles to diverge.  I think this 
is analogous to a \q{twin earth XYZ} scenario in that the 
thought experiment is set up as if we have access to 
expert confirmation that twin earth's 
XYZ is not physically the same substance as water.  
Projecting from earthly practice, we accordingly 
accept that \q{externalist} considerations may need to come 
to bear, and \q{XYZ} may need to be classifid as a different 
concept that water \i{notwithstanding} the lack 
of any concptual role difference between XYZ for 
twin earthers as compared to water for us.  This is 
consistent with our tolerance for including 
factors beyond just concptual roles in more mundane 
circumstances: we accept that sufficiently divergent 
notions of \q{Medicare for All} \i{could} be 
most appropriately classified as two different concepts.  
Such is not mandated \mdash{} we could certainly 
describe the Sanders and Harris platform as 
\q{two different Medicare for All plans}, subsuming 
them under one concept but acknowldging their 
differences \mdash{} as token differences, like the conceptual 
difference between this apple and that apple, rather 
than concept-differencs like apple vs. cherry.  
Analogously, we \i{could} subsume XYZ under the concept 
\i{water} \mdash{} XYZ being a kind of water insofar as 
samples of XYZ (tokens of the XYZ-concept) bear some 
physical differences to tokens of ordinary water (like 
heavy-water samples do), but we can handle this variation 
on a token-token level (analogous to comparing 
two apples).  But we can \i{also} split rather than 
expand the conceps \mdash{} \i{divorce} rather 
than \i{dilate} \mdash{} making XYZ a different concept than 
water, just as we can make Sanders supporters' Medicare 
for All a different concept that Harris supporters.  
The key point is that our choice of \q{divorce or dilate} 
may be driven by factors wholly external to 
some concept-bearers' internal concept-uses.  
Two different concepts \mdash{} recognized by us as different 
\mdash{} may play identical conceptual roles for some people.  
}
\p{This stance is at least minimally Externalist in that 
I don't insist on internal conceptual-role 
similitude being an immovable criteria selecting 
\q{dilate} over \q{divorce}.  We as a language 
community can and sometimes should override the tendency 
for concepts to expand under role considerations.  
As I pointed out earlier, a corkscrew and even a 
hammer can sometimes satisfy the role 
\q{bottle opener} in specific contexts.  Usually 
we distinguish context-specific 
concptual role-playing from general concept dilation 
\mdash{} I think this is the gist of Zhaohui Luo's 
analysis of \q{situations} and \q{Manifest entries}.  
We can adopt a temporary frame of reference wherein, say, 
hammers are bottle openers \mdash{} or in Luo's example 
(in a single zoo exhibit) all animals are snakes 
\mdash{} without mutating the concepts so 
wildly that \q{hammers} become expanded to including 
anything that may open a capped bottle, or 
\q{snakes} become all animals.  Yet such situational 
dilations can recur and eventually spill beyond 
their situational guard rails.  In a 
vegan cafe I can imagine the staff converging on a 
usage that soy, almond, and cashew milks are 
collectively called just \q{milk}.  If veganism becomes 
entrenched in some English-speaking community I can similarly 
imagine that in their dialect \q{milk} will mean anything 
that can be used like milk in a culinary context.  
The warrants for such expansions sem to be drivn by concptual 
roles \mdash{} situations present \q{slots}, 
like \i{that which opens this bottle} or \i{that which 
I pour on cereal}, and existing concepts tend to 
expand to fit these slots.  
}
\p{These considerations follow the \i{internalist} 
line: we take attitudes based on conceptual role 
more than external natural-kinds when adjudicating 
conceptual boundaries.  Thus situationally we may 
present almond milk and agave to satisfy a request for 
milk and honey.  But superimposed on \q{centrigual} 
tendency for concepts to expand into \q{under-lexified} 
concptual niches we have a counter tendency to 
question conceptual uses where functional 
resemblance strays \i{too far} from common sense.  
Someone may accept agave in lieu of honey, or a hammer 
as a bottle opener, in the context of how one 
situation plays out; but they are less likly 
to accept these uses becoming entrenched, compared to, 
say, refiguring \q{milk} to include almond and 
cashew milk.  And our hesitation to 
accept concept-expansion in these latter kinds 
of cases seems to implicitly look beyond 
concptual roles \mdash{} we may insist on limiting 
concept dilations even if there are many people for 
whom there will never be situations where the 
differences between concept referents, over and 
above functional resemblance, would be important.  
In short, even if a community could do 
just fine with some dialect idiosyncracy that 
ignores a conceptual distinction we would ordinarily 
make, we don't tend to take this as evidence 
that our multiple concepts can be merged into one more 
diverse concept.
}
\p{Of course we \i{can} merge concepts, and the fact that 
many people can live their lives without a conceptual 
coarsening may render such merger likelier, but 
it seems we evaluate potential mergers more by 
reference to entire speech-communities, not 
isolated parts.  Note that I am speifically talking here 
about merging or splitting concepts, not word-senses or 
lexemes or any purely linguistic artifacts.  
Certainly we have variegated \q{water} concepts 
\mdash{} salt, tap, distilled, heavy \mdash{} but we have an 
overarching water concept that includes these as 
subconcepts.  We can make a conscious decision to modify 
concept/subconcept relations \mdash{} which is 
different from changing how concepts are mappped to 
lexemes.  So I take it that Conceptual Role Semantics 
prioritizes role factors in drawing concept/subconcept 
relations and boundaries, and the consequence is a 
mostly Internalist intuitive model: we should 
accept concept maps where concepts are mostly 
drawn together when there is a functional resemblance 
between their roles: our concept/subconcept renderings 
should witness and help us exploit functional analogies.
}
\p{At the same time, however, I think we instinctively 
project notions of conceptual role outward from 
individual perople or subcommunities to the social 
totality.  Even if technically distinct Medicare for 
All plans play similar conceptual roles in different 
voters' conceptions, we understand that such similarity 
may break down as we expand the community outward.  Sanders 
and Haris supporters don't live on thier own islands.  
There are factors outside their own minds that weigh 
on whether their functionally similar Medicare for 
All concepts are indeed \i{the same concept} from the 
larger communiy's point of view.  But these external 
factors are not necessarily \i{extramental}: we 
can zoom outside the concptual patterns of one 
subcommunity and argue that concptual differences 
appear in the overall speech community that supercede 
functional resemblance in some subcommunity.  
Conceptual roles are not solipsistic: the role of 
the concept Medicare for All for a Sandrs supporter 
is not just a role in \i{his} mind, but it bcomes a role 
in \i{our} minds if we dialogically interact 
with him.
}
\p{Insofar as people can make inferences about other peoples' 
conceptual role \q{system} \mdash{} we can figure out 
the role which a concept plays in somone else's 
mind, to some approximation, even if analogous 
concepts play a different role in our own minds 
\mdash{} conceptual roles are not private affairs; the 
have some public manifestation and there is a need 
for collective reconciliation of role differences, 
just as we need to identify when different people 
are using the same words in different ways and 
use lexical conventions to diminish the 
chance of confusion.  To the extent that they have this public 
dimension, conceptual roles are 
not \i{internal}.  But \q{externalism} in this sense is 
waranted because we want to look philosophically 
at entire speech or cognitive communities \mdash{} it is not 
automatically a philosophy of conceptual content 
being external to \q{mind in general}.  Conceptual 
differences that could \i{potentially} become 
publicly observable from the vantage point 
of the \i{entire} cognitive community warrant 
consideration for conceptual divorce over dilation 
\mdash{} overriding similar roles in some \i{part} 
of the community. 
}
\p{In the case of XYZ, insofar as the twin earth cognitive 
community and our own could \i{potentially} become part 
of a single overarching cognitive community, we have 
potential grounds for drawing comparisons 
between water and XYZ.  Merely by contemplating 
their planet here on earth we are performatively 
drawing twin earthers into our cognitive community.   
By postulating that twin earthers 
think about XYZ the same way we think about water \mdash{} 
and that we know this \mdash{} we implicitly assume 
that their conceptual role patterns are public observables 
in the context of our own community.  If conceptual 
roles are observable, then there is a concept 
of a conceptual role: pundits can 
conceptually analyze how \q{Medicare for All} plays 
identical conceptual roles for Sanders and Harris 
supporters even if the candidates' plans are 
consequentially different.  But this merely says 
that there are latent differences in two people's 
conceptual roles that they themselves may not actually 
experience.  The public facet of conceptual roles 
complicates the notion of conceptual role similarity 
\mdash{} two persons' patterns of conceptual rols may 
be observably different as public phenoemena 
even if the people lack resources to realize the 
difference.  Conceptual roles are therefore external 
to individual minds \mdash{} but this is by scoping 
outside individual minds to holistic cognitive 
communities who can publicly observe our 
cognitive tendencies.  We are still reasoning 
\q{internalistically} in the sense of considering 
cognitive patterns at the scale of an overall 
cognitive community. 
}
\p{In short, I will take the mantra of an \q{Externalist} 
when passing from individual minds and subcommunities 
to the public nature of conceptual roles and 
overaching cognitive communities.  Once we get 
to the maximal possible community, however, 
I am inclined to revert to Internalism: if there 
is no broadening of communal scope that could 
make putative external diffrences meaningful to 
\i{anyone's} concptual roles, I see no reason to 
account for \i{these} erstwhile externalities in  a
theory of concepts.  If XYZ has \i{some} 
not-water-like qualities that a sufficiently large 
cognitive community could confront \mdash{} even if 
XYZ-conceptual-role and earthly-water-concptual-role 
is identical for the two isolated communities \mdash{} 
I am happy to accept that twin earthers' XYZ-concept 
is a different concept than earthers' water-concept.  
Similarly, I accept that Sanders supporters' 
Medicare for All concept may be a different concept 
than Harris supporters'.  But in both cases I accept 
concept splitting to ovrride role-similarity because 
I believe in an overarching cognitive community 
which has an interest in detecting differences or 
potential differences in conceptual roles qua 
public obsrvables, which transcendes our own internal 
awareness of what our conceptual roles entail. 
The fact that earthers and twin-earthers might never 
\q{discover} a water/XYZ difference is a contingent 
fact, not an essential structure in policing 
conceptual maps.   When establishing how 
we should consider redrawing thse maps, we should 
work from the picture of an overarching comunity that 
can subsume isolated communities as an abstract 
posit; the parts of the twin earth story that imply 
earthers and twin earthers could never actually discover 
their differences are not, I think, compelling as 
intrinsic features of the analysis.  In short, if 
water and XYZ have some potentially observable 
differences, then we need to proceed as the community which is 
aware that these differences exist and that therefore, for us, 
water and XYZ need different conceptual slots.   
The only analysis then is how to reconcile the fact 
that we have multiple conceptual slots whereas twin earthers 
(and earthers who have not read Hillary Putnam) have just one.
}
\p{But if we take a \i{maximal} cognitive community 
\mdash{} the sum total of earthers and twin 
earthers and philosophers \mdash{} this community \i{does} 
distinguish XYZ from water (surrely XYZ plays a different 
role in Putnam's mind than water).  And we should 
scope to the maximal community when determining whether 
smaller communities' conceptual roles are 
truly identical, because conceptual roles are, in part, 
potential public obervables for any possible supercommunity.
}
\p{On the other hand, if XYZ is so much like 
water that \i{no} community would 
\i{ever} have reason to contrast twin-earthers's 
XYZ-conceptual-role with our water-concptual-role, then 
I think these roles are not just 
\i{internally} identical for each (twin-) earther, but 
\i{publicly} identical for any conceivable 
cognitive community for whom public observations 
of (twin-) earthers' conceptualizations are 
consequential givens.  And in \i{that} case 
I think XYZ is the same concept as water 
notwithstanding putative compositional differences.  
}
\p{The whole idea that conceptual roles can be 
\i{public} complicates the Internalist/Externalist 
distinction, because each person's conceptual 
patterns can be evaluated from a vantage point 
external to \i{their} mind but still within 
the proclivities of a \q{maximal} cognitive community.  
Concptual roles are not private to each person, but 
are private inclinations that get reshaped, 
corrected, influenced, or reinterpreted by a 
larger community.  If we understand conceptual 
roles to include the totality not just of each 
person's conceptual role attitudes but the totality 
of how these attituds are observed by others, 
then we should consider that concepts are not 
\q{external} to the \i{maximal} cognitive comunity.  
Externalism about \i{individual} minds can be 
wrapped inside Internalism at the \i{maximal} 
inter-cognitive level. 
}
\p{But, complicating matters further, the maximal 
community's observations of conceptual-role attitudes 
is often driven by at least our \i{belifs} about 
external (i.e., extramental, natural-kind) 
criteria.  For example, some companies 
want to rechristen \q{corn syrup} as \q{corn sugar}, 
to make it seem more like a sugar-subconcept.  Meanwhile, 
some dairy companis want laws restricting the use of 
\q{milk} for vegan products.  In both cases our larger 
community has a chance to weigh the proper conventions 
for how our conceptual maps should be drawn.  
As I argued earlier, both functional and naturalistic 
criteria play a role in such deliberations.  We are 
poised to distinguish transient situation-specific 
roles \mdash{} that one time someone used a hammer 
as a bottle opener \mdash{} from functional 
parallels that stretch across many contexts.  
Within the parameters of that contrast, we are 
receptive to redrawing maps on role 
criteria \mdash{} allowing milk to subsume 
vegan milk-substitutes, for instance.  But this 
tendency is balanced by a respect for some 
notion of coherent natural kinds \mdash{} the 
distinct biological properties of vegan milks 
work against a \i{maximal} community subsuming 
them under \q{milk} outside of special contexts. 
}
\p{Both the Externalist and Internalist points of 
view have some traffic in the considerations that 
cognitive communities bring to bear on 
which concptual maps should be enorsed by convention.  
Because ad-hoc conceptual roles 
can be established for particular situations, 
we can be conservative about \i{conventionalizing} 
concept maps driven by functional correspondancs 
too far removed from (what we think 
to be) scintifically endorsed, natural-kind 
boundaries.  In other words, I think we \i{do} and 
\i{should} allow \q{naturalistic} considerations 
to be a factor in what concept maps we endorse.  
But this is not a claim about Externalism as 
a philosophical paradigm shaping how we 
should construe the triangulation between 
mind, world, and language, as a matter of 
metaphysical ideology.  Rather I believe 
that \q{externalist} factors should and do come 
to bear on the deliberations \i{internal} 
to cognitive communities' (sometimes but 
not always explicit) evaluations of how to draw 
concept and subconcept boundaries and relations 
\mdash{} when to split concepts and when to dilate them.  
Dilate-or-divorce options are are pulled by 
both externalist and internalist considerations, 
sometimes in competing ways.  
}
\p{As as case-study, the wording \q{corn sugar} \mdash{} which implies a 
\q{redistricting} wherein the concept \q{corn syrup} becomes 
part of the territory \q{sugar} \mdash{} may be credible 
on purly biochemical grounds.  But our community may feel 
that there is enough functional difference between sugar 
and corn syrup from a commercial and nutritional sense to 
reject a proposed merger \mdash{} here functional considerations 
trump natural-kind ones.  Conversely, the community may be 
sympathetic to claims that milk substitutes should be labeled 
to clearly indicate how they are are not \i{literally} milk 
\mdash{} here natural-kind considerations trump functional ones.
}
\p{If we consider language \mdash{} and communally-endorsed 
conceptualizations \mdash{} evolving in practice, then 
by light of my claims until here there is material 
for both Externalist and Internalist readings.  
This perhaps leaves room for a theory which 
accepts that both are partially true \mdash{} each 
being logically founded under consideration of two 
different aspects of how concepts evolve.  I will 
explore this possibility further, but first I 
want to shore up my account of conceptual roles 
themselves.  One complication I have 
glossed so far is that \i{functional} roles in an 
enactive and \q{pragmatic} (in the everyday-world 
sense) spheres are not \i{ipso facto} the 
same as either conceptualizations 
(conceptual-role-attitudes) or lexicosemantic 
conventions.  Thse three are interrelated, but we 
need social and cognitive practices to get 
situational understandings entrenched in 
language and in communal concept-maps.  
Without a theory of this process, to speak of 
functional roles like \i{hammer} for 
\i{bottle opener} is not a substitute for 
speaking of conceptual roles \i{per se}.  
How to properly link \q{functionality} in an 
enactive quotidian sense \mdash{} the data that 
various narural and man-made artifacts are 
used by people for cocrete tasks, and 
we often talk about this \mdash{} to the cognitive 
realm of concepts (and their boundaries and 
subconcept relations)?  This is the main theme 
of my next section. 
}
\p{}
