\subsection{The illogic of syntax}
\p{Let us agree that \mdash{} beneath surface-level 
co-framing complxity \mdash{} many language acts have a 
transparent content as \q{doxa} that gets conveyed between 
people with sufficiently resonant 
cognitive frames.  This is still not enough to 
elevate doxa to the \i{meaning} of such language-acts.  
For a logic-driven theory, we have 
to thematize not only \i{that} expressions refer 
to or convey propositional content, but \i{how} they 
do so.  It is not only the content itself but the 
\q{how} that should be analyzed through a logical structure.  
}
\p{Since it is widely understood that the essence of language 
is compositionality, the clerest path to 
a logic-based theory would be via the 
\q{syntax of semantics}: a theory of how 
language designates propositional content by 
emulating or iconifying propositional structure 
in its own structure (i.e., in grammar).  
This would be a theory of how linguistic 
connectives reciprocate logical connectives, 
phrase hierarchies reconstruct propositional 
compounds, etc.  It would be the kind of theory motivated 
by cases like 
\begin{sentenceList}\sentenceItem{} This wine is a young Syrah.
\sentenceItem{} My cousin adopted one of my neighbor's dog's puppies.
\end{sentenceList}
where morphosyntactic form \mdash{} possessives, adjective/noun 
links \mdash{} seems to transparently recapitulate predicate 
relations.  Thus the wine is young \i{and} 
Syrah, and the puppy if the offspring of a dog who 
is the pet of someone who is the neighbor of the speaker.  These 
are well-establishd logical forms: predicate conjunction, here; 
the chaining of predicate 
operators to form new operators, there.  Such are embedded in 
language lexically as well as grammatically: the conjunction 
of husband and \q{former, of a prior time} yields ex-husband; 
a parent's sibling's daughter is a cousin.  
}
\p{The ineteresting question is to what extent \q{morphosyntax 
recapitulates predicate structure} holds in general 
cases.  This can be considered by examining the logical 
structure of reported assertions and then the structures 
via which they are expressed in language.  I'll 
carry out this exercise \visavis{} several sentences, 
such as these:
\begin{sentenceList}\sentenceItem{} The majority of students polled were 
opposed to tuition increases.
\sentenceItem{} Most of the students expressed disappointment 
about tuition increases.
\sentenceItem{} Many students have protested the tuition increases.
\end{sentenceList}
}
\p{There are several logically significant elements here that 
seem correspondingly expressed in linguistic 
elements \mdash{} that is, to have some model 
in both prelinguistic predicate structure 
and in, in consort, semantic or syntactic principles.  
All three of ()-() have similar but not 
identical meanings, and the differences are 
manifest both propositionally and 
linguistically (aside from the specific superficial 
fact that they are not the same sentence).  
I will review the propositional differences first, 
then the linguistic ones.
}
\p{One obvious predicative contrast is that () and () 
ascribes a certain \i{quality} to students (e.g., disappointment), 
whereas () and () indicate \i{events}.  As such 
the different forms capture the contrast between \q{bearing 
quality $Q$} and \q{doing or having done action 
$A$}: the former a predication and the latter an event-report.  
In the case of (), both forms are available because we can 
infer from \i{expressing} disappointment 
to \i{having} disappointment.   
There may be logics that would map one 
to the other, but let's assume we can 
analyze language with a logic expressive enough 
to distinguish events from quality-instantiations.
}
\p{Other logical forms evident here involve how the 
subject noun-phrases are constructed.  
\q{A majority} and \q{many} imply a multiciplicity 
which is within some second multiplicity, and 
numerically significant there.  The sentences differ 
in terms of how the multiciplicities are circumscribed.  
In the case of \q{students polled}, an extra determinant is 
provided, to construct the set of students forming the 
predicate base: we are not talking about students in 
general or (necessarily) students at one school, 
but specifically students who participated in a poll.   
}
\p{Interrelated with these effects are how the 
\q{tuition increases} are figured.  Using 
the explicit definite article suggests that there is 
\i{some specific} tuition hike policy 
raising students' ire.  This would also favor a 
reading where \q{students} refers collectively to those 
at a particular school, who would be directly affected by the 
hikes.  The \i{absence} of an article on 
\q{tuition increases} in () leaves open an interpretation 
that the students are not opining on some specific policy, but on 
the idea of hikes in general.
}
\p{Such full details are not explicitly laid out in the sentences, 
but it's entirely possible that they are clear in context.  
Let's take as given that, in at least some cases where they would 
occur, the sentences have a basically pristine 
logical structure given the proper contextual framing \mdash{}  
context-dependency, in and of itself, does not weaken 
our sense of language's logicality.  In particular, 
the kind of structures constituting the sentences' 
precise content \mdash{} the details that seem context-dependent 
\mdash{} have bona fide logical interpretations.  For 
example, we can consider whether students are 
responding to \i{specific} tuition hikes 
or to hikes in general.  We can consider 
whether the objectionable hikes have already 
happened or are just proposed.  Context 
presumably identifies whether \q{students} are 
drawn from one school, one governmental 
jurisdiction, or some other aggregating criteria 
(like, all those who took a poll).  
Context can also determine whether aggregation is 
more set- or type-based, more extensional 
or intensional.  In ()-() the implication 
is that we should read \q{students} more as a set or 
collection, but variants like \i{students 
hate tuition hikes} operates more at the level 
of students as a \i{type}.  In \q{polled students} 
there is a familiar pattern of referencing a set by 
marrying a type (students in general) with a descriptive 
designation (e.g., those taking a specific 
poll).  The wording of () does not 
mandate that \i{only} students took the poll; it 
does however employ a type as a kind of operator on a set: 
of those who took the poll, focus on students 
in particular.
}
\p{These are all essentially logical structures and can 
be used to model the propositional 
content carried by the sentences \mdash{} their \q{doxa}.  
We have operators and distinctions like past/future, 
set/type, single/multiple, subset/superset, and 
abstract/concrete comparisons like tuition hikes \i{qua} idea 
vs. \i{fait accompli}.  A logical system could 
certainly model these distinctions and accordingly capture  
the semantic differences between ()-().  So such details 
are all still consistnt with a truth-theoretic paradigm, although 
we have to consider how linguistic form actually 
conveys the propositional forms carved out via 
these distinctions.
}
\p{Ok, then, to the linguistic side.  My first observation is that some 
logically salient structurs have fairly clear analogs 
in the linguistic structure.  For instance, the logical operator for 
deriving a set from criteria of \q{student} merged with \q{taking a 
poll} is brought forth by the verb-as-adjective 
formulation \q{polled students}.  Subset/superset arrangements 
are latent as lexical norms in senses like \q{many} and 
\q{majority}.  Concrete/abstract and past/future distinctions 
are alluded to by the presence or absence of a definite 
article.  So \q{\i{the} tuition increases} connotes that 
the hikes have already occurred, or at least been 
approved or proposed, in the past relative to the 
\q{enunciatory present} (as well as that they are a 
concrete policy, not just the idea), 
whereas articleless \q{tuition increases} can be read as 
referring to future hikes and the idea of hikes 
in general: past and concrete tends to 
contrast with future and abstract.
}
\p{A wider range of 
logical structures can be considered by subtly varying 
the discourse, like: 
\begin{sentenceList}\sentenceItem{} Most students oppose the tuition increase.
\sentenceItem{} Most students oppose a tuition increase.
\end{sentenceList}
These show the possibility of \i{increase} being singular 
(which would tend to imply it refers to a concrete 
policy, some \i{specific} increase), although in 
() the \i{in}definite article \i{may} connote 
a discussion about hikes in general.
}  
\p{But maybe not; cases like these are perfectly plausible:
\begin{sentenceList}\sentenceItem{} Today the state university system 
announced plans to raise tuition by 
at least 10\%.  Most students oppose a tuition increase.
\sentenceItem{} Colleges all over the country, facing 
rising costs, have had to raise tuition, but 
most students oppose a tuition increase.
\end{sentenceList}
In () the definite article could also be used, but saying 
\q{\i{a} tuition increase} seems to reinforce the 
idea that while plans were announced, the details 
are not finalized.  And in () the plural \q{increases} 
could be used, but the indefinite singular connotes the 
status of tuition hikes as a general phenomenon 
apart from individual examples \mdash{} even though the 
sentence also makes reference to concrete examples.  In other 
words, these morphosyntactic cues are like 
levers that can fine-tune the logical designation 
more to abstract or concrete, past or future, as 
the situation warrants.  Again, context should 
clarify the details.  But morphosyntactic forms 
\mdash{} e.g., presence or absence of articles (definite 
or indefinite), and singular/plural \mdash{} are 
vehicles for language, through its own 
forms and rules, to denote propositional-content structures 
like abstract/concrete and past/future.
}
\p{Other logical implications may be more circuitous.  For instance, 
describing students as \i{disappointed} 
implies that the disliked hikes have already occured, 
whereas phraseology like \q{students are gearing for a fight} 
would have the opposite effect.  The mapping from 
propositional-content structure to surface language here 
is less mechanical than, for instance, merely 
using the definite article on \q{the tuition increases}.  
Arguably \q{dissapointment} \mdash{} rather than just, say, 
\q{opposition} \mdash{} implies a specific timeline 
and concreteness, an effect analogous to the definite 
article.  But the semantic register 
of \q{dissapointment} bearing this implication is a 
more speculative path of conceptual resonances, compared 
to the brute morphosyntactic \q{the}.  There is subtle 
conceptual calculation behind the scenes in the former 
case.  Nonetheless, it does seem as if via this 
subtlety linguistic resources are expressing 
the constituent units of logical forms, like 
past/future and abstract/concrete.
}
\p{So, I am arguing (and conceding) that there are units of 
logical structure that are conveyed by units of 
linguistic structure, and this is partly how 
language-expressions can indicate propositional 
content.  The next question is to explore this 
correspondance compositionally \mdash{} is there a 
kind of aggregative, hierarchical order in terms 
of how \q{logical modeling elements} fit together, 
on one side, and linguistic elements fit 
together, on the other?  
There is evidence of compositional concordance 
to a degree, examples of which I have cited.  In 
\q{polled students}, the compositional structure of 
the phrase mimics the logical construct \mdash{} deriving 
a set (as a predicate base) from a type crossed with 
some other predicate.  Another example is the 
phraseology \q{a/the majority of}, which directly 
nominates a subset/superset relation and so 
reciprocates a logical quantification (together 
with a summary of relative 
size; the same logical structure, but with 
different ordinal implications, is seen in cases 
like \q{a minority of} or \q{only a few}).  
Here there is a relatively mechanical translation 
between propositional structuring elements 
and linguistic structuring elements.
}
\p{However, varying the examples \mdash{} for instance, 
varying how the subject noun-phrases are conceptualized 
\mdash{} points to how the synchrony between 
propositional and linguistic composition can break down: 
\begin{sentenceList}\sentenceItem{} Student after student came out against the tuition hikes.
\sentenceItem{} A substantial number of students 
have come out against the tuition hikes.
\sentenceItem{} The number of students protesting the tuition hikes 
may soon reach a critical mass.
\sentenceItem{} Protests against the tuition hikes 
may have reached a tipping point.
\end{sentenceList}
Each of these sentences says something about a large number 
of students opposing the hikes.  But in each 
case they bring new conceptual details to the fore, and 
I will also argue that they do so in a way that 
deviates from how propositional structures are composed.
}
\p{First, consider \q{student after student} as a way of 
designating \q{many students}.  There's a little more 
rhetorical flourish here than in, say, \q{a majority 
of students}, but this is not just a matter of 
eloquence (as if the difference were stylistic, 
not semantic).  \q{Student after student} creates a 
certain rhetorical effect, suggesting via how it 
invokes a multiplicty a certain recurring or 
unfolding phenomenon.  One imagines the speaker, time 
and again, hearing or encountering an angry student.  
To be sure, there are different kinds of contexts 
that are consistent with (): the events could 
unfold over the course of a single hearing 
or an entire semester.  Context would foreclose 
some interpretations \mdash{} but it would do so in any 
case, even with simpler designations like \q{majority 
of students}.  What we \i{can} say is that the speaker's  
chosen phraseology cognitively highlight a 
dimension in the events that carries a certain 
subjective content, invoking their temporality and 
repetition.  The phrasing carries an effect of 
cognitive \q{zooming in}, each distinct event 
figured as if temporally inside it; the sense 
of being tangibly present in the midst of the event is 
stronger here than in less temporalized language, 
like \q{many students}.  And then at the same time 
the temporalized event is situated in the context of many 
such events, collectively suggesting a recurring presence.  
The phraseology zooms in and back out again, 
in the virtual \q{lens} our our cognitively figuring 
the discourse presented to us \mdash{} all in just 
three or four words.  Even if \q{student after student} 
is said just for rhetorical effect \mdash{} which 
is contextually possible \mdash{} \i{how} it 
stages this effect still introduces a subjective 
coloring to the report.  
}
\p{Another factor in () and () is the various possible 
meanings of \q{come out against}.  This could be 
read as merely expressing a negative opinion, or as a 
more public and visible posturing.  In fact, a similar 
dual meaning holds also for \q{protesting}.  Context, 
again, would dictate whether \q{protesting} means 
actual activism or merely voicing displeasure.  Nonetheless,  
the choice of words can shade how we frame situations.  
To \q{come out against} connotes expressing disapproval 
in a public, performative forum, inviting 
the contrast of inside/outside (the famous example 
being \q{come out of the closet} to mean publicly 
identifying as LGBT).  Students may not literally 
be standing outside with a microphone, but \mdash{} even 
if the actual situation is just students 
complaining rather passively \mdash{} using \q{come out against} 
paints the situation in an extra rhetorical hue.  The students 
are expressing the \i{kind} of anger that can goad 
someone to make their sentiments known theatrically and 
confrontationally.  Similarly, using \q{protest} in lieu 
of, say, \q{criticize} \mdash{} whether or not students are actually 
marching on the quad \mdash{} impugns to the students a level 
of anger commensurate with policized confrontation.   
}
\p{All these sentences are of course \i{also} compatible with 
literal rioting in the streets; but for sake of argument 
let's imagine ()-() spoken in contexts 
where the protesting is more like a few comments to a school 
newspaper and hallway small-talk.  The speakers have still 
chosen to use words whose span of meanings 
includes the more theatrical readings: \q{come out against} 
and \q{protest} overlap with \q{complain about} or \q{oppose}, 
but they imply greater agency, greater intensity.  These lexical 
choices establish subtle conceptual variations; for instance, 
to \i{protest} connotes a greater shade of anger than to \i{oppose}.
}
\p{Such conceptual shading is not itself unlogical; one 
can use more facilely propositional terms to evoke 
similar shading, \q{like very angry} or \q{etremely angry}.  
However, consider \i{how} language like ()-() 
conveys the relevant facts of the mater: there is 
an observational, in-the-midst-of-things staging at 
work in these latter sentences that I find missing 
in the earlier examples.  \q{The majority of} sounds 
statistical, or clinical; it suggests journalistic reportage, 
the speaker making an atmospheric effort to sound like someone 
reporting facts as established knowledge rather than 
observing them close-at-hand.  
By contrast, I find ()-() to be more \q{novelistic} than 
\q{journalistic}.  The speaker in these cases is reporting 
the facts by, in effect, \i{narrating} them.  She is building 
linguistic constructions that describe propositional 
content through narrative structure \mdash{} or, at least, cognitive structures 
that exemplify and come to the fore in narrative understanding.  Saying 
\q{a substantial number of students}, for example, rather than 
just (e.g.) \i{many} students, employs semantics  
redolant of \q{force-dynamics}: the weight of student 
anger is described as if a \q{substance}, something with the 
potency and efficacy of matter.
}
\p{This theme is also explicit in 
\q{critical mass}, and even \i{tipping point} has material 
connotations.  We can imagine different versions of what 
lies one the other side of the tipping point 
\mdash{} protests go from complaining to activism?  The school 
forced to reverse course?  Or, contrariwise, the 
school \q{cracking down} on the students 
(another partly imagistic, partly force-dynamic metaphor)? 
Whatever the case, language 
like \q{critical mass} or \q{tipping point} is language 
that carries a structure of story-telling; 
it tries tie facts together with a narrative coherence.  The 
students' protests grew more and more strident until ... 
the protests turned aggressive; or the school dropped its plans; 
or they won publicly sympathy; or attracted media attention, etc.  
Whatever the situation's details, describing the facts in 
force-dynamic, storylike, spatialized language (e.g. \q{come \i{out} 
against}) represents an implicit attempt to report 
observations or beliefs with the extra fabric and completeness 
of narrative.  It ascribes causal order to how the 
situation changes (a critical mass of anger could \i{cause} the 
school to change its mind).  It brings a photographic or 
cinematic immersion to accounts of events and descriptions: 
\i{student after student} and \i{come out} invite us 
to grasp the asserted facts by \i{imagining} situations. 
}
\p{The denoument of my argument is now that these narrative, cinematic, 
photographic structures of linguistic reportage \mdash{} signaled 
by spatialized, storylike, force-dynamic turns of phrase \mdash{} 
represent a fundamentally different way of signifying 
propositional content, even while they 
\i{do} (with sufficient contextual grounding) carry 
propositional content through the folds of the narrative.  
I don't dispute that hearers understand logical forms 
via ()-() similar to those more \q{journaslistically} 
captured in ()-().  Nor do I 
deny that the richer rhetoric of ()-() play a logical 
role, capturing granular shades of meaning.  My 
point is rather that the logical picture painted by the latter 
sentences is drawn via (I'll say as a kind of suggestive 
analogy) \i{narrative structure}.
}
\p{I argued earlier that elements of propositional 
structure \mdash{} for example, the 
set/type selective operator efficacious in \q{students polled} \mdash{} can 
have relatively clean morphosyntactic manifestation in 
structural elements in language, like the verb-to-adjective 
mapping on \q{polled} (here denoted, in English, by unusual 
word position rather than morphology, although the 
rules would be different in other languages).  Given my 
subsequent analysis, however, I now want to claim that 
the map between propositional structure and linguistic structure 
is often much less direct.  I'm not arguing that 
\q{narrative} constructions lack logical structure, or 
even that their rhetorical dimension lies outside 
of logic writ large: on the contrary, I believe 
that they use narrative effects to communicate granular 
details which have reasonable logical bases, like 
degrees of students' anger, or the causative 
interpretation implied in such phrases as \q{critical mass}.  
The rhetorical dimension does not prohibit a reading 
of ()-() as expressing propositional content \mdash{} and using 
rhetorical flourishes to do so. 
}
\p{I believe, however, that \i{how} they do so unzips any neat 
alignment between linguistic and propositional structure.  
Saying that students' protests \q{may have reached a critical mass} 
certainly expresses propositional content (e.g., that enough 
studnts may now be protesting to effectuate change), 
but it does so not by mechanically asserting its propositional 
idea; instead, via a kind of mental imagery which portrays 
its idea, in some imaginative sense, iconographically.   
\q{Critical mass} compels us to read its meaning imagistically; 
in the present context we are led to actually visualize 
students protesting \i{en masse}.  Whatever the actual, empirical 
nature of their protestation, this language paints a picture 
that serves to the actual situation as an interpretive 
prototype.  This is not only a conceptual image, but 
a visual one.
}
\p{Figurative language \mdash{} even if 
it is actually metaphorical, like \q{anger boiling over} 
\mdash{} has similar effect.  Alongside the analysis 
of metaphor as \q{concept blending}, persuasively 
articulated by writers like Gilles Fauconnier and 
Per Aage Brandt, we should also recognize how metaphor 
(and other rhetorical effects) introduces into 
discourse language that invites visual imagery.  Sometimes 
this works by evoking an ambient spatiality (like 
\q{come out against}) and sometimes by figuring 
phenomena that fill or occupy space (like \q{students protesting} 
\mdash{} one salience of this language is that we imagine 
protest as a demonstrative gesture expanding outward, as if 
space itself were a theater of conflict: 
protesters arrayed to form long lines, fists splayed 
upward or forward).  There is a kind of visual patterning 
to these evocations, a kind of semiotic grammar: 
we can analyze which figurative senses work via connoting 
\q{ambient} space or via \q{filling} space, 
taking the terms I just used.  But the details of such a 
semiotic are tangential to my point here, which is that 
the linguistic structures evoking 
these visual, imagistic, narrative frames are not 
simply reciprocating propositionl structure 
\mdash{}- even if the narrative frames, via an \q{iconic} 
or prototype-like modeling of the actual situation, 
\i{are} effective vehicles for \i{communicating} 
propositional structure. 
}
\p{What breaks down here is not propositionality but \i{compositionality}: 
the idea that language signifies propositional content \i{but 
also} does so compositionally, where we can 
break down larger-scale linguistic elements to smaller parts 
\i{and} see logical structures mirrored in the parts' 
combinatory maxims.  In the later examples, I have argued that 
the language signifies propositional content by creating narrative 
mock-ups.  The point of these imagistic frames 
is not to recapitulate logical structure, but to have a kind 
of theatrical coherence \mdash{} to evoke visual and narrative order, 
an evolving storyline \mdash{} from which we then understand 
propositional claims by interpreting the imagined scene.  
Any propositional signifying in these kinds of 
cases works through an intermediate stage of 
narrative visualization, whose structure is 
holistic more than logically compositional.  
It relies on our faculties for imaginative 
reconstruction, which are hereby drafted into 
our language-processing franchise.
}
\p{This kind of language, in short, leverages its 
ability to trigger narrative/visual framing as a 
cognitive exercise, intermediary to the eventual 
extraction of propositional content.  As such 
it depends on a cognitive layer of narrative/visual 
understanding \mdash{} which, I claim, belongs 
to a different cognitive register than building
logical models of propositional content directly.
}
\p{In the absence of a compelling analysis of \i{compositionality} 
in the structural correspondance between narrative-framed language 
and logically-ordered propositional content, I consequently think we 
need a new theory of how the former signifies the latter.  My own 
intuition is that language works by trigering \i{several 
different} cognitive 
subsystems.  Some of these hew closely to predicate 
logic; some are more holistic and narrative/visual.  
Cognitive processes in the second sense may be informed 
and refined by language, but they have an extralinguistic 
and prelinguistic core: we can exercise faculties of narrative 
imagination without explicit use of 
language (however much language orders our imaginations 
by entrenching some concepts more than others, via lexical 
reinforcement).  
}
\p{I'm not just talking here about \q{imagination} in the sense 
of fairy tales: we use imaginative cognition to make sense 
of any situation dscribed to us from afar.  When presented with 
linguistic reports of not-directly-obsevrable situations, 
we need to build cognitive frames modeling the context 
as it is discussed.  In the terms I suggested earlier, we 
build a \q{doxa inventory} tracking beliefs and 
assertions.  Sometimes this means internalizing 
relatively transparent logical forms.  But sometimes 
it means building a narative/visual account, playing 
an imaginary version of the situation in our minds.  
Language could not signify in its depth and 
nuance without triggering this \i{interpretive-imaginative} 
faculty.  Cognitively, then, language is an 
\i{intermediary} to this cognitive system, 
an \i{interface}.  To put it as a slogan, 
\i{language is an interface to interpretive-imaginative 
cognitive capabilities}. 
}
\p{If this claim about \i{language} seems plausible, it 
has some ramifications for \i{linguistics}: 
insofar as language has a formal articulation, it 
is the formality of an \i{interface}, 
which is not necessarily the same thing as 
the formality of a \i{logical system}.  Insofar as 
linguistics is a science, it would then be the science 
of the intermediate space between grammatical plus lexical 
observations and interpretive-imaginative 
cognition.  Framing linguistics in these 
terms is, I believe, analogous to describing 
Biology as the interface from medicine to chemistry and 
physics \mdash{} with analogous philosophical justifications 
and metatheortical consequences.  Both can be seen as a 
larger metascientific exploration of what it means \mdash{} 
as a philosophical claim, on the one hand, and as a normative 
proscription on scientific practice, on the other \mdash{} for a 
\i{science} to be an \i{interface}.   
}
\p{In the specific context of linguistics, one consequence is 
that any linguistic structuring element becomes an 
intermediary eventually handed off to interpretive-imaginative 
cognitive processes \mdash{} analogous, in the 
computational realm, to application-level function 
implementions setting up kernel system calls.  On  
this comparison, intermediate structures of language 
understanding are like source code, and interpretive-imaginative 
cognition corresponds to the system kernel.  Insofar 
as intermediate linguistic structures are analyzed 
via type theory, we can accordingly model this 
situation as type properties modulating how 
language elements carry over to the interpretive-imaginative 
cognitive \q{kernel} \mdash{} which is a faithful analogy 
to how types work when applied to, say, \Cpp{} 
coding structures.  Insofar as linguistic understanding 
is viewed through the lense of Conceptual Roles, 
we can analyze how the \i{conceptual role} 
which some referent plays \mdash{} more than its specific 
referenced nature \mdash{} determines how it is \q{communicated} 
to the \q{interpretive-imaginative kernel}.  In fact, 
this last topic is an analysis I will now consider further.
}
