%\OldSection*{Addendum}
\spsubsection{Addendum}
\p{The following are more inclusive code samples
demonstrating compilation and evaluation at both
a syntactic and semantic level.  Figure \ref{lst:rzsyn}
shows a list of \q{prerules} (\OneOverlay{}) in a special parser-definition
format for Context-Sensitive grammars,
which translate to Regular Expression code, and
full-fledged \q{rules} which combine Regular Expressions,
named captures (\TwoOverlay{}),
contextual tests (\ThreeOverlay{}, \FourOverlay{}, \FiveOverlay{}), and \Cpp{}
callbacks (\SixOverlay{}).
Figure \ref{lst:rzsem} samples the code which leads up to dynamic
method-invocations, and then the actual invocation-points (\OneOverlay{}):
the latter is
wrapped in macros, for the code to work with various numbers
of arguments.  A key feature of this code is various sets
of \q{conventions} \mdash{} for input arguments, return values, binary
encoding, etc. \mdash{} which deal with differences in how the
\Qt{} Meta-Object System (and generic function pointers) can
handle the encoding of disparate data types into
a small set of fundamental binary structures, such as
\QString{}s; \QObject{}- and \ptrv{}-pointers; and
64-bit unsigned integers.
}
\itcl{rzsyn}
\vspace{-1em}
\itcl{rzsem}
\vspace{-2em}