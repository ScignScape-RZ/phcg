\subsubsection{Channelized-Type Interpretations of
Larger-Scale Source Code Elements}
\p{By intent, Channel Algebras provide a machinery for modeling
function-call semantics more complex than \q{pure} functions
which have only one sort of input parameter (as in lambda
abstraction) \mdash{} note that this is unrelated to
parameters' \i{types} \mdash{} and one sort of (single-value) return.
%As just discussed,
Examples of a more complex paradigm
come from Object-Oriented code, where there are two varieties of
input parameters (\q{lambda} and \q{sigma}).
Consider method calls in Object-Oriented languages: these
typically have a special syntax with one
distinguished carrier.  This
carrier is in a sense privileged: the type of
the \q{\thisc{}} carrier establishes the class to which
function belongs, influencing when the function may be
called and how polymorphism is resolved.  Moreover,
\q{chaining} method calls means that the result of
one method becomes an object that may then receive 
another method (the following one in the chain).  Such 
chaining allows for an unambiguous function-composition operator:
since functions in general take multiple arguments, there is
no single operator to pass the result of one function to
another; but since most methods have one return value and
one \thisc{} object, it is straightforward to
define a method-chain operator.
}
\p{Another case-study
is offered by exceptions.  The option to \q{throw} an exception
can be considered an alternative kind of output channel.
A function throws an exception instead of returning a value.
As a result, \returnc{} and \errorc{} channels typically
evince a semantic requirement (which earlier \mdash{} see
the notations on page \hyperref[retexc]{\pageref{retexc}} \mdash{} I sketched
as an algebra stipulation): when functions
have both kinds of channels, only one
may have an initialized carrier after the function
returns.
Usually, thrown-exception values can only be bound to carriers
in \catchddd{} formations \mdash{} once held in a carrier
they can be used normally, but carriers in \exceptionch{}
channels themselves can only transfer values to
other carriers in narrow circumstances
(this in turn depends on things like code
blocks, which in turn will be reviewed below).
So \errorc{} channels are not a sugared
form of ordinary returns, any more than objects are sugar for
functions' first parameter; there are axiomatic criteria
defining possible formations of \errorrc{} and \returnrc{}
channels and carriers, criteria which are more transparently
rendered by recognizing \errorrc{} and \returnrc{} as
distinct channels of communication available
within function bodies.
}
\p{%This terminology serves as a basis for exploring the
%variation in how functions receive and share values.
In general, extensions to \lambdaCalculus{} are
meaningful because they model semantics other than
ordinary lambda abstraction.  For example, method-calls
(usually) have different syntax than non-method-calls,
but \sigmaCalculi{} aren't trivial extensions or
syntactic sugar for \lambdach{}s; the more significant difference is
that sigma-abstracted symbols and types have different
consequences for overload resolution and function
composition than \lambdach{}-abstractions.  Similarly,
exceptions interact with calling code differently
than return values.  These differences do not belong precisely
to \lambdaCalculus{}, because they are consequential more
in the calling context than in the called implementation ---
though not entirely, since (for example) throwing exceptions
aborts lexical finalization.  Nor do they belong
precisely to process calculi, because they are most complex
in the context of sequential procedures
(for example, a function cannot catch exceptions thrown from
functions it has spawned in new threads).  As intended here,
\q{Channel Algebra} suggests an intermediate formalization
combining features of both (generalized) lambda and
process calculi (or algebras).  Instead of scattered
\mOldLambda{}-extensions, Channel Algebra unifies multiple
expansions by endowing functions (their signatures,
in the abstract, and function-calls, in the concrete)
with multiple channels, each of which can be
independently modeled by some \mOldLambda{}-extension
(objects, exceptions, captures, and so forth).
}
\p{Specific examples of unorthodox \lambdas{} (objects, exceptions, captures)
suggest a general case: relations or operators between
functions can be modeled as relations between their respective
channels, subject to certain semantic restrictions.  A \i{method} can
be described as a function with several different channels:
say, a \q{\lambdach{}} channel with ordinary arguments (as in
\mOldLambda{}-calculus); a \q{\sigmach{}} channel with a distinguished
\thisc{} carrier (formally studied via \q{\sigmaCalculus{}}); and
a \returnc{} channel representing the return value.  Because the contrast between
these channels is first and foremost \i{semantic} \mdash{} they have different
meanings in the semantics of the programs where they appear \mdash{}
channels may therefore have restrictions governed by
programs' semantics.  For example, as I mentioned
in the context of \q{method chaining}, it may be stipulated that both
\sigmac{} and \returnc{} channels can have at most one carrier;
as a result, a special channel-to-channel operator can be defined which
is specific to passing values between the carriers of
\returnc{} and \sigmac{} channels.  This operator is available because
of the intended semantics of the channel system.
}
\p{In general, a Channel Algebra identifies
several \i{kinds} of channels which
each have their own semantic interpretation, and accompanying axioms
or restrictions.  On the basis of these semantic details,
channel-to-channel operators can be defined, derived from
underlying carrier-to-carrier operators.  A Channel Algebra in this sense
is not a single fixed system, but an outline for modeling
function-call semantics in the context of different programming
languages and environments.
}
\p{As the preceding paragraphs have presupposed, different
functions may have different kinds of channels, which may
or may not be reflected in functions'
types (recall the question, can
two functions have the same type, if
only one may throw an exception)?  This may vary between type
systems; but in any case the contrast between channel \q{structures}
is \i{available} as a criteria for modeling type descriptions.
On this basis, as I will now argue, we can provide type-system
interpretations to source code structures beyond just
values and symbols.
}
\subsubsection{Statements, Blocks, and Control Flow}  
\p{The previous paragraphs discussed expanded channel structures \mdash{} with,  
for example,  
objects and exceptions \mdash{} that model call semantics more complex
than the basic lambda-and-return (of classical \mOldLambda{}-Calculus).  
A variation on this theme, in the
opposite direction, is to \i{simplify} call structures: functions
which lack a return channel have to communicate solely through
side-effects, whose rigorous analysis demands a
\q{type-and-effect} system.  Even further, consider functions
with neither \lambdac{} nor \returnc{} (nor \sigmac{}
nor, maybe, \errorc{}).
As an alternate channel of communication, suppose function bodies
are nested in overarching bodies, and can
\q{capture} carriers scoped to the enclosing function.  \q{Capture semantics}
specifications in \Cpp{} are a useful example, because
\Cpp{} (unlike most languages that support anonymous or \q{intra-expression}
function-definitions) mandates that symbols are explicitly captured
(in a \q{capture clause}), rather than allowing functions to access
surrounding lexically-scoped with no further notation:
this helps visualize the idea that captured symbols are a kind of
\q{input channel} analogous to \lambdac{} and \sigmac{}.
}
\p{I contend this works just as well for code blocks.  Any language which has
blocks can treat them as unnamed function bodies, with
a \q{\capturech{}} channel (but not \lambdach{} or \returnch{}).
When (by language design) blocks \i{can} throw exceptions, it is reasonable to
give them \q{\exceptionch{}} channels (further work, that I put off for
now, is needed for loop-blocks, with \codebreak{} and \codecontinue{}).
Blocks can then be typed as function-like values, under
the convention that function-types can be expressed through
descriptions of their channels (or lack thereof).
}
\p{Consider ordinary source-code
expressions to represent a transfer of values between graph structures: let
\gammaOne{} and \gamaTwo{} be code-graphs compiled
from source at a call site and at the callee's implementation,
respectively.  The function call transfers values from carriers
marked by \gammaOne{} nodes to \gammaTwo{} carriers; with the further
detail of \q{Channel Packages} we can additionally say that the
recipient \gammaTwo{} carriers are situated in a graph structure which
translates to a channel description.  So the morphology of
\gammaOne{} has to be consistent with the channel structure of \gammaTwo{}.
For regular (\q{value}) expressions, we can introduce a new kind
of channel (which I'll call \q{lookup}) acknowledging that the function
called by an expression may itself be evaluated by its own expression, rather
than named as a single symbol (as in a pointer-to-function call
like \pfx{} in \CLng{}).  A segment of source code represents a
value-expression insofar as an equivalent graph representation
comprises a \Gamma{} semantically and morphologically
consistent with the provision of values to channels required
by a function call \mdash{} including the lookup channel on the
basis of which the proper implementation (for overloaded functions)
is selected.  How the graph-structure maps to the appropriate
channels varies by channel kind: for instance the \returnc{} channel
is not passed \i{to} the callee, but rather bound to a carrier as the
right-hand-side of an assignment (an \i{rvalue}) \mdash{} or else passed to
a different function (thus an example of channel-to-channel connection
without an intervening carrier).  A well-formed \Gamma{} represents
part of a function implementation's code graph, specifically that
describing how a Channel Package is concretely provisioned with values
(i.e., a payload).
}
\p{I will use the term \i{call-clause} to designate the
portion of a code graph, and the associated collection of
source code elements, describing a Channel Payload.
Term a call-clause \i{grounded} if its
resulting value is held in a carrier (as
in \yeqfx{}), and \i{transient} if this value
is instead passed on (immediately) to another function
(as in \hfx{}); moreover a call-clause
can be \i{standalone} if it has no result value or this
value is not used; and \i{multiply-grounded} if it has
several grounded result values \mdash{} i.e., a multi-carrier
\returnc{} channel, assuming the type system allows as much.
Grounded and standalone call-clauses can, in turn, model \i{statements};
specifically, \q{assignment} and \q{standalone} statements, respectively.
}
\p{This vocabulary can be useful for interpreting program flow.
Assignment statements with no other side effects can be delayed
until their grounding carrier is convoluted with some other carrier.
Of course, the default choice of \q{eager} or \q{lazy} evaluation is
programming-language-specific, but for abstract discussion
of source code graphs, we have no \i{a priori} idea of temporality;
of a program executing in time.  This is not a matter of
concurrency \mdash{} we have no \i{a priori} idea of procedures running
at the \i{same} time any more than of them running sequentially
(cf. \q{detached}
evaluation as on page \hyperref[detachedeval]{\pageref{detachedeval}}).
\mbox{Any temporal} direction through a graph is an interpretation
\i{of} the graph, and as such it is useful to assume that
graphs in and of themselves assert no temporal
ordering among their nodes or edges.  When modeling
eager-evaluation languages, particular edge-types
can be designated as forcing a temporal order or else
edges can be annotated with additional temporalizing
details.  Without this extra documentation, however,
execution order among graph elements can be
evaluated based on other criteria.
}
\p{In the case of statements, an assignment without side effects
has temporalizing relations only with other statements
using its grounding carrier.  In particular, the order
of statements' runtime need not replicate the order in which
they are written in source code.  For sake of argument, consider
this the default case: Channel Algebras, in principle, model
\q{lazy} evaluation languages, in the absence of any
temporalizing factors.  The actual runtime order among
sibling statements \mdash{} those in the same block \mdash{} then depends,
in the absence of further information, on how their grounding carriers are
used; this in turn works backward from a function's return channel
(in the absence of exceptions or effectual calls).  That is,
runtime order works backward from statements that initialize
carriers in the return channel, then carriers used in those
statements, etc.
}
\p{This order needs to broken, of course, for statements with side-effects.
A case in point is the expansion of \q{\doNotation{}} in Haskell: without
an \i{a priori} temporality, Haskell source code relies on
the asymmetric order of values passed into lambda abstractions to
enforce requirements that effectual expressions evaluate before other
expressions (Haskell does not have \q{statements} per se).  Haskell's
\doH{} \q{blocks} can be modeled (in the techniques used here) as
a series of assignment statements where the grounding carrier of each
statement becomes (i.e., transfers its value to) the sole occupant of a
\lambdach{} channel marking a new function body, which includes all the
following statements (and so on recursively).  There are two concepts
in play here: interpreting any sequence of statements (plus one
terminating expression, which becomes a statement
initializing a \returnch{} carrier) as a function body (not just those covering
the extent of a \q{block}); and interpreting assignment statements
as passing values into \q{hidden} lambda channels.  Of course,
Haskell backs this syntactic convention with monad semantics \mdash{}
the value passed is not the actual value of the monad-typed
carrier but its \q{contained} value.
For sake of discussion, let's call this a \i{monad-subblock} formation.
}
\p{The temporalizing elements in this formation are the \q{hidden lambdas}.
In a multi-channel paradigm, we can therefore consider
\q{monad-subblocks} with respect to other channels.  Consider how
individual statements can be typed: like blocks, statements can
select from symbols in scope and can potentially
result in thrown expressions, so their channel structure is something
like \capturech{} plus \errorch{}.  Even without hidden lambdas, observe
that the runtime order of statements can be fixed in situations
where an earlier statement can affect the value (via non-constant capture)
of a carrier whose value is then used by a later statement.  So
for languages with a more liberal treatment of side-effects than Haskell,
we can interpret chains of statements \i{in fixed order} as successively
capturing (and maybe modifying) symbols which occur in multiple
statements.  Having discussed convoluted \i{carriers},
extend this to channels: in particular, say two \capturech{} channels are
convoluted if there is a modifiable carrier in the first
which is convoluted with a carrier in the second (this is an ordered relation).
One statement must run before a second
if their \capturech{} channels are convoluted, in that order.
}
\p{This is approaching toward a \q{monad-subblock} formation using \i{\capturech{}}
in place of \lambdach{}.  To be sure,
Haskell monad-subblock does have the added gatekeeping
dimension that the symbol occurring after its appearance as grounding
an assignment statement is no longer the symbol with a monad
type, but a different (albeit visually identical) symbol
taken from the monad.  Between two statements, if the
prior is grounded by a monad, the implementation of
its \bind{} function is silently called, with the subsequent
(and all further) statements grouped into a block passed in
to \bindFn{}, which in turn (by design) both extracts its
wrapped value and (if appropriate) calls the passed function.
But this architecture can certainly be emulated on
non-\lambdach{} channels \mdash{} a transform that would belong to the
larger topic of treating blocks as function-values
passed to other functions, to which I now turn.
}
\subsubsection{Code Blocks as Typed Values}
\p{Insofar as blocks can be typed as functions (in a sense),
they may readily be passed around: so loops, \ifthenelse{},
and other control flow structures can plausibly be modeled
as ordinary function calls.  This requires some extra semantic
devices: consider the case of \ifthenelse{} (I'll use this
also to designate code sequences with potential \q{\elseif{}}s), which
has to become an associative array of expressions and functions
with \q{block} type (e.g., with only \capturech{} and \errorch{}
channels).  We need, however, a mechanism to suppress
expression evaluation.  Recall that expressions are
concretized channel-structures which include a
\lookupch{} channel providing the actual implementation to call.
Assume that \lookupch{} can be assigned a flag which suppresses
evaluation.  Assume also that carriers can be declared which
hold (or somehow point to) \i{expressions} that evaluate
to typed values,
in lieu of holding these values directly
(note that this is by intent orthogonal
to a type system: the point is not that carriers can hold values
whose type is designed to encapsulate potential
computations yielding another type, like \stdFuture{} in \Cpp{}).
Consider again the nested-expression variant of
\carrOneOpTransferTwo{}: when the result
of one function call becomes a parameter to another
function, the value in the former's
\returnch{} carrier (assume there is just one) gets
transferred to a carrier in the latter's \lambdach{} channel
(or \sigmach{}, say).  This handoff can be described before being
effectuated: a language runtime is free to vary the
order of expression-evaluation no less than of statements.
The semantics of a carrier-transfer between \Twof{}'s return
and \Onef{}'s lambda does not stipulate
that \Twof{} has to \i{run} before \Onef{}; language
engines can provide semantics for \carrOneOpTransferTwo{} allowing \carOne{} to hold a delayed capability to
evaluate the \Twof{} expression.  Insofar as this is an
option, functions can be given a signature \mdash{} this
would be included in the relevant \TXL{} \mdash{} where
some carriers are of this \q{delayed} kind.
Functions like \ifthenelse{} can then be declared in
terms of these carriers.
}
\p{To properly capture \ifthenelse{} semantics, it is also
appropriate to notate conditions on how blocks
passed as function values are used.  In the
case of \ifthenelse{}, the implementation
will receive multiple function values with the caller
expecting \i{exactly one} function to be called.
This constitutes a requirement on how carriers are
used, analogous to mandating that a \returnch{}
(or either \returnch{} or \errorch{}) carrier be initialized,
or that mutable references passed
to a function for initialization (as an alternative
to returning multiple values) are initialized.
While such requirements can potentially be
described as annotations on carriers/channels,
this is a general issue outside the scope of
blocks-as-function-values.
}
\p{A thorough treatment of blocks-as-functions also needs to
consider standard procedural affordances like \codebreak{}
and \codecontinue{} statements.  Since blocks
can be nested, some languages allow inner blocks to express the
codewriter's intention to \q{break out of} an outer block from
an inner block.  One way to model this via Channel Algebra is
to introduce a special kind of return channel for blocks
(called a \q{\breakch{}}, perhaps) which, when it has
an initialized carrier, uses this channel to hold a value that the
enclosing block interprets in turn: by examining the
inner \breakch{} the immediately outer block can decide whether it
itself needs to \q{break} and, if so, whether its own
\breakch{} channel needs to have an initialized carrier.  The
presence of such a \breakch{} can type-theoretically
distinguish loop blocks from blocks in (say)
\ifthenelse{} contexts.
}
\p{Further discussion of code models via Channel Complexes and
Channel Algebras is outside the scope of this chapter, but
is demonstrated in greater detail in the accompanying code-set.
Hopefully, the best way to present Channel Semantics outside
the basic \lambdach{}/\sigmach{}/\returnch{}/\exceptionch{}
quartet is via demonstrations in live code.
In that spirit, the demo code-set focuses more
on practical engineering and problem-solving where Channel models
can be useful, and I'll briefly review its structure and its
organizing rationales in the Conclusion.
}
