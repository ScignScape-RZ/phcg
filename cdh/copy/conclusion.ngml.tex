\section{Conclusion}
\vspace{-.5em}
\p{There are several tactics to practically employ techniques I have 
reviewed here in concrete projects.  For sake of discussion, I will focus 
on the approach used by the demo code-set provided with this chapter.  
This code actually has multiple dimensions, because it has been designed 
to integrate with data sets accompanying other chapters in the 
present volume, not merely this chapter.  This has led to the code being 
planned as follows: code published with \i{this} chapter directly uses 
its Hypergraph and Channel Complex libraries to parse Interface Definition 
files associated with \i{other} chapters.  Those chapters' data sets in 
turn employ \Cpp{} code whose 
construction is influenced by Channel Algebra 
\mdash{} notably in the design of function groups which are overloaded 
via types that are reasonable for \q{hidden channels}, 
and in how functions are exposed to external scripting and reflection.  
The implementations are in normal \Cpp{}, 
but the associated \IDL{} annotations 
describe them via more sophisticated channel formations, which are 
mimicked (in terms of conventional input parameters) in 
the production code.
}
\p{The \q{hidden} values constructed while routing between logically 
related functions, as seen in the various code-sets, tend to 
fit into two classifications.  On the one hand, these may be values 
constructed as a test that function arguments conform to protocol: 
for instance, one way to check that a numeric value falls within some 
desired range is to cast this value to a range-checked type 
\mdash{} whether or not the second value is actually used.  
Alternatively, in some places supplemental values are 
enumerations of typestates \mdash{} allowing function bodies 
to achieve something resembling pattern-matching as a 
code-branching structure (by \switch{}ing on enum values).
}
\p{Manually creating residual values along these lines is 
admittedly cruder than pure type-level engineering, as exemplified 
by Idris's synthesis of Dependent Types, typestate, and 
effect-typing.  The solutions chosen for the present context are 
less automated; they require deliberate choice by developers.  
Nevertheless, the Interface Definition technology makes these choices 
explicit and documented, which can help guide developers toward 
embracing these more fine-grained coding conventions and 
confirming that they have done so consistently.
}
\p{It is a legitimate question whether establishing coding paradigms 
\mdash{} in a data set management context, for example \mdash{} in this 
convention-driven spirit is better or more maintainable than 
approaching data sets in a language like Haskell or Idris whose 
rigors are more formal than conventional.  But the 
\Cpp{}-based approach has the benefit of more immediate integration 
with \GUI{} code, reflected in the \Qt{} components published 
as wrappers and visualizers for data-set-specific datatypes.  
}
\p{In the best-case scenario, \GUI{} code is a natural extension 
of programming languages' and individual languages' 
type systems \mdash{} with rigorous mapping between value types and 
visual-object types that graphically display associated 
values.  So long as Functional Programming languages remain 
operationally separated from \GUI{} drivers \mdash{} needing a 
\q{smoke binding} or foreign-function 
interface to marshal data between a 
\CLang{}-oriented User Interface and the non-visual data 
(and database) management components of a project \mdash{} the strongly-typed 
rigor of Functional Programming environments will be 
incomplete.  
}
\p{Whether or not by technical necessity or just entrenched culture, 
\GUI{} frameworks are still predominantly built in procedural 
or \OO{} languages like \CLang{}, \Java{}, and \Cpp{}.  It seems likely 
that a truly integrated type system, covering User Interface as 
well as data management logic, will need a hybrid functional/procedural 
paradigm at some stratum \mdash{} either the underlying \GUI{} framework or 
at application-level code.  So long as \GUI{} frameworks remain 
committedly procedural, the most likely site for such hybrid 
paradigms to emerge is at the application level, and in the 
context of application-development \SLE{} tools.
}
\p{Implementing \GUI{} layers in a Functional environment is usually 
approached from the perspective of \i{functional reactive} 
programming, which emphasizes how the interface between visual 
components and controller logic can be structured in terms 
of \i{event-driven} programming.  In this paradigm, there is 
no \i{immediate} linkage between \GUI{} events and the 
functions called in response \mdash{} for example, no 
single function that automatically gets called when the user 
clicks a mouse button.  Instead, events are entered into a pool 
wherein each event may have a varying number of handlers 
(including being ignored entirely).  This style of 
programming accords well with paradigms that try to minimize 
the number of functions with side effects.  Event-handlers are free 
to post new signals (these are interpreted as events) which 
may in turn be handled by other functions \mdash{} so that 
signals may be routed between multiple functions entirely 
without side-effects.  That said, most events \i{should} cause 
side effects eventually \mdash{} for instance, after all, 
a user does not typically 
initiate an action (triggering an event) without intending 
to change something in the application data or display.  But 
events can be routed between pure functions until an eventful 
handler is called, so side-effects can be localized in a 
proportionately small group of functions.
}
\p{Moreover, certain qualities of \GUI{} design can be expressed as 
logical constraints rather than as application-level state enforced 
by procedural code.  For example, optimal design may stipulate that some 
graphical component must automatically be resized and repositioned to remain 
centered in its parent window, sustaining that geometry even when 
the window itself is resized.  
Functional-Reactive frameworks allow many of these constraints to be 
declared as logical axioms on the overall visual layout and properties 
of an application, constructing the procedures to maintain this 
state behind-the-scenes \mdash{} which minimizes the extent of 
procedural code needing to be explicitly maintained by application developers.
}
\p{But while Functional Reactive Programming is a strategy for 
providing \GUI{} layers on a Functional code base, it can equally 
be treated as an Event-Driven enhancement to Object-Oriented 
programming.  In an \OO{} context, events and signals constitute an 
alternative form of non-deterministic method-call, where signal-emitting 
objects send messages to receiver functions \mdash{} except indirectly, 
passing through event-pools.  Indeed, as documented 
by the demo code, events and signals 
have a natural expression in terms of Channel Algebra, where both signal 
emitters and receivers are represented via special \q{Sigma} channels.  
On this evidence, Functional Reactive Programming should be assessed not 
just as a \GUI{} strategy for Functional languages but as a 
hybrid methodology where Functional and Object-Oriented 
methodologies can be fused, and integrated.
}
