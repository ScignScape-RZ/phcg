\section{Channel Hypergraph Grammar and Cognitive State Semantics}
\p{In my opinion, a well-motivated philosophy of language will be oriented 
to other sciences, both formal and humanistic \mdash{} language as a 
\q{bridge} between multiple disciplines.  Linguistics should not 
aspire to analytic completeness, or explanatory autonomy: language 
understanding requires many cognitive faculties which are \i{not} 
governed by linguistic theories; and the pragmatic and situational 
contexts surrounding discourse likewise need extra-linguistic 
sociological treatments.  So a valid goal for a philosophy of language 
is to establish which theories \i{are} proper to language.  
Ideally, these theories should engender both philosophical 
analysis and formal models: there is a systematic, structural 
dimension to language whose properties can be revealed through 
rigorous (even computational) representations.  At the same time, 
there is also a subtlety and intersubjectivity to language 
which call for analyses grounded in philosophy rather than any 
computational or logicomathematical commitments.
}
\p{Having presented objections and, I believe, counter-examples to 
\q{truth-theoretic} semantics, I now want to present some ideas 
for an alternative theory which retains some formal or logical 
(or at least mathematical) structuration \mdash{} but not so as 
to \q{reduce away} linguistic nuances; nor structures in language 
whose semiotic foundation draws from a spectrum of cognitive 
(perceptual, situational, narrative, empathic) facilities, which cannot be 
reduced to deterministic logic.  My plan is to describe a grammar 
framework which integrates such perspectives as Link Grammar, 
Applicative and Cognitive Grammar (\ACG{}), and Type Logical Grammar, but 
with a distinct underlying formal model.  Like other theories, 
this model proposes a broad categorization for semantics by 
identifying a small set of primitive types, and defining other types 
as functional or combinatory derivates on those base types 
(from which the type system overall is \q{generated}).     
}
\p{A common paradigm is to consider natural-language types as generated 
by just two bases \mdash{} a noun type \N{} and a proposition type 
\Prop{}, the type of sentences and of sentence-parts which are 
complete ideas \mdash{} having in themselves a propositional content 
(see e.g. \cite{BarkerShanTG} or \cite{KubotaLevine}).    
Different models derive new types on this basis in different ways.
One approach, inspired by mathematical \q{pregroups}, establishes 
derivative types in terms of word pairs \mdash{} an adjective followed 
by a noun yields another noun (a noun-phrase, but \N{} is the phrase's 
\i{type}) \mdash{} e.g., \i{rescued dogs}, like \i{dogs}, is conceptually 
a noun.  Adjectives themselves then have the type of words which form 
nouns when paired with a following noun, often written as 
\NOverN{}.  Pregroup grammars distinguish left-hand and right-hand 
adacency \mdash{} \i{bark loudly}, say, demonstrates an adverb \i{after} a 
verb, yielding a verb phrase: so \q{loudly} here has the type of a 
word producing a verb in combination with a verb to its \i{left} 
(sometimes written \VUnderV{}); by contrast adjectives combine with 
nouns to their \i{right}.
}
\p{A related formalization, whose formal 
analogs lie in Typed Lambda Calculus, abstracts from left-or-right 
word order to models derived types as equivalent (at least 
for purposes of type attribution) to \q{functions}.  
This engenders a fundamental notion of functional \i{application} 
and operator/operand distinctions: 
\begin{quote}\i{Categorial Grammars} make the connection between the 
first and the second level of the ACG.  These models are 
typed applicative systems that verify if a linguistic 
expression is syntactically well-formed, and then construct 
its functional semantic interpretation.  They use the 
\i{fundamental operation of application} of an operator to an 
operand.  The result is a new operand or a new operator. 
This operation can be repeated as many times as necessary 
to construct the linguistic expressions' functional semantic 
representation.
\cite[p. 1]{AurelieRossi}
\end{quote}
So, e.g., an adverb becomes a function which takes a verb and produces another verb; 
an adjective takes a noun and produces another noun; 
and a verb takes a noun and produces a proposition 
(see \i{students complained}).  By \q{function} we can consider 
some kind of conceptual tranformation: \i{loudly} transforms the 
concept \i{bark} into the concept \i{loud bark}.  Another 
feature of this \q{function type} interpretation is that 
words can be treated as functions taking multiple arguments 
\mdash{} a verb, for example, can have as many as three 
\q{inputs} (subject, direct object, and indirect object).  
}
\p{While informed by the aforementioned theories, 
instead of pregroups or lambda calculus, 
the type system I propose here is inspired by 
computer programming languages, and specifically 
by frameworks for assigning types to 
computational procedures.  Implemented \q{functions} in 
this sense are different than functions in mathematical 
contexts, like the lambda calculus.  Mathematical 
functions have input and output parameters, but 
\i{procedures} (functions implemented in computer code) 
have different \i{kinds} of input and output.  
Procedures can draw information from many sources 
\mdash{} e.g. files, databases, and network resources 
\mdash{} and likewise can produce many sorts of effects 
(changing the state of some software or physical 
component; e.g., writing text to a monitor).  
These sources and effects lie alongside explicit 
inputs and outputs, as in $x$ as input to $f$ 
(and $y$ carrying $f$'s output) in \yeqfx{}.  Modern programming 
languages also have special features, like \q{objects} 
and exceptions (special kinds of inputs and outputs 
respectively) which complicate the picture of functions 
\q{mapping} inputs to outputs.  One strategy for 
analyzing computer code is to take all procedural 
inputs and outputs \mdash{} including any means by which 
a procedure can obtain data or cause effects outside 
its own internal execution environment \mdash{} as 
\q{channels of communication}, so the \i{type} of 
a procedure is defined by the types of data carried 
in each of its channels.  
A formal model of channels can then play a role analogous 
to input-to-output maps in lambda-calculus inspired grammars, 
and to adjacent-word left-or-right pairs in pregroup grammars.    
}
\p{Complementary to a \q{channel-oriented} theory of 
procedures is the idea of channel \i{states}, representing 
procedural effects on any data accessed or modified 
by procedures.  The data in an 
output channel, for instance, is in a \q{pre-initialized} state 
until the procedure terminates and control returns to a
calling procedure, where that output data is then 
typically used as input data to other procedures and/or 
held in a \q{carrier}, embodied by a source-code symbol 
like $y$ in \yeqfx{}.  Channels as such can be formally 
defined as higher-scale structures on (labeled and directed) 
hypergraphs (where procedures are expressed via hypernodes, 
making this a variation on \q{hypergraph grammars} as in 
\cite{BrendanFongThesis} or \cite{BrendanFong}).  
Transforms in channel state are, 
in this framework, informally analogous to \q{beta-reduction} in 
the lambda calculus and to \q{identity laws} in pregroup grammars.
}
\p{A detailed outline of channels is beyond the scope of this 
paper (I have written about \q{channel algebra} in 
\cite{NathanielChristenCyberphysical} and \cite{NathanielChristen}).  
Briefly, though, I believe that this \q{Channel Hypergraph Grammar} 
(formulated in particular to analyze \i{computer} languages) can 
be applied to \i{natural} language because of how 
the entirety of a sentence can influence our interpretation of 
specific word-pairs and inter-word relations.  Assuming we accept 
the intuitive analogy of words as \i{procedures} 
(maybe scripts or conceptualizing operations in some cognitive 
processing space), then \q{non-local} semantic effects correspond 
to alternative procedural channels.  
} 
\p{For this model, assume we have a baseline lamba-calculus-like 
functional summary of sentences and derived types.  That is, 
any sentence can be rewritten as if a sequence of \q{function calls}, 
assuming an underlying representational vocabulary of a typed 
lambda calculus, with sentences having overall \Prop{} types:
\begin{sentenceList}\sentenceItem{} I believe that he is at school.
\sentenceExample{(believe I (that (is he (at school))))}
\sentenceItem{} Student after student complained about tuition hikes.
\sentenceExample{((about (tuition hikes) complained) (after student student))}
\sentenceItem{} Three times, students asked an interesting question.
\sentenceExample{((three times) (asked students (an (interesting question))))}
\end{sentenceList}
In this representation \q{function-like} types (verbs, adjectives, 
adverbs, and so on) are notated preceding one or more words or 
expressions which are functional \q{arguments} 
(sometimes the \q{functions} are expressions with their own 
inner function-argument structure, as in \i{(about (tuition hikes) complained)} 
qua verb-phrase).  This reconstruction presents one layer of structure 
\mdash{} for example, the connection of verbs to their one, two, or three arguments 
\mdash{} which can then be overlaid with Link-Grammar style pairs, 
like verb-to-subject and verb-to-direct-object. 
}
\p{These word-links are then \q{local} connections in that they nest within 
a functionally reconstructed phrase hierarchy.  We can call them 
\i{local} channels.  Meanwhile, semantic effects outside the 
local phrase structure \mdash{} like anaphora or \q{space-building} insofar 
as it influences nonlocal morphosyntax \mdash{} can then be 
modeled via \i{nonlocal} channels.  To clarify these claims I will 
demonstrate this style of analysis with respect to examples like 
I have just cited.  
}
\p{My overall goal is to embrace a hybrid methodology \mdash{} 
accepting formal analyses when they shed light on linguistic 
processes, but not going so far as to treat logical, mathematical, 
or computational models as full explanations for linguistic 
rationality qua scientific phenomenon.  
The path toward such a hybrid methodology, as I will sketch it 
here, takes its inspiration and orientation from Cognitive Grammar.  
This perspective, in particular, challenges our assumption that 
grammar and semantics are methodologically separate.  Received 
wisdom suggests that grammar concerns the \q{form} of sentences 
whereas semantics considers the meaning of words \mdash{} implicitly 
assuming that \i{word combinations} produce new meanings, and 
that the \i{order} by which words are combines determines how 
new meanings are produced.  This notion, in turn, is allied 
with the essentially logical or propositional picture of 
signifying via doxa: the idea that inter-word relations cue up 
different logically salient transformations of an underlying 
predicate model.  Thus \i{many students} as a phrase is 
more significatorily precise than \i{students} as a word, because 
the phrase (with intimations of quantitative comparison) has 
more logical detail.  Similarly \i{many students complained} 
is more logically complete because, provisioning both a verb-idea 
and a noun-idea, it represents a whole proposition. 
}
\p{In general, then, phrases are more complete than words because 
they pack together more elements which have some logical role, 
establishing individuals, sets, spatiotemporal setting, and 
predicates which collectively establish sufficiently 
completed propositional attitudes.  On this account the 
key role of phrase-structure is to establish phrases as 
signifying units on a logical level analogous to 
how lexemes are signifying units on a referential or conceptual 
level.  Moreover, phrases' internal structure are understood 
to be governed by rule defining \i{how} word-combinations 
draw in extra logical detail.  A link between words is not a 
random synthesis of concepts, but rather implies a certain 
logical connective which acts as a de facto \q{third party} in 
a double-word link, proscribing with orientation to predicate 
structures \i{how} the words' semantic concepts are to be 
joined.  In \i{many students} the implied connector is 
the propositional act of conceiving a certain quantitative 
scale to a conceptualized set; in \i{students complained} the 
implied connector is a subject-plus-verb-equals-proposition 
assertiveness.  Phrases acquire logical specificity by 
building up word-to-word connections into 
more complex aggregates.
}
\p{One implication of this model is that phrases are semantically 
substitutable with individual lexemes that carry similar meanings, 
having been entrenched by convention to capture a multipart concept 
which would otherwise be conveyed with the aggregation of a 
phrase: consider \q{MP} for \i{member of parliament}, or 
\q{primaried} for \i{subject without your own party to a primary 
challenge}.  Conversely, phrases can be repeatedly used in a 
specific context until they function as quasi lexical units in 
their own right.  These patterns of entrenchment imply that we 
hear language in term of phrases bearing semantic content; and 
insofar as we are comfortable with how we parse a sentence, each 
word sited in its specific phrasal hierarchy, we do not tend to 
consider individual words semantically outside of their 
constituent phrases.
}
\p{This theory of the syntax-to-semantic relationship is a paradigmatic 
partner, at the grammatic level, to truth-theoretic semantics and as
such, I maintain, is subject to similar critiques as I developed 
semantically in the last two sections.  A critique of grammatic theories 
based on, let's say, \q{\i{proprositional semantics of phrase-structure}} 
can address two concerns: first, the idea that lexemes retain some 
syntactic and semantic autonomy even within clearly defined phrases 
where they are included; and, second, that the shape of phrases 
insofar as they are perceived as holistic signifying units 
is often driven by figurative or \q{gestalt} principles rather than 
neat logical structuration.  I'll call the former the issue of 
\q{phrasal isolation} (or lack thereof): syntactic and semantic effect 
often cross phrasal boundaries, even outside the overarching hierarchy 
whose apex is the whole sentence.  Both of these lines of reasoning 
\mdash{} arguably especially the second \mdash{} are developed in 
Cognitive Grammar literature.   
}
\p{}
