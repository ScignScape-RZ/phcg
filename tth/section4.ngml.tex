\spsubsectiontwoline{Cognitive Grammar and Type Theoretic Semantics}
\p{In Ronald Langacker's \i{Foundations of Cognitive Grammar}, the sentence
\begin{sentenceList}\sentenceItem{} \label{itm:threetimes} Three times, students asked an interesting question.
\end{sentenceList}
is used to demonstrate how
grammatical principles follow from cognitive \q{construals} of the relevant situations,
those which language seeks to describe or takes as presupposed context.\footnote{For example, \cite[pp. 119 and 128]{LangackerFoundations},
discussed by \cite[p. 189]{LineBrandt}, and \cite[p. 9]{EstherPascual}.
}
In particular, Langacker argues that \q{students} and \q{question} can both be either singular or
plural: syntax is open-ended here, with neither form more evidently correct.  Langacker uses this
example to make the Cognitive-Linguistic point that
we assess syntactic propriety relative to cognitive frames and conversational context.  In this
specific case, we are actually working with two different cognitive frames which are interlinked
\mdash{} on the one hand, we recognize distinct events consisting of a student asking a question, but
the speaker calls attention, too, to their recurrence, so the events can also be understood
as part of a single, larger pattern.  There are therefore two different cognitive foci, at two
different scales of time and attention, a \q{split focus} which makes both singular and plural
invocations of \q{student} and \q{question} acceptable.
}
\p{Supplementing this analysis, however, we can additionally focus attention directly on
grammatical relations.  The words \i{student} and \i{question} are clearly linked as the subject and
object of the verb \i{asked}; yet, contrary to any simple presentation of rules,
no agreement of singular or plural is required between them (they can be singular and/or plural in
any combination).  Moreover, this anomaly is only in force due to the context established
by an initial phrase like \i{three times}; absent some such framing, the singular/plural
relation would be more rigid.  For example, \q{A student asked interesting questions} would
(in isolation) strongly imply \i{one} student asking \i{several} questions.  So the initial
\q{Three times} phrase alters how the subsequent phrase-structure is understood while remaining
structurally isolated from the rest of the sentence.  Semantically, it suggests a
\q{space builder} in the manner of Gilles Fauconnier or Per Aage Brandt
\cite{Fauconnier}; \cite{PerAageBrandt}, but we
need to supplement Mental Space analysis with a theory of how these spaces
influence syntactic acceptability, which would seem to be logically prior to the stage where Mental Spaces
would come in play.
}
\p{The mapping of (\ref{itm:threetimes}) to a logical
substratum would be more transparent with a case like: 
\begin{sentenceList}\sentenceItem{} \label{itm:threes} Three students asked interesting questions.
\end{sentenceList}
(\ref{itm:threes}) is a more direct translation of the facts
which the original sentence conveys.  But this \q{more logical} example has different
connotations than the sentence Langacker cites; (\ref{itm:threetimes}) places the emphasis
elsewhere, calling attention more to the idea of something temporally drawn-out,
of a recurrence of events and a sense of time-scale.  The \q{more logical} sentence
lacks this direct invocation of time scale and temporal progression.
}
\p{We can say that the \q{Three students} version is a more direct statement of fact, whereas
Langacker's version is more speaker-relative, in the sense that it elaborates more
on the speaker's own acknowledgment of belief.  The speaker retraces the steps of her
coming to appreciate the fact \mdash{} of coming to realize that the \q{interesting questions}
were a recurrent phenomenon and therefore worthy of mention.  By situating expressions
relative to cognitive processes rather than to the facts themselves, the sentence
takes on a structure which models the cognition rather than the states of affairs.
But this shift of semantic grounding from the factual to the cognitive also apparently
breaks down the logical orderliness of the phrase structure.  \q{Three times}, compared
to \q{three students}, leads to a morphosyntactic choice-space which is
\q{underdetermined} and leaves room for speakers' shades of emphasis.
}
\p{This is not an isolated example.  Many sentences can be provided with similar
phrase-structure complications, particularly with respect to singular/plural agreement.
\begin{sentenceList}\sentenceItem{} Time after time, tourists (a tourist) walk(s) by this building
with no idea of its history.
\sentenceItem{} The streets around here are confusing; often people (someone)
will ask me for directions.
\sentenceItem{} Student after student came with their (his/her)
paper to compain about my grade(s).
\sentenceItem{} Student after student \mdash{} and their (his/her) parents
\mdash{} complained about the tuition increase.
\end{sentenceList}
On a straightforward phrase-structure reading, \i{student after student} reduces to an
elegant equivalent of \i{many students}, with the rhetorical flourish abstracted away
to a logical form.  But our willingness to accept both singular and plural agreements
(his/her/their parents, grades, papers) shows that clearly we don't simply substitute
\i{many students}; we recognize the plural as a logical gloss on the situation but
engage the sentence in a more cogntively complex way, recognizing connotations of temporal
unfolding and juxtapositions of cognitive frames.  The singular/plural underdeterminism
is actually a signification in its own right, a signal to the listener that the
sentence in question demands a layered cognitive attitude.  Here again, syntactic
structure (morphosyntactic, in that syntactic allowances are linked with
variations in the morphology of individual words, such as singular or plural form)
serves to corroborate conversants' cognitive frames rather than to model logical
form.
}
\p{The contrast between the phrases \i{Student after student} and \i{Many students}
cannot be based on \q{abstract} semantics alone \mdash{} how the evident temporal implications of the
first form, for example, are concretely understood, depends on conversants' mutual recognition
of a relevant time frame.  The dialog may concern a single day, a school year, many
years.  We assume that the speakers share a similar choice of time \q{scale}
(or can converge on one through subsequent conversation).  \i{Some} time-frame
is therefore presupposed in the discursive context, and the first phrase invokes
this presumed but unstated framing.  The semantics of the phrase are therefore somewhat open-ended:
the phrase \q{hooks into} shared understanding of a temporal cognitive framing without referring
to it directly.  By contrast, the second phrase is less open-ended: it is consistent with both
a more and less temporally protracted understanding of \i{many}, but leaves such details (whatever
they may be) unsignified.  The factual circumstance is designated with a level of abstraction that sets
temporal considerations outside the focus of concern.  The second (\q{\i{Many students}}) phrase 
is therefore both
less open-ended and also less expressive: it carries less detail but accordingly also relies
less on speaker's contextual understanding to fill in detail.\footnote{The examples I have used so far may also imply that a choice of phrase structure is
always driven by semantic connotations of one structure or another;
but seemingly the reverse can happen as
well \mdash{} speakers choose a semantic variant because its grammatic realization lends a useful
organization to the larger expression.  There are many ways to say \q{many},
for example: \i{a lot of},
\i{quite a few}, not to mention \q{time after time} style constructions.  Whatever their
subtle semantic variations, these phrases also have different syntactic properties:
\i{Quite a few} is legitimate as standalone (like an answer to a question);
\i{A lot of} is not, and \i{A lot} on its own is awkward.  On the other hand the \q{of} in
\i{A lot of} can \q{float} to be replicated further on: \q{A lot of students, of citizens,
believe education must be our top priority} sounds more decorous than the equivalent sentence with the
second \q{of} replaced by \q{and}.  If the cadence of that sentence appeals to the speaker, then
such stylistic preference will influence taking \q{A lot of} as the \q{many} variant of choice.
So speakers have leeway in choosing grammatic forms that highlight one or another aspect of
situations; but they also have leeway in choosing rhetorical and stylistic pitch.  Both cognitive
framings and stylistic performance can be factored when reconstructing what compels the
choice of one sentence over alternatives.
}
}
\p{One consequence of these analyses is that grammar
needs to be approached holistically: the grammatic structure of phrases cannot,
except when deliberate oversimplification is warranted, be isolated from
surrounded sentences and still larger discourse units.  Semantic roles
of phrases have some effect on their syntax, but phrases are nonetheless chosen
from sets of options, whose variations reflect subtle semantic and syntactic
maneuvers manifest at super-phrasal scales.  The constituent words of phrases retain some
autonomy, and can enter into inter-word and phrasal structures with other words outside their
immediate phrase-context.  We can still apply formal models to phrase
structure \mdash{} for example, Cognitive and Applicative Grammar (\CAG{}) considers phrases
as \q{applications} of (something like) linguistic or cognitive \q{functions},
in the sense that (say) an adjective is like a \i{function} applied to a noun,
to yield a different noun (viz., something playing a noun's conceptual role)
\cite{Descles2010}.
But we should not read these transformations
\mdash{} like \i{rescued dogs} from \i{dogs} \mdash{}
too hastily as a purely semantic correlation within a space of denotable concepts
\mdash{} \i{such that} the new concept wholly replaces the
contained parts, which then cease to have further linguistic role and effect.
Instead, applicative structures represent shifts or evolutions in
mental construal, which proceed in stages as conversants form
cognitive models of each others' discourse.  Even if phrase structure
sets landmarks in this unfolding, phrases do not wholly subsume their
constituents; the parts within phrases do not \q{vanish} on the higher scale,
but remain latent and may be \q{hooked} by other, overlapping phrases.
}
\p{Consider the effect of \q{Many students complained}.  Propositionally, this appears
to say essentially that \i{students} complained; but, on hermeneutic charity, the
speaker had \i{some} reason to say \q{many}.  The familiar analysis is that
\q{many} suggests relative size; but this
is only half the story.  If the speaker chose merely \i{students complained}, we would hear an assertion
that more than one student did, but we would also understand that there were several
occasions when complaints happened.  Adding \q{many} does not just
imply \q{more} students, but suggests a mental shift away from the particular episodes.
In the other direction, saying \i{a student complained} is not just
asserting how at least one student did so, but
apparently reports one specific occasion (which perhaps the speaker wishes to
elaborate on).  In other words, we cannot really capture the singular/plural semantics,
or different varieties of plural, just by looking at the relative size of implied
sets; we need to track how representations of singleness or multitude imply
temporal and event-situational details.
}
\p{Against this backdrop, \i{Student after student complained} captures both dimensions,
implying both a widespread unrest among the student body and also
temporal recurrence of complainings.
By way
of illustration, Figure ~\ref{fig:ESA} shows a
destructuring in the fashion of Dependency Grammar, 
along with implicit type annotations.  \begin{figure}
\caption{Dependency-style graph with argument repetition}	
\label{fig:ESA}
\vspace{1em}
\hspace{0.15\textwidth}	
\begin{minipage}{0.7\textwidth}
\begin{tikzpicture}

%\tikzset{snake it/.style={decorate, decoration=snake, segment length=5mm, amplitude=15mm}}

%\draw

%\node [s1] at (0,0) {Student};

\node (s1) at (1,1) {\textbf{Student}};
\node (after) [right=5mm of s1] {\textbf{after}};
\node (s2) [right=5mm of after] {\textbf{student}};
\node (complained) [right=5mm of s2] {\textbf{complained}};

\node (s1Rep) [double,draw=black,shape=circle,thick,fill=gray!50,inner sep=.5em,below=3.5cm of s1] {};
\node (afterRep) [double,draw=black,shape=circle,thick,fill=gray!50,inner sep=.5em,below=2cm of after] {};
\node (s2Rep) [double,draw=black,shape=circle,thick,fill=gray!50,inner sep=.5em,below=3.5cm of s2] {};
\node (complainedRep) [double,draw=black,shape=circle,thick,fill=gray!50,inner sep=.5em,below=2.5cm of complained] {};

\draw [ |-,-|, <->, line width = .8mm, draw=gray!70, 
 dashed, double equal sign distance, >= stealth, shorten <= .25cm, shorten >= .25cm ]
 (s1) to (s1Rep);

\draw [ |-,-|, <->, line width = .8mm, draw=gray!70, 
 dashed, double equal sign distance, >= stealth, shorten <= .25cm, shorten >= .25cm ]
(after) to (afterRep);
 
\draw [ |-,-|, <->, line width = .8mm, draw=gray!70,  
 dashed, double equal sign distance, >= stealth, shorten <= .25cm, shorten >= .25cm ]
(s2) to (s2Rep);

\draw [ |-,-|, <->, line width = .8mm, draw=gray!70, 
 dashed, double equal sign distance, >= stealth, shorten <= .25cm, shorten >= .25cm ]
(complained) to (complainedRep);
 
\draw [shorten <= .25cm, shorten >= .25cm ] 
(afterRep) to node [draw=black,shape = star,star points=4,thick,inner sep = 0mm, above] {1} (s1Rep);

\draw [shorten <= .25cm, shorten >= .25cm ] 
(afterRep) to node [draw=black,shape = star,star points=4,thick,inner sep = 0mm,above] {1} (s2Rep);

\node (s1E) [left=-2mm of s1Rep.east]{};
\node (s2E) [left=1mm of s2Rep.west]{};

\draw [shorten <= 2cm, shorten >= .3cm, double, % bend left = 5, 
decorate, decoration={snake, segment length=5mm, amplitude=.5mm}] %, amplitude=15mm]  
(s1E) to node [draw=black,shape = star,star points=4,thick,inner sep = 0mm, below] {2} (s2E);

\draw [shorten <= .5cm, shorten >= .5cm ] 
(afterRep) edge [bend left=20,looseness=1] node [draw=black,shape = star,star points=4,thick,inner sep = 0mm, 
 above, near end ] {3} (complainedRep);

\node (frameTopLeft) [below left = 1.5cm and -.65 cm of s1] {};
\node (frameBottomLeft) [below = 2.5cm of frameTopLeft] {};
\node (frameBottomRight) [right = 3.95cm of frameBottomLeft] {};
\node (frameTopRight) [above = 2.5cm of frameBottomRight] {};

\draw [shorten <= 0.15cm, shorten >= 0.15cm ] 
(frameTopLeft) edge [bend right=30,looseness=1] (frameBottomLeft);

\draw [shorten <= 0.15cm, shorten >= 0.15cm ] 
(frameBottomLeft) edge [bend right=30,looseness=1] (frameBottomRight);

\draw [shorten <= 0.15cm, shorten >= 0.15cm ] 
(frameBottomRight) edge [bend right=35,looseness=1] (frameTopRight);

\draw [shorten <= 0.15cm, shorten >= 0.45cm ] 
(frameTopRight) edge [bend right=30,looseness=1]
node [draw=black,shape = regular polygon,regular polygon sides=3,thick,inner sep = .2mm, 
above, near start, shape border rotate = 180] {4} (frameTopLeft);

%node [draw=black,shape = star,star points=4,thick,inner sep = 0mm, above, 
%bend left=100,looseness=3] {3}

%;


\end{tikzpicture}
\end{minipage}

\hspace{0.1\textwidth}
\begin{minipage}{0.8\textwidth}
			\renewcommand{\labelitemi}{$\blacklozenge$}
	
\begin{itemize}\setlength\itemsep{-.3em}
\item 1 \hspace{12pt}  Head/dependent relation
\item 2 \hspace{12pt}  Argument repetition ({}\BlankAfterBlank{} idiom)
\item 3 \hspace{12pt}  Propositional completion ({}\VisNtoS{})	
			\renewcommand{\labelitemi}{$\blacktriangledown$}
\item 4 \hspace{12pt}  Phrase (modeled as applicative structure), typed as {}\NPl{} 
\end{itemize}
\end{minipage}
\end{figure}
As this shows, the \q{Student after Student} idiom can be notated as, say,
\AfterNSingAndNSingToNPl{} (using \NSing{} and \NPl{} to mean singular
and count-plural nouns, respectively), but with the special case that the \q{argument} to
\i{after} is repeated in both positions, suggesting an unusual degree of repetition,
something frustratingly recurrent: \i{He went on and on}; \i{Car after car passed
us by}; \i{Time after time I got turned down}.
Although I have no problem
treating these constructions as idiomatic plurals, I also contend (on the
premise of phrase-overlap) that the dependent constituents in the \BlankAfterBlank{}
construction can be hooked to other phrases as well (which is why
\q{and [their/his/her] parents} can also be singular, in this case).  I dwell on
this example because it shows how type/functional accounts of phrase structure
can be useful even if we treat phrases more as frames which overlay linguistic
structure, not as rigid compositional isolates.  Each \q{students} variation uses
morphology to nudge cognitive attention in one direction or another, toward events or the
degree to which events are representative of some global property (here of
a student body), or both.  The \NSingToNPl{} transformation is not \i{the}
morphosyntactic meaning, but instead the skeleton on which the full meaning
(via cognitive schema) is designed.
}
\p{If this analysis has merit, it suggests that a \CAG{} or type-logical 
approach to phrases like \i{many students} or \i{student after student}
(singular-to-plural or plural-to-plural mappings) should be understood not just
as functions among Part of Speech (\POS{}) types but as adding cognitive shading, foregrounding
or backgrounding cognitive elements like events or typicality in some context.
In other words, \i{many students} is type-theoretically \NtoN{} or \NpltoNpl{};
but, in more detail, it adds a kind of cognitive rider attached to the mapping which focuses
cognition in the subsequent discourse onto events (their recurrence and temporal distribution);
similarly \q{student after student} has a \q{rider} suggesting more of a temporal
unfolding.  The second form implies not only that many students complained, but that
the events of these complainings were spread out over some stretch of time.
Each such functional application (mappings between \POS{} understood as linguistic types)
produces not only a resulting \POS{} \q{type}, but also a reconfiguration of cognitive
attitudes toward the relevant situation and context.
Language users have many ways to craft a sentence with similar meanings, and arguably one
task for linguistic analysis is to model the space of choices which are available in a
given situation and represent what specific ideas and effects are invoked by one
choice over others.
}
%\spsubsectiontwoline{Link Grammar and Type Theoretic Semantics}