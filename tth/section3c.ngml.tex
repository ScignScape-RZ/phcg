\subsection{Three tiers of linguistic type theory}
\p{By three \q{tiers} of linguistic organization, I am thinking of
different levels of granularity, distinguished by relative scales of
resolution, amongst the semantic implications of
putative type representations for linguistic phenomena.
As I argued earlier,
type-related observations can be grouped (not necessarily
exclusively or exhaustively) into those I will call
\i{macrotypes} \mdash{} relating mostly to Parts of Speech and the functional treatment
of phrases as applicative structures; \i{mesotypes} \mdash{} engaged with
existential/experiential qualities and \q{Ontological} classifications
like sentient/nonsentient, rigid/nonrigid, and
others I have discussed; and \i{microtypes} \mdash{} related to lexemes and word-senses.
This lexical level can include  \q{microclassification}, or
gathering nouns and verbs by the auxiliary prepositions they allow and
constructions they participate in (such as, different cases), and
especially how through this they compel various spatial and
force-dynamic readings; their morphosyntactic resources for describing states
of affairs; and, within semantics, when we look toward even more fine-grained classifications
of particular word-senses, to reason through contrasts in usage.\footnote{So, conceiving microclasses similar in spirit to Steven Pinker in
Chapter 2 of \cite{Pinker}, though I'm not committing to using the
term only in the way Pinker uses it.  Cf. also \cite{AnneVilnat}, which
combines a microclass theory I find reminiscent of \i{The Stuff of Thought} with
formal strategies like Unification Grammar.
}  Microclasses can point out similarities
in mental \q{pictures} that explain words' similar behaviors, or
study why different senses of one word succeed or fail to be acceptable in particular phrases.
There are \i{stains all over the tablecloth} and \i{paint splattered all over the tablecloth},
but not (or not as readily) \i{dishes all over the tablecloth}.  While \q{stains} is count-plural and
\q{paint} is mass-aggregate, they work in similar phrase-structures because both
imply extended but not rigid spatial presence; whereas \q{dishes} can work for
this schema only by mentally adjusting to that perspective, spatial construal shifting
from visual/perceptual to practical/operational (we might think of dishes \q{all over} the
tablecloth if we have the chore of clearing them).  Such observations support
microclassification of nouns (and verbs, etc.) via Ontological and
spatial/dynamic/configuration criteria.
}
\p{Type-theoretic semantics can also apply Ontological tropes to unpack the overlapping mesh of word-senses,
like \i{material object} or \i{place} or \i{institution}.
This mode of analysis is especially well illustrated when competing senses
collide in the same sentence.  Slightly modifying two examples:\footnote{\cite[p. 40]{ChatzikyriakidisLuo} (former) and
\cite[p. 4]{MeryMootRetore} (latter).
}
\begin{sentenceList}\sentenceItem{} The newspaper you are reading is being sued.
\sentenceItem{} Liverpool, an important harbor, built new docks.
\end{sentenceList}
Both have a mid-sentence shift between senses, which is analyzed
in terms of \q{type coercions}.  The interesting detail of this treatment
is how it correctly predicts that such coercions are not guaranteed to
be accepted:
\begin{sentenceList}\sentenceItem{} The newspaper fired the reporter and fell off
the table (?).
\sentenceItem{} Liverpool beat Tottenham and built new docks (?).
\end{sentenceList}
(again, slightly modifying the counter-examples).  Type coercions are
\i{possible} but not \i{inevitable}.  Some word-senses \q{block} certain coercions
\mdash{} that is, certain sense combinations, or juxtapositions, are disallowed.
These preliminary, motivating analyses carry to more
complex and higher-scale types, like plurals (the plural of a type-coercion
works as a type-coercion of the plural, so to speak).
As it becomes structurally established that type rules at the
simpler levels have correspondents at more complex levels, the use of
type notions \i{per se} (rather than just \q{word senses} or other
classifications) becomes more well-motivated.
}
\p{Clearly, for example,
only certain kinds of agents may have beliefs or desires, so
attributing mental states forces us to conceive of their referents
in those terms:
\begin{sentenceList}\sentenceItem{} Liverpool wants to sign a left-footed striker.
\sentenceItem{} That newspaper plans to fire its editorial staff.
\end{sentenceList}
This \i{can} be analyzed as \q{type coercions}; but the type-theoretic machinery should contribute
more than just obliquely stating linguistic wisdom, such as
maintaining consistent conceptual frames or joining only suitably
related word senses.  The sense of \i{sign} as in \q{employ to play on
a sports team} can only be linked to a sense of Liverpool as the
Football Club; or \i{fire} as in
\q{relieve from duty} is only compatible with newspapers as
institutions.  These dicta can be expressed in multiple ways.
But the propagation of classifications
(like \q{inanimate objects} compared to
\q{mental agents}) through complex type structures lends credence to the
notion that type-theoretic perspectives are more than just an expository tool;
they provide an analytic framework which integrates grammar and semantics, and
various scales of linguistic structuration.
For instance, we are prepared to accept some examples of dual-framing
or frame-switching, like thinking of a newspaper as a physical object and a city government
(but we reject other cases, like \i{Liverpool voted in a new city government and signed a
new striker} \mdash{} purporting to switch from the city to the Football Club).  The rules for
such juxtapositions appear to reveal a system of types with some parallels to
those in formal settings, like computer languages.
}
\p{In short, \q{Ontological} types like \i{institution} or \i{place} serve in some
examples to partition senses of one multi-faceted word.  Here they reveal
similar cognitive dynamics to reframing-examples like \i{to the press}, where
Ontological criteria (like reading something as a place) are triggered by
phrase-scale structure.  But there are also interesting contrasts:
the \i{newspaper} and \i{Liverpool} examples
imply that some words have multiple framings which are well-conventionalized;
newspaper-as-institution feels less idiomatic and metaphorical than
press-as-place.  So these examples suggest two \q{axes} of variation.
First, whether the proper Ontological framing follows from other word-choices
(like \q{fire} in \i{the newspaper fired the reporter}, which has
its own semantic needs), or from morphosyntax
(like the locative in \i{to the press}); and, second, whether triggered framings work
by selecting from established word senses or by something more metaphorical.
Metaphors like \i{to the press} do have an element of standardization;
but apparently not so much so to be distinct senses: note how \i{the press} as metaphorical place
does not work in general: \qmarkdubious{}\i{at the press}, \qmarkdubious{}\i{near the press}
(but \i{at the newspaper}, \i{near the newspaper}
\mdash{} imagine two journalists meeting outside the paper's offices \mdash{} sound quite reasonable).
}
\p{The \q{type coercion} analysis works for mid-sentence frame-shifts; but other
examples suggest a more gradual conceptual \q{blending}.  For example, the
place/institution dynamic is particularly significant for \i{restaurant}
(whose spatial location is, more so, an intrinsic part of its
identity).  Being a \i{place} implies both location and extension; most places are not single
points but have an inside where particular kinds of things happen.  I am not convinced
that restaurant as place and as institution are separate word senses; perhaps, instead,
conversations can emphasize one aspect or another, non-exclusively.  As I have argued,
we need not incorporate all framing effects via \q{subtypes} (restaurant as either
subtype of hypothetical \q{types of all} places or institutions, respectively).  But
\q{placehood}, the Ontological quality of being a place \mdash{} or analogously being
a social institution \mdash{} identify associations that factor into cognitive frames; types
can then be augmented with criteria of tolerating or requiring one association or another.
So if \q{restaurant} is a type, one of its properties is an institutionality that \i{may}
be associated with its instances.  In conversation,
a restaurant may be talked about as a business or community, foregrounding this
dimension.  Or (like in asking for directions) its spatial dimension may be foregrounded.
The availability of these foregroundings is a feature of a hypothetical restaurant type,
whether or not these phenomena are modeled by subtyping or something more sophisticated.
The \q{newspaper} examples suggest how Ontological considerations
clearly partition distinct senses marked by properties like objecthood or
institutionality (respectively).  For \q{newspaper} the dimensions are less available for
foregrounding from a blended construal, than \q{unblended} by conventional usage; that
is why reframings evince a type \i{coercion} and not a gentler shift of emphasis.
The example of \i{restaurant}, in contrast, shows that competing routes for
cognitive framing need not solidify into competing senses, though they trace
various paths which dialogs may follow.
But both kinds of examples put into evidence an underlying
cognitive-Ontological dynamic which has potential type-oriented models.
}
\p{At the most general level \mdash{} what I called \i{macrotype} modeling \mdash{} a type
system recognizes initially only the grammatical backbone of expressions, and
then further type nuances can be seen as shadings and interpretations which add substance
to the syntactic form.  So in type-theoretical analysis at this more grammatic level,
to which I now turn, we can still keep the more fine-grained theory in mind:
the relation of syntax to semantics is like the relation of a spine to its flesh,
which is a somewhat different paradigm than treating syntax as a logical or temporal
stage of processing.  Instead of a step-by-step algorithm where grammatical parsing
is followed by semantic interpretation, the syntax/semantics interface can be seen
as more analogous to stimulus-and-response: observation that a certain grammatic
configuration appears to hold, in the present langauge artifact, triggers a marshaling
of conceptual and cognitive resources so that the syntactic backbone can be filled in.
Perhaps a useful metaphor is grammar as gravitation, or the structure of a gravitational
field, and semantics is like the accretion of matter through the interplay of multiple
gravitational centers and orbits.  For this analogy, imagine typed lambda
reductions like \PropToNYieldsN{} taking the place of gravitational equations;
and sentences' grammatic spine taking the place of curvature pulling mass into a planetary center.
}
\p{Parts of speech have \q{type signatures}
notionally similar to the signatures of function types in programming languages: a verb
needing a direct object, for example, \q{transforms} two nouns (Subject and Object)
to a proposition, which I have been notatating with something like \NNtoS{}.
At the most basic level, the relation of Parts of Speech to \q{type signatures}
seems little more than notational variants of conventional linguistic
wisdom like a sentence requiring a noun and a verb (\SeqNPplVP{}).
Even at this level, however, type-theoretic intuitions
offer techniques for making sense of more complex, layered sentences,
where integrating link and phrase structures can be complex.
Even the most broadly scoped analysis of type signatures, dealing only with
generic Parts of Speech like nouns and verbs, can lead to surprising
complications.  One example I have alluded to several times, and will return to shortly:
the problem of applying Dependency Grammar where phrases do not seem
to have an obviously \q{most significant} word for linkage with other phrases.
}
\p{A tendency in both dependency and phrase-oriented perspectives is to define
structures around the most \q{semantically significant} words \mdash{} so that a phrase
like \i{many students} becomes in some sense collapsible to its semantic
core, \i{students}.  Some of my earlier examples, however, argued that
phrases cannot just be studied as replacements for semantic units.  Incorporating
type theory, we can instead model phrases through the perspective of
type signatures: given Part of Speech annotations for phrasal units and then for
some of their parts, the signatures of other parts, like verbs or adjectives
linked to nouns, or adverbs linked to verbs, tend to follow automatically.
A successful analysis yields a formal tree, where if (in an act of semantic
abstraction) words are replaced by their types, the \q{root} type is something like
\Prop{} and the rest of a tree is formally a reducible structure in
Typed Lambda Calculus: \NNtoProp{} \q{collapses} to \Prop{}, \ProptoN{} collapses
to \N{}, and so forth, with the tree \q{folding inward} like a
fan until only the root remains \mdash{} though a more subtle analysis would
replace the single \Prop{} type with variants that recognize different
forms of speech acts, like questions and commands.  In Figure ~\ref{fig:Iknow},
this can be seen via the type annotations: from right to left \NtoN{} yields the
\N{} as second argument for \i{is}, which in turn yields a \Prop{} that is mapped
(by \i{that}) to \N{}, finally becoming the second argument to \i{know}.  This calculation
only considers the most coarse-grained classification (noun, verb, proposition) \mdash{} as I
have emphasized, a purely formal reduction can introduce finer-grained grammatical or
lexico-semantic classes (like \i{at} needing an \q{argument} which is somehow an expression
of place \mdash{} or time, as in \i{at noon}).  Just as useful, however, may be analyses
which leave the formal type scaffolding at a very basic level and introduce
finer type or type-instance qualifications at a separate stage.
}
\p{In either case, Parts of Speech are modeled as (somehow analogous to) functions, but the important
analogy is that they have \i{type signatures} which formally resemble functions'.
Phrases are modeled via a \q{function-like} Parts of Speech along with one or more
additional words whose own types match its signature; the type calculations
\q{collapsing} these phrases can mimic semantic simplifications
like \i{many students} to \i{students}, but here the theory is explicit
that the simplification is grammatic and not semantic: the collapse
is acknowledged at the level of \i{types}, not \i{meanings}.  In addition,
tree structures can be modeled purely in terms of inter-word relations
(this is an example of embedding lambda calculii in process algebras),
so a type-summary of a sentence's phrase structure can be notated and
analyzed without leaving the Link Grammar paradigm.
}
\p{As a concrete example, in the case of \q{many students},
both \q{students} and the semantic role of
the phrase are nouns (count-plural nouns, for where that's relevant).  Accordingly,
\q{many} has a signature \NtoN{} (or \NpltoNpl{}, dependending on how
narrowly we want to notate the types in context).
Once we assign types and signatures to all words in a sentence, we can also see a
natural hierarchy resembling an expression in typed lambda calculus, where some words
appear as \q{functions} and others as \q{arguments}.  Often the less
semantically significant words appear as \q{higher} in the structure,
because they serve to modify and lend detail to more significant words.
The kind of structure or \i{Charpente} which falls out of a sentence \mdash{}
adopting a term from \Tesniere{}
(cf. \cite[p. 181]{ElkeTeich}) \mdash{} is typically different from a link-grammar
\q{linkage}, although the two structures can be usefully combined.
}
\p{To return to the example of \i{Student after student}, where designating
one word to \q{represent} the phrase seemed arbitrary, we can analyze the
situation via type-signatures.  I have teased a proposed solution repeatedly;
here's what I had in mind.  Insofar as \i{after} is the only non-noun,
the natural conclusion is that \q{after} should be typed \NNtoN{}
(which implies that \q{after} is analogous to the \q{functional} position, and
in a lambda-calculus style reconstruction would be considered the \q{head}
\mdash{} Figure ~\ref{fig:ESA} is an example of how
the sentence could be annotated, for sake of discussion).
This particular idiom depends however
on the two constituent nouns being the same word (a pattern I've also alluded to with
idioms like \i{time after time}), which can be accommodated by invoking the (computationally rather complex
and topical) concept of \i{dependent types} \cite{BernardyEtAl}, \cite{TanakaEtAl}
\mdash{} in other words the parameters
for \i{after} are a dependent type pair satisfied by an identity comparison between
the two nouns.  The signature for \q{after} has this added complication, but
the nuances of this example can still be accommodated within the overall
architecture of type theory.  I would pair this argument with my earlier
analysis of \q{many} variations which suggested how apparent complications
can be accommodated largely within the extant theoretical resources of
Link Grammar, and in combination suggest that the union of Link Grammar
with Type-Theoretic Semantics seems poised to accommodate many
complex real-world linguistic cases with a coherent abstract perspective.
}
\p{Consider alternatives for \q{many students}.  The phrase as
written suggests a type signature (with \q{many} as the \q{function-like} or
derivative type) \NpltoNpl{}, yielding a syntactic interpretation of the phrase; this
interpretation also suggests a semantic progression, an accretion of intended detail.
From \i{students} to \i{many students} is a conversion between two plural nouns
(at the level of concepts and semantic roles); but it also implies relative size,
so it implies some \i{other} plural, some still larger group of students from which
\q{many} are selected.  While rather abstract and formal, the \NpltoNpl{} representation
points toward a more cognitive grounding which considers this \q{function} as a form
of thought-operation; a refinement of a situational model, descriptive resolution,
and so forth.  If we are prepared to accept a cognitive underpinning to semantic
classification, we can make the intuition of part of speech signatures as \q{functions}
more concrete: in response to what \q{many} (for example) is a function \i{of},
we can say a function of propositional attitude, cognitive schema, or attentional
focus.  The schema which usefully captures the sense and picture of \i{students} is
distinct (but arguably a variation on) that for \i{many students}, and there is a
\q{mental operation} triggered by the \i{many students} construction which
\q{maps} the first to the second.  Similarly, \i{student after student} triggers a
\q{scheme evolution} which involves a more explicit temporal unfolding
(in contrast to how \i{many students} instead involves a more explicit
quantitative \i{many/all} relation).  What these examples show is that
associating parts of speech with type signatures is not just a formal
fiat, which \q{works} representationally but does not necessarily capture
deeper patterns of meaning.  Instead, I would argue, type signatures
and their resonance into linkage acceptability structures
(like singular/plural and mass/count agreement) \i{point toward} the
effects of cognitive schema on what we consider meaningful.
}
\p{In \i{Student after student came out against the proposal},
to \i{come out}, for/against, lies in the semantic frame of attitude and expression
(it requires a mental agent, for example), but its reception
carries a trace of spatial form: to come out \i{to} a public place, to
go on record with an opinion (I analyzed this case in Section 6).  Usually
\q{come out [for/against]}, in the context of a policy or idea, is similarly
metaphorical.  But the concrete spatial interpretation remains latent, as a kind
of residue on even this abstract rendition, and there sustains a chance that this
undercurrent will actually figure in conversants' mutual understanding \mdash{} if
there were not just columns being written and opinions voiced but demonstrations
on the quad.  The spatial undercurrent is poised to emerge
as more literal, should the context warrant.  However literally or metaphorically
the \q{space} of the \q{coming out} is
understood, however explicit or latent its cogitative figuration,
is not something internal to the language; it is a potentiality which
will present in different ways in different circumstances.  This is not to say that
it is something apart from linguistic meaning, but it shows how linguistic meaning
lies neither in abstract structure alone, nor contextual pragmatics, but in their cross-reference.
}
\subsection{Levels of formalization}
\p{Of the three type levels I have proposed, the macrotype \q{functional} level is the most
quasi-mathematical; for other levels, formal type theory may provide interpretive
tools and methodological guides, but formally representable framings and
transformations may be only approximations of how people actually think, while
they are understanding language.  From this perspective, we are left with the
metatheoretical question of clarifying how different kinds of analyses, which
put different degrees of weight on formal or on interpretive argumentation,
are to be joined in overarching theories.  In particular, are the
linguistic phenomena which seem to demand more \q{interpretive} treatment actually
beyond formalization, or is it just impractical (but possible in theory) to provide
formal analysis of each individual case-study, each real-world language formation?
Is Natural Language actually no less formal than (for example) computer programming
languages, except that the former have a much larger set of semantic and syntactic
rules such that any analysis can uncover them only partially?  Or is any rule-based
model of language, no matter how complete, necessarily partial relative to real language?
}
\p{Computer languages are a good case-study in what I might call \q{semiotic computability}.
This designates the question of whether the
operations of sign-systems \mdash{} how sign-users express intentions by forming or modifying
structured networks of signs that explicitly exhibit or are understood to have been
formed according to collectively recognized signifying rules \mdash{} can be modeled,
at least to some substantial degree, by computable algorithms.  Our
notion of computation can be based on modern computer code, not just academic
topics like pure functions: the behavior of computing systems
where many functions run concurrently, with possible side-effects, is often
non-computable via static analysis; such systems can only be understood by
actually running them.  Nevertheless the capabilities of software programmed
in modern languages certainly deserve to be characterized as \q{computable}
behaviors.  A single function, which embodies a computable
calculation, may be part of a process space whose evolution through
time is nondeterministic, and computing environments which employ
functional side-effects are difficult or impossible to evaluate in the abstract.
I use \q{computability} therefore in this wider sense: operationally implementable
according to theories underlying mainstream programming languages, which is
conceptually (if perhaps not mathematically) distinct from \q{computability} in
subjects like algorithm analysis.
}
\p{Natural Langauge Processing, working with human languages from a computing platform,
is then a step further, continuing beyond logico-mathematic abstractions and toward
empirical language-use.  We can consider at what point formal and computational methods reach a limit,
beyond which they fail to capture
the richess and expressiveness of Natural Language, or whether this limit itself
is an illusion \mdash{} whether even fully human
language competence is (perhaps in principle if not in practice) no less reducible
to formalizable patterns.  Using the wording I just proposed, we can speculate on
whether all language is \q{semiotically computable} or whether language merely
depends on faculties which in some neurological and/or presentational sense are
\q{computable} in those terms \mdash{} faculties that, measured against linguistic fluency,
are necessary but not sufficient.
Whatever one's beliefs on this last question,
a progression of subdisciplines \mdash{} from formal-logical semantics through programming
langauges and computational Natural Langauge Processing \mdash{} is a reasonable
scaffolding for a universe of formal methods that can build up, by progressive theoretical
sophistication or assembly of distinct analyses which piece together jigsaw-like, to model
real-world language understanding.  Perhaps real language is an \q{emergent property} of
many distinct algorithms that run and combine in the mind; or perhaps the relevant
algorithms are a precondition, presenting cognition with essential signifying givens
but fleshed out in other, more holistic ways, as we become conscious of language not
just as a formal system but an interactive social reality.
}
\p{I have (in the last two sections) sketched a
similar theoretical progression, starting with a theory of
grammar (Link Grammar), transitioning to a form of semantics (a type-theoretic semantics
defining type hierarchies and signatures over linkage graphs), and finally proposing a
cognitive interpretation of the resulting semantics.  I will refer to this
\i{interpretation} as  \q{Cognitive State Semantics}, meaning that such a theory
adopts its \i{formal} structures from Link Grammar and type-theory but also attempts
to \i{motivate} these structures by appeal to cognitive considerations.  Both Link Grammar
(through its specific Category of labeled graphs modeling sentence linkage-structures) and
Type-Theoretic Semantics work with rigorous, algebraically formal models satisfying criteria
I referenced at the end of the last section: translation of language content into these
formats and subsequent review or transformation of the target structures can be programmed
as a purely mechnical space of operations.
}
\p{By itself, the superposition of
type-theoretic semantics on link-grammar graphs does not cross a hypothetical \q{barrier} between
the formal and the cognitive.  But I intend here to suggest a cognitive \i{interpretation}
for the formal structures; that they represent an outline of cognitive schema, or progressions,
or represent linguistic \q{triggers} that a cognitive language ability (taking language
as part of an environing world and produced by others, in rule-bound social situations,
to communicate ideas and sentiments) responds to.  This range of interpretations is
deliberately open-ended: we can say that a formal infrastructure grounds the cognitive
reception of language givens, without arguing specifically that formal structures identified
in language therefore model cognitive operations directly, or that these are instead
patterns identified in language that trigger a cognitive response, or any other
paradigm for mapping cognition as process and activity to language structure as model and
prototype.  Leaving these options open, however, I will focus in the remainder of this
paper on one interpretation, considering formal structures as \q{triggers} which
get absorbed into language understanding via observatory propensities: as language
users (on this proposal) we are disposed to identify certain formal structurations
operating in language as we encounter it, and respond to these observations by building
or refining mental models of the situations and signifying intentions we believe have been
implied by the discourse, in evolving and intersubjective dialogic settings that involve
joint practical activity as well as communication.
}
\p{In this sense, I believe natural language reveals mutually-modifying juxtapositions
of concepts whose full semantic effects
are probably non-computable: I would work on the assumption that language
\i{as a whole} and as human social phenomena is not \q{computable} in a
semiotic sense, or any related practical sense (although
I make no metaphysical claims about the \q{abstract} computability of mental
processes merely by virtue of their neurophysical materiality).
The aforementioned \q{linguistic side effects} can be \i{modeled} by tracing our reception
of linguistic meaning through syntactic and semantic formations, like Link Grammar
and Type Theory, but I argue for such models not as models \i{of} cognitive processes,
but rather models of \i{observations} which trigger cognitive follow-up.  Even if we
believe in and practice a rigorous formalization of morphosyntactic structure,
where the \i{pattern} of conceptual \q{side-effects} can be seen as
unfolding in algorithmic ways, the cognitive \i{details} of these
effects are too situational, and phenomenologically rich, for
computability as ordinarily understood.
}
\p{But the formal structure is
not wholly irrelevant: to call up nuanced cognitive schema
\mdash{} or so I submit for consideration \mdash{} may not be possible without
algorithmically reproducible lexicosemantic and morphosyntactic triggers,
at least modulo some approximation.  A (perhaps non-computable) space
of cognitive schema may be projected onto a (perhaps computable)
set of affiliated morphological patterns, using notations like
link-grammar pairs and type signatures to catalog them.  For example, there may be a non-computable
expanse of possible construals of pluralization; but any such construal,
in context, is called into focus in conversants' minds by morphosyntactic
invitations, by speakers' choices of, say, \mbox{\NSingToNPl{}}-pattern
phrases.  The important balance is to take formalization as far as is reasonable
without being seduced into logico-symbolic reductionism \mdash{} a
methodological pas de deux I will explore further in the next section, a brief
concluding coda to this paper.
}
\p{Any word or usage invites various facets to either
emphasize or deemphasize, and these subsumed concepts or foci are
latent in potential meanings, brought into linguistic space
by the play of differentiation \footnote{Alluding, in part, to Sausurrean \q{system of differences}
\cite[p. 15]{EfePeker} \mdash{} to choose a reference which introduces
Sausurre in a rather unexpected context.
}
: \i{baked}, not \i{made}; \i{flew}, not \i{traveled};
\i{spill}, not \i{pour}.
These under-currents of subsidiary concepts and foci are selectively hooked onto by
morphosyntactic selection, so in analyzing phrase
structure we also have to consider how using syntax
which constructs a given structure also brings to the forefront certain
nested concepts and construals, which are latent in word-sense options;
in the topos of lexicosemantic possibilia.
}
\p{So, any talk about \q{side effects} of morphosyntactic functions
\mdash{} mapping verb-space to adjective-space, noun-space to
proposition-space, singularity to plurality, and so forth \mdash{} should consider
a type-theoretic gloss like \NtoN{} as sketching just the motivating
scaffold around an act of cognitive refocusing.  The interesting semantics
lies with \i{how} a sense crosses over, in conversants' minds,
to some other sense or concept, wherein other aspects are foregrounded
\mdash{} for example, within temporal event plurality: multiplicity as
frequency, or episodic distribution relative to some time span;
or suggesting something that is typical
or predominant; or relative count against some other
totality \mdash{} each such refocusing triggered by a phrasal construction
of the form \NtoNpl{} or \mbox{\NpltoNpl{}}.
Or we can map singulars, or count plurals, to mass nouns, and vice-versa (\i{shrubs} become \i{foliage};
\i{water} becomes \i{a glass of water}).
The plural and the singular are a coarse-grained semantic that has not yet arrived as \i{meaning}.
Conceptual spaces guide attention to classes and properties, defining a path of ascending
precision as speakers add descriptive detail;
cognitive construals negotiate relations between different kinds
of aggregates/individuals; individuality, aggregation and multiplicity as phenomena and
disposition.  These construals are practical and embodied, \i{and}
phenomenological \mdash{} they direct attention (\i{qua} transcendental universal of
mentality, if we like), to and fro, but in the course of intersubjective and
goal-driven practical action (and in that sense particular, world-bound, historicized).
}
\p{Given these considerations, I propose a \q{Cognitive State Semantics}
\mdash{} understanding phrase structure in terms of (or analogous to) functional effects
(like \cite{ShanThesis}), but cognitive: word and syntax choice effectually
steering cognitive appraisals of jointly experienced situations
in specific directions.  Cognitive State Semantics also has formal implications:
the inner structuration of data \q{spaces},
including unknown and undefined values, and including (side-effects-bearing) function types,
can be understood as dynamic \i{states of knowledge} and their changes, grounding datatype semantics in human
use/interactions.  Linguistically, the \q{effects} of language \q{functions} are
mutations/modifications in cognitive state, respondant to concrete
or abstract scenarios which are topics of dialog.  Sometimes, effects may
tolerate mathematical analysis; but such analytical thematics tend to peter out into the
ambient, chaotic worldliness of human consciousness.
}
