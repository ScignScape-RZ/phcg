\section{The Illogic of Syntax}
\p{As I understand it, a non-trivial truth-theoretic semantics 
requires more than a holistic association between 
sentences and propositional content: it requires that this 
association be established \i{by linguistic means} and 
\i{on linguistic grounds} (syntax, semantics, pragmatics).  
I will present several arguments against this possibility, in 
the general cases \mdash{} that is, against the possibility that 
for \i{typical} sentences we can analyze syntactic form through 
the lens of the logical structure of propositions signified 
via a sentence; or analyze natural-language semantics through a 
logically well-structured semantics of propositions.  
I will emphasize to issues: first, that the architecture of 
linguistic performances \i{does not}, in the general case, 
\i{recapitulate propositional structure}; and, second, 
that language acts work through gaps in logical specificity that 
complicate how we should theorize the triangular relation between 
surface language, propositional content, and side-effect meanings.
}
\p{}
\p{}
