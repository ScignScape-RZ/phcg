\p{A leitmotif of Cognitive Linguistics is critiquing \q{truth-theoretic} semantics 
\mdash{} part of a broader objection to linguistic and philosophical paradigms which 
appear to underappreciate language's subtlety, its lack of mechanical determinism, 
and its rootedness in human consciousness and sociality.  To make the critique 
precise, I think we can define \q{truth-theoretic semantics} with an emphasis 
on two dimensions: first, the idea that linguistic meaning is closely 
associated with propositionality \mdash{} that meaning is grounded in states of affairs 
that make (or would make) statements true; and, second, that there is a 
formalizable (and potentially computationally reproducible) process whereby 
language-users parse utterances or sentences to recover their signified propositional 
content \mdash{} which is the most important \i{deep structure}, or organizing principle, 
determining the form of language.  To explain language we have to consider both 
semantics and grammar: the rules distinguishing senses of one word and 
selecting among related words; and the rules for combining 
words into phrases and sentences.  Truth-theoretic semantics 
goes beyond the obvious fact that language often expresses ideas 
which can be reasonably summarized in logical terms \mdash{} as a 
guiding metatheory these paradigms have to argue that 
the logical forms of such meanings retroactively shapes 
syntax and semantics.  In this paper I will try to lay out a rigorous theory of the limitations
and motivations for these paradigms and then suggest alternative \q{metatheories} 
for semantics, or the philosophy of language, which are philosophically stronger.
} 
\p{It is easy to summarize how logic \i{can} in principle, and in some 
cases, provide an outline to syntactic and semantic structures.  
Plurals, for instance, correspond to the logical mapping between 
individuals and sets: assuming we have a theory of the concept 
\i{dog}, we then have via logic (qua essential cognitive faculty, 
perhaps) a correlated concept of a \i{set} of dogs; isomorphically, 
insofar as \q{dog} signifies the singular concept, the plural 
\q{dogs} signifies the \i{set}.  Meanwhile, given discrete 
ideas like 
\begin{sentenceList}\sentenceItem{} Those dogs are Staffordshires.
\sentenceItem{} Those dogs are females.
\end{sentenceList}
we can form compounds using logical disjunction:
\begin{sentenceList}\sentenceItem{} Those dogs are female Staffordshires.
\end{sentenceList}
Broadening the expressive scope, \i{modal} logic can represent 
hypotheticals and speaker-sentiment:
\begin{sentenceList}\sentenceItem{} The window may be open (it is possible that the window is open).
\sentenceItem{} I'd like the window closed (I'd prefer to be in a world where 
the window is closed).
\sentenceItem{} If the window is open, can you close it  
(if we are in a world where 
the window is open, I request we move to a world where it is closed)?
\end{sentenceList}
 
\noindent{}In short, familiar syntactic and semantic patterns have analogous, formalizable 
constructions in logic and set theory.
} 
\p{The question is then how significantly these formal constructions 
explain or regulate the informal linguistic analogs.  
If we believe that the brain is constrained to 
process language according to logical maxims, we may believe 
that articulating language-to-logic analogies is a meritous 
step toward a scientific theory of language.  Conversely, if 
we believe that linguistic understanding spans many cognitive registers 
\mdash{} most of which are only tangentially governed by logical 
parameters \mdash{} the relatively simple and context-neutral 
models of (such formations as) sets/plurals, inter-propositional 
operators (e.g. conjunction and disjunction), and modal possibilia 
are at most approximately relevant to a robust cognitive 
theory of language. 
} 
\p{It is easy to understand why scientists would find \q{truth-theoretic} 
paradigms and logical language models appealing: there certainly 
is a level of nuance, a contextuality and interpersonal dynamic, 
which feels somehow beyond and more granular than formal logic.  
We cannot understand someone's language without some degree of 
emotional insight and \q{theory of other minds}, and without 
grasping the environing situations where language-acts are 
carried out.  But human cognition on that emotional and situational 
level seems holistic and \q{emergent}; merely observing human behavior 
(as if sociologically or anthropologically) at the everyday level 
does not take us to the cognitive or neurophysical particulars 
that can explain \i{how} reason and consiousness emerges in the brain.  
Targeting attention on the logical structures within language, 
while downplaying some of language's subtlety and \q{humanness}, 
can be perceived as a strategy to foreground cognitive processes 
involving language that we can actually hope to understand 
on a neurological level, and therefore approach from a goal 
of scientific (physical and/or logico-mathematical) \i{explanation}, 
as opposed to philosophical or humanistic \i{interpretation} \mdash{} 
to uncover the physical/causal and logical/regulative mechanisms 
of the phenomena to be explained.   
} 
\p{My perspective here is not to dispute the importance of 
physical/causal and logical/regulative \i{explanations}, or that 
a \i{perfect} theory of language would need (perhaps in 
conjunction with an overarching cognitive science) 
such an explanatory base.  However, I believe that trying to 
isolate a logical model of \i{language} specifically is essentially 
a philosophical gambit: a hope that language can be isolated from 
the totality of cognition and consciousness and have its own 
reductive theory.  When for instance I talk about \i{dogs} in the 
plural I am phenomenologically and practically engaged with 
specific dogs, or else with mental impressions or intentions 
of dogs, and their plurality is manifest in multiple 
perceptual and enactive manners \mdash{} e.g., as a diversified 
visual expanse, a multi-point attentional (and/or kinaesthetic) 
focus.  Any language production or reception in enmeshed with 
this phenomenologically and enactive immediacy, and the plurality 
of \i{dogs} connects to this pluralized intentional disclosure, 
not just to the logical reading of plurals as sets.  To analyze 
the morphosyntax of plurals as \i{only} the appearance, within 
language-processing, of the logical individual/set distinction is 
to assume that our diagram of the stages of linguistic understanding 
can be lifted outside the context of our overall conscious-cognitive 
engagements.  Ditto for other logical patterns like modality/hypotheticals 
and predicate operations.     
} 
\p{Truth-theoretic paradigms are not entirely off-base: certainly there is 
\i{some} propositional content which is usually essential to meaning.  
Although not every sentence baldly asserts some proposition, there 
is typically a propositional content for which each sentence stands 
as a kind of complex signifier, and then the sentence as a whole 
acquires its full meaning in orientation to that signified: 
\begin{sentenceList}\sentenceItem{} Please close the window?
\sentenceItem{} I think the window is closed.
\sentenceItem{} The window is closed.
\end{sentenceList}
There are indeed several issues which complicate a \q{mechanical} translation of 
sentences to propositional contents.\footnote{Henceforth I will talk about sentences 
as proxy for linguistic utterances (or written discourse segments) in general.}  
The diversity of illocutionary forms (like questions, imperatives, and statements 
of belief as well as direct assertions) is one complication.  A second is 
context-sensitivity: normal sentences are truth-theoretically incomplete:  
taken out of context, they do not internally have sufficient precision 
to fully specify propositional content in isolation.  Anaphoric resolution, 
pronouns, deictic references (\i{here}, \i{there}, \i{now}), and so forth, 
all are open-ended semantic units that demand pragmantic processing to 
be completed.  Subtler contextual effects are also in play \mdash{} in a 
vegan restaurant, say, someone requesting 
\begin{sentenceList}\sentenceItem{} Can I have some milk in this coffee?
\end{sentenceList}
is presumably intending a vegan milk substitute.  Making it the case 
that \i{actual} milk is in her coffee is \i{not} fulfilling her 
request, as a cooperative conversational partner.
}
\p{Such pragmantic and contextual nuances, however, do not intrinsically 
point against a semantics grounded in propositional meanings.  We cannot 
reject the possibility that contextual processing is just one stage in 
mapping language content to propositional constituents.  If for instance 
\i{her} is in some context a designation for Alexandria Ocasio-Cortez, 
U.S. Representative from New York, the relevant sentence can be 
generalized by substituting for \q{her} the definitive phrase.  
Contextuality is a factor for any semantic theory; we should assume 
that it specifically disfavors truth-theoretic approaches unless 
there is reason to believe that context is \i{more} difficult to 
model truth-theoretically than on other paradigms.  
Here, instead, I will concede that for typical sentences there 
\i{is} a propositional content that determines sentence-meaning and 
that can be designated, at least under reconstructive analysis, 
in context-neutral ways \mdash{} in effect a kind of semantic 
\q{deep structure} which resolves pronouns, anaphora, deixis, and 
other open-ended linguistic surface structures.  I take it 
that such a logically refined \i{intermediate representation} is 
necessary but not sufficient for truth-theoretic semantics.  
Since I also accept this intermediate representation as 
appropriate for normal language, I will here focus on the 
\i{not sufficient} part. 
}
\p{My critique of truth-theoretic paradigms, in a nutshell, is that 
language-processing can \i{not} be isolated from the totality 
of our conscious-cognitive engagements.  So although the mappings 
between sentences and propositions are real, they do not lead 
us toward a useful heuristic model of language as a kind of 
\q{informalized} logic \mdash{} because the aspects of cognition wherein  
such a logical gloss is explanatorily useful cannot be isolated 
as a grounding theory for language in general.  
}
\p{In a typical case, I contend that each sentence has a corresponding 
propositional content, based off of the relevant contextual and 
enactive situations where the language use is involved.  Certain 
interpretive processes may be needed to uncover propositional 
details from surface language, but I will argue that related 
interpretations are intrinsic to language-understanding on 
any theory.  So I will argue that an important dimension to 
language is the designation and negotiation of propositional 
content.
}
\p{At the same time, I also believe that 
cognitive engagement with propositional 
content is endemic to any purposeful, 
goal-directed, social human activity, including 
but hardly limited to language.  A defence of 
truth-theoretic semantics would need to 
demonstrate that linguistic performance, in 
some essential or formal manner, thematizes 
propositional content \i{above and beyond} the 
latent logicality of human reason and practice 
in general \mdash{} or at least that language 
embodies an especially structured and formalizable 
manifestation of this logicality.  
Conversely, a \i{critique}  of 
truth-theoretic semantics should present arguments 
that language's logical order is derivative 
of the ambient rationality of human action in 
general, and that the specific \i{linguistic} 
structures of syntax and semantics do not 
derive, directly, from the priority of signifying 
propositional content.  For the latter perspective, 
the question is then from whence syntax and semantics 
\i{do} derive, and how that origin connects to 
the logicality which (I contend and conceed) \i{is} 
certainly present in language. 
}
\p{So my second goal in this paper is to sketch a theory of 
this other origin \mdash{} drawing from various perspectives, 
including phenomenology, Cognitive Linguistics, and the 
philosophy of science.  My overall approach is to, 
rather than treat natural language as a self-contained 
system, theorize language as a bridge between 
multiple \q{ontological} and scientific registers: 
social action and private cognition; practical 
embodied-enactive engagements and structurally 
analyzable cerebral routines; consciously mediated 
rationality and neurophysical substrata.  
Linguistics itself spans a subdisciplinary 
tableau which stretches from the social/quotidian to 
the mathematical or microscale \mdash{} trying to uncover 
the formal underpinnings of syntactic and semantic 
systems at the neurocognitive level (or at least via 
formalizations we can see as potentially realizable at 
the neurophysical scale) or at a computational 
level where we try to establish language's 
functional organization and schematic regularities. 
}
\p{Intrinsic to this disciplinary spread is the question of 
what qualifies as a scientific theory of language.  Ideally 
linguistics would take us to a causally closed and 
complete account of language as a phenomenon in the world, 
analogous to how biology provides a theory of life, 
disease, and reproduction grounded in phyical and chemical 
explanations.  On this analogy we can then debate whether the 
best scientific theory would be ultimately reducible to 
neurocognitive principles \mdash{} taking brain activity as the 
physical realization of linguistic processes akin to 
cells and organic molecules as the realization of 
biological phenomena \mdash{} or whether language is 
best treated, in the scientific vein, more as a 
structured system; i.e., that the system-properties constituting 
language which our language-using brains must instantiate 
are more scientifically salient than their 
neurocognitive implementation.  But whether on a neurophysical 
or computational track, this philosophy implies that the 
essential goal of linguistic \i{science} is to strip it, 
at least for explanatory purposes, of its human and 
societal context.  The distinctly human qualities of 
language are perhaps an \q{emergent property} of 
structured neurophysical systems, and linguistics 
needs to bring us to the threshold of explaining the 
laws governing the reductive base of that emergence, 
as opposed to itemizing the rules of the emergent 
system on its own terms. 
}
\p{I believe similar metatheoretical intuitions have buttressed 
the intuitive appeal of truth-theoretic semantics, because 
excavating the logical rules of language can seem like 
a path for us to understand language on the terms 
by which it manifests at the neurocognitive level 
(as opposed to how it appears to us consciously and 
intersubjectively).  However, while I agree that on 
some level a scientifically complete theory of 
language (and of all human reason) needs to explain down 
to neurophysical substrata (on pain of implying some 
sort of \i{elan vitale}-like rational essentialism), I 
believe it is erroneous to site this reduction \i{inside linguistics 
itself}.  Our world-view may call for an overall 
reduction of \i{all} reason and consciousness to neurophysical substrata; 
but I question whether linguistically inflected cognition has a 
reductive analysis separate and apart from this ambient, 
metaphysical reductionism.  Thus, I will argue that 
logico-mathematical analyses of language, 
even if they are motivated by an understandable commitment 
to physicalism or functionalism as an overarching mind-body 
paradigm, are inappropriately reductionistic \visavis{} 
\i{language} as such.  Our reductionistic commitments 
should not compel us toward reductive theories of 
syntax and semantics within the confines of the study 
of language itself.    
}
\p{In short, I will try to outline a theory of language which 
would be a philosophical complement to the 
\q{nonreductive physicalism} of John Searle or 
David Woodruff Smith: we can take mind-body 
reductionism as an ideological axiom but not a 
humanistic \i{methodology}.  This includes critiquing 
paradigms in \i{computational} linguistics, and 
the larger \q{\AI{}} trend of \q{mind as computer}; the 
assumption that cognitive processes are reproducible 
or can be modeled via computer software.  I think 
there is a role for computational models which can 
shed some light on linguistic structures, but not to 
the extent that language understanding can be 
fully replicated on a computer (\AI{} systems right 
now evince \i{some} realistic language-behavior but
hardly reveal a human-like competence).  I will therefore 
consider what class of computer models \i{can} be useful.       
}
\p{Overall, I dispute the assumption that 
linguistic analysis needs to be a self-contained 
science; or to engender self-contained 
\q{Natural Language Processing} software.  Instead, it 
is appropriate for practitioners to 
understand linguistics in federation with other 
fields and sciences, both in the humanities and 
natural or mathematical sciences.  The proper metatheoretical 
role for linguistic analysis is to isolate properly 
\i{linguistic} (syntactic, semantic, and pragmatic) 
structures as theoretical targets, strategically defering to 
other sciences outside the circle of language proper (whether in 
the sociological or neurocognitive direction).  In this guise 
linguistics can act as a \q{bridge} theory, connecting 
sociology to neurophysics, for example, by analogy to 
biology bridging medicine and chemistry.  If linguistics' 
proper metascentific role is to serve as a social-to-natural 
science (and humanities-to-mathematics) bridge, I will 
argue that truth-theoretic perspectives are less than ideal 
for this role, and will propose an alternative more grounded 
in Cognitive Semantics and Cognitive Grammar.
}
\p{}
