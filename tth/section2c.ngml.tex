\spsubsectiontwoline{Semantics and Logical Underdetermination}
\p{I believe similar effects of underdeterminism 
can be found even in straightforward, locutionary speech-acts.  Consider
\begin{sentenceList}\sentenceItem{} The Maple Leafs failed to win in overtime for the
first time this year.
\sentenceItem{} The Maple Leafs failed for the first time this year
to win in overtime.
\end{sentenceList}
The first can mean either that the Leafs had won \i{all} or \i{none}
of their prior overtime games.  From a phrase-structure perspective,
we have to image that \i{to win in overtime} can \q{migrate} so we
hear it as in the second version of the sentence.  For more inter-word
grammars, the alternation is simpler: \q{for}, initiating the phrase
\i{for the first time}, can be linked with either \i{failed} or
\i{win} \mdash{} notationally, it amounts to the presence or absence of one
graph-edge, when the syntax is represented as a graph with
inter-word labels for link kinds.  This could be a distinction without
a real difference, since choosing which inter-word link to recognize
triggers linking in the rest of the phrase along with it.
But perhaps reflecting on how we process the ambiguity \mdash{} realizing
that there are two competing parses and deciding which is the one
intended \mdash{} we picture the alternatives more as \q{horizontal}
options for connecting threads across the sentence,
more so than a \q{vertical} organization where
we hear \i{for the first time} as \q{contained} in a larger phrase.
My own feeling is of exploring competing relational patterns more than
exploring different ways that the phrases can be nested inside each
other.
}
\p{That being said, how much of our sense of ambiguity (or clarity, for
that matter) is driven by meaning, not form?  The \q{double parse}
just examined does not always generalize to similar cases:
\begin{sentenceList}\sentenceItem{} The Maple Leafs failed to win two consecutive games
for the first time this year.
\end{sentenceList}
The reading as in \q{this is the first time they failed to win two
consecutive games} makes no sense \mdash{} unless you've won every game,
but perhaps the first, you've at some point lost after a win.  Is this
case anomalous, where a syntactic ambiguity idiosyncratically fails to
yield logically plausible readings?
The ambiguity is found in \i{failed to make the playoffs
for the first time since 2013}, and many \i{for the first time this
season} cases, like \i{beat the Habs}, \i{sell out the arena},
\i{score a goal in the first period}.  But \q{failed to score a goal}
is almost surely read that they \i{did} score in every prior game.
Do we hear the construction as intrinsically ambiguous, and reject one
reading only when it is clearly flawed pragmatically?
}
\p{If we believe that language understanding unfolds in a predictable
operational sequence, then we should assume that both parses are
deemed plausible, and semantic considerations only retroactively
eschew one reading (if they do so at all).  This would explain why in
many cases the ambiguity persists enough to cast the practically
intended meaning in doubt.  But that account does not consider the
temporality of language itself; the hearer does not know in advance
that a trailing phrase like \i{for the first time this season} is coming,
and starts to make sense of the sentence up to there; once then hearing or
reading the addendum, the audience instinctively has to interpret the
final phrase as deliberately inserted to modify an already-complete
idea.  On this analysis, the addendum is initially approached as a
performative detail, something said for a reason to be determined \mdash{}
it is not structurally necessary to make the sentence well-formed.
Perhaps we then try to fit the last phrase into the sentence both
syntactically and semantically, together, triggered by a pragmatic
phenomenon (the speaker's choice to add on to a seemingly complete
thought) which then becomes logically prior to both syntax and
semantics.  If this is plausible, it supports an inter-word relational
model because we are forming a picture of language structure
relationally, assimilating new words and phrases to those already
heard by linkings referring back in time, rather than waiting until we
are sure we have a complete sentence and then treating it as a static
structure to vertically reconstruct.
}
\p{Of course, then, a host of further effects
are bound to morphological details once we have a complete sentence.
Case in point are plurals: for each plural usage we have a conceptual
transformation of an underlying singular to a collective, but
how that collective is pictured varies in context.  One
dimension of this variation lies with mass/count: the
mass-plural \i{coffee} (as in \q{some coffee}) figures the
plurality of coffee (as liquid, or maybe coffee grounds/beans)
in spatial and/or physical/dynamic terms.  So we have:
\begin{sentenceList}\sentenceItem{} There's coffee all over the table.
\sentenceItem{} She poured coffee from an ornate beaker.
\sentenceItem{} There's too much coffee in the grinder.
\sentenceItem{} There's a lot of coffee left in the pot
\mdash{} should I pour it out?
\end{sentenceList}
These sentences use phrases associated with plurality (\i{all over},
\i{a lot}, \i{too much}) but with referents that on perceptual and
operational grounds can be treated as singular \mdash{} as in the
appropriate pairing of \i{a lot} and \i{it} in the last sentence.
With count-plurals the collective is figured more as an
aggregate of discrete individuals:
\begin{sentenceList}\sentenceItem{} There are coffees all over the far wall at the expresso bar.
\sentenceItem{} She poured coffees from an ornate beaker.
\sentenceItem{} There are a lot of coffees left on the table
 \mdash{} shall I pour them out?
\end{sentenceList}
}
\p{So mass versus count \mdash{} the choice of which plural form to use \mdash{}
triggers an interpretation shaping how the plurality is pictured and
conceived (which is itself trigggered by the use of a plural to
begin with).  But if we restrict attention to just, say,
count-plurals, there are still different schemata for intending
collections:
\begin{sentenceList}\sentenceItem{} \label{itm:borough} New Yorkers live in one of five boroughs.
\sentenceItem{} New Yorkers reliably vote for Democratic presidential candidates.
\sentenceItem{} \label{itm:commute} New Yorkers constantly complain about how long it takes to commute
to New York City.
\end{sentenceList}
The first sentence is consistent with a reading applied to
\i{all} New Yorkers \mdash{} the five boroughs encompass the whole
extent of New York City.  The second sentence is only reasonable
when applied exclusively to the city's registered voters \mdash{} not all
residents \mdash{} and moreover there is no implication that the
claim applies to all such voters, only a proportion north of one-half.
And the final sentence, while perfectly reasonable, uses \q{New Yorkers}
to name a population completely distinct from the
first sentence \mdash{} only residents from the metro
area, but not the city itself, commute \i{to} the city.
}
\p{These examples demonstrate that we need
prelinguistic
background knowledge to understand what \i{sort} of plurality
the speaker intends.  To be sure, the more subtle plurals can
still be read in logical terms, and we can imagine sentences that
hew more crisply to a logical articulation:
\begin{sentenceList}\sentenceItem{} All New York City residents live in one of five boroughs.
\sentenceItem{} The majority of New York voters support Democratic
presidential candidates.
\sentenceItem{} Many New York metro area residents complain about how
long it takes to commute to New York City.
\end{sentenceList}
According to truth-theoretic semantics, sentences compel addressees to believe
(or at least consider) logically structured propositions \i{by virtue of}
linguistic shape replicating the architecture of the intended propositional
complexes, as these would be represented in first-order logic.  This
view on linguistic meaning is consistent with the last three sentences,
which are designed to map readily to logical notations (signaled by
quasimathematical phrases like \i{the majority of}).  But most sentences
do not betray their logical form so readily: these latter sentences actually
sound less fluent, more artificial, than their prior equivalents.
}
\p{It is also true that the more \q{logical} versions are more, we might say,
dialectically generalized because they do not assume the same level
of background knowledge.  Someone who knew little about
New York geography could probably make sense of the latter
sentences but might misinterpret, or at least have to consciously
think over, the former ones.  So we may grant that exceptionally
logically-constructed sentences can be clearer for a broad
audience, less subject to potential confusion, and indeed such
logically cautious language is a normal stylistic feature of
technical, legal, and journalistic discourse.  But for this reason
such discourse comes across as self-consciously removed from
day-to-day language.  As I argued from multiple angles earlier,
in the typical case \mdash{} i.e., stylistically
neutral, day-to-day language \mdash{} syntactic
composition does not neatly recapitulate logical form.
}
\p{My prior analysis demonstrated warrants for this idea by
highlighting narrative and imagistic aspects
of language used to convey ideas, like \q{come out against}
providing the verb-phrases in reports of people
criticizing something.  Here, examples like
(\ref{itm:borough})-(\ref{itm:commute}) point to a similar
conclusion, but from a more lexico-semantic orientation:
words like \i{borough} and \i{commute} carry a space
of logical details that tend to force logical
interpretations one way or another
(e.g., the detail that the territory of a city
is fully partitioned by boroughs, \i{so}, it
is \i{all} citizens who live in a borough).  This part of the
logic is however not reflected in sentence-structure; it is,
rather, latent
in lexical norms and assumed part of understanding
relevant sentences only because linguistic
competence is understood to include
familiarity with the logical
implications of the lexicalized concepts:
e.g. that the quantification in \q{New Yorkers
live in one of five boroughs} is \i{all},
but the quantification in \q{New Yorkers
vote democratic} is \i{most}.
}
\p{Here I'll also add the
following: the current examples show
how if addressees \i{have} the requisite background knowledge,
linguistic structure does not have to replicate logical structure very
closely to be understood.  The content which addressees undrstand may
have a logical form, and language evokes this form \mdash{} guides
addressees toward considering specific propositional content \mdash{} but
this does not happen because linguistic structure in any precise way mimics,
replicates, reconstructs, or is otherwise organized propositionally.  Instead,
the relation of language to predicate structures is evidently
oblique and indirect: language triggers interpretive processes
which guide us toward propositional content, but the structure
of language is shaped around fine-tuning the activation of
this background cognitive dynamics more than around any need to model
predicate organization architecturally.   In the case of plurals,
the appearance of plural forms like \i{New Yorkers} or \i{coffees}
compels us to find a reasonable cognitive model for the
signified multitude, and this model will have a logical form
\mdash{} but the linguistic structures themselves do not in general
model this form for us, except to the limited degree needed to
activate prelinguistic interpretive thought-processes.
}
\p{I make this point in terms of plural \i{forms}, and earlier made
similar claims in terms of lexical details.  A third group of triggers I
outlined involved morphosyntactic \i{agreement}, which establishes
inter-word connections which themselves trigger interpretive processing.
Continuing the topic of plurals, how words agree with other words
in singular or plural forms evokes schema which guide situational
interpretations.  So for instance:
\begin{sentenceList}\sentenceItem{} My favorite band gave a free concert last night.
They played some new songs.
\sentenceItem{} There was some pizza earlier, but it's all gone.
\sentenceItem{} There were some slices of pizza earlier, but it's all gone.
\sentenceItem{} There were some slices of toast earlier, but there's none left.
\sentenceItem{} There was some toast earlier, but they're all gone.
\sentenceItem{} That franchise had a core of talented young players, but it
got eroded by trades and free agency.
\sentenceItem{} That franchise had a cohort of talented players, but they
drifted away due to trades and free agency.
\sentenceItem{} Many star players were drafted by that franchise, but it
has not won a title in decades.
\sentenceItem{} Many star players were drafted by that franchise, but they
failed to surround them with enough depth.
\sentenceItem{} Many star players were drafted by that franchise, but they
were not surrounded with enough depth.
\sentenceItem{} Many star players were drafted by that franchise, but they
did not have enough depth (around them).
\end{sentenceList}
Plurality here is introduced not only by isolated morphology (like \i{slices},
\i{players}, \i{songs}), but via agreements marked by
word-forms in syntactically significant pairings: was/were, it/they,
there is/there are.  Framing all of these cases is how we can usually schematize
collections both plurally and singularly: the same set can be cognized as a
collection of discrete individuals one moment and as an integral whole the next.
This allows language some flexibility when designating plurals
(as extensively analyzed by Ronald Langacker: see his discussion of examples
like \i{Three times, students asked an interesting question}).  A sentence
discussing \i{slices of pizza} can schematically shift to treating
the pizza as a mass aggregate in \i{it's all gone}.  Here the antecedent
of \i{it} is \i{slices} (of pizza).  In the opposite direction, the mass-plural
\q{toast} can be refigured as a set of individual pieces in \i{they're all gone}.
The single \i{band} becomes the group of musicians in the band.
In short, how agreements are executed invites the addressee to reconstruct
the speaker's conceptualization of different referents discussed by
a sentence, at different parts of the sentence: linking
\i{it} to \i{slices} or \i{cohort}, or \i{they} to \i{the band} or
\i{the toast}, evokes a conceptual interpretation shaped in
part by how morphosyntactic agreement overlaps with \q{semantic}
agreement.  Matching \i{they} to \i{the band} presents agreement
in terms of how we conceive the aggregate (as a collection of
musicians); using \q{it} would also present an agreement, but
one schematizing other aspect of the band conecpt.
}
\p{In the last five above cases, \i{it} similarly binds (being singular) to
\i{the franchise} seen as a single unit \mdash{} here basic grammar and
conceptual schema coincide \mdash{} but \i{it} also binds
to the \i{core of young players}.  The
players on a team can be figured as a unit or a multiple.  The franchise itself
can be treated as a multiple (the various team executives and decision-makers),
as in \i{they failed to surround the stars with enough depth}.  The last sentence
is ambiguous between both readings: \q{they} could designate either the players
or the franchise.  Which reading we hear alters the sense of \q{have}: asserting
that the star \i{players} lack enough depth implies that they cannot execute
plays during the game as effectively as with better supporting players;
asserting that the \i{franchise} lacks depth makes the subtly different point
that there is not enough talent over all.  The variant which would include
\q{around them} nudges toward the second reading; but it is
still permissible to, according to speaker intent, parse the last
\i{they} as designating the \i{franchise} and the last \i{them} as
the \i{players} \mdash{} i.e., that final \i{they}/\i{them} pair
having different antecedents.
}
\p{The unifying theme across these cases is that when forming sentences we often
have a choice of how we figure plurality, and moreover these choices can be
expressed not only in individual word-forms but in patterns of agreement.
Choosing to pronominalize \i{slices of pizza} or \i{cohort of players}
as \i{it}, or alternatively \i{they}, draws attention to either the more
singular or more multitudinal aspects of the aggregate in question.
But this effect is not localized to the individual
\i{it}/\i{they} choice; it depends on
tracing the pronoun to its antecedent and construing
how the antecedent referent has both individuating and
multiplicity-like aspects.
Thus both individuation and plurality are latent in phrases like
\i{slices of} or \i{cohort of}, and this singular/plural co-conceiving
is antecedently figured by how subsequent morphosyntax agrees
with the singular or, alternatively, the plural.
}
\p{Moreover, these patterns of agreement invoke new layers of
interpretation to identify the proper conceptual scope of plurals.
In \i{The band planned a tour, where they debuted new songs} we hear
the scope of \q{they} as narrower than its antecedent \q{the band},
because only the band's \i{musicians} (not stage crew, managers, etc.)
typically actually perform:
\begin{sentenceList}\sentenceItem{} The band planned a tour, where
they debuted new songs.
\sentenceItem{} \label{itm:teamflew} The team flew to
New York and they played the Yankees.
\sentenceItem{} \label{itm:theflies} The city's largest theater company
will perform \q{The Flies}.
\end{sentenceList}
Likewise in (\ref{itm:teamflew}), only the athletes are referenced via
\i{they played}; but presumably many other people (trainers, coaches, staff)
are encompassd by \q{the team flew}.  And in (\ref{itm:theflies})
we do not imply that the Board of Directors will
actually take the stage (the President
as Zeus, say).  Even in the course of one sentence, plurals
are reinterpreted and redirected:
\begin{sentenceList}\sentenceItem{} The city's largest theater company
performed \q{The Flies} in French, but everyone's accent
sounded Quebecois.
\sentenceItem{} The city's largest theater company
performed \q{The Flies}; then they invited a professor
to discuss Sartre's philosophy when the play was over.
\end{sentenceList}
In the first sentence, the \q{space} built by the sentence is wider
initially but narrows to encompass only the actual actors on stage.
In the second, the \q{space} narrows in a different direction, since
we hear a programming decision like pairing a performance with a
lecture as made by a theater's administrators rather than its actors.
I discussed similar modulation in conceptual schemas related to plurality
and pluralization earlier; what is distinct in these last examples is how
the interpretive processes for cognizing plurality are shaped by
agreement-patterns (like \i{it} or \i{they} to a
composite antecedent) as
much as by lexical choice, or morphology, in isolation.
}
\p{I have accordingly outlined a theory where lexical, morphological, and
morphosyntactic layers all introduce \q{triggers} for cognitive processes,
and it is these procsses which (via substantially prelinguistic perception
and conceptualization) ultimately deliver linguistic meaning.  What is
\i{linguistic} about these phenomena is how specifically linguistic
formations \mdash{} word choice, word forms, inter-word agreements in form \mdash{}
trigger these (in no small measure pre- or extra-linguistic) interpretations.
But as I suggested this account is only preliminary to analysis of
how multiple interpretive processes are \i{integrated}.  Linguistic
\i{structure} contributes the arrangements through which the
crossing and intersecting between interpretive
\q{scripts} are orchestrated.  Hence at the higher linguistic scales and
levels of complexity, the substance of linguistic research, on this view,
should gravitate toward structural intgration of interpretive
processes, even more than individual interpretive triggers themslves.
}
\p{}
