\section{Truth-Theoretic Semantics and Enaction}
\p{I will start by reviewing illocutionary pragmatics, to 
identify some of the contextual and interpretive 
transformations that pertain to mapping surface language 
to propositional contents.  My point is to establish 
what should be a common theory of logicality that can be
shared by both critics and defenders of \q{truth-theoretic} 
paradigms, on which basis their legitimate disputes 
can be investigated.
}
\p{Many linguists (on both sides, I would say, of my 
central truth-theoretic pro/con), 
seem to analyze hedges like \q{could you please}
as merely dressing over crude commands: we don't
want to come across as giving people orders, but
sometimes we do intend to ask pople to do specific
things.  As a result, we feel obliged to couch the
request in conversational gestures that signal
our awareness of how bald commands may lie outside
the conversational norms.  These ritualistic
\q{could you please}-like gestures may have
metalinguistic content, but \mdash{} so the theory
goes \mdash{} they do not \i{semantically} alter
the speech-act's directive nature.
}
\p{The problem with this analysis is that sometimes
directive and \q{inquisitive} dimensions can
overlap:
\begin{sentenceList}\sentenceItem{} \label{itm:almond} Do you have almond milk?
\sentenceItem{} Can you get MsNBC on your TV?
\sentenceItem{} \label{itm:needcorkscrew} This isn't a screw-cap bottle: I need a corkscrew.
\end{sentenceList}
These \i{can} be read as bare directives, and would
be interpreted as such if the hearer believed the
speaker already knew that yes, he has almond milk, and yes,
he gets MsNBC.  In (\ref{itm:needcorkscrew}), if both parties
know there's one corkscrew in the house,
the statement implies a directive to fetch \i{that} corkscrew.
But, equally, (\ref{itm:almond})-(\ref{itm:needcorkscrew}) can \i{also} be read as bare
questions with no implicature: say, as fans of
almond milk and MsNBC endorsing those selections,
or pointing out that opening the bottle
will need \i{some} corkscrew.
And, meanwhile, (\ref{itm:almond})-(\ref{itm:needcorkscrew})
can \i{also} be read as a mixture of the
two, as if people expressed themselves like this:
\begin{sentenceList}\sentenceItem{} I think the window is open, can you close it?
\sentenceItem{} I see you have almond milk, can I have some?
\sentenceItem{} If you get MsNBC, can you turn on Rachel Maddow?
\sentenceItem{} If there is a corkscrew in the house, can you get it?
\end{sentenceList}
}
\p{I think the mixed case is the most prototypical, and pure
directives or inquiries should be treated as degenerate
structures where either directive or inquisitive content
has dropped out.  After all, even a dictatorial
command includes the implicit assumption that the order
both makes sense and is not impossible.  On
the other hand, we don't ask questions for no
reason: \q{do you have almond milk} may be a
suggestion rather than a request, but it still
carries an implicature (e.g., that the addressee
\i{should} get almond milk).
}
\p{Ordinary requests carry the assumption that addressees
can follow through without undue inconvenience,
which includes a package of assumptions about both
what is currently the case and what is possible.
\q{Close the window} only has literal force if the
window is open.  So, when making a request, speakers
have to signal that they recognize the request involves
certain assumptions and are rational enough to
accept modifications of these assumptions in
lieu of literal compliance.  This is why
interrogative forms like \q{can you} or
\q{could you} are both semantically nontrivial
and metadiscursively polite: they leave open the
possibility of subsequent discourse framing the original
request just as a belief-assertion.  Developments
like \i{can you open the window} \mdash{} \i{no, it's closed}
are not ruled out.  At the same time, interrogative forms
connote that the speaker assumes the addressees can
fulfill the request without great effort: an implicit
assumption is that they \i{can} and also \i{are
willing to} satisfy the directive.  This is an
assumption, not a presumption: the speaker
would seem like a bully if he acted as if he
gave no thought to his demands being too much
of an imposition \mdash{} as if he were taking
the answer to \q{can you} quesions for granted.
This is another reason why requests
should be framed as questions.  So, in short,
\q{commands} are framed as questions because the speaker
literally does not know for sure whether the command is
possible; given this uncertainty a command \i{is} a question,
and the interrogative form is not just a non-semantic
exercise in politesse.
}
\p{Sometimes the link between directives and
belief assertions is made explicit.  A common
pattern is to use \i{I believe} or \i{I believe that} as an
implicature analogous to interrogatives:
\begin{sentenceList}\sentenceItem{} I believe you have a reservation for Jones?
\sentenceItem{} I believe this is the customer service desk?
\sentenceItem{} I believe we ordered a second basket of garlic bread?
\sentenceItem{} I believe you can help me find computer
acessories in this section?
\end{sentenceList}
These speakers are indirectly signaling what they want
someone to do by openly stating the requisite
assumptions \mdash{} \i{I believe you can} in place
of \i{can you?}.  The implication is that
such assumptions translate clearly to a
subsequent course of action \mdash{} the guest who
\i{does} have that reservation should be checked in;
the cashier who \i{can} help a customer find
accessories should do so.  But underlying these
performances is recognition that
illocutionary force is tied to background
assumptions, and conversants are reacting to
the propositional content of those assumptions
as well as the force itself.  If I \i{do} close the
window I am not only fulfilling
the request but also confirming that the window
\i{could} be closed (a piece of information
that may become relevant in the future).
}
\p{In sum, when we engage pragmatically with other
language-users, we tend to do so cooperatively,
sensitive to what they wish to achieve with
language as well as to the propositional
details of their discourse.  But this often means
that I have to interpret propositional
content in light of contexts and implicatures.
Note that both of these are possible:
\begin{sentenceList}\sentenceItem{} Do you have any milk?
\sentenceItem{} Yes, we have almond milk.
\sentenceItem{} No, we have almond milk.
\end{sentenceList}
A request for milk in a vegan restaurant could plausibly be
interpreted as a request for a vegan milk-substitute.
So the concept \i{milk} in that context may actually be
interpreted as the concept \i{vegan milk}.  
Responding to the force of speech-acts
compels me to treat them as not \i{wholly}
illocutionary \mdash{} they are in part statements of
belief (like ordinary assertions).  One reason I need
to adopt an epistemic (and not just obligatory)
attitude to illocutionary acts is that I need to
clarify what meanings the speaker intends, which
depends on what roles she is assigning to
constituent concepts.
}
\p{Suppose my friend says this, before and after:
\begin{sentenceList}\sentenceItem{} \label{itm:put} Can you put some almond milk in my coffee?
\sentenceItem{} \label{itm:after} Is there milk in this coffee?
\end{sentenceList}
Between (\ref{itm:put}) and (\ref{itm:after}) I do put almond milk
in his coffee and affirm \q{yes} to (\ref{itm:after}).  I feel it
proper to read (\ref{itm:after})'s \q{milk} as really meaning
\q{almond milk}, in light of (\ref{itm:put}).  Actually
I should be \i{less} inclined to say \q{yes}
if (maybe as a prank) someone had instead
put real (cow) milk in the coffee.  In responding
to his question I mentally substitute what
he almost certainly \i{meant} for how
(taken out of context) (\ref{itm:after}) would usually
be interpreted.  In this current
dialog, the \i{milk} concept not only
includes vegan milks, apparently, but
\i{excludes} actual milk.
}
\p{It seems \mdash{} on the evidence of cases like this one \mdash{}
as if when we are dealing with
illocutionary force we are obliged to subject
what we hear to extra interpretation, rather
than resting only within \q{literal} meanings
of sentences, conventionally understood.
This point is worth emphasizing because it complicates
our attempts to link illocution with propositional
content.  Suppose grandma asks us to close the
kitchen window.  Each of these are plausible and
basically polite responses:
\begin{sentenceList}\sentenceItem{} It's not open, but there's still some
cold air coming through the cracks.
\sentenceItem{} It's not open, but I closed the window in
the bedroom.
\sentenceItem{} I can't \mdash{} it's stuck.
\end{sentenceList}
In each case I have not fulfilled her request \visavis{}
its literal meaning, but I \i{have} acted benevolently
in terms of conversational maxims.  
}
\p{Part of reading propositional content is
syncing our conceptual schemas with our fellow
conversants.  The illocutionary
dimension of a request like \i{can I have some milk?}
makes this interpretation especially important,
because the addressee wants to make a good-faith effort
to cooperate with the pragmatic intent of the
spech-act.  But cooperation requires the
cooperating parties' conceptual schemas to
be properly aligned.  I therefore have to
suspend the illocutionary force of a directive
temporarily and treat it as locutionary
statement of belief, interpret its apparent
conceptual underpinnings in that mode, and
then add the illocutionary force back in: if I
brought \i{real} milk to a vegan customer who
asked for \q{milk} I would be \i{un}-cooprative.
}
\p{The upshot is that conversational implicatures
help us contextualize the conceptual negotiations
that guarantee our grasping the correct
propositional contents, and vice-versa.  This means
that propositionality is woven throughout both
assertive and all other modes of language, but it
also means that propositional content can be
indecipherable without a detailed picture
of the current context (including illocutionary
content).  The proposional content of,
say, \i{there is milk in this coffee} has to be
judged sensitive to contexts like \i{milk}
meaning \i{vegan milk} \mdash{} and this
propagates from a direct propositional
to any propositional attitudes which may
be directed towards it, including requsts like
\i{please put milk in this coffee}.
}
\p{Suppose the grandkids close grandma's bedroom window
when she asks them to close the kitchen window.
The propositional content at the core of grandma's
request is that the kitchen window be closed; the
content attached to it is an unstated belief that
this window is open.  Thus, the truth-conditions
satisfying her implicit understanding would be
that the kitchen window went from being open to being
closed.  As it happens, that window is already
closed.  So the truth-conditions that would satisfy
grandma's initial belief-state do not obtain \mdash{} her
beliefs are false \mdash{} but the truth conditions satisfying
her desired result \i{do} obtain.  The window
\i{is} closed.  Yet the grandkids should not thereby
assume that her request has been properly responded to;
it is more polite to guess at the motivation behind
the request, e.g., thay she felt a draft
of cold air.  In short, they should look outside
the truth conditions of her original
request taken literally, and \i{interpret}
her request, finding different content
with different truth-conditions that are both consistent
with fact and address whatever pragmatic goals
grandma had when making her request.  They might
infer her goal is to prevent an uncomfortable
draft, and so a reasonable \q{substitute content} is
the proposition that \i{some} window is open,
and they should close \i{that} one.
}
\p{So the grandkids should reason as if translating
between these two implied meanings:
\begin{sentenceList}\sentenceItem{} I believe the kitchen window
is open \mdash{} please close it!
\sentenceItem{} I believe some window
is open \mdash{} please close it!
\end{sentenceList}
They have to revise the simplest reading of
the implicit propositional content of grandma's
\i{actual} request, because the actual request is
inconsistent with pertinent facts.  In short, they
feel obliged to explore propositional alternatives
so as to find an alternative, implicit request whose
propositional content \i{is} consistent with
fact and also meets the original request's illocutionary
force cooperatively.
}
\p{In essence, we need to express a requester's desire as
itslf, in its totality, a specific propositional content,
thinking to ourselves (or even saying to others) things
like
\begin{sentenceList}\sentenceItem{} Grandma wants us to close the window.
\sentenceItem{} He wants a bottle opener.
\end{sentenceList}
But to respond politely we need to modify
the parse of their requests to capture the
\q{essential} content:
\begin{sentenceList}\sentenceItem{} Grandma wants us to eliminate the cold draft.
\sentenceItem{} He wants something to open that bottle.
\end{sentenceList}
We have to read outside the literal interpretation
of what they are saying.  This re-reading is something
that may be appropriate to do with respect to
other forms of speech also; 
but our conversational responsibilty to infer
some unstated content is especially pronounced
when we are rsponding to an explicit
request for something.
}
\p{Certainly, in many cases, meanings are not literal.
But how then do we understand what people are saying?
Trying to formulate a not-entirely-haphazard
account of this process, we can speculate
that interpreting what someone is \q{really} saying
involves systematically mapping their apparent
concepts and references to some superimposed
inventory designed to mitigate false beliefs or
conceptual misalignments among language users in some
context.  That means, we are looking for mappings
like \i{milk} to \i{almond milk} in (\ref{itm:can}) from a
vegan restaurant, or \i{kitchen window} to
\i{bedroom window} in (\ref{itm:close}) if it is the latter
that is open:
\begin{sentenceList}\sentenceItem{} \label{itm:can} Can I have some milk?
\sentenceItem{} \label{itm:close} Can you close the kitchen window?
\end{sentenceList}
The point of these \q{mappings} is that they
preserve the possibility of
modeling the \i{original} propositional content
by identifying truth conditions
for that content to be satisfied.
}
\p{A \i{literal} truth-condition model doesn't work in
cases like (\ref{itm:can}) and (\ref{itm:close}): the diner's request
is \i{not} satisfied if it is the case
that there is now (real) milk in her coffee; and
grandma's request is not necessarily satisfied if it is
the case that the kitchen window is closed.  The
proposition \q{the kitchen window is closed} only bears on
grandma's utterance insofar as she believes that
this window is open and causing a draft.  So if we want
to interpret the underlying locutionary content
of (\ref{itm:can}) and (\ref{itm:close}) truth-theoreticaly, we need to
map the literal concepts appearing
in these sentences to an appropriate translation,
a kind of \q{coordinate transformation} that
can map concepts onto others, like milk/almond milk
and kitchen window/bedroom window.
}
\p{In sum, a theory of sentences' logical nexus can only 
be complete with some model of discursive context 
\i{structured in such a way} that we can repesent the 
interpretations and concept-transforms internal to 
parsing sentences to their propositional core.  
I will now consider what such a \q{theory of context} 
might look like.
}
