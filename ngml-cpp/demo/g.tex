\documentclass[oneside,11pt]{book}
\usepackage{fancyhdr,geometry,keyval,ifthen,makeidx,multicol,mparhack,poemscol}
\usepackage{times}

\newenvironment{minipagestanza}{\begin{stanza}\begin{minipage}{13cm}}{\end{minipage}\end{stanza}}

%\usepackage{caption}
\usepackage{enumitem}

\usepackage[rgb]{xcolor}

\definecolor{bmainColor}{rgb}{.7,.2,.3}
%\colorlet{LightBlue}{BlueLUH!20!white}
\colorlet{mainColor}{bmainColor!50!black}

\definecolor{blineColor}{rgb}{.2,.7,.7}
%\colorlet{LightBlue}{BlueLUH!20!white}
\colorlet{lineColor}{blineColor!80!white}


\colorlet{bheadColor}{blineColor!80!blue}

\colorlet{headColor}{bheadColor!30!black}


\usepackage{eso-pic}
\usepackage[nolinks]{qrcode}

\colorlet{br}{blue!30!red}	

\AddToShipoutPicture{%
	\AtPageLowerLeft{%
		\raisebox{3cm}{\hypersetup{
				colorlinks=false}
			\hspace{16cm}{\textcolor{br!40!black}{\qrcode[height=2cm]{MichaelLesher2019-\thepage}}
				%\includegraphics[width=5cm]{example-image}
			}}}}
			
\newlength{\footnoterulelen}
\setlength{\footnoterulelen}{\textwidth}
\addtolength{\footnoterulelen}{-4pt}


\renewcommand{\footnoterule}{%
  \kern 5pt
  {\color{blineColor!60!black}{\hrule width \footnoterulelen height 2pt}}
  \kern 4pt
}

\newcommand{\xepigraph}[1]{\hspace*{3.5em}\begin{minipage}{13cm}\epigraph{#1}\vspace*{1em}\end{minipage}}
\newcommand{\xfepigraph}[1]{\hspace*{3.5em}\begin{minipage}{13cm}\epigraph{#1}\vspace*{-.5em}\end{minipage}}

\newcommand{\ardagbotrot}{{\rotatebox{-90}{$\blacktriangleright$}}}

\newcommand{\lowardag}{{\raisebox{-3pt}{\makebox[.5pt][l]{\ardagbotrot}}\dag}}

\newcommand{\ardag}{{\raisebox{3pt}{\lowardag}}}

\let\oldciterange\citerange
\renewcommand{\citerange}{line \oldciterange}

\usepackage{amssymb}
\usepackage{graphicx}

\newcounter{xmargrefnumber}
\setcounter{xmargrefnumber}{1}
\newtoks{\xmargrefmarker}
\xmargrefmarker={\ardag}
\newcommand{\setxmargrefmarker}[1]{\xmargrefmarker={#1}}
\newcommand{\makexmargreflabel}{xmargref}
\newif{\ifxmargrefstomargin}
\xmargrefstomargintrue
\newcommand{\setxmargref}[1]{%
\marginpar{\ifxmargrefstomargin\hspace*{.5\leftmargin}\fi\scriptsize {\hyperlink{#1}{\ardag}}}}

\usepackage{calc}

\newlength{\rulelen}
\newlength{\negrulelen}

\newcommand{\xmargreftextnote}[1]{%
\iftextnoteson\addtocounter{xmargrefnumber}{1}%
\immediate\write\textnotes{\makexmargreflabel}%
\setxmargref{#1}\else\relax\fi%
}

\newlength{\xtextnotelen}
\setlength{\xtextnotelen}{\textwidth}
%\addtolength{\xtextnotelen}{-4em}


\newcommand{\xtextnote}[2]{%
\textnote{\hypertarget{#1}%
\protect\raisebox{-6pt}{\protect\parbox{\xtextnotelen}{{\color{lineColor}{%
\setlength{\rulelen}{\footnoterulelen}%
\setlength{\negrulelen}{\widthof{\citerange:}}%
\addtolength{\negrulelen}{1.5pt}%
\addtolength{\rulelen}{-\negrulelen}%	
\rule{\rulelen}{0.1pt}}}\\#2}}
}}


\newlength{\dxtextnotelen}
\setlength{\dxtextnotelen}{\textwidth}
\addtolength{\dxtextnotelen}{-3em}
\addtolength{\dxtextnotelen}{-3pt}

\newcommand{\dxtextnote}[2]{%
	\textnote{\hypertarget{#1}%
		\protect\raisebox{-12pt}{\protect\parbox{\dxtextnotelen}{{\color{lineColor}{%
						\setlength{\rulelen}{\footnoterulelen}
						\setlength{\negrulelen}{\widthof{\citerange:}}
						\addtolength{\negrulelen}{2.5pt}
						\addtolength{\rulelen}{-\negrulelen}
						\rule{\rulelen}{0.1pt}}}\\#2}}
	}}


\usepackage[raiselinks=true,pdftex]{hyperref}
%\hypersetup{linktocpage}

\usepackage{wasysym}

\geometry{textwidth=6in,textheight=7.8in,headsep=32pt}
\makeverselinenumbers
\global\includeaccidentalstrue
\global\includetypescriptstrue
\global\redundantemendationsfalse


\newcommand{\lX}{\Large{X}}

\stanzaatbottom{$\curlyeqprec$end stanza$\curlyeqsucc$}
\nostanzaatbottom{$\preccurlyeq$stanza continues$\succcurlyeq$}

%\nostanzaatbottom{\relax}

\reversemarginpar

\usepackage{needspace}

\newcommand{\cq}[1]{{``}#1{''}}
\newcommand{\q}[1]{{``}#1{''}}


\newcommand{\poemnewpage}{\newpage}

\newenvironment{prestanza}{\needspace{3\baselineskip}\hspace{-1em}}{}

\newenvironment{itprestanza}{\begin{prestanza}\begin{itshape}}{\end{itshape}\end{prestanza}}

\newcommand{\rar}{{\raisebox{-2pt}{{\Large$\Rightarrow$}}}}

\newcommand{\poemfl}[2]{\item[#2] %
 \hspace*{1.5em} \raisebox{2pt}{\hyperlink{#1}{\rar}} \hfill \getpagerefnumber{#1} %
 \hspace*{5em} }

\renewcommand{\contentspoemtitlefont}{\fontsize{11}{14}\fontseries{b}\selectfont}

\makeatletter
\newcommand{\basicpoemtitlear}[3]{%
	\set@p@emtitle{#3}{\poemtitlepenalty}{\poemtitlefont}%
	{\relax}{\nobreak\par\nobreak\afterpoemtitleskip\nobreak}{\z@}%
	\c@ntentsinfo{#2\hspace{1.5em}\hyperlink{#1}{\rar}}%
	{\contentsindentone}{\contentspoemtitlefont}%
	{\contentsindenttwoamount}%
	\t@xtnotesinfo{#2}%
}
\makeatother

\makeatletter
\renewcommand*{\setcounterfrompageref}[2]{%
	\ifrefundefined{#2}{%
		\protect\G@refundefinedtrue
		\@latex@warning{Reference `#2' on page \thepage \space
			undefined}%
		\setcounter{#1}{5}%
	}{%
	\setcounter{#1}{5}%
}%
}
\makeatother

%\definecolor{bmainColor}{rgb}{.2,.7,.3}
%\colorlet{mainColor}{bmainColor!50!black}

%	colorlinks,

\hypersetup{
	colorlinks,
	linkcolor={mainColor},
	urlcolor=black,
pdfstartpage={1},
pdfstartview={FitH \hypercalcbp{\paperheight-\topmargin-1in}},
pdfview={FitH 0}
}

%	citecolor={blue!50!black},
%	urlcolor={red!80!black},

\newcommand{\poemtitlear}[3]{\hypertarget{#1}{\label{#1}}\basicpoemtitlear{#1}{#2}{#3}}


\newcommand{\mypageref}[1]{\hfill\getpagerefnumber{#1}\hspace*{6em}}

\newcommand{\paraspace}{

\vspace{9pt}}

\let\oldpmpara\pmpara
\renewcommand{\pmpara}{\paraspace\oldpmpara}


\usepackage[ruled,para*]{manyfoot}
\textnotesatfoot

\usepackage{tcolorbox}

\newcommand{\xvolumeheadervalue}{\parbox{\textwidth}{%
\begin{tcolorbox}[colframe=headColor,
                  boxsep=4pt,
                  left=0pt,
                  right=0pt,arc=1pt,outer arc=1pt
                  top=0pt,colback=red!5!white,
        boxrule=0.5pt,bottomrule=0pt,rightrule=2pt
                  ]%%
  \hfill\volumeheadervalue
\end{tcolorbox}}
\vspace{28pt}
}
	
	
\usepackage{mdframed}

\definecolor{logoGreen}{RGB}{85, 153, 89}
\definecolor{logoCyan}{RGB}{66, 206, 244}

\newenvironment{ccframed}{\begin{mdframed}[backgroundcolor=logoGreen!5,linecolor=logoCyan!50!black,linewidth=3pt,topline=false]}{\end{mdframed}}

\colorlet{ob}{orange!30!gray}	

\newenvironment{cctframed}
{\begin{mdframed}[backgroundcolor=logoGreen!5,linecolor=logoCyan!50!ob,linewidth=8pt]}{\end{mdframed}}


\newenvironment{cframed}{\begin{mdframed}[linecolor=ob,
leftline=false,
  rightline=false,
linewidth=2pt]}{\end{mdframed}}
	
\colorlet{oy}{orange!80!yellow}	
\colorlet{oyr}{oy!40!red}	
\colorlet{bluegr}{blue!80!green}	


\setlength{\fboxsep}{1.5pt}

\usepackage{microtype}

\newcommand{\xtheverselinenumber}{\protect\raisebox{-1pt}{\protect\parbox{1cm}{
\begin{tcolorbox}[colframe=oyr!50!black,
	boxsep=-4pt,
	colback=white,
	hbox,
	sharp corners,
	boxrule=1pt,rightrule=0pt,
	toprule=0pt,leftrule=2pt,
	]%%
{\centering{\color{bluegr!35!black}{\textbf{\textls*[120]{\theverselinenumber}}}}}
\end{tcolorbox}	
}}}

\newcommand{\vrsindent}{\hspace{3em}}
	
\newcommand{\divider}{\vspace{.5em}{\vrsindent}***\vspace{.5em}}

\usepackage{navigator}
\embeddedfile{txh}{gaza.txh}


%\usepackage{makeidx}
%\makeindex

\begin{document}
\maketextnotes

%\indexingontrue

%\makepoemindex

%\makeemendations
%\makeexplanatorynotes
\resetpagestyle
\normalsize
\raggedbottom

%\leftheader{{KADDISH FOR GAZA}}
%\rightheader{{KADDISH FOR GAZA}}
%\protect\color{blue}

\setcontentsleaders{ }

	
%\setcounter{lineindexrepeat}{4}

%\setcounter{page}{}

\begin{volumetitlepage}
\volumetitle{KADDISH FOR GAZA}
\volumeheader{KADDISH FOR GAZA}
\volumesubtitle{by Michael Lesher}
\volumededication{There is a relationship between war and words; there is a relationship between love and war.\\
\textnormal{\hspace*{2.8em}--- Yehia Jaber}}

\vfill
\hfill\begin{minipage}{.5\textwidth}
\begin{cctframed}
Michael Lesher\\
285 Aycrigg Avenue, Apt. 21C\\
Passaic NJ 07055\\
(862) 290-7806\\
michaellesher@optonline.net
\end{cctframed}
\end{minipage}\hfill


\end{volumetitlepage}


\putpoemcontents
\pagenumbering{arabic}%
\makepoemcontents

%\putpoemindex

%\setcounter{prosepage}{4}

%\prosesectiontitle{Introduction}
\prosesectiontitlenotitle{Introduction}
{\begin{center}{INTRODUCTION\\
A Few Words about Writing and Resistance}\end{center}}



\begin{pmsection}
{\linespread{1.2}\selectfont\parindent32pt 
{\pmpara}
I have written the poems that follow with tormented thoughts of Gaza. And if anyone asks why I, an American Jew, should have written so much about a place I have never been, my short answer is that I write about Gaza because I cannot write any other way. For me, to think of Gaza is to touch the drama of human resistance: not only resistance to oppression but resistance to the most insidious powers of intellectual and spiritual destruction, the lying and the sophistry that modern communication technology has honed to something almost ineffably sinister. In this respect all human beings are equally vulnerable. The dehumanizing of Gaza by way of propaganda has its counterpart in the deafening blandishments of modern media, replete with militarism and consumerism: both aim to corrupt the moral core of their hearers, and neither gives any quarter.
{\pmpara}
If language is the most distinctively human capability, it's easy to understand why the contamination of language must accompany the most inhuman actions of governments and their corporate masters. And the reverse holds true as well. With humanity under such fundamental threat, poetry, the most intense form of language, is itself an act of resistance --- in fact, it is a means of human survival, the ultimate index of freedom.

{\pmpara}
How well these poems live up to such standards is for others to decide. But I don't think anyone can question the urgency of the threats we face as human beings in a world of propaganda. And it seems obvious to me that there is, and must be, an indelible bond between each individual's battle to retain his or her humanity and the battle to secure a place like Gaza from the assaults that target every aspect of its people, from its children’s lives to the visibility of their faces on American television. If Israeli PR experts can get away with speaking as if nothing but \q{terrorists} existed in Gaza; if the Hillary Clintons of the world can get away with blaming Gaza's government for its death toll at the very moment Israeli shells are obliterating apartment buildings, mosques, schools, hospitals and U.N. shelters; if a Nobel Peace laureate like Elie Wiesel can get away with castigating Gazans for \q{child sacrifice} while an illegal occupying power kills hundreds of their little ones; then something irretrievable will have been lost.

{\pmpara}
What can we do to prevent it? That depends a good deal on who we are, and where we either have or can cultivate the kind of power that can bring justice. But as the great singer/songwriter and activist Phil Ochs said, during the Vietnam carnage, \q{in such an ugly time the true protest is beauty.} And true beauty, one might add, is protest.

{\paraspace}
{\noindent}Michael Lesher
}\end{pmsection}


%\setcounter{page}{1}	
{\poemnewpage}
%\hypertarget{ctg:b}{}
\poemtitlear{ctg:Thinking}{Thinking of Gaza from Thousands of Miles Away}{THINKING OF GAZA FROM THOUSANDS OF MILES AWAY}
%poemtitlear{ctg:}{X}{THINKING OF GAZA FROM THOUSANDS OF MILES AWAY}

%\poemonlycontents{Thinking of Gaza FROM THOUSANDS OF MILES AWAY}

\begin{poem}

\begin{stanza}
If I were to be born again,\verseline
it might be as the white moth\verseline
whose slow, flexing heartbeat\verseline
of wings
\end{stanza}

\begin{stanza}
declares its pittance\verseline
in a vast anonymity of snow...
\end{stanza}

\begin{stanza}
It might be as the spore\verseline
borne on mimosa tendrils\verseline
in an uncertain breeze,
\end{stanza}

\begin{stanza}
alone in a tiny pivot of air,\verseline
all the earth a mystery hovering below.
\end{stanza}

\begin{stanza}
For I am my place, and\verseline
I have nowhere to go,
\end{stanza}

\begin{stanza}
and all hearts are my heart,\verseline
and none knows me,\verseline
every breath shakes my world\verseline
though not a syllable is mine ---
\end{stanza}

\begin{stanza}
not a glance my glance, yet in every one\verseline
I disappear behind its silence...
\end{stanza}

\begin{stanza}
And where a petal drops onto the fixed eyes\verseline
of the child whose bracts are\verseline
already in earth,\verseline
whose face\verseline
is cold with death, her eyes
\end{stanza}

\begin{stanza}
blue and blank\verseline
as the flower that was ---
\end{stanza}

\begin{stanza}
where,\verseline
at dawn, a boy rises from another tear\verseline
to test the blue air\verseline
left behind by the bomb,
\end{stanza}

\begin{stanza}
and cannot find any path\verseline
to a door, or womb, or nightmare\verseline
clearer than smoke\verseline
or louder than a shroud,
\end{stanza}

\begin{stanza}
and not even the first word\verseline
of mourning can be said ---
\end{stanza}

\begin{stanza}
I also rise;
\end{stanza}

\begin{stanza}
That is me.
\end{stanza}

\begin{stanza}
Because I am not dead.\verseline
Because I am not there.
\end{stanza}

\begin{stanza}
Because I cannot breathe\verseline
the air motionless forever\verseline
in the child's breast,
\end{stanza}

\begin{stanza}
and cannot touch the sky\verseline
that is all that's left\verseline
to the last boy's famished eyes.
\end{stanza}

\begin{stanza}
Because I wander that sky unseen,\verseline
never to touch their earth.
\end{stanza}

\begin{stanza}
And because,\verseline
whatever I touch,\verseline
it is their faces I will feel,\verseline
their silence my breath will trace.
\end{stanza}


\end{poem}


{\poemnewpage}
\poemtitlear{ctg:Burlesque}{Burlesque at the Barrier}{BURLESQUE AT THE BARRIER}
\begin{poem}



\begin{stanza}
What do you write on a wall?\xmargreftextnote{bot:write}\verseline%
\xtextnote{bot:write}{Alludes to the wall surrounding and partitioning Gaza from Israel.}
What can you smear that will fall\verseline
upwards of your fist, and sting\verseline
sharper than a tear?
\end{stanza}

\begin{stanza}
What can you draw\verseline
that will laugh\verseline
away the soldier you think\verseline
of (if you think), whose calf\verseline
bears up the sky's iron dome,\verseline
whose finger kills for a mile\verseline
but whose heart's not at home ---
\end{stanza}

\begin{stanza}
what could wipe the smile\verseline
off that misshapen maw?
\end{stanza}

\begin{stanza}
Another length, another place ---\verseline
the damn thing's all the same.\verseline
Besides, you can't get close.
\end{stanza}

\begin{stanza}
They've fixed the game,\verseline
but that doesn't stop the eye\verseline
from hating its way into the cracks\verseline
of the concrete. If there were bricks\verseline
to throw, you'd come up with something\verseline
to say.
\end{stanza}

\begin{stanza}
Or, once you got by it,\verseline
you'd stand on the sand\verseline
and pee into the air ---\verseline
let them do something\verseline
to you then, if they dared.
\end{stanza}

\begin{stanza}
But the dome shifts slowly overhead\verseline
as always, and the prison wall is everywhere.\verseline
What can you piss\verseline
that will tell it like it is?
\end{stanza}

\begin{stanza}
Damn it all, I've been seeing it in my dreams.\verseline
A little bit\verseline
of imagination, and there's a fetor all through me\verseline
that giggles, then screams...
\end{stanza}

\begin{stanza}
Atop the wall's gray brink\verseline
I've been longing to write\verseline
something fiercely ugly, wordless as a smell,\verseline
something that would gloat\verseline
there when I'm gone\verseline
like a slow death, with pulse\verseline
to match, bleeding all over the light.
\end{stanza}

\begin{stanza}
A stain impossible to dispel\verseline
that would stink\verseline
through the night...
\end{stanza}

\begin{stanza}
I reach into my pants\verseline
to see what can be done.\verseline
But a pimply-faced guard quickly raises his gun\verseline
and I lose my chance;\verseline
I gather he thought I meant something else.
\end{stanza}

\end{poem}


{\poemnewpage}
\poemtitlear{ctg:Drinking}{Drinking It}{DRINKING IT}
\poemdedication{(for Amira Hass)}
\begin{poem}


\begin{stanza}
The sea is lover to the sand,\verseline
the sand's the refuse of a stagnant day.\verseline
All earth's a captive of the sky.
\end{stanza}

\begin{stanza}
You dip your hand into the waves\verseline
to feel the slow, eternal surge\verseline
that drives the water on. Unaware,
\end{stanza}

\begin{stanza}
it breaks against the shore to sigh,\verseline
caress, fall back unsatisfied\verseline
all day between the beaches' thighs.
\end{stanza}

\begin{stanza}
This beach is empty as the heart is wide.\verseline
Will anyone come to taste its plight\verseline
under a pitiless sky that pins
\end{stanza}

\begin{stanza}
it here, dead or dying, as the earth\verseline
turns brittle and every sorrow dries?\verseline
No one will come. It's worth
\end{stanza}

\begin{stanza}
their lives to try to run away;\verseline
they stay in Gaza, helpless as the sand,\verseline
where the sun's a sentry tower, and a gun
\end{stanza}

\begin{stanza}
takes aim at anyone who drinks.\verseline
No use. There's no relief in salt or tears,\verseline
and only the sea's too slow to think
\end{stanza}

\begin{stanza}
of anything but out, and in, and out\verseline
in its dumb mockery, silent mimicry,\verseline
a mouthing rape that makes a fool of care.
\end{stanza}

\end{poem}


{\poemnewpage}
\poemtitlear{ctg:Silent}{Silent Movie}{SILENT MOVIE}

\begin{minipage}{14cm}%
\xfepigraph{Every where at the approach of the white man {\lbrack}the Indians{\rbrack} fade away.... We hear the rustling of their footsteps, like that of the withered leaves of autumn, and they are gone for ever.
\\\textnormal{--- Joseph Story, U.S. Supreme Court justice}}
\vspace{1em}

\xepigraph{We never really conquered Lydda.... {\lbrack}T{\rbrack}here was really no city to conquer. The whole place...was empty.
\\\textnormal{--- Amos Kenan, writing about the scene of a notorious massacre by Zionist forces in 1948}}
\end{minipage}

\begin{poem}


\begin{stanza}
Soundless on aging acetate,\verseline
a building was standing one moment,\verseline
gone the next. Next to it\verseline
were grainy human forms that disappeared\verseline
in the wink of a wrinkling frame.\verseline
The projector jammed; an organist\verseline
filled the time, keeping time, suspense!\verseline
On the screen\verseline
the bodies hung, like dancers poised on invisible pins\verseline
impossible to repeat, impossible to imitate.
\end{stanza}

\begin{stanza}
When the end comes without a sound\verseline
it looks like a miracle, it's uncaused art.\verseline
On the news, behind a talking head\verseline
I saw apartments crumble noiselessly like cards\verseline
and the people fly in a flash.\verseline
And one of those, framed at a window,\verseline
balanced on one toe while the film stayed put\verseline
for the anchorman to take the view;\verseline
then, silent, we watched\verseline
him as he pirouetted down.
\end{stanza}

\end{poem}

{\poemnewpage}
\poemtitlear{ctg:Jewish}{Jewish Journal and Bad Dreams}{JEWISH JOURNAL AND BAD DREAMS}
\begin{poem}


\begin{itprestanza}
July 22, 2014
\end{itprestanza}

\begin{stanza}
Summer has its clammy skin against my window.\verseline
Even here, at dawn you can smell death ---\verseline
yes, smell it in the silence of\verseline
the casual newspaper,\verseline
the purring of the first neighborhood\verseline
car, a moving gloom\verseline
rabid with secrets.\verseline
No one is saying anything.\verseline
But we all know of more dead children in Gaza,\verseline
people walk the three blocks to synagogue\verseline
to nag and natter about it,\verseline
if not to pray for the victims. Of course\verseline
we cannot hear the bombs ---\verseline
we only built them.\verseline
No, I'm hearing nothing new.\verseline
I've swallowed a stench of silence and cannot cough it up.\verseline
Awakening today, I could not at first remember\verseline
the reason of my dread. There's something maniacal\verseline
about an alarm that shrieks,\verseline
{\vrsindent}Begin, begin, begin!
\end{stanza}

\begin{itprestanza}
July 23
\end{itprestanza}

\begin{stanza}
I'm afraid to leave my room.\verseline
Am I thinking of the kids who risk a sniper's bullet\verseline
if they take a walk?\verseline
But this is New Jersey --- above me is a vague blue wall,\verseline
heat without sound, no shapes, no words. That's\verseline
what terrifies: reality going about its business\verseline
while its distant products turn fatal to it.\verseline
Silence poisons every calm.\verseline
This shirt sticks to my back's hair --- a penance?\verseline
I find myself tearing bits out of the newspaper\verseline
as if to patch with stalled time what demons have pierced;\verseline
at night the sweating sky purples with rage\verseline
and there is danger in the gimlet stars.
\end{stanza}


\begin{itprestanza}
July 24
\end{itprestanza}

\begin{stanza}
I'm lying, of course. I'm not afraid,\verseline
I'm only saying it to convince myself I am --- because\verseline
it's an outrage not to be afraid.\verseline
What ought I to feel? Do the dead\verseline
turn in their sleep, those ghostly child faces\verseline
averted as they pass the word, row on silent row,\verseline
{\vrsindent}It's nothing, never mind?\verseline
What have I given?\verseline
I remember now, \cq{New Jersey} was the name\xmargreftextnote{bot:beirut}\verseline%
\xtextnote{bot:beirut}{The USS New Jersey fired on Beirut in December 1983}
of the battleship that mauled Beirut for\verseline
Ronnie the Ripper and his friends...\xmargreftextnote{bot:reagan}\verseline%
\xtextnote{bot:reagan}{Ronald Reagan}
{\vrsindent}Sixteen-inch guns!\verseline
the newspapers kvelled, while\verseline
boys and girls choked on their guts and\verseline
hearts zipped shut, once and for all,\verseline
tiny dolls in a body bag. Sailors rocked\verseline
their playthings on the swells.\verseline
Today I've only got myself to rock, overgrown baby.\verseline
Mother, I'm tearing at your womb,\verseline
I'm another Israeli, wild child,\verseline
I will destroy where I cannot climb back in.
\end{stanza}

\begin{itprestanza}
July 25
\end{itprestanza}

\begin{stanza}
Heat is hate. The sun's the angry\verseline
wizard whose redness will consume us all.\verseline
Some wind today lifted wings of discarded\verseline
newsprint to coast above the lawns. I strained but could not\verseline
read the words --- it was silly of me to think I might.\verseline
Someone I know is an Israeli soldier, part of a column piercing the Gaza border.\verseline
I hear rabbis are calling for \cq{solidarity} with the State.\verseline
Nothing in me is solid; have I missed a lesson somewhere?\verseline
Charred wings float away from me still.\verseline
Lines distort in swimming eyes,\verseline
sense gone, prayer nothing but hallucination\verseline
as here and now, all words betray us:\verseline
Hear, O Israel, the War is our God,\verseline%
the War is Fun.\xmargreftextnote{bot:hear}
\end{stanza}\xtextnote{bot:hear}{A play on the liturgical 
\q{Hear, O Israel: the Lord is our God, the Lord is One} (Sh'ma Yisra'el)}



\begin{itprestanza}
July 26
\end{itprestanza}

\begin{stanza}
Sabbath. Forbidden to write until after dark,\verseline
now shackled with migraine. Funny,\verseline
I've lost touch with the use of words,\verseline
they're insects on my skin as they scurry away, unheard,\verseline
a mystery to me, all of them.\verseline
I try to say to the dead,\verseline
{\vrsindent}I have tasted your unmoving air, and retched.\verseline
Will anyone answer me?\verseline
How I wish I were really sick! It would feel better than this dull ache,\verseline
I would have an occupation of sorts.\verseline
But only children can be sick\verseline
without responsibility, tucked into bed, patted on the forehead.\verseline
The children in my mind's eye are too still for sickness,\verseline
too silent for innocence. (Hearing nothing, I watch my fingers write\verseline
in a swim of pain.)\verseline
Whose?
\end{stanza}

\begin{itprestanza}
July 27
\end{itprestanza}

\begin{stanza}
In dawn is the sputum of defeat.\verseline
No point pretending, even trying to speak.\verseline
Today I watched a film made in the ruins of Shuja'iya\xmargreftextnote{bot:Shuja}\verseline%
\xtextnote{bot:Shuja}{Historic Gaza City neighborhood, just outside the Old City.}
after artillery flattened the town. I saw\verseline
the arm of a young man, trapped, bleeding to death,\verseline
extended from the rubble in helpless appeal\verseline
towards two bulky-vested rescuers\verseline
who dared not reach him because of sniper fire\verseline
from the soldiers. (There was a \cq{ceasefire} on.)\verseline
The wounded man died slowly while the camera danced\verseline
and the two medics, stamping in the dust, roared dark harmonies\verseline
at an unpictured sun.
\end{stanza}

\end{poem}


{\poemnewpage}
\poemtitlear{ctg:Kaddish}{Kaddish}{KADDISH}
\begin{poem}


\begin{stanza}
Sometimes it makes sense to think about God.\verseline
Look at your watch, and see death has moved an hour closer.\verseline
Look upwards --- it's about to rain.
\end{stanza}

\begin{stanza}
So what are we going to do?\verseline
What will we eat this evening? Next week? Next year?\verseline
What would happen if I forgot my address, and had to shelter under the sky?
\end{stanza}

\begin{stanza}
The Law: ten men have scoured its pages, ten men owe\verseline
thanks to the Creator. Might as well say it, then ---\verseline
except that we say, Magnified and hallowed may His great name be!\verseline
\textit{Yisgadal v'yisqadash}...but\xmargreftextnote{bot:Glorified}\verseline%
\xtextnote{bot:Glorified}{From the Mourner's Kaddish: \q{Exalted and sanctified be God's name ...}}
who are we? What name do we mean?
\end{stanza}

\begin{stanza}
Through the muffled tiers that ring the stage\verseline
their eyes sparkle through darkness, the watchful dead,\verseline
training their memories on the scuffed, illuminated boards
\end{stanza}

\begin{stanza}
where each of us sings for his supper, heads craning\verseline
toward the encircling gloom.\verseline
My heart's racing as if the room\verseline
were on fire,\verseline
but I know I've got to be calm, follow the rules ---
\end{stanza}

\begin{stanza}
it's nothing to be proud of, having eyes, having a voice,\verseline
it's nothing to stand here on two legs.\verseline
It's no big deal to breathe, or to grieve.\verseline
It's everything to aspire\verseline
to a place in that pantheon, beyond desire.
\end{stanza}

\begin{stanza}
Well, are the dead beckoning now? Is He? How can we hear\verseline
anything, while we pack the room with words and fear?
\end{stanza}

\begin{stanza}
Maybe, somewhere, they've already told us what they want...\verseline
what we all want. One must do one's best. Panic won't help,\verseline
but --- when might I hope for another chance
\end{stanza}

\begin{stanza}
to get this right? How much do I have to read, how much do I beg\verseline
for answered prayers? (I'll try again.)
\end{stanza}

\begin{stanza}
Through eyes magnified\verseline
with dread I see a fog spread in dusk-light, descending, closing in.
\end{stanza}

\begin{stanza}
At various times I try to think about its name.\verseline
Each instance brings me closer to the ultimate ignorance ---\verseline
but the words run on, and it's only the same\verseline
nails against the palm,
\end{stanza}

\begin{stanza}
the long dusty dance,\verseline
always the end of things slowly rising towards us,\verseline
always the same echo drying against the tide.
\end{stanza}

\end{poem}


{\poemnewpage}
\poemtitlear{ctg:Oranges}{Oranges}{ORANGES}
\begin{poem}


\begin{stanza}
Piled in slanting supermarket bins,\verseline
Spray-painted like tennis balls,\verseline
oranges lack history.
\end{stanza}

\begin{stanza}
Oranges can't tell you\verseline
where they came from, each colored globe,\verseline
each sectioned world.
\end{stanza}

\begin{stanza}
Did this one hang\verseline
over a ruin near Bethlehem, or sway alone in Jaffa?\verseline
Did its darkened branches feel
\end{stanza}

\begin{stanza}
a soldier's touch, as boots\verseline
scuffed the sand of exile\verseline
and rinds, drying, hid deep in the friendless shade?
\end{stanza}

\begin{stanza}
Smell the orange and forget all that,\verseline
forget there was a past\verseline
or a place,
\end{stanza}

\begin{stanza}
forget everything but the sweetness\verseline
of possession, the aroma\verseline
of imperial distances.
\end{stanza}

\begin{stanza}
Touch it: it's round and full\verseline
as a young life --- hold it up to sunlight,\verseline
it glistens.
\end{stanza}

\begin{stanza}
Devour it. Now youth and life\verseline
dissolve in you, lobe by lobe.\verseline
The startled scent rises, but the fruit
\end{stanza}

\begin{stanza}
of victory escapes the tongue.\verseline
Vagrant, you are everywhere\verseline
and nowhere now;
\end{stanza}

\begin{stanza}
how alone you've grown,\verseline
prism of horizons and dew!\verseline
You wipe the damp of ghostly orchards
\end{stanza}

\begin{stanza}
from your white forehead\verseline
as the sun clears and the mind returns to Tel Aviv,\verseline
and you carry home one more orange
\end{stanza}

\begin{stanza}
to crush with your hand\verseline
for the yield of the juice,\verseline
the discarding of the pulp ---
\end{stanza}

\begin{stanza}
as the past bleeds its sweetness,\verseline
and the earth of every beauty\verseline
dies, forgotten and unseen.
\end{stanza}

\end{poem}

{\poemnewpage}
\poemtitlear{ctg:Lecture}{Lecture}{LECTURE}
\begin{poem}


\begin{stanza}
First, is there anything wrong with you?\verseline
Have you been to the doctor, married a new\verseline
bride? Proud of a freshly-built house,\verseline
or maybe you're timid as a mouse?
\end{stanza}

\begin{stanza}
We've got to know.\verseline
Not for ourselves, you understand; it's all in the Law\verseline
what you must do.
\end{stanza}

\begin{stanza}
We're not the ones you're killing for.\verseline
We're not the ones who give the orders.\verseline
It's God who made us, fixed the borders\verseline
between the strong and the profane,\verseline
the privileged and the plain ---
\end{stanza}

\begin{stanza}
that is, if you follow the rules.\verseline
Now let's get going.\verseline
You can choose your own tools,
\end{stanza}

\begin{stanza}
you can have lots of things here,\verseline
carry armfuls of bullets\verseline
and a rifle like a yardarm.\verseline
Don't look so shocked. It's all in a day's work.
\end{stanza}

\begin{stanza}
True, you still don't know how to fight.\verseline
Well, wrap your scrawny arms in this. Here's\verseline
a new skin to cover up all the tender years ---\verseline
tough and hairy, ram-scented, brushed and oiled\verseline
till it's smooth; you'll hardly notice it,\verseline
it moves when you move, but \verseline
the Arabs will be fooled.
\end{stanza}

\begin{stanza}
Couldn't ask for more than \textit{that}.\verseline
You're anybody now, you're blessed, you're everywhere.\verseline
Today a steel helmet, tomorrow you'll scare\verseline
them all, just wait.
\end{stanza}

\begin{stanza}
You wanted a football? There's a goal.\verseline
You looked for a quest: here's a dare,\verseline
and an ooze of virtue for the blood spoor in the air.\verseline
Get moving --- don't be late.
\end{stanza}

\end{poem}


{\poemnewpage}
\poemtitlear{ctg:Three}{Three Words for the Dead}{THREE WORDS FOR THE DEAD}
\begin{poem}


\begin{stanza}
I saw young men scramble from an earthen tunnel\verseline
bootless, unhelmeted, guns in their hands. They meant\verseline
to strike Israeli tanks attacking Gaza --- \& they were doomed\verseline
from the start, it was the army that had made the film\verseline
I watched (black \& white, with white\verseline
cross-hairs stenciled over the middle of the frame,\verseline
so we'd know\verseline
the invaders had those brave ones in their sights) ---\verseline
\& soon, sure enough, explosions\verseline
buffeted them, \& (when they tried\verseline
to retreat) destroyed their tunnel too.
\end{stanza}

\begin{stanza}
But I had time for three thoughts while the picture lasted.
\end{stanza}

\begin{stanza}
First: wonder at the freshness of their hearts --- \verseline
running (not marching, thank the Lord)\verseline
so eagerly into battle, not to die really, but to contend\verseline
proudly for their homes, their mothers \& their fate,\verseline
\& to sacrifice (just as proudly) if need be, but not to mourn.\verseline
Too often jaded by the joyless human carnival\verseline
of lies, cruelty \& folly, I caught my breath\verseline
at such hope, such carefree license with the gift of lives ---\verseline
my heart rose at their generosity, but not, alas, for long.
\end{stanza}

\begin{stanza}
Next I knew deep sadness. I saw\verseline
that the young were lost, \& saw, what's more,\verseline
how such young loves were lost for nothing;\verseline
no child survived because of them, no prisoner escaped,\verseline
no one would visit their graves with thanks from the living.\verseline
Having hoped merely to scratch a scar or momentary\verseline
mark on the monster's tail, what could\verseline
they have done --- what could any creature have done ---\verseline
to atone for the wasting of such cherished life?\verseline
I chided my heart's pride in them then,\verseline
I scorned myself for having waved a handkerchief\verseline
with pointless tears to decorate a crime.
\end{stanza}




\begin{stanza}
Then anger came; it elbowed grief aside \&\verseline
stared me down. \cq{How dare you mourn?} it said,\verseline
\cq{\& reproach yourself with mourning?\verseline
Do you scorn the spurned\verseline
when he rises, just to be kicked again?\verseline
Do you blame the face that yields to blows\verseline
only because a man won't turn his back?\verseline
Is it for the hopeless to cast away freedom, too?\verseline
Did their hands dig the stony channels\verseline
that turned the current of their loves,\verseline
either to cowardice or to death?}
\end{stanza}

\begin{stanza}
Then I knew that I must rage, \& knew\verseline
the curse of silence, for I saw\verseline
I could not say what I felt.\verseline
Why must generosity run unheeded into death?\verseline
Why a tunnel, not a grateful eye, to draw such fruits \& sorrows in?\verseline
Why this squalid power over fragile youth?\verseline
Why such puny sunsets before\verseline
the immensity of night?
\end{stanza}

\begin{stanza}
The picture faded, \& without a sound\verseline
those young men, buried \& unspeaking,\verseline
left me without words, bereft of time.
\end{stanza}

\end{poem}


{\poemnewpage}
\poemtitlear{ctg:Clowns}{Clowns with Swastikas}{CLOWNS WITH SWASTIKAS}
\begin{poem}

\begin{prestanza}
1
\end{prestanza}

\begin{stanza}
Ink that \cq{bomb,} Netanyahu, play the public fool!\verseline
But massacres germinate in a comic school.
\end{stanza}

\begin{prestanza}
2
\end{prestanza}


\begin{stanza}
With bagels, \textit{litterateurs} wolf down the crimes\verseline
the IDF parades in the \textit{New York Times}.\xmargreftextnote{bot:parade}
\end{stanza}\xtextnote{bot:parade}{The Israel Defense Forces conducted 
several parades, meant to display its military force, up until 1973.}
\end{poem}


{\poemnewpage}
\poemtitlear{ctg:Dead}{Dead God, Dying God}{DEAD GOD, DYING GOD}
\begin{poem}

\begin{stanza}
This one's on a pedestal,\verseline
all marble and glare.\verseline
A marvelous pair of shoulders, a frown\verseline
that could shame every brazen whisperer\verseline
and tamp loose tongues down.
\end{stanza}

\begin{stanza}
But I don't care what anybody ever felt for him,\verseline
or for the orations he might have growled\verseline
through those stone lips, under those raised hands ---\verseline
not even for the storm\verseline
that slathered sea-spray at his command\verseline
and flayed the faithless beaches\verseline
while priests hung their heads, and Hellas howled. 
\end{stanza}

\begin{stanza}
I don't even care when the other one\verseline
stares at me with soft, reproachful eyes\verseline
from that splintered perch,\verseline
and bids me honor his silence with my own.\verseline
His blood is fantasy, his wounded flesh never\verseline
bone of my bone,\verseline
no matter how he suffers, no matter how I try\verseline
to feel, to think of him as something real,\verseline
no matter the despair, no matter the calm.
\end{stanza}

\begin{stanza}
Let them both die.
\end{stanza}

\begin{stanza}
All that matters is what starves inside.\verseline
And that one dies every day,\verseline
the carking infant's born in each dream, to die at dawn.\verseline
That's the one that never survives\verseline
and never goes away,\verseline
doomed every tomorrow with tomorrow's bomb.
\end{stanza}

\end{poem}


{\poemnewpage}
\poemtitlear{ctg:AnUntimely}{An Untimely Man}{AN UNTIMELY MAN}
\begin{poem}


\begin{stanza}
I could not die at twelve.\verseline
They gave me drugs to swill\verseline
and told me it was all my fault ---\verseline
myself,\verseline
I could never really tell\verseline
what made me halt.
\end{stanza}

\begin{stanza}
I never felt at home\verseline
with comforts that disease\verseline
or childhood craves; too bored to hoard\verseline
for long,\verseline
I faintly hoped to be released\verseline
from what I stored.
\end{stanza}

\begin{stanza}
And yet I never gave\verseline
the stroke that would have killed\verseline
the pretense I could hardly keep,\verseline
or salve;\verseline
I merely waited, with enlarging guilt,\verseline
for the last sleep.
\end{stanza}

\begin{stanza}
So I've stumbled to the end ---\verseline
and now? As death draws near,\verseline
I mourn that I can neither lose\verseline
nor find\verseline
a martyrdom to purge the fear\verseline
of all I choose.
\end{stanza}

\end{poem}


{\poemnewpage}
\poemtitlear{ctg:Cyrano}{Cyrano de Marianne}{CYRANO DE MARIANNE}
\xepigraph{Welcome to Israel!  It seems you got lost. Perhaps you meant to sail to a place not far from here --- Syria. There the Assad regime slaughters his people every day with the support of the murderous Iranian regime.\\\textnormal{--- Israeli Prime Minister Benjamin Netanyahu, in a letter delivered to humanitarian activists when they were kidnapped by IDF sailors from their fishing boat, the Marianne, while attempting to break Israel's siege of Gaza and deliver medicine to the inhabitants (June 2015)}}

\begin{poem}


\begin{stanza}
Ah no, sir, you are too simple! Why, you might have said --- oh,\verseline
a great many things! Come, let me help you...thus.
\end{stanza}

\begin{stanza}
Didactic: \cq{Why were you seeking Israel's occupation in Gaza?\verseline
After all, we occupy the whole West Bank as well!}
\end{stanza}

\begin{stanza}
Confidential: \cq{Humanitarians don't hold office here; perhaps you didn't know that?}
\end{stanza}

\begin{stanza}
Sensible: \cq{In this country, we don't try anything\verseline
with one small boat --- we bomb in force\verseline
and flatten everything in sight.}
\end{stanza}

\begin{stanza}
Dramatic: \cq{Is this the ship that launched a change of heart\verseline
and curbed the boundless tyranny of Israel? Naaah!}
\end{stanza}

\begin{stanza}
Witty: \cq{Now, isn't that just like a bunch of terrorists.\verseline
Trying to take on Israel without even an AK-47.}
\end{stanza}

\begin{stanza}
Colloquial: \cq{Charity, huh? Man, are you lost!}
\end{stanza}

\begin{stanza}
Rhetorical: \cq{Gaza! --- Gaza, you pathetic simpletons? Why,\verseline
cast your eyes northward and recall what we did to Lebanon!}
\end{stanza}

\begin{stanza}
Forthright: \cq{Since you came in peace, it's clear you came to the wrong address.}
\end{stanza}

\begin{stanza}
Comic: \cq{Human rights? Did you say \textit{human rights}? I mean,\verseline
what kind of school did you graduate from? Do you\verseline
know where you \textit{are}? Have you ever, like, visited\verseline
one of our prisons for Palestinians?\verseline
Oh my God, \textit{human rights}??}
\end{stanza}


\begin{stanza}
Pedantic: \cq{Does not the legal maxim \textit{res ipsa loquitur}\verseline
impute demonstration of intent to an act's own form?\verseline
Then, given your demonstrable impotence against Israel's might,\verseline
surely you meant to accomplish nothing here at all!}
\end{stanza}

\begin{stanza}
Simple: \cq{Were you guys looking for a decent government\verseline
and just took a wrong turn?}
\end{stanza}

\begin{stanza}
Ironic: \cq{Have I got this right? I'm supporting al-Qaeda in Syria,\verseline
I'm occupying the Golan, I'm threatening Iran with nuclear bombs,\verseline
I'm killing protesters in the West Bank, ethnically cleansing\verseline
Jerusalem, and you fellows have to make trouble in Gaza?}
\end{stanza}

\begin{stanza}
Passionate: \cq{Never again will Jews endure\verseline
the blight of conscience! Never again will Jews permit\verseline
justice that our elites disdain! Never again will Jews refrain\verseline
from lording it over their neighbors\verseline
when the spirit moves them!}
\end{stanza}

\begin{stanza}
Picturesque: \cq{My family didn't flee the pogroms\verseline
of Russia for me to be lectured over enslaving\verseline
a few million people.}
\end{stanza}

\begin{stanza}
Candid: \cq{About that Syria crack,\verseline
I could have mentioned the American\verseline
slaughter in Iraq, or the attacks on Yemen\verseline
by U.S. client states --- and you know that Obama and I\verseline
are both killing Syrians, too. But I couldn't\verseline
write anything like that under the circumstances ---\verseline
you understand, don't you?}
\end{stanza}

\begin{stanza}
Or even --- since you mention \cq{the murderous Iranian regime} ---\verseline
\cq{You know Iran's navy never killed unarmed civilians\verseline
the way ours does; why couldn't you sail \textit{there}\verseline
and teach them some sort of lesson?}
\end{stanza}

\begin{stanza}
All this you might have said, if you had one tenth\verseline
the virtue of the average gangster, or at least\verseline
some of Goebbels' graceful prose.\xmargreftextnote{bot:Goebbels}\verseline%
\xtextnote{bot:Goebbels}{Alludes to Nazi-era German minister Joseph 
Goebbels's talents for public speaking and propaganda.}
But no,\verseline
even propaganda stales with overuse,\verseline
and the rotten net shreds in the wind.\verseline
One might have known.
\end{stanza}

\end{poem}



{\poemnewpage}
\poemtitlear{ctg:Drone}{Drone}{DRONE}
\begin{poem}


\begin{stanza}
We believe in what knows us,\verseline
not in what we see.\verseline
Each low moon or eavesdropping tree\verseline
at dusk erodes our poses
\end{stanza}

\begin{stanza}
of self-possession; God\verseline
lives in the patient walls\verseline
whose mirrors catch us naked, all cells\verseline
on display. They say that's good
\end{stanza}

\begin{stanza}
for something --- who knows what?\verseline
Well, what makes us is what's known,\verseline
I guess, and all is known; except we're not.
\end{stanza}

\begin{stanza}
If God were a foreign general\verseline
he'd be puzzled, I think, by these rites\verseline
of awe. Why should the sight\verseline
of a hovering drone draw all
\end{stanza}

\begin{stanza}
faint thoughts to it, subdue each voice?\verseline
It's only a spy who sees us,\verseline
and all he sees are cross-hairs --- Jesus,\verseline
that's not life, when just his choice
\end{stanza}

\begin{stanza}
(thumbs down) decides one's fate!\verseline
And yet I can't quite believe in soul,\verseline
knowing I'm spread across some private's bombsight.
\end{stanza}

\end{poem}

{\poemnewpage}
\poemtitlear{ctg:TheStorm}{The Storm Scene}{THE STORM SCENE}
\poemdedication{(after James Merrill)}
\begin{poem}

\begin{stanza}
Last night I dreamed about a place called Sabbath.\verseline
There, we had left a pile of things we were going to use,\verseline
but never did use: lids and saucers and can openers\verseline
whose silence beckoned towards a misty column\verseline
that wafted from the waste, filled slowly with night.\verseline
The smell reminded me of every other failure, every sought-for\verseline
respite. Then the sky quickened and thickened, clouds sobbed\verseline
and a blurred howl rose from the branches.\verseline
Staring down from a height, I watched trembling\verseline
as embers winked out under rain, my fingers trained\verseline
toward five holes in the dike, one final hope masked\verseline
as a separate peace. Nothing would suffice.\verseline
The homeless roar that stalked the wilderness grew less human\verseline
as it rose to its most personal pitch. Alone again,\verseline
betrayed, where am I, was I?...and I turned back, as in its mad\verseline
despair those vowels drenched every shard of autumn.
\end{stanza}

\end{poem}


{\poemnewpage}
\poemtitlear{ctg:CrimeScene}{Crime Scene}{CRIME SCENE}
\begin{poem}


\begin{stanza}
Item: one wreck of a car.\verseline
Split open, like a crowbar\verseline
had wrenched the roof.
\end{stanza}

\begin{stanza}
One flame lapping at the charred chassis.\verseline
Two medics,\verseline
one hose.
\end{stanza}

\begin{stanza}
It seems this one died alone.\verseline
But now it's done\verseline
it's a happening, it's a freak,\verseline
they're all talking at once:\verseline
the bewildered grocer, still in his apron,\verseline
the boy on the bike.
\end{stanza}

\begin{stanza}
Who was he? What was he doing?\verseline
Where did the missile come from?
\end{stanza}

\begin{stanza}
Well,\verseline
this one's mum --- he's sworn an oath\verseline
and answers all chatter with the same\verseline
impenetrable, stolid stare,\verseline
his torso oozing, blue lips facing the earth,\verseline
splayed out and proud.
\end{stanza}

\begin{stanza}
Slowly, two of them lift him\verseline
and carry him away:\verseline
now there's one less for the crowd.
\end{stanza}

\begin{stanza}
The news in Khan Yunis: a new martyr!\xmargreftextnote{bot:KhanYunis}\verseline%
\xtextnote{bot:KhanYunis}{Site of a massacre by Israel Defense Forces on November 3, 1956}
In Israel, it's a terrorist \cq{neutralized.}
\end{stanza}

\begin{stanza}
What's left to see?
\end{stanza}

\begin{stanza}
The band of a wristwatch,\verseline
untouched, in the middle of the road.
\end{stanza}

\begin{stanza}
The shriveled driver's door prized\verseline
from the wreck\verseline
before they go...
\end{stanza}

\begin{stanza}
The car's acrobatic tilt,\verseline
one wheel still spinning free.
\end{stanza}

\end{poem}


{\poemnewpage}
\poemtitlear{ctg:TheSurvivor}{The Survivor}{THE SURVIVOR}
\begin{poem}


\begin{stanza}
After the dancing stopped, \& she\verseline
(her cool daydreams on her cheeks)\verseline
could slowly drift around\verseline
the busy eddying of air\verseline
(that might have turned her bridal gown\verseline
to something less\verseline
than time had wanted it to be),\verseline
\& setting sun from windows left motes \& flecks\verseline
across the curtain of her hair ---
\end{stanza}

\begin{stanza}
\& late came forward then, alone, to show\verseline
my eager eyes (that hardly seemed to know\verseline
her) what she was, \& hid ---\verseline
what was the veil that slid before my eyes\verseline
to baffle surprise\verseline
when the spun white gauze that made the gown\verseline
gave pause\verseline
as night looked down\verseline
when to me she raised her head,\verseline
\& I, the shadow of a doubt undid?
\end{stanza}

\begin{stanza}
Did her hair lying flat,\verseline
\& that enigma of lips underneath\verseline
(that made me strain for breath)\verseline
know how the last act lay,\verseline
when death would guarantee\verseline
that no more deliberate play\verseline
of discovered things could resurrect a fact\verseline
to rise with clarity\verseline
from the tidal shadow of a woman's face?
\end{stanza}

\begin{stanza}
A moment poised to fall ---\verseline
\& did,\verseline
where in the graveyard slid\verseline
\& angled down the mouth that swallowed,\verseline
with its alien dream,\verseline
all, till down it went\verseline
replete \& silent\verseline
closed up against the word\verseline
I forgot to ask in time\verseline
\& never heard.
\end{stanza}


\begin{stanza}
Impregnable now. \& nothing tames\verseline
the brute of unperception here\verseline
or drought\verseline
(with always the factual meaningless air\verseline
no matter how dark the danger\verseline
or in despair the shout) ---\verseline
everything too complete for names\verseline
shadows or remainder.
\end{stanza}

\begin{stanza}
Was I so ignorant?\verseline
Will it help now if I lament ---\verseline
\& what else anyway\verseline
\& what exclaim for?
\end{stanza}

\end{poem}


{\poemnewpage}
\poemtitlear{ctg:TheWalls}{The Walls}{THE WALLS}
\begin{poem}


\begin{stanza}
From Gaza they can't touch it,\verseline
but in Jerusalem I've seen the Wall.\verseline
Two of them, actually --- two prison lines,\verseline
deadly, both, in what they define\verseline
and what they keep out.
\end{stanza}

\begin{stanza}
First there's the long one,\verseline
the razor-tipped, the sinuous, the strong.\verseline
We all know what it means,\verseline
from our grandmothers we've learned\verseline
about ghettos, of walls that poison
\end{stanza}

\begin{stanza}
each horizon, every hope. But here's\verseline
another one, no less cruel: the prayer wall\verseline
whose sullen stones say \cq{no} to all\verseline
but the victors. Every fool worships\verseline
empire, makes monuments of his fears ---
\end{stanza}

\begin{stanza}
but I see here, locked away in\verseline
this fustian's dry, high stains,\verseline
a fierce negation that leaves the acolytes blind,\verseline
a hate they hide even from the captives\verseline
penned in dark rain,
\end{stanza}

\begin{stanza}
waiting at a checkpoint\verseline
out of Gaza. \textit{They're} not wishing\verseline
for love or even life any more; but where am I\verseline
going, with my curse of freedom, I\verseline
who've seen so much future vanishing?
\end{stanza}

\end{poem}


{\poemnewpage}
\poemtitlear{ctg:MyDevil}{My Devil Curses Me}{MY DEVIL CURSES ME}
\begin{poem}


\begin{stanza}
Thought I was gone, did you?\verseline
My shrunken head still\verseline
hisses evil nothings\verseline
in your ear.\verseline
For all your pious will,\verseline
I'm not killed ---
\end{stanza}

\begin{stanza}
in fact, I'm never far.\verseline
Crush me, I'm a residue\verseline
on the tongue.
\end{stanza}

\begin{stanza}
Idiot Jew!\verseline
You've got your own Jew now,\verseline
bitter but strong ---\verseline
bottle me, there's more\verseline
where this one came from.
\end{stanza}

\begin{stanza}
Talk of virtue,\verseline
I'm the catch in your voice.\verseline
Progress --- I'm the sacrifice.
\end{stanza}

\begin{stanza}
And now you think\verseline
you can fix things\verseline
if you screw off the top,\verseline
pull me out by the hair\verseline
to show me to your friends\verseline
one day a year.
\end{stanza}

\begin{stanza}
Ladies and gentlemen,\verseline
it's confession time,\verseline
watch me drop a tear\verseline
for the stillborn child.
\end{stanza}

\begin{stanza}
But what they see is my distorted face,\verseline
aborting the glow like a horrid moon\verseline
usurping every civilized room ---\verseline
the audience is gone\verseline
from your Laoco\"on.\xmargreftextnote{bot:laocoon}
\end{stanza}\dxtextnote{bot:laocoon}{Greek and Roman mythological 
figure during the Trojan war; in several stories, he 
was attacked by snakes sent by Athena (or another god) 
in retaliation for his suspicions toward the Trojan Horse.}

\begin{stanza}
I tell them your Promise is misery,\verseline
your divine dream, my loss.
\end{stanza}


\begin{stanza}
Do you like what you see?
\end{stanza}

\begin{stanza}
And now they've gone,\verseline
and now we're alone.
\end{stanza}

\begin{stanza}
Face to face in this tiny, hollow cell,\verseline
you and me.
\end{stanza}

\begin{stanza}
Where's your act now?\verseline
No one to listen,\verseline
no one to sympathize.\verseline
No one to admire your groans\verseline
or your patient pleas.
\end{stanza}

\begin{stanza}
Nothing to tell.\verseline
What will you do,\verseline
blind Jew?
\end{stanza}

\end{poem}


{\poemnewpage}
\poemtitlear{ctg:Delirium}{Delirium}{DELIRIUM}
\begin{poem}


\begin{stanza}
Where does dreaming end\verseline
and the new day begin? The ash that rises\verseline
in muzzy columns
\end{stanza}

\begin{stanza}
towards the sky becomes the\verseline
thundercloud that\verseline
will threaten us all again,
\end{stanza}

\begin{stanza}
that drenches the crematoria\verseline
as the dead blur between worlds.\verseline
They, they are gone
\end{stanza}

\begin{stanza}
and unintelligible ---\verseline
but I?\verseline
Where to hide from the eye
\end{stanza}

\begin{stanza}
that pierces the living\verseline
as stubborn smoke erases\verseline
all that moves, all that divides?
\end{stanza}

\begin{stanza}
A displaced tongue\verseline
invests the vowels of sleep,\verseline
a wind laps at dawn's gray puddles
\end{stanza}

\begin{stanza}
and rain beads coldly on\verseline
an open window... Alone, every word\verseline
hurts me. I do not want to speak
\end{stanza}

\begin{stanza}
or to listen. But the accuser\verseline
demands: Where were you?\verseline
How did you survive so long? And where?
\end{stanza}

\begin{stanza}
As if to be heard\verseline
were to be cursed, to stand unique\verseline
were to live and to live
\end{stanza}

\begin{stanza}
to earn the taunt\verseline
of the ultimate ones,\verseline
who are bold, and scabrous, and gone...
\end{stanza}

\begin{stanza}
Should I awaken at all?\verseline
It must be truer\verseline
to sink
\end{stanza}

\begin{stanza}
below the bottom, before the fall,\verseline
the first deadly rhyme --- there to drink,\verseline
without hope or fear,
\end{stanza}

\begin{stanza}
the Lethe poured in\xmargreftextnote{bot:lethe}\verseline%
\xtextnote{bot:lethe}{Underworld river in Greek mythology, 
whose waters caused forgetfulness when consumed.}
between anonymous lips,\verseline
relieved of time,\verseline
as ignorant as they are poor.
\end{stanza}

\end{poem}


{\poemnewpage}
\poemtitlear{ctg:Farewell}{Farewell}{FAREWELL}
\begin{poem}


\begin{stanza}
Each time you leave could be the last.\verseline
Each gentle touch of sunlight on your eyes\verseline
might prevent my every touch,\verseline
might sketch your final image in my brain,\verseline
impalpable, then, as a moment's glare.
\end{stanza}

\begin{stanza}
I never know which detail of your body\verseline
to hold on to. If I take the ends of your hair\verseline
what becomes of your fingers, your knees?\verseline
Will I remember the way your smile turned\verseline
aside as you dissolved into twilight?
\end{stanza}

\begin{stanza}
And then there are words --- which ones are we\verseline
to choose, before your absence closes in,\verseline
knowing the silence may have no limit\verseline
and the sounds I say may brush against\verseline
mere shadows after you've gone? No,
\end{stanza}

\begin{stanza}
words cannot save this moment; no touch\verseline
will preserve the scent or shading\verseline
of your bare skin, as it is just now,\verseline
not for a second longer. Turn upward,\verseline
love --- look quietly into my eyes.
\end{stanza}

\begin{stanza}
Will you? Even now you're leaving me,\verseline
even now your beauty recedes\verseline
from the glance that longs to take you in\verseline
and hold you... I'm afraid to voice the final plea\verseline
that might clutch, vaguely, the darkening scene.
\end{stanza}

\end{poem}


{\poemnewpage}
\poemtitlear{ctg:Fanatics}{Fanatics}{FANATICS}
\begin{poem}

\vspace{1em}
\begin{minipagestanza}
\textit{Occupied Palestine, March 2011}. Ehud \cq{Udi} Fogel was killed in his home in the West Bank settlement of Itamar, along with his wife and three young children, on the night of March 11, 2011.
\end{minipagestanza}

\begin{minipagestanza}
No other Israeli Jews were killed in occupied Palestine that year.
\end{minipagestanza}

\begin{minipagestanza}
Udi Fogel, an Orthodox rabbi, was a tank officer in the Israel Defense Forces (IDF). During and after his service in Israel's military occupation, Fogel raised his family in Gaza, in the exclusively Jewish settlement of Netzarim. Until 2005, Netzarim was part of a bloc of Israeli colonies known as Gush Katif, which occupied much of Gaza's most valuable land and broke the 1.5 million Palestinians inhabiting Gaza into scattered enclaves.
\end{minipagestanza}

\begin{minipagestanza}
After Netzarim was closed by Israel, Fogel's religious convictions led him to the Jewish settlement of Itamar in the occupied West Bank. There he served as a teacher, under the supervision of a former chief rabbi of the IDF, telling students of the holy obligation of settling the West Bank.
\end{minipagestanza}

\begin{minipagestanza}
Hakim and Amjad Awad, 17 and 18 at the time of the killings, were both convicted in Israeli military courts and sentenced to multiple life terms. Virtually the only evidence against them consisted of their confessions, though both had initially denied involvement and members of their families stated that they had been elsewhere at the time of the attack. (Those family members were not asked to testify at the trial.) It is unclear how the youths' confessions were obtained.
\end{minipagestanza}

\begin{minipagestanza}
Hakim told the military judges that Israelis had tied up and killed two men from his village. \cq{This is what the state does to me every day,} he said. \cq{When I want to leave my village I have to undergo a search which always involves beatings.} The judges told him to refrain from discussing politics.
\end{minipagestanza}

\begin{minipagestanza}
Defense Attorney Raed Erda enraged the judges by pointing out that even Israel's Supreme Court had ruled that murders committed inside the territories are considered acts of war. To observers' disgust, he added, \cq{Houses are being built on their lands, there's no work, no education, the occupation is pushing people from all directions and a boy like the defendant goes out and does things like this without realizing their consequences.}
\end{minipagestanza}

\begin{minipagestanza}
Jewish media expressed deep sympathy for Udi Fogel's father, who lamented that Palestinians enjoy favorable conditions in Israeli jails.
\end{minipagestanza}

\begin{minipagestanza}
\textit{Jewish Week} editor Gary Rosenblatt suggested that international media had refused to treat the Fogels' murder as their lead story because \cq{we expect Palestinians to act in inhuman ways in expressing their hatred of Jews.} Rosenblatt also blamed Israeli newspapers for treating the Fogels as \cq{second-class Jews} because they were \cq{religious.}
\end{minipagestanza}

\begin{minipagestanza}
Rabbi Elyakim Levanon, a prominent clergyman among the Jewish settlers, told Udi's older brother that Udi's death was \cq{not private, it's public,} because the killing of his family would encourage Israel to build more settlements in occupied Palestine. A former Chief Rabbi of Israel eulogized the Fogels by comparing Palestinians to Nazis and promising: \cq{We will not bend, we will not give up...and nothing will prevent our faith in the righteousness of our path.}
\end{minipagestanza}

\begin{minipagestanza}
The military prosecutor called on the judges to disregard the youth of the boys convicted of the killings. They \cq{acted on a malicious and satanic ideology,} he said.
\end{minipagestanza}

\begin{minipagestanza}
Surviving members of the Fogel family, including a 12-year-old daughter, have pledged to \cq{be strong} and to continue expanding Israel's occupation of Palestinian land.
\end{minipagestanza}

\begin{minipagestanza}
In 2014, Israel's activities in the Gaza Strip, West Bank and East Jerusalem resulted in the deaths of 2,314 Palestinians and 17,125 injuries, compared with 39 deaths and 3,964 injuries in 2013, according to the annual report by the U.N. Office for the Coordination of Humanitarian Affairs.
\end{minipagestanza}

\end{poem}


{\poemnewpage}
\poemtitlear{ctg:Pines}{Pines in the Suicides' Wood}{PINES IN THE SUICIDES' WOOD}
\dedication{(see Inferno, Canto XIII)}
\xepigraph{If there is no future, there is no hope.
\\\textnormal{--- Adel Hamdona, speaking of his son's suicide attack on the Jewish settlement of Netzarim}}

\begin{poem}

\begin{stanza}
Tread lightly, stoic, these needles bleed.\verseline
Wonder --- if you must --- but ask no question.\verseline
If the mute corpses shock, it's you that need
\end{stanza}

\begin{stanza}
more than they can bear: not even groans\verseline
from them but move in fresh wounds, flow of pain ---\verseline
their grief has no words, their night no moon
\end{stanza}

\begin{stanza}
and souls that died with them won't rise again,\verseline
for death's all those who die for death can know.\verseline
Revenge is their root, blood their only rain.
\end{stanza}

\begin{stanza}
Mortal, you dare not follow where they go,\verseline
nor they explain themselves except in pain\verseline
written in a language known to the few
\end{stanza}

\begin{stanza}
who died as they did, embraced as kin\verseline
by strangers who killed them; forever joined\verseline
in loss, they break now just to break again ---
\end{stanza}

\begin{stanza}
as in death, they threw arms around\verseline
a hopeless love, so now it cuckolds them\verseline
in shame, as you in shame invade their wounds.
\end{stanza}

\begin{stanza}
Who are you to tease out the final flame\verseline
of souls that burned, to thumb shut eyes\verseline
whose tears you've never touched? Your home
\end{stanza}

\begin{stanza}
is with the strong, your step too coarse for dry\verseline
twigs, ashen bones that bear their weight\verseline
in moments lost, loves spurned, lies
\end{stanza}

\begin{stanza}
like bloody tendrils where lips once met\verseline
to close forever, yet never to forget\verseline
the grieving that kills, the hopes that hate.
\end{stanza}

\end{poem}


{\poemnewpage}
\poemtitlear{ctg:DeathInPalestine}{Death in Palestine}{DEATH IN PALESTINE}
\begin{poem}


\begin{stanza}
Where blood drowns\verseline
the memory of this stubborn ground,\verseline
steep in its age,\verseline
let the dead go.
\end{stanza}

\begin{stanza}
Let mule-footed wind,\verseline
pale with sacrificial rage,\verseline
run dry as our ruined eyes\verseline
above each sleeping stone.
\end{stanza}

\begin{stanza}
Let the desert draw tears,\verseline
if there are tears,\verseline
from an imageless brain.
\end{stanza}

\begin{stanza}
And let pain\verseline
bury what cannot be undone.
\end{stanza}

\begin{stanza}
Safe in their sorrows lie the grave's few.\verseline
Numb to a blind sky,\verseline
unflagging sun,\verseline
where noon's dim coliseum\verseline
is hushed\verseline
and burdened as a lightless pew.
\end{stanza}

\begin{stanza}
Let them go.\verseline
Cradled in our dark tread\verseline
they, at least, have earned the dust\verseline
we cannot own\verseline
except to know its end,\verseline
and loss.
\end{stanza}

\begin{stanza}
Deep in the pith\verseline
where columns of the dead\verseline
have marked the sand,\verseline
they stay.\verseline
We are the exiles of the land.\verseline
Over us, the pitch and curse of day.
\end{stanza}

\end{poem}


{\poemnewpage}
\poemtitlear{ctg:WhyItMattered}{Why It Mattered}{WHY IT MATTERED}
\xfepigraph{On Tuesday, August 19 [2014], the Israel Defense Forces...announced that Cpl. David Menachem Gordon...was found dead in central Israel, his weapon at his side.... We understood from the code phrase, \cq{weapon at his side,} that Gordon had committed suicide.\\\textnormal{--- Rabbi Yehoshua Looks}\\}

\xepigraph{Unable to disclose his mortifying secret, the boy can only fantasize revenge on those vile men whose twisted lustful current raged through their veins.... He dreams of an escape from his Hell... As much as he tried, he could not ignore the scattered scars that sexual abuse left on his Soul.%
\\\textnormal{--- David Menachem Gordon, writing of himself as an Orthodox Jewish child}}


\begin{poem}


\begin{stanza}
If I try to understand a dead man I never knew\verseline
from the relative safety of ignorance, it is to stress\verseline
the contradictions I will never be able to resolve.\verseline
First: I know I cannot share the origins\verseline
by which a boy was born, a boy learned the alphabet\verseline
of clothes, roles, manners, a Jew's youth,\verseline
nor the sickening plunge of that youth's last lesson ---\verseline
that rabbis rape, that saviors turn away.\verseline
But I'm just as shy with the boy at twenty-one,\verseline
helmeted, camouflaged, fatigued, carrying\verseline
the scar on his heart and a six-pointed medallion\verseline
around his neck, lying across the sand, like a lover,\verseline
near Gaza behind a swiveled gun.
\end{stanza}

\begin{stanza}
The face is young and blank in the only photo I've seen.\verseline
Perhaps the boy dreamed of lining up\verseline
his tormentors in that olive-drab gunsight;\verseline
maybe he saw old lechers, not panicking women and their\verseline
sons, when his brigade's guns were torching houses with\verseline
{\vrsindent}\cq{repeated shelling} as the victims fled...\verseline
maybe he believed his colonel when that maniac\verseline
{\vrsindent}pronounced Palestinians the enemies of God.\verseline
Or, maybe he was maddened with the mumbling myth\verseline
{\vrsindent}of soldiers who believe themselves invincible,\verseline
though of course he wasn't --- no more than the \cq{six men}\verseline
his unit killed in a \cq{summary execution} in Khuza'a\verseline
{\vrsindent}one August day, nor the hundred or more helpless who fell\verseline
to his brigade's artillery in Rafah.\verseline
But how much, then, did the young sufferer see?\verseline
Could there have been guilt, as his fire raged out in agony?
\end{stanza}

\begin{stanza}
Or did his binoculars reveal a frightened boy\verseline
who offered only terror against the ugly threat\verseline
{\vrsindent}(a barrel shaft upright, burning to invade\verseline
{\vrsindent} a child, impale his innocence)...\verseline
Did he hear his own voice in the collective cry of pain?\verseline
I hardly dare conceive how conscience,\verseline
seeping over the scene, might have enlarged the stain:\verseline
\cq{You did well, for larger men than you have doomed the young ---\verseline
though once the victim, you are among the strong.}\verseline
{\vrsindent}(Some who were his friends now carried suspect memories\verseline
{\vrsindent}and had to go.) Or, heard in a whisper, a taunt in his ear,\verseline
\cq{Have no pride. What men did to you, you've done.}\verseline
{\vrsindent}(And would do again, for no slogan could shatter\verseline
that circle of guilt.)\verseline
Or perhaps the voice framed the eeriest, most sinister words of all:\verseline
\cq{Nothing matters, all is death, and in death there is no difference\verseline
between the bullet and the brain, beaten and betrayer, all are one.\verseline
Forget good and evil, leave life, know only what is gone.}
\end{stanza}

\begin{stanza}
I do not know. But he is gone,\verseline
gone beyond mourning. Self-victimed now,\verseline
dumb to accuse, dumb to suffer, he cannot throw\verseline
this taint from his breast. And if I labor to peel\verseline
the web of violence from the wounded rest, it is to show\verseline
how numb is my heart, that cannot learn to feel\verseline
remorse enough for the brute he was, and was not;\verseline
for the bloody deed he suffered, and did;\verseline
for the innocence he ravaged and the terror he bequeathed\verseline
when the deluge that drowned all good and all bad\verseline
closed, forever, over his trailing grief.
\end{stanza}

\end{poem}


{\poemnewpage}
\poemtitlear{ctg:MyKaddish}{My Kaddish}{MY KADDISH}

\xepigraph{Magnified and sanctified may His great name be
in the world He created according to His will; and may His kingship reign,
in our lives and in our days, and within the life of the whole house of Israel,
speedily and soon, amen!
May His great name be blessed always, forever and ever!...
He who makes peace in his high places --- may He bestow peace
upon us and upon all Israel, and let us say, Amen.
\\\textnormal{--- From the Mourner's Kaddish}}

\begin{poem}

\begin{stanza}
Magnify his name I dare not.\verseline
For if great, where is his saving power?\verseline
Sanctify it?\verseline
Shall I then join the killers who murder children in that name?\verseline
And may he reign\verseline
{\vrsindent}(if anything should reign)\verseline
in a world he first made differently,\verseline
I hope,\verseline
from what he might have willed...\verseline
may a better one come soon,\verseline
in our lifetime, within our days! ---\verseline
{\vrsindent}in this I join.
\end{stanza}

\begin{stanza}
May the great name, Love, be blessed forever and ever!
\end{stanza}

\begin{stanza}
Let it be blessed, let it be praised, raise it on high!\verseline
Though we have never known words to\verseline
praise it, let alone prize it...\verseline
though we lack a tongue to try\verseline
variations on the final, inimitable phrase.
\end{stanza}

{\divider}

\begin{stanza}
Every inward cupboard hides a space\verseline
for loneliness that accumulates in crooked corners,\verseline
filling up too awkward a place\verseline
for the owner readily to clean ---\verseline
year by year, the odd thoughts make a quaint\verseline
and creeping heap of pain\verseline
until the hoarder dies, and mourners\verseline
dig out all the detritus from the shelves.\verseline
That is what it means to be \cq{ourselves.}
\end{stanza}


\begin{stanza}
I cannot cast it out, but I can rant,\verseline
{\vrsindent}(and ranting, raise a pure shout\verseline
{\vrsindent}heavenward),\verseline
and I can say: life must be found somewhere apart\verseline
from age and staleness and waste, without\verseline
cruelty, without cant.
\end{stanza}

\begin{stanza}
Otherwise there isn't any life, only creeping death,\verseline
and there can be no love except love of ignorance,\verseline
of apartness, irrelevance.
\end{stanza}

\begin{stanza}
\cq{Praise} for this? No. Doubt seasons pain\verseline
under the slapdash shed that keeps us paused\verseline
for a few short winters, fearful of rain,\verseline
while desire wrings the heart, uncaused\verseline
by virtue, unsolved by time.\verseline
Liturgically, my struggle's all a game.
\end{stanza}

\begin{stanza}
But if I flay the truths that tell me so\verseline
I might survive the wounding of my own words,\verseline
find some self-respect with which to know\verseline
a last, terrible music in these drifting surds ---
\end{stanza}

\begin{stanza}
at least I can confess\verseline
the helplessness of my blood\verseline
to rise to the humanness in another's blood...\verseline
the need to infuse
\end{stanza}

\begin{stanza}
with life what cannot be real, to know\verseline
for certain what has never been true,
\end{stanza}

\begin{stanza}
to cross a horizon I have not seen,\verseline
and to return.
\end{stanza}

\begin{stanza}
There, to roll my life's own precious stone\verseline
forward again, and forward again,
\end{stanza}

\begin{stanza}
for the sake of love alone ---
\end{stanza}

\begin{stanza}
and let me say, Amen.
\end{stanza}

\end{poem}


{\poemnewpage}
\poemtitlear{ctg:Elegy}{Elegy for a Child}{ELEGY FOR A CHILD}
\poemdedication{(in memory of Mo'ayyad al-A'raj, aged three, killed at Khan Yunis on August 24, 2014)}

\begin{poem}
\begin{stanza}
You were the child whose life closed before your youth.\verseline
I am the clumsy crier who cannot bring you home:\verseline
I, the father of others, who never knew or loved you,\verseline
the friend who spoke your enemy's tongue,\verseline
the neighbor who had other things to do while you lived.
\end{stanza}

\begin{stanza}
For you, I write a few words of grief and guilt --- for it's\verseline
all I can do; for you will hear nothing I can say;\verseline
for I cannot pierce the bloody tangle of hates\verseline
that strangled your youth, nor even single you out\verseline
among the dead, there are so many, and so much unsaid.
\end{stanza}

\begin{stanza}
When the missiles came and went, and shadows thronged\verseline
in the ash of what remained, I drowned my shame\verseline
in shell-shrieks screaming midnight all day long\verseline
in the blaze of the mind's eye, hung the torpor of my \verseline
ignorance between my heart and your unknown name.
\end{stanza}

\begin{stanza}
Dumb at your death, I strained to pray, but could not.\verseline
I looked for blue skies, but saw they mocked\verseline
your poisoned air and breath --- your city's gaping war ---\verseline
I could not speak, could not explain how I was locked\verseline
in unable dreams, in sickness stranded old and far. 
\end{stanza}

\begin{stanza}
Dishonored in silence, do I wound your silence\verseline
now with words unasked, unheard? Is it pride\verseline
that pricks my hurt, makes helplessness my penitence?\verseline
I only know each word falls farther from your side ---\verseline
thieving my grief, a dull heart's unsacred rite.
\end{stanza}

\end{poem}

\newpage
\vspace*{1em}
\prosesectiontitlenotitle{Index of First Lines}

\poemtitlebaretitle{INDEX OF FIRST LINES}
\poemsubtitle{(links via arrows)}

\begin{pmsection}
\begin{description}[labelindent=2.5em]
\setlength{\itemsep}{0pt}
\poemfl{ctg:Thinking}{If I were to be born again,}
\poemfl{ctg:Burlesque}{What do you write on a wall?}
\poemfl{ctg:Drinking}{The sea is lover to the sand,}
\poemfl{ctg:Silent}{Soundless on aging acetate,}
\poemfl{ctg:Jewish}{Summer has its clammy skin against my window.}
\poemfl{ctg:Kaddish}{Sometimes it makes sense to think about God.}
\poemfl{ctg:Oranges}{Piled in slanting supermarket bins,}
\poemfl{ctg:Lecture}{First, is there anything wrong with you?}
\poemfl{ctg:Three}{I saw young men scramble from an earthen tunnel}
\poemfl{ctg:Clowns}{Ink that \cq{bomb,} Netanyahu, play the public fool!}
\poemfl{ctg:Dead}{This one's on a pedestal,}
\poemfl{ctg:AnUntimely}{I could not die at twelve.}
\poemfl{ctg:Cyrano}{Ah no, sir, you are too simple! Why, you might have said --- oh,}
\poemfl{ctg:Drone}{We believe in what knows us,}
\poemfl{ctg:TheStorm}{Last night I dreamed about a place called Sabbath.}
\poemfl{ctg:CrimeScene}{Item: one wreck of a car.}
\poemfl{ctg:TheSurvivor}{After the dancing stopped, \& she}
\poemfl{ctg:TheWalls}{From Gaza they can't touch it,}
\poemfl{ctg:MyDevil}{Thought I was gone, did you?}
\poemfl{ctg:Delirium}{Where does dreaming end}
\poemfl{ctg:Farewell}{Each time you leave could be the last.}
\poemfl{ctg:Fanatics}{Occupied Palestine, March 2011}
\poemfl{ctg:Pines}{Tread lightly, stoic, these needles bleed.}
\poemfl{ctg:DeathInPalestine}{Where blood drowns}
\poemfl{ctg:WhyItMattered}{If I try to understand a dead man I never knew}
\poemfl{ctg:MyKaddish}{Magnify his name I dare not.}
\poemfl{ctg:Elegy}{You were the child whose life closed before your youth.}
\end{description}
\end{pmsection}


\newpage
%\raisebox{-6em}
%{\parbox{0.5\textwidth}}

\vspace{-1em}
\hspace*{-3em}\begin{minipage}{1.1\textwidth}
\begin{ccframed}
\vspace{2em}	
\begin{description}[labelindent=\parindent,itemsep=10pt]
\item[QR Codes]
QR Codes on each page can be used by readers switching 
between print and digital versions of this document.  The QR figures 
encode bibliographic and page information which can orient 
conformant PDF software in showing the interactive PDF version of each page.
\item[Embedded Data]
For users with conformant cross-referencing software, 
this page has an embedded file in TxHSL (Type Expression and Hypergraph 
Serialization Language) format.  The contents of this file can 
also be accessed \openfilelink{txh.pdf}{{\color{logoCyan!60!black}{here}} 
(if your PDF viewer can read embedded files)}.
\item[Data Description]
The following TxHSL excerpt indicates 
the type interface and sample values of the embedded data.
\begin{scriptsize}
\begin{verbatim}
&type CiteGroup [8;5;2] 
:i:1 :t:2 :p:3 :dp:4 :x:5 ;

&type DataCite {6} 
:i:1 :x:2 :p:3 :dp:4 :ln:5 :t:6 ;

&/
!/ CiteGroup
!/ DataCite
$i: 1 
$x: bot:write
$p# 6 4
$ln: 1
$t. Alludes to the wall surrounding and partitioning Gaza from Israel.
.
/!
<>>
$i: 1 
$t: Burlesque at the Barrier 
$p# 6 4
$x: ctg:Burlesque 
/!
<+>
/&
\end{verbatim}
\end{scriptsize}
\end{description}
\end{ccframed}
\end{minipage}



%\putemendations

%\putexplanatory

%\puttextnotes

\end{document}

