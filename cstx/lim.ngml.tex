\spsubsectiontwoline{Phenomenology and the Limits of Formalization}
\p{This tension is not limited to consolidating
Analytic Philosophy's meta-logical intuition.
The trajectory through Algebraic Topology and Category Theory
has been one vector of post-1950 mathematical progress.
Another has been computer proofs, proof-assistants, and
computer-inspired approaches to mathematical foundations, like
Homotopy Type Theory.  Regarding \i{types} rather than
\i{sets} as the primordial elements of logic and mathematics
positions mathematics and computer science as two different
worlds evolving from a common type-theoretic universe.
In this environment, mathematics as abstract thought gets
mixed with mathematics as a technological discipline;
specialists for example debate the merits of mathematical
systems formalized with a priority to automated proof-checking
vs. presentations more aligned with mathematical precedents.
The proper place for technology and automation becomes
debated.  Computer theorem-provers after all are not oracles;
they need their own quality-checking, their own design theory.
Do we trust the theorem-prover (itself a human artifact) more
than the conventional deliberations validating a proof
in the eyes of mathematicians (and if so, why)?  Are design
principles of proof-assistant software and languages topics
for mathematics or computer science?  Does a test suite to
demonstrate a proof assistant's trustworthiness bear witness
to mathematics becoming suddenly an empirical rather than
eidetic science?  Does the empirical contingency of the
correctness of software used to provde the 4-color theorem
make the theorem something other
than \q{synthtic \i{a priori}}?
}
\p{There are analogs to these questions in the realm of
Analytic-Philosophical logic.  For example, the idea that cognitive
or linguistic meanings can be reduced to first-order predicate
formulae can be investigated by considering computer technology
which does in turn model digital content, in effect, as predicate
assertions together with rules of inference (that can
in principle be modeled via first-order logic): the Prolog
programming language, for instance; certain database
query systems; the Semantic Web and \q{formal Ontologies}.
These technologies operationalize logistic paradigms, but they
also put an empirical face on a logico-mathematical background
that for Russell, Frege, and Wittgenstein would have been
abstract speculation, akin to how proof assistants
put an emprical tie on abstract mathematics.  Similarly,
formal linguistic approaches (like Categorial Combinatory
Grammar) base themselves on abstract logico-mathematical
paradigms (like Hindley-Milner type sysetms), but by 2019 we
have a panoply of linguistic schools whose models
are realized in computational linguistic technologies,
digital platforms that represent and/or automatically
identify semantic and grammatic patterns according to a
specific thory of language.  In this context, type theory
informs linguistic research at a theoretical level as part
of its logico-mathematical underpinnings, but it also belongs
to the technological infrastructure wherein linguistic
technology is implemented.  These two roles can be complimentary:
insofar as Classical Hindley-Milner type theory has been
supplanted by competing paradigms in software which proves
technologically advantageous for modeling natural language,
we can explore whether more modern type theory is
superior for linguistic analysis at a purely theoretical level too.
}
\p{Insofar as Analytic Philosophy converged in the last century upon, it
seems, a fairly consistent nucleus of logical paradigms
\mdash{} covering the various intrinsic facets of quantifier-predicate
logic \mdash{} the heritage of that philosophy is similarly
rebased by 21st-century technology.  The persistence of first-order
logic as a reified example and medium of logic as such was driven by
the abstract reality that almost any logical system can be translated
first- or secod-order logic in principle (at least if we can
tolerate whatever metaphysical posits may be needed
to ground the resulting logic expressions, like Possible Worlds).
As long as logic belongs just to pure thought, the formal/theortical
isomorphism between superficially different theoretical
architectures can be treated as a challenge to overcoming:
converging on first-order predicate-quantifier logic
as a common ground beneath various shallower logics may seem
(and perhaps did seem to early Analytic Philosophers)
to be explanatory progress, a theoretical repositioning
which really does evince a situation where philosophical
disputation really can progress toward greater clarity.  However,
now that different logics are indeed manifest in a concrete sense
in different digital technologis, their contrasts no longer seem
so shallow.  It could be argued that the illusion of a \q{deep}
logic underlying superficial variation (like the illusion
of a \q{deep grammar} stabilizing the morphosyntactic surface,
perhaps) is itself the superficial presumption, failing
to duly appreciate surface contrasts which are only manifest
in the register of applied technology, rather
than abstract thought.
}
\p{For example, one can make a strong
semi-theoretical and semi-implementational case that Graph Theory
or Process Calculii are a better foundation for projects like
programming-language compilers and database query systems than
conventional predicate logic \mdash{} notwithstanding that
propositions in graph or process algebras can potentially be
reformulated in first-order logic (and vice-versa).  The
issue with such in-principle reductions is that they are of
dubious explanatory merit; the pro-forma restatement of
graph structures in predicate logic, for instance, has no
particular importance unless we are prior committed to
predicate logic being somehow fundamntal, in which case
the reducibiloty of graph theory (since this reducibility
can be inverted \mdash{} we can provide graph models of
attribute predication, boolean operators, etc.) can't
justify the commitment on pain of circularity.  What
\i{can} be observed, however, is that graph-database
queries cannot be optimized in their first-order
translations (which is why graph databases are a different
technology than relational databases), which in turn suggests that
the structural differences btween graphs and propositions are
philosophically non-trivial, and consequential.
}
\p{Similarly, concrete technologies can point
us to an intuition of
sentence-comprehension as a calculus of interlocking
cognitive processes, not the mechanical unzipping
of a predicate compact, seeing that there are substantial
implementational differences between process calculii and
predicate logic as paradigms for programming languages.
In short, technology reveals how formal systems may be
substantially divergent from predicate logic even if the
logical structures governing their operation have
isomorphs in predicate logic: technology reveals that such
\q{in principle} isomorphs are less important, less
indicative of philosophical depth, than we may have believed a
century ago.  But this is revealed not within philosophy
alone, but in an interdisciplinary spirit where insights are
mined from fields like compiler design and database engineering;
and from practical experience as well as speculation.  We can
appreciate the substantiveness of
\q{surface-level} logico-structural variation when we actually
\i{write} compilers or database query engines.
}
\p{The phenomenological de-centering of metaphysics does
not mean a complete abdication of metaphysical priorities,
of the need for a ledger to assess the reach and limits
of thought, or science or intellectual explanation;
to place the norms of philosophical exposition and
conversation on a disputable, theorizable conceptual bedrock.
Bracketing metaphysics does not mean suspending what
metaphysics is supposed to do to philosophical performance.
But Phenomenology seeks to bracket metaphsyics as something
received, habitual, or institutioanlized; something that
influences scholarship more than it is studied.
Phenomenology brackets metaphysics in the hope that it can
be recreated and return, but more transparent and self-aware
than before, like a professional athlete who returns
from a league suspension chastened and technically sounder.
Metaphysics is suspended but not expelled.  There is an
analogous dialectic in open-source progrommaing;
institutional norms for software quality are both suspended
and resinstated in a more transparent, technical
incarnation.  The de-centering of \i{institutional} controls
does not foreclose a re-centering of scientific
convergence in a shared technical understanding of the
project.  Similarly, the decentering of \i{institutional}
metaphysics \mdash{} not just the ideas, but the academic workflows
and publishing conventions that support them \mdash{} does
not foreclose a recovery of metaphysics in the
philosopher's relation to her own consciousness, in
writings as an existential trace more
than a social production.
}
\p{All of this then produces the question: what should metaphysics
be if it is not the paradigms we have received?  Where
does a phenomenological analysis take us that we were not before
it started?  As with any writing, we have to start with
the question: now that the writing has finished,
what has changed?  Aniticpating the writings of the future:
what do we want to have changed?
}
\p{We have to communicate this to Analytic Philosophers
(and practitioners of many other branches of scholarship,
like cognitive science and linguistics): what changes
when a phenomenological study is complete?  What
changed when there was a \i{Thing and Space} or a
\i{Phenomenology of Perception} compared to before
there was?  Failure to understand what a writing
wants to change can easily yield to failure
to understand the writing.
}
\p{I have often felt that Analytic and phenomenological
practitioners were talking past each other precisely in
this sense: a specialist in the philosophy of mind or
of science might approach a reading or conversation
thinking the phenomenologists is working to one kind
of telos, while the other is going somewhere else.
In apparently the most common iteration, the
Philosopher of Mind thinks the phenomenologist is aiming
toward a kind of skeptically inflected science of
perception, a theory of the deep structures behind
perceptual experience that potentially takes us further toward
an unbiased and even scientific theory by bracketing
presupposions about how consciousness and cognition
are supposed to work.  The Philosopher of Mind might
then think that phenomenology should, let's say,
uncover the synthesizing and interpretive acts that
constitute my raw sensate awareness as the grasping
and judging that Hugo is sleeping on the sofa, or that
it is half-past four and the sun is setting outside, or that
the car double-parked outside is red.  By attending
in a hopefully presupposionless way to the immediacy
of consciousness, we can see the underlying stitching
and unifying that normally passes beneath the threshold
of awareness, as if we were observing our
own consciousness like a shaman on ayahuasca.
}
\p{I have never in a philosophy classroom, nonetheless, felt
that my role was to issue oracular pronouncements as if from
a drug-induced trance.  Consciousness is not usually
extravagant and cosmic; my awareness of Hugo sleeping
on the couch is, really, not much more than
just Hugo sleeping on the couch.  Notice the structural
analogue to logistic philosophies of language: just as
adding logical symbols to \q{Hugo is sleeping on the sofa} does
not guarantee we are uncovering some deeper structure in the
language, some precious scientific insight, nor does
superimposing talk of a Passive Synthesis of Perception or
Hyletic Immediacy on the mundane episode wherein I see
Hugo sleeping on the sofa make that perception suddenly
scientifically tractable, as if we can solve the
mind-body problem just by concentrating hard enough.
It is what it is.  The things themselves are the things
themsevles; if we're going to them we can't assume we're
en route to some revelation or science, as if the
laws of consciousness are present in the room alongside
Hugo, the sofa, and the lamp.
}
\p{But a reframing of the sentence \q{Hugo is sleeping on the sofa}
\i{can} take us at least modestly further toward a good theory
of language, something beyond the disquotational
verity that a sentence means what it means, just as
\q{it is what it is}; \q{Snow is white} means that snow is white.
Similarly, I believe reframing perceptual episodes can
shed light on the formal \mdash{} and perhaps even scientific
\mdash{} structures of cognition, but we have to clarify ahead
of time what magnitude of clarity we are hoping for;
what precision of understanding landmarks a successful
analysis.  If a scientist reads phenomenology
hoping for an \expose{} of consciosuness that is both
first-person accurate and naturalistically revelatory
\mdash{} something experientially faithful like a
physchological novel but clinically analyzble
like a Brain Simulation \mdash{} then what is actually on
offer will fall short.  Both attending to consciousness
as it is experienced and honoring professional and
scientific standards of intellectual rigor are guiding
principles of phenomenological research, but these
are principles, not minimal standards like
a one-drink minimum at a night club: a very good
phenomenological writing will \i{not} truly be
either stream-of-consciousness realistic like a
novel nor scientifically complete like a Neural Network
platform presented at a computer science conference.
The point is via phenomenology to move a little closer
to both experiential honesty (to accept
consciousness as it is experienced, not as something
we hypothetically transform to fit cognitive and
physical explanations) and scientific rigor
(both the science of neurophysical explanations and
the science of cognitive or intelligence simulations),
but while doing so acknowledging the incompleteness of
both projects and the tensions between them.
}
\p{In the case of perceptual synthesis, we have good reseon
to believe that biophysiology and experimental psychology
gives us a well-informed acccount of perceptual phenoema,
in their episodic unity, at the neurological and
sensori-motor levels; presumably I have some brain
function that unifies white-sensations and brown-sensations
and distributes the color regions to different concepts
(sofa, Hugo).  Since I'm not aware of these
levels they are not phenomenological subjects per se
(which doesn't mean we should reject them entirely,
since I have reasonably firm beliefs about the scientific
basis of perception, beliefs which I don't deliberately
question any more than I deliberately apply them to
mundane perceptions).  Perhaps our scientific training
casuses our actual perceptual content, at last sometimes, to be
subtly different in our experience of it.  I am reminded
of the rainbow color of a thin slick of spilled oil:
knowing to some approximation how we think that phenokenon
occurs makes me explicitly associate the rainbow with the
oil, and therefore to almost unconsciously direct my
gaze to the border lines where the colors are brownish and
less pronounced.  Somone who did not impose a mental
explanation on what they're seeing, even if unintentionally,
would perhaps attend more intently to the color and
experience it more vigorously.  So scientific beliefs
are relevant in the limited sense that their trace can
sometimes be detected in actual experience.
With this qualification, however, any scientific
story about how percption \i{works} lies outside
phenomenology proper.
}
\p{Over-analysis, \i{excess} clarity, \i{too much}
resolution distorts experience even if it aims
to explain it.  So phenomenology does not neatly
fit within a scientific paradigm where magnification
often leads to elucidation and discovery: we cannot take
a microscope to consciousness and find new candidates
for scientific treatment, the way we can advance science
by making more powerful telescopes.  For a scientist
or someone with scientific intiutions it can seem
hard how to go forward: you don't study cancer by staring
at tumors; so how can you study consciousness without some
analysis, some distortion and defamiliarization?  Otherwise
we seem fated to endless circularity: \q{it is what it is}.
The things themselves are what they are.
}
\p{This point, relative to perceptual synthesis, has
similar forms in other aspects of cognitivity activity:
judging that the animal on the sofa is a cat, indeed Hugo,
that he's sleeping, that it's daytime becauuse it is
light outside, that the men in uniform skating on
an ice rink on television are hockey players, and so forth.
Phenomenology cannot ignore that many of the perceptual
and conceptual judgments that we are aware of take the
form in consciousness of settled fact; we can't do first-person
analysis of chains of reasoning that we don't actually
experience ourselves performing.  So while
presumably there is a detailed story to be told about
how optical and inferential thought-processes lead
us to the conclusions that it's daytime and that Hugo
is sleeping, the perceptual record shows that
none of this labor was special enough to warrant conscious
attention.  It's not like I first speculated that some
unfamiliar creature was on the sofa, and subsequently
concluded that it's just Hugo; or that I had to stand
and ponder who the cat is until I could assemble
enough visual detail to reach that conclusion.
I honestly don't know what I was unconsciously
thinking to presume passively that it's Hugo \mdash{} honestly,
I usually don't know what I'm unconsciously thinking.
I cannot, accordingly, use unconscious or
too-mundane-to-experience judgments as part of
phenomenology at least during argumentational stretches
where I'm sticking to phenomenological reports.
}
\p{Insofar as this is characeristic of phenomenology in general,
we should not expect that phenomenology will advance
us further to some sort of rigorous philosophy
or experientially faithful science by revealing things
about experience that we have not yet noticed due to
lack of attention: we cannot magnify consciousness
the way a microscope enlarges different things and a
telescope reveals distant ones; metacognition
and first-person reflection don't enlarge or
polish the things in consciousness so that we can
describe them more thoroughly or scientifically.
Phenomenology does take us to some new place by
making us aware of pieces of experiential synthesis
that we don't experience directly otherwise,
as we don't experience the molecules in
water; or minutiae of conceptual
reasoning that we would otherwise take for
granted, like how we figure out when an
animal is a cat.
}
\p{Of course, however, often we \i{are} aware of
perceptual synthesis and conceptual judgment.  If
after seeing Hugo I walk toward him,
with my perspecive gradually changing, I
consciously register a change in how Hugo appears to me
which produces an explicated synthsis,
a self-conscious accumulation of visual evidence telling
me about Hugo's current state, at least if I am
looking at him with express interest him as opposed to looking in
his directly and passively noticing him while attending to
something else, like the television.  Likewise I may passively
observe that he is sleeping if I previously thought he
was sleeping and just happened to glance at him again,
but there's an alternative scenario where I don't
specifically assume he is sleeping, but upon looking
at him rather attentively, with his lack
of movement and his relaxed posture, I reach the
conclusion self-consciously.  Walking outside I may
see a familiar stray cat and put no thought into recognizing
her, but I may also spot a movement in the distance
and look purposefully before identifying the animal as a cat.
In short, perceptual synthesis and conceptual
judgments are neither \i{intrinsically} conscious
or pre-conscious, active or passive; many
cognitive episodes that we are explicitly aware of in their
unfolding could, in slightly different circumstances,
remain below the threshold of conscious awareness.
Conversely, many judgements we make without
much conscious attention could alternately be
experienced with more deliberation and attention.
Indeed, sometimes we revise our passive judgments
when something goes awry in our passive estimation of
what's going on around us \mdash{} we assume we hear
the dog barking in the back yard, but then
we see her eating pizza crusts in the kitchen, so now
we have to account for the mysterious barking outside.
We all have the faculty to quickly adopt the
mode of conscious investigation, trying to gather specific
information, when the passive judgments we have just
assumed to be true turn out to have some incongruity, some
gap in their logical order.
}
\p{Now we are moving toward some insight, since although we
can't give a phenomenological account of preconscious
judgment, we can at least confirm that \i{some}
judgments are \q{active} and part of
explicit awareness, while others are passive and
not (in any inner details) noticed; we can also say that
at least some passive jugments seem similar in form
and outcome to active ones.  This raises several
avenues for further, maybe scientific continuation:
why do we have both active and passive judgment?  What
benefits accrue to us biologically as animals that depend
on precise judgments to survive that we make many
instinctive judgments and some additional deliberated ones?
Why are active and passive judgments often similar?
What triggers a decision to actively attend to specific
components of experience so that associated judgments in
that one area will, at least in the immediate aftermath,
be more active than passive?  What is actually
neurophysically different in the two cases so that,
in a pure subjective and first-person sense, I know when
a judgment is active and passive?
}
\p{Furthermore, since many judgments \i{do} rise to a level
of consciousness \mdash{} and not just the minimal awareness
of believing such-and-such but some explicit experience of
how the judgment is formed, of their inner pieces and
intemediate stages \mdash{} these conceptual and perceptual episodes
\i{are} candidates for phenomenological analysis.   We are
certainly aware of perceptual episodes in their
unfolding; how perception scans over a room, or
any other surrounding, like the pan of a camera;
while, also, different episodes piece together,
with shifts in attention (and, often, kinaesthtic
actions) \mdash{} like looking up from a book \mdash{}
juncturing disparate perceptual phases.  It seems self-evident
that perception unfolds according to multiple time scales,
where episodes measured in seconds join to become the perceptual
background as we engage in activities over the course of minutes,
within errands that stretch toward hours and engagements
over the course of a day (we cook, go to the store, ride
a subway, watch a show), and of course we plan for a future
and remember a past over monrhs and years.  Cognitive acts and
perceptual interludes can be slotted into different time
scales, affecting how we experience them and should analyze
(or even name) them: a perceptual episode, brief enough to appear
in consciousness as a single event, needs a different
theoretical registration than experiences implying lengthier
segments.  Practical actions, also, bridging
perception, conceptualization, and planning, sort into different
time-spans: opening a bottle vs. preparing a meal;
petting the dog vs. walking the dog.
}
\p{We have, as such, I would argue, a collection of observations
and distinctions we can make about consciousness and
conscious experience, that emananate from consciousness
itself and not from prior scientific or
metaphysical commitments.  These observations
seem presuppositionless, or at least common-sensical.
We should not take that too far: the \q{suspension of
belief} should not commmit us to a virgin brilliance
exorcising millenia of thought, posturing us as priests
of intellectual purity that can be defrocked by the
smallest trace of naturalistic thinking \mdash{} gotcha, a
presupposition!  Philosophy as volumes forging profound
insight from pure thought alone seems silly in our time,
more art created from ideas than rational exposition.
But we can make a good-faith effort to present
patterns in consciousness based on rather casual and
common-sense reflections not explicitly aligned with
scientific thories, or with schools of philsophy,
and I claim at least some of the intellectual structure
we can get to via that route takes the form of observations
I have highlighted: passive vs. active judgment; the
various scales of perceptual and enactive temporality; the
individuality, aggregation, and junctioning between
perceptual episodes.
}
\p{Insofar as these observations
and distinctions form the rudimentary constructs in
a theory of consciousness, the explanatory value
of that theory will depend on how it can be
consistently extended.  Remaining within phenomenology,
we cannnot analyze deeper into consciousness \mdash{}
capturing more fine-grained layers of perception, uncovering
more details of passive synthesis \mdash{} because we cannot
impose scientific-empirical data or philosophical speculation
on subjective experience.  So if we can only articulate a
handful of organized but commonsensical comments about
experience, we need to find away to make the analysis meaningful:
for it to make a difference.
}
\p{One way to do this, I believe, is to make the common-sensical
observations more than just observations; not in the
sense of importing interpretive claims that don't
unfold from them organically, but in the sense of trying to
express their as a formal structure, something which can
be then set in relation to other structures and other
accounts of mental processes.  There are many formal systems
that have been proposed to cover various features of
cognition and intelligence; but only a few
of these, I suspect, are formal views on experientially vivid
acts of mind; on judgments or percpetual syntheses that
we can experience in their unfolding and reflect upon
first-personally.
}
\p{There is a considerable body of cognitive-scientific
literature that aspires to formalize (and sometime simulate)
mental activity via structures that have some blend
of logical, mathematical, and computational
specification.  There is also a considerable body of
literature that remains true only to the first-person
perspective, not explicitly trying to align with
scientific explanations and world-view.  But where
these bodies overlap I believe lays the potential for
overlapping theories, like a Naturalized Phenomenology.
}
\spsubsectiontwoline{Phenomenology as Experiment}
\p{Phenomenology has not usually attended to formal analysis of
cognitive models, or models of mental phenomena overall.
This is entirely undersandable, since exploration of formal
systems surely does not belong to phenomenological
analysis itself, at least outside the exotic sense of studying
what it is like to learn or think about mathematics, say,
from a first-person perspective.  Nevertheless, even a
writer committed to phenomenology does not solely do
phenomenological analysis, any more than a Structuralist
poeticist devotes all paragraphs to close-readings.
Equally intrinsic to the philosophical process is
how intra-methodological analyses (\q{readings}
on consciousness or literary works, say, respectively) are
placed in a larger context, which can have multiple
dimensions: relations to other scools of philosophy, but
also other academic disciples and other regimes of knowledge
and practice.  Pursuit of formal models can help ground and
orient phenomenological investigations, and vice-versa.
}
\p{As a case in point, consider how Phenomeneology was among the
inspirations for what computer scientists call the Semantic
Web and Formal Ontologies \mdash{} protocols for sharing information
and aggregating \q{knowledge} across computer networks,
paricularly the World Wide Web.  Such information is not only
abstract mathematical or technical data: our world is founded
on communicating structured knowledge from many human and scintific
domains, like medicine and government.  We cannot in this
context construe data as just bytes and numbers, but rather
encodings of human concepts and judgments.  Trying to
map essentially informal human concepts, like \i{illness},
to a precise scientific formulation, is of course a
foundational concern in philosophy; but computer networks and
technology-enhanced knowledge sharing reveal practical applications
of these perhaps once purely abstract problems.  Scientists
use Formal Ontologies to codify conceptual systems
underlying human knowledge and beliefs.  This, moreover,
spans both fairly narrow and concrete propositions (e.g.,
that the pericardium surrounds the heart) and deeper, more
abstract, more cognitively primeval concepts
(like what it means for one thing to surround another thing).
}
\p{In short, part of the requiements of building knowledge
\q{engineering} or \q{sharing} systems was to give a technical
specification for apparently innate concepts or gestalts
like \q{part of}, \q{inside of}, \q{surrounding}, and
so on.  At least one method for appracohing
this problem was via phenomenology, meaning that
phenomenology serves as one tributary among others that can
be followed into the technical context of data modeling and
data sharing protocols.  At the same time, the Semantic Web
community has converged on several preferred models and
formats \mdash{} bearing acronyms like OWL and RDF \mdash{} which
in turn have proved controversial.  Some
researchers, notably G\"ardenfors,
have critiqued the Semantic Web for a conceptual simplification
that does not really capture the semantics of technical domains
(like science and medicine), still less of Natural Language.
Others from a more implementation-minded perspetive
can highlight technical limitations of Semantic Web models,
such as how data sharing \i{between} computing environments can
best intergrate with data managment \i{within} particular
databases and applications.  The Semantic Web \mdash{} whose
underlying representations are based on labeled, directed graphs
\mdash{} has been critiqued by advocates for modestly different
graph-like structurres, like Conceptual Graph Semantics and
Directed Hypergraphs.  These alternative models are arguably
more conceptually accurate and/or more practically efficacious,
from an engineering perspective, than the paradigms for
representing Formal Ontologies and Informations Spaces
or \q{linked data} which emerged as predominant in the
community of Semantic Web developers and researchers.
}
\p{These unfolding narratives are relevant to phenomenology
partly because competing representational paradigms
can seem more or less phenomenologically faithful;
can intersect with phenomenological accounts in different
ways.  For example, suppose we judge that an
alternative representational model, like the Hypergraph
framework associated with the OpenCog project \mdash{} itself
oreinted to Artificial General Intelligence \mdash{} is a
more realistic model of conceptual sructures insofar as
these intrinsically emerge from and regulate
cognitive/perceptual processes as articulated
in phenomenology.  That is just a claim, of course, but
assuming for sake of argument it is plausible, then we
have a case of two competing formal models \mdash{} both
reflecting some measure of at least informal influence
frpm phenomenological ideas, as far as seeking
philosophically well-grounded accounts of ontology and
intelligence.  These models can be contrasted on
philosophical grounds, but also technologically.
Neither OoenCog nor the overall Semantic Web are academic
theories per se, but technology projects with their
own software ecosystems and engineering norms,
alongside academic literature and philosophical
attitudes or intuitions that can be evaluated
theoretically.
}
\p{Even though phenomenology is \i{philosophy}, I believe,
doesn't mean that considerations from other disciplines
as they bear on \mdash{} to take this one specific case \mdash{}
OpenCog vs. the canonical Semantic Web are not potentially
relevant and interesting to the phenomenological project.
The relevant contrast here presents two competing
computational frameworks, and the ecosystems
can be scrutinized side-by-side with an eye to
the merits of their technology, to how they
are used, extended, and integrated into practical software
and respond to technical requirements.  These engineering
comparisons can co-exist with more philosophical ones:
if one paradigm seems superior \i{both} practicallly
and philosophically, this deserves consideration \mdash{} do
the two horns reinforce each other?  Conversely,
is the more phenomenologically faithful theory
proves less implementationally useful, does this
shed light on philosophically interesting issues
\mdash{} the weakness of \q{mind as compuer} analogies,
for instance?  We may not be able to specify \i{a priori}
what kind of significance to attach to comparisons between
formal models on phenomenological vs. engineering
grounds.  But we should recognize
that technical comparsions are at the least nontrivial data
points that should at the least be acknoedged in the background
while investigating formalizations that incoporate,
and insofar as they incorporate, phenomenological
intuitions.
}
\p{If this is plausible, then the disciplinary frontier of
phenomenology significantly expands.  Phenomenologists
can in any cases engage with the \i{academic} face
of, say, OpenCog and Semantic Web projects
\mdash{} texts by Ben Goertzle or G\"ardenfors,
for instance \mdash{} read as at least tangentially philosophical
ouevres.  But any discipline which finds relevance in
these \i{writings} should also find relevance in
the technology they are writing \i{about},
in their concrete form as technical artifacts and
(products of) engineering processes.  We can therefore
approach technologies like AtomSpace (a database associate
with OpemCog) and Fact++ (a Semantic Web engine)
from an engineering as well as theoretical perspective
\mdash{} how they are implemented, compiled, and interoperate with
other applications.  Or, as this one example illustrates,
we can incorporate within the phenomenological discipline an
interest in technical comparison between formal systems
which also manifest phenomenological intuitions, so the
phenomenological and engineering dimensions of their
juxtaposition can be juxtaposed in turn.  This represents
a new avenue for engagement with formalizations following
the example of, let's say, Husserl's \i{Formal and
Transcendental Logic}, which approached from a
phenomenological perspective then-contemporary issues in
logic and mathematics.  But a key difference is that the
formalizations Husserl entertained were fully abstract, while
the formal systems that can be approached phenomenologically
in our century are more concrete and enmeshed in
social-scientific practice (health care, environmental
policy, etc.).  Engaging with these \q{concretized}
formal systems is not a matter of proving theorems,
or undersanding proofs of prior theorems; it
more involves compiling computer code,
or writing new code, and understanding the tchnical
structure of programming languages and data representations.
}
\p{In effect, computer code \mdash{} software and formal
languages (including markup and database query as well as
general-purpose programming languages) \mdash{} has in many
contexts supplanted abstract logic as formal foundations
of well-structured thought.  This even applies
to mathematics, where type-theoretic and proof-assistant
methods take the place of set theory and predicate logic
as foundations.  This new reality should be confronted
by philosophers as well \mdash{} what is the philosophical
analog of the Univalent Foundations project in mathematics?
How should Analytic Philosophers \mdash{} or indeed
phenomenologists \mdash{} re-evaluate the last century
with computers replacing logic as the instuitional
mechanization of thought?  How should Philosophy
be reconsidered if some founding books were reimagined
as, let's say, the \i{Formal and Computational
Computer Science} or the
\i{Tractatus Computational-Philosophical}?  To speak more
precisely, what are the consequences propagating across
Philosophy if quantification is foundationally
conceived as over type-instances rather than sets?  What
changes when the domain of a quantification has to have a
conceptual unity at least to the level of what can be modeled via
a formal type theory rather than being open-ended sets?
What changes when our reigning paradigm of perfect
reasoning is not proofs as a mental exercise, but the
engineering of computer systems and then the engineering
of (formal representations of) theories and then of apparent
theorems so as feed theories and theorems  into the aforementioned
computer systems for verification?  What changes when even mathematics
becomes a rather empirical domain that can be experimented
with on a computer?  What kind of Philosophy can
be done by experimenting on a computer?
}
\p{}
