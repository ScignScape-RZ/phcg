\spsubsectiontwoline{Phenomenology and the Limits of Formalization}
\p{Considering how Analytic and Continental Philosophy 
has diverged, it is easy to forget that the early 
mature works of Husserl \mdash{} arguably the root 
stock of 20th-century \q{Continental} thought \mdash{} 
emerged from the same circle of logico-mathematical 
questions and evolutions as the ideas 
of Frege, Russerl, Wittgenstein, et. al. that 
would evolve into \q{Analytic Philosophy}.  
Nor is it necessarily true that analytic philosophers 
have remained closer in spirit to the unfolding 
horizons of mathematical knowledge in
the 20th century.  
}
\p{The perpetuation of 20th-century philosophical tropes
\mdash{} Analytic vs. Continental Philosophy; humanities vs.
logicao-mathematics \mdash{} perhaps diagnoss something incomplete in
20th century thought that the 21st century needs to transcend, but
it also bears witness to intellectual structures that directly
abut phenomenology itself.  The idea, for example, that
logico-mathematical analysis is distinct from cultural-humanities
theory \mdash{} that works of art, for example, do not have logical
or mathematical forms; that the structures evinced by cultural
and artistic productions (and any system of signs and communication
necessarily has \i{some} rigorous structure) does not itself
have a logical and mathematical dimension, or a structural
form that if specified in isolation from its signifying context
would reveal logical and/or mathematical properties.  A parallel
assumption is that logical and mathematical structures need to be
expressed \mdash{}and indeed \i{located} \mdash{} via a few canonical
theories, such as first-order predicate logic, first-order
quantification, and (perhaps modally inflected) set theory.
So 20th-century Analytic Philosophy unfolded in a
kind of semi-explicit dialog with disciplines such as
linguistics and computer science which promoted, from the
full spectrum of mathematical domains, a handful of
specific paradigms \mdash{} like Zermelo-Frankel Set Theory,
Tarskian quantificational-predicate logic, and
Hindley-Milner type theory \mdash{} that in turn became equally
paradigmatic in fields like formal language analysis and
\q{knowledge enginering}.  This is not to imply that
linguists and computer scientists took their marching orders
from philsophers \mdash{} it's probably equally true that Analytic
Philosphers gravitated to these kinds of mathematical frameworks
because they were proving empircally useful in other fields.
But we can say that a group of disciplines aiming to
technically examine subjects at the intersection between
human thought and formal systems \mdash{} language, knowledge,
valid inference, Artificial Intelligence \mdash{} collectively
progressed toward a vision of what logicomathematical
structure \i{is}, a paradigm (or network of paradigms)
which undergirds any reception of phenomenology insofar
as a \q{Naturalizing Phenomenology} project seeks to
initiate dialog and comparisons between how phenomenology
and Analytic Philosophy, respectively, treat logic,
mathematics, or formal-systematic explanations in general.
}
\p{Interestingly, the logico-mathematical formulations that
arguably have come to dominate methodology on the scholarly
\i{frontier} between science and humanities \mdash{} lingustics,
cognitive science, computer science \mdash{} are different from
those that seem to dominate mathematics and physical sciences
themselves.  Of course, some logical form is intrinsic to
any formal method since formal systems cannot be blatantly
self-contradictory.  But the vehicles by which criteria
of logical consistency are expressed in, say,
linguistics and philosophy \mdash{} presenting sentence-meaning via
first-order logical glosses of their predicate
(locutionary, assertorial) semantics, for instance, or
equating meanings to the conditions in the world
(or possible worlds) that would make linguistic expressions true
\mdash{} these first-order and set-theoretic intuitions do not
necessarily coincide with working mathematicians' visualization of
the spaces where their own axioms and formulae are in force.
Category theory, for example, recognizes that set and predicate logic
are in a sense emergent properties of the different formal domains
where mathematics is enacted \mdash{} sheaves, topoi, manifolds, sites, and
so on \mdash{} which are not isomorphic, propagating to variation
in how logic itself is understood.  Insofar as there is no single
formalization to specify what a \i{predicate} is, or a \i{set},
we are on shakier foundation if we want to apply these notions to
conceptual or linguistic investigations \mdash{} to consolidate the concept
\q{cat}, say, as the set of all \i{felix catus}, or
the meaning of \i{Hugo is sleeping on the sofa} as a particular state
of affairs pertaining wherein a certain tabby cat sleeps on a
certain piece of furniture.  The problem with this
sort of logico-mathematical gloss is that it either employs sets and
predicates in an informal manner \mdash{} which brings us back to
the circularity problems I analyzed earlier, since
an honest assertion that Hugo is sleeping on the sofa obvious
intends to report, modulo the imperfections of informal logic,
that as a point of fact Hugo is sleeping on the sofa \mdash{} or
it tries to elevate the logic from informal to formal,
which legitimately then adds explanatory value to our technical
grasp of the sentence, but then we get tangled in questions like
what sets are.
}
\p{To put it differently, it certainly seems as if the
sentence \q{Hugo is sleeping on the sofa}, as an intended and
sensed artifact in a human world, in some fashion encapsulates and
\i{designates} a proposition, and/or the state of affairs, that this
tabby cat is in repose on that piece of furniture.  The sentence
(deliberately produced sound patterns, perhaps) provides us
with mental access or handle to that proposition, making it an
unitary and isolated object of thought, as well as a thought that
can be shared between people via language.  There is a predicate
there that seems to be \i{intended} in the phenomenological sense
by the sentence as a linguistic gestalt, just as Hugo is intended
in our visual consciousness when we explicitly see him on the couch,
or at least analogous to that regime of intentionality
insofst as Husserl finds both parallels and contrasts between
\i{apophantic} and preceptual-integrative analysis.
That is, the predicate or fact of Hugo's sleeping at that spot
is the noema to the sentence's phenomena, just as Hugo is the
noema to our seeing-Hugo.  In this sense it is non-circular to
proceed with \q{Hugo is sleeping on the sofa} to mean the
fact of Hugo's sleeping on the sofa, since we are exploiting the
seemingly reundant repetition of the sentence to designate
indirectly a predicate which we have no way of expressing except
through the sentence.  So there is \i{something} there that seems to
have a logical form we need (at least if our goal is theoretical
completeness) to cast light on.
}
\p{But when we then pass from informal to formal logic
we encounter the gap separating mathematics circa 2019 from
mathematics circa 1919.  Hugo, for example, is a cat,
which (in the precincts of informal logic, i.e. a logic
solid enough to shape communication and coordinations of
thought but backgrounded as only one of many structural
parameters delimiting how thoughts and the world avolves)
means that Hugo is a member of the set of all cats.  But
what is a set?  Garfield (a fictional, comic-strip cat)
is also a cat but he does not \q{exist}, so the set of all cats
cannot be enumerated just by (even as a mental exercise) pointing
to them one-by-one.  Moreover, what is referenced by the word
or idea \q{Hugo} in the thought that Hugo belongs to the set of
cats?  We can look at and point to Hugo to get a rough idea; but
the fur Hugo just shed is not a member of the set of cats, even
though it is (or was) a part of Hugo.  It is by no means
obvious that we can, even as another mental exercise, define
\q{Hugo} by an enumeration (metaphysically possible even
if logistically infeasible) of hairs, or cells or molecules, that
via mereological summation grounds the name in a
logically well-established referential framework.
}
\p{On course, it is possible to logically restate a sentence like
\q{Hugo, who is a cat, sleeps on the couch} in first-order logic,
which seems to be arriving at a formal exposition of the predicate
informally intended by the sentence \mdash{} the noema to the
sentence's phenomena.  But formal machinery
only papers over the logical incompleteness of concepts like
\q{set} and \q{predicate}: a logical denotation of the sentence
\i{seems} to refer to the state of affairs that the sentence
tokens (and not just the sentence itself) via quantifier
symbols and other logic artifacts.  I'd argue, however, that these
elements are meta-discursive cues rather than theoretical posits:
they invite us to entertain the thought of the predicate lurking
beyond the sentence via a departure from discursive norms,
the same way that simply repeating the sentence does.  In short,
it may be non-circular to gloss \q{Hugo, who is a cat,
sleeps on the couch} via formations like \ExistsX{} and the
logical form gets us closer to apprehending the abstract predicate,
but this is not because of logical symbols like \q{\Exists{}}
in themselves, but because of their rhetorical effects as
interrupting the flow of conventional philosophical writing.
Symbols of quantification, set membership, or part/whole
inclusion only carry the imprimateur of logical rigor
insofar as we can synchronize what flavor of mathematics
we are using with which to theorize quantification or parthood.
}
\p{We can, for example, argue that Hugo is an extended
three-dimensional thing (or maybe a four-dimensional
perdurant, since he's still here day after day), and
any reasonably compact and continuous \ndim{} manifold can
potentially be designated in thought and language.  This is
reasonable, but it positions our infrastructure for the semantics
of proper names (and a large class of referring expressions)
against a topological backdrop in which we can talk about sets
of points, or perhaps regions spanned by extremum points
(Hugo's fur as the convex hull of Hugo's solid mass) or just
reasons as first-class posits in a point-free algebraic
topology.  We can then present a case for the semantics of
\q{Hugo} as a conventionalized sound adopted by a small group
of humans to designate a particular four-dimensional manifold
\mdash{} at least if we have a sufficient topological or Categorical
foundation to give logical exactitude to concepts like
\q{four-dimensional manifold}.  This still neglects
merological problems, like Hugo losing some fur
but still being himself; either the manifold Hugo
is scattered wherever his fur (and, well, peepee and poopie)
are or the Hugo manifold has fragmented into lots of disjoint
other manifolds, only one of which is \q{Hugo}.  This latter
option suggests that we need to identify the \q{essential parts}
of Hugo, not the inessential parts like those in the litter box;
then the Hugo manifold is the one that contains those parts
\mdash{} i.e. we need a mereological formalization to embed in
the topological one.
}
\p{It is certainly possible that by patching philosophical incompletions
and importing mathematical vocabularies we can piece together a
reasonably compelling theory of mereological/topological
complexes, maybe with graded sets (some parts are more essential than
others), to logically model the semantics of singular referents.
By taking sets of such complexes, maybe
with a modal machinery (for, say, fictional cats) or
graded membership (some things are only partially
subsumed by their concept: is a pet malawolf a dog?) we can
similarly flesh out a set-theoretic semantics, and combining
the singular and set analyses perhaps achieve a logically
inflected semantics applicable to a broad class of linguistic events.
But have we therefore truly surpassed informal logic,
coiled within dialogic conventions, toward a rigorous
theory of linguistic meanings via formal logic?  Do we need
topology, mereology, graded sets, and modal logic because these
are intrinsic to the epistemic domain where signification
happens, or are we just grabbing an assortment of mathematical
tools to plug the leaks in a philosophical attitude wherein
there is a relatively straightforward formalization of
quantifier-predicate logic that can polish the informal
logic of language, but which perhaps failed to
anticipate the mathematical complexities of formalizing what
exactly are the \q{thises} that quantifiers can quantify over?
}
\p{These questions are directly relevant to phenomenology \mdash{}
and to Analytic Philosophy's reception of and dialog with
phenomenology \mdash{} because a phenomenological account of
issues like manifold synthesis and mereological essentialism
can easily bring forward formulations that overlap
with a \q{logistic} paradigm in some details but not others,
and profitably pursuing the dialog depends on identifying both
similarities and differences.  The story of Hugo as a
four-dimensional manifold might not be told in a phenomenological
context, but we might study our visual episodes involving Hugo
in terms of perceptual synthesis: seeing Hugo from the front
we anticipate that we \i{could} see Hugo's tail if we were
standing behind him and it was not as much tucked beneath him;
this anticipation is also a formula for how we integrate
successive impressions as our gaze travels from his ears to
his tail that we see partially.  There is a certain continuity and
topography of our accumulation of successive visuals that
retraces a sort of manifold, even if now we're talking
not about Hugo (or his body) himself but rather the
interplay between the space where he's laying and my own
perspctive and vantage.  So, even
on a phenomenological basis we encounter fragments of analysis
that point toward something like a mathematical treatment \mdash{}
but that doesn't mean we are necessarily committed to a
quantitative-predicate logical theory of preception, or
equally important, of the communicative acts that
perception yields.
}
\p{In pursuit of logical rigor, 20th-century Analytic Philosophy
(plus, say, linguistics, cognitive science, and fields related
to knowledge engineering and information technology, like
AI-driven medical diagnosis) have highlighted
specific logico-mathematical representation-regimes,
building up a paradigm of logicality (as a philosophical
quality, something one senses in analytic methods)
and propositionality (as a corresponding analyzand).
The various layers of this construction address different
aspects of logic, or \q{logicality} as such, such as
predication (that which can be qualified as true or
false in the first place), quantification (establishing
a scope for empirical truthness, so logic does not not decay
into tautology \mdash{} that \i{one} thing or \i{some}
members of set have a property where others don't, modeling
contingency and a posteriority), modality
(capturing the further contingency that most facts
could have been otherwise), mereology and individuation,
quantification scope, supervenience, and other logical
themes that seem intrinsic to any intellectual territory
where predication and true-falseness pertains.
Codification of these facets of logic not only
fine-tunes philosophical analysis; it also provides a
semantics for natural language insofar as we accept that
meaning is centered on locutionary truth-conditions (from
which phenomena like illocutionary force and polarity
\mdash{} our feelings and pragmatic intentions \visavis{}
states of affairs at the predicate substrate of language \mdash{}
can be reached in supplemental analysis) and a philosophical
basis for the formal/computational representation of facts
and knowledge (such as, in database design).  So
insofar as a philosophical-computational-linguistic consensus
emerged on the broad outlines of what logic entails,
this paradigm spread into multiple disciplines,
scientific as well as humanistic, and applied as well
as purely speculative.  Insofar as Naturalizing Phenomenology
involves dialog with Analytic Philosophy, it
intersections with these different faces of logicality and
the mathematical systems which prior generations believed best
captured them.
}
\p{But this does not mean phenomenology needs to endorse the full
stack of paradigms layered in Analytic Philosophy's
excavation of logic; that we need to approach logical-mathematical
eidetics via quantifier-predicate logic, for example, or to
prefer the mathematical frameworks through which
quantification and predication are usually modeled.
Mathematics itself pulls in different directions: quantification,
precication, and set-membership have different
theories in different contexts, e.g. mereotopology compared to
algebraic topology compared to Category theory.  These
lines of development are largely tangential to Analytic
Philosophy partly because the philosophical (and, say,
cognitive-scientific) push toward formalized logic is not really
a push toward \i{mathematics}, but rather a vision of
bringing rigor to what is still essentially \i{informal} logic.
That is, pursuing for instance a formal treatment of
quantification to undergird natural language semantics
does not mean building up to a rigorous, maybe
Categorial theory of quantification in terms
of (say) dependent coproducts, because what \q{logicality}
does in language does not depend on that degree of technical
detail.  But that means in effect that operationalizing
the \q{logicalness} of language is not so much a problem
in \i{formal} (mathematical) logic but a formalized
\i{informal} logic, logico-linguistic or logico-mental
(as in \q{folk theory of other minds}) more than
logico-mathematical.  Fair enough; but this then places other
analytic conveniences (like referents as 4-dimensional
manifolds) on shakier ground, because we're no longer
in an all-encompassing logico-\i{mathematical} paradigm.
Topology and Category Theory, in effect, reveals the difference
between formal-mathematical and formalized-informal logic.
Trying to negotiate with mathematics as source of methodological
rigor, mathematics proves to be an elusive partner.
}
\spsubsectiontwoline{Phenomenology and the Limits of Logic-Compositionality}
\p{The trajectory through Algebraic Topology and Category Theory
has been one vector of post-1950 mathematical progress.
Another has been computer proofs, proof-assistants, and
computer-inspired approaches to mathematical foundations, like
Homotopy Type Theory.  Regarding \i{types} rather than
\i{sets} as the primordial elements of logic and mathematics
positions mathematics and computer science as two different
worlds evolving from a common type-theoretic universe.
In this environment, mathematics as abstract thought gets
mixed with mathematics as a technological discipline;
specialists for example debate the merits of mathematical
systems formalized with a priority to automated proof-checking
vs. presentations more aligned with mathematical precedents.
The proper place for technology and automation becomes
debated.  Computer theorem-provers after all are not oracles;
they need their own quality-checking, their own design theory.
Do we trust the theorem-prover (itself a human artifact) more
than the conventional deliberations validating a proof
in the eyes of mathematicians (and if so, why)?  Are design
principles of proof-assistant software and languages topics
for mathematics or computer science?  Does a test suite to
demonstrate a proof assistant's trustworthiness bear witness
to mathematics becoming suddenly an empirical rather than
eidetic science?  Does the empirical contingency of the
correctness of software used to provde the 4-color theorem
make the theorem something other
than \q{synthtic \i{a priori}}?
}
\p{There are analogs to these questions in the realm of
Analytic-Philosophical logic.  For example, the idea that cognitive
or linguistic meanings can be reduced to first-order predicate
formulae can be investigated by considering computer technology
which does in turn model digital content, in effect, as predicate
assertions together with rules of inference (that can
in principle be modeled via first-order logic): the Prolog
programming language, for instance; certain database
query systems; the Semantic Web and \q{formal Ontologies}.
These technologies operationalize logistic paradigms, but they
also put an empirical face on a logico-mathematical background
that for Russell, Frege, and Wittgenstein would have been
abstract speculation, akin to how proof assistants
put an emprical tie on abstract mathematics.  Similarly,
formal linguistic approaches (like Categorial Combinatory
Grammar) base themselves on abstract logico-mathematical
paradigms (like Hindley-Milner type sysetms), but by 2019 we
have a panoply of linguistic schools whose models
are realized in computational linguistic technologies,
digital platforms that represent and/or automatically
identify semantic and grammatic patterns according to a
specific thory of language.  In this context, type theory
informs linguistic research at a theoretical level as part
of its logico-mathematical underpinnings, but it also belongs
to the technological infrastructure wherein linguistic
technology is implemented.  These two roles can be complimentary:
insofar as Classical Hindley-Milner type theory has been
supplanted by competing paradigms in software which proves
technologically advantageous for modeling natural language,
we can explore whether more modern type theory is
superior for linguistic analysis at a purely theoretical level too.
}
\p{Insofar as Analytic Philosophy converged in the last century upon, it
seems, a fairly consistent nucleus of logical paradigms
\mdash{} covering the various intrinsic facets of quantifier-predicate
logic \mdash{} the heritage of that philosophy is similarly
rebased by 21st-century technology.  The persistence of first-order
logic as a reified example and medium of logic as such was driven by
the abstract reality that almost any logical system can be translated
first- or secod-order logic in principle (at least if we can
tolerate whatever metaphysical posits may be needed
to ground the resulting logic expressions, like Possible Worlds).
As long as logic belongs just to pure thought, the formal/theortical
isomorphism between superficially different theoretical
architectures can be treated as a challenge to overcoming:
converging on first-order predicate-quantifier logic
as a common ground beneath various shallower logics may seem
(and perhaps did seem to early Analytic Philosophers)
to be explanatory progress, a theoretical repositioning
which really does evince a situation where philosophical
disputation really can progress toward greater clarity.  However,
now that different logics are indeed manifest in a concrete sense
in different digital technologis, their contrasts no longer seem
so shallow.  It could be argued that the illusion of a \q{deep}
logic underlying superficial variation (like the illusion
of a \q{deep grammar} stabilizing the morphosyntactic surface,
perhaps) is itself the superficial presumption, failing
to duly appreciate surface contrasts which are only manifest
in the register of applied technology, rather
than abstract thought.
}
\p{For example, one can make a strong
semi-theoretical and semi-implementational case that Graph Theory
or Process Calculii are a better foundation for projects like
programming-language compilers and database query systems than
conventional predicate logic \mdash{} notwithstanding that
propositions in graph or process algebras can potentially be
reformulated in first-order logic (and vice-versa).  The
issue with such in-principle reductions is that they are of
dubious explanatory merit; the pro-forma restatement of
graph structures in predicate logic, for instance, has no
particular importance unless we are prior committed to
predicate logic being somehow fundamntal, in which case
the reducibiloty of graph theory (since this reducibility
can be inverted \mdash{} we can provide graph models of
attribute predication, boolean operators, etc.) can't
justify the commitment on pain of circularity.  What
\i{can} be observed, however, is that graph-database
queries cannot be optimized in their first-order
translations (which is why graph databases are a different
technology than relational databases), which in turn suggests that
the structural differences btween graphs and propositions are
philosophically non-trivial, and consequential.
}
\p{Similarly, concrete technologies can point
us to an intuition of
sentence-comprehension as a calculus of interlocking
cognitive processes, not the mechanical unzipping
of a predicate compact, seeing that there are substantial
implementational differences between process calculii and
predicate logic as paradigms for programming languages.
In short, technology reveals how formal systems may be
substantially divergent from predicate logic even if the
logical structures governing their operation have
isomorphs in predicate logic: technology reveals that such
\q{in principle} isomorphs are less important, less
indicative of philosophical depth, than we may have believed a
century ago.  But this is revealed not within philosophy
alone, but in an interdisciplinary spirit where insights are
mined from fields like compiler design and database engineering;
and from practical experience as well as speculation.  We can
appreciate the substantiveness of
\q{surface-level} logico-structural variation when we actually
\i{write} compilers or database query engines.
}
\p{The phenomenological de-centering of metaphysics does
not mean a complete abdication of metaphysical priorities,
of the need for a ledger to assess the reach and limits
of thought, or science or intellectual explanation;
to place the norms of philosophical exposition and
conversation on a disputable, theorizable conceptual bedrock.
Bracketing metaphysics does not mean suspending what
metaphysics is supposed to do to philosophical performance.
But Phenomenology seeks to bracket metaphsyics as something
received, habitual, or institutioanlized; something that
influences scholarship more than it is studied.
Phenomenology brackets metaphysics in the hope that it can
be recreated and return, but more transparent and self-aware
than before, like a professional athlete who returns
from a league suspension chastened and technically sounder.
Metaphysics is suspended but not expelled.  There is an
analogous dialectic in open-source progrommaing;
institutional norms for software quality are both suspended
and resinstated in a more transparent, technical
incarnation.  The de-centering of \i{institutional} controls
does not foreclose a re-centering of scientific
convergence in a shared technical understanding of the
project.  Similarly, the decentering of \i{institutional}
metaphysics \mdash{} not just the ideas, but the academic workflows
and publishing conventions that support them \mdash{} does
not foreclose a recovery of metaphysics in the
philosopher's relation to her own consciousness, in
writings as an existential trace more
than a social production.
}
\p{All of this then produces the question: what should metaphysics
be if it is not the paradigms we have received?  Where
does a phenomenological analysis take us that we were not before
it started?  As with any writing, we have to start with
the question: now that the writing has finished,
what has changed?  Aniticpating the writings of the future:
what do we want to have changed?
}
\p{We have to communicate this to Analytic Philosophers
(and practitioners of many other branches of scholarship,
like cognitive science and linguistics): what changes
when a phenomenological study is complete?  What
changed when there was a \i{Thing and Space} or a
\i{Phenomenology of Perception} compared to before
there was?  Failure to understand what a writing
wants to change can easily yield to failure
to understand the writing.
}
\p{I have often felt that Analytic and phenomenological
practitioners were talking past each other precisely in
this sense: a specialist in the philosophy of mind or
of science might approach a reading or conversation
thinking the phenomenologists is working to one kind
of telos, while the other is going somewhere else.
In apparently the most common iteration, the
Philosopher of Mind thinks the phenomenologist is aiming
toward a kind of skeptically inflected science of
perception, a theory of the deep structures behind
perceptual experience that potentially takes us further toward
an unbiased and even scientific theory by bracketing
presupposions about how consciousness and cognition
are supposed to work.  The Philosopher of Mind might
then think that phenomenology should, let's say,
uncover the synthesizing and interpretive acts that
constitute my raw sensate awareness as the grasping
and judging that Hugo is sleeping on the sofa, or that
it is half-past four and the sun is setting outside, or that
the car double-parked outside is red.  By attending
in a hopefully presupposionless way to the immediacy
of consciousness, we can see the underlying stitching
and unifying that normally passes beneath the threshold
of awareness, as if we were observing our
own consciousness like a shaman on ayahuasca.
}
\p{I have never in a philosophy classroom, nonetheless, felt
that my role was to issue oracular pronouncements as if from
a drug-induced trance.  Consciousness is not usually
extravagant and cosmic; my awareness of Hugo sleeping
on the couch is, really, not much more than
just Hugo sleeping on the couch.  Notice the structural
analogue to logistic philosophies of language: just as
adding logical symbols to \q{Hugo is sleeping on the sofa} does
not guarantee we are uncovering some deeper structure in the
language, some precious scientific insight, nor does
superimposing talk of a Passive Synthesis of Perception or
Hyletic Immediacy on the mundane episode wherein I see
Hugo sleeping on the sofa make that perception suddenly
scientifically tractable, as if we can solve the
mind-body problem just by concentrating hard enough.
It is what it is.  The things themselves are the things
themsevles; if we're going to them we can't assume we're
en route to some revelation or science, as if the
laws of consciousness are present in the room alongside
Hugo, the sofa, and the lamp.
}
\p{But a reframing of the sentence \q{Hugo is sleeping on the sofa}
\i{can} take us at least modestly further toward a good theory
of language, something beyond the disquotational
verity that a sentence means what it means, just as
\q{it is what it is}; \q{Snow is white} means that snow is white.
Similarly, I believe reframing perceptual episodes can
shed light on the formal \mdash{} and perhaps even scientific
\mdash{} structures of cognition, but we have to clarify ahead
of time what magnitude of clarity we are hoping for;
what precision of understanding landmarks a successful
analysis.  If a scientist reads phenomenology
hoping for an \expose{} of consciosuness that is both
first-person accurate and naturalistically revelatory
\mdash{} something experientially faithful like a
physchological novel but clinically analyzble
like a Brain Simulation \mdash{} then what is actually on
offer will fall short.  Both attending to consciousness
as it is experienced and honoring professional and
scientific standards of intellectual rigor are guiding
principles of phenomenological research, but these
are principles, not minimal standards like
a one-drink minimum at a night club: a very good
phenomenological writing will \i{not} truly be
either stream-of-consciousness realistic like a
novel nor scientifically complete like a Neural Network
platform presented at a computer science conference.
The point is via phenomenology to move a little closer
to both experiential honesty (to accept
consciousness as it is experienced, not as something
we hypothetically transform to fit cognitive and
physical explanations) and scientific rigor
(both the science of neurophysical explanations and
the science of cognitive or intelligence simulations),
but while doing so acknowledging the incompleteness of
both projects and the tensions between them.
}
\p{In the case of perceptual synthesis, we have good reseon
to believe that biophysiology and experimental psychology
gives us a well-informed acccount of perceptual phenoema,
in their episodic unity, at the neurological and
sensori-motor levels; presumably I have some brain
function that unifies white-sensations and brown-sensations
and distributes the color regions to different concepts
(sofa, Hugo).  Since I'm not aware of these
levels they are not phenomenological subjects per se
(which doesn't mean we should reject them entirely,
since I have reasonably firm beliefs about the scientific
basis of perception, beliefs which I don't deliberately
question any more than I deliberately apply them to
mundane perceptions).  Perhaps our scientific training
casuses our actual perceptual content, at last sometimes, to be
subtly different in our experience of it.  I am reminded
of the rainbow color of a thin slick of spilled oil:
knowing to some approximation how we think that phenokenon
occurs makes me explicitly associate the rainbow with the
oil, and therefore to almost unconsciously direct my
gaze to the border lines where the colors are brownish and
less pronounced.  Somone who did not impose a mental
explanation on what they're seeing, even if unintentionally,
would perhaps attend more intently to the color and
experience it more vigorously.  So scientific beliefs
are relevant in the limited sense that their trace can
sometimes be detected in actual experience.
With this qualification, however, any scientific
story about how percption \i{works} lies outside
phenomenology proper.
}
\p{Over-analysis, \i{excess} clarity, \i{too much}
resolution distorts experience even if it aims
to explain it.  So phenomenology does not neatly
fit within a scientific paradigm where magnification
often leads to elucidation and discovery: we cannot take
a microscope to consciousness and find new candidates
for scientific treatment, the way we can advance science
by making more powerful telescopes.  For a scientist
or someone with scientific intiutions it can seem
hard how to go forward: you don't study cancer by staring
at tumors; so how can you study consciousness without some
analysis, some distortion and defamiliarization?  Otherwise
we seem fated to endless circularity: \q{it is what it is}.
The things themselves are what they are.
}
\p{This point, relative to perceptual synthesis, has
similar forms in other aspects of cognitivity activity:
judging that the animal on the sofa is a cat, indeed Hugo,
that he's sleeping, that it's daytime becauuse it is
light outside, that the men in uniform skating on
an ice rink on television are hockey players, and so forth.
Phenomenology cannot ignore that many of the perceptual
and conceptual judgments that we are aware of take the
form in consciousness of settled fact; we can't do first-person
analysis of chains of reasoning that we don't actually
experience ourselves performing.  So while
presumably there is a detailed story to be told about
how optical and inferential thought-processes lead
us to the conclusions that it's daytime and that Hugo
is sleeping, the perceptual record shows that
none of this labor was special enough to warrant conscious
attention.  It's not like I first speculated that some
unfamiliar creature was on the sofa, and subsequently
concluded that it's just Hugo; or that I had to stand
and ponder who the cat is until I could assemble
enough visual detail to reach that conclusion.
I honestly don't know what I was unconsciously
thinking to presume passively that it's Hugo \mdash{} honestly,
I usually don't know what I'm unconsciously thinking.
I cannot, accordingly, use unconscious or
too-mundane-to-experience judgments as part of
phenomenology at least during argumentational stretches
where I'm sticking to phenomenological reports.
}
\p{Insofar as this is characeristic of phenomenology in general,
we should not expect that phenomenology will advance
us further to some sort of rigorous philosophy
or experientially faithful science by revealing things
about experience that we have not yet noticed due to
lack of attention: we cannot magnify consciousness
the way a microscope enlarges different things and a
telescope reveals distant ones; metacognition
and first-person reflection don't enlarge or
polish the things in consciousness so that we can
describe them more thoroughly or scientifically.
Phenomenology does take us to some new place by
making us aware of pieces of experiential synthesis
that we don't experience directly otherwise,
as we don't experience the molecules in
water; or minutiae of conceptual
reasoning that we would otherwise take for
granted, like how we figure out when an
animal is a cat.
}
\p{Of course, however, often we \i{are} aware of
perceptual synthesis and conceptual judgment.  If
after seeing Hugo I walk toward him,
with my perspecive gradually changing, I
consciously register a change in how Hugo appears to me
which produces an explicated synthsis,
a self-conscious accumulation of visual evidence telling
me about Hugo's current state, at least if I am
looking at him with express interest him as opposed to looking in
his directly and passively noticing him while attending to
something else, like the television.  Likewise I may passively
observe that he is sleeping if I previously thought he
was sleeping and just happened to glance at him again,
but there's an alternative scenario where I don't
specifically assume he is sleeping, but upon looking
at him rather attentively, with his lack
of movement and his relaxed posture, I reach the
conclusion self-consciously.  Walking outside I may
see a familiar stray cat and put no thought into recognizing
her, but I may also spot a movement in the distance
and look purposefully before identifying the animal as a cat.
In short, perceptual synthesis and conceptual
judgments are neither \i{intrinsically} conscious
or pre-conscious, active or passive; many
cognitive episodes that we are explicitly aware of in their
unfolding could, in slightly different circumstances,
remain below the threshold of conscious awareness.
Conversely, many judgements we make without
much conscious attention could alternately be
experienced with more deliberation and attention.
Indeed, sometimes we revise our passive judgments
when something goes awry in our passive estimation of
what's going on around us \mdash{} we assume we hear
the dog barking in the back yard, but then
we see her eating pizza crusts in the kitchen, so now
we have to account for the mysterious barking outside.
We all have the faculty to quickly adopt the
mode of conscious investigation, trying to gather specific
information, when the passive judgments we have just
assumed to be true turn out to have some incongruity, some
gap in their logical order.
}
\p{Now we are moving toward some insight, since although we
can't give a phenomenological account of preconscious
judgment, we can at least confirm that \i{some}
judgments are \q{active} and part of
explicit awareness, while others are passive and
not (in any inner details) noticed; we can also say that
at least some passive jugments seem similar in form
and outcome to active ones.  This raises several
avenues for further, maybe scientific continuation:
why do we have both active and passive judgment?  What
benefits accrue to us biologically as animals that depend
on precise judgments to survive that we make many
instinctive judgments and some additional deliberated ones?
Why are active and passive judgments often similar?
What triggers a decision to actively attend to specific
components of experience so that associated judgments in
that one area will, at least in the immediate aftermath,
be more active than passive?  What is actually
neurophysically different in the two cases so that,
in a pure subjective and first-person sense, I know when
a judgment is active and passive?
}
\p{Furthermore, since many judgments \i{do} rise to a level
of consciousness \mdash{} and not just the minimal awareness
of believing such-and-such but some explicit experience of
how the judgment is formed, of their inner pieces and
intemediate stages \mdash{} these conceptual and perceptual episodes
\i{are} candidates for phenomenological analysis.   We are
certainly aware of perceptual episodes in their
unfolding; how perception scans over a room, or
any other surrounding, like the pan of a camera;
while, also, different episodes piece together,
with shifts in attention (and, often, kinaesthtic
actions) \mdash{} like looking up from a book \mdash{}
juncturing disparate perceptual phases.  It seems self-evident
that perception unfolds according to multiple time scales,
where episodes measured in seconds join to become the perceptual
background as we engage in activities over the course of minutes,
within errands that stretch toward hours and engagements
over the course of a day (we cook, go to the store, ride
a subway, watch a show), and of course we plan for a future
and remember a past over monrhs and years.  Cognitive acts and
perceptual interludes can be slotted into different time
scales, affecting how we experience them and should analyze
(or even name) them: a perceptual episode, brief enough to appear
in consciousness as a single event, needs a different
theoretical registration than experiences implying lengthier
segments.  Practical actions, also, bridging
perception, conceptualization, and planning, sort into different
time-spans: opening a bottle vs. preparing a meal;
petting the dog vs. walking the dog.
}
\p{We have, as such, I would argue, a collection of observations
and distinctions we can make about consciousness and
conscious experience, that emananate from consciousness
itself and not from prior scientific or
metaphysical commitments.  These observations
seem presuppositionless, or at least common-sensical.
We should not take that too far: the \q{suspension of
belief} should not commmit us to a virgin brilliance
exorcising millenia of thought, posturing us as priests
of intellectual purity that can be defrocked by the
smallest trace of naturalistic thinking \mdash{} gotcha, a
presupposition!  Philosophy as volumes forging profound
insight from pure thought alone seems silly in our time,
more art created from ideas than rational exposition.
But we can make a good-faith effort to present
patterns in consciousness based on rather casual and
common-sense reflections not explicitly aligned with
scientific thories, or with schools of philsophy,
and I claim at least some of the intellectual structure
we can get to via that route takes the form of observations
I have highlighted: passive vs. active judgment; the
various scales of perceptual and enactive temporality; the
individuality, aggregation, and junctioning between
perceptual episodes.
}
\p{Insofar as these observations
and distinctions form the rudimentary constructs in
a theory of consciousness, the explanatory value
of that theory will depend on how it can be
consistently extended.  Remaining within phenomenology,
we cannnot analyze deeper into consciousness \mdash{}
capturing more fine-grained layers of perception, uncovering
more details of passive synthesis \mdash{} because we cannot
impose scientific-empirical data or philosophical speculation
on subjective experience.  So if we can only articulate a
handful of organized but commonsensical comments about
experience, we need to find away to make the analysis meaningful:
for it to make a difference.
}
\p{One way to do this, I believe, is to make the common-sensical
observations more than just observations; not in the
sense of importing interpretive claims that don't
unfold from them organically, but in the sense of trying to
express their as a formal structure, something which can
be then set in relation to other structures and other
accounts of mental processes.  There are many formal systems
that have been proposed to cover various features of
cognition and intelligence; but only a few
of these, I suspect, are formal views on experientially vivid
acts of mind; on judgments or percpetual syntheses that
we can experience in their unfolding and reflect upon
first-personally.
}
\p{There is a considerable body of cognitive-scientific
literature that aspires to formalize (and sometime simulate)
mental activity via structures that have some blend
of logical, mathematical, and computational
specification.  There is also a considerable body of
literature that remains true only to the first-person
perspective, not explicitly trying to align with
scientific explanations and world-view.  But where
these bodies overlap I believe lays the potential for
overlapping theories, like a Naturalized Phenomenology.
}
\p{}
