\section{Cognitive Grammar and Type Theoretic Semantics}
\p{The emergence of Dependency Grammar as a \i{computational} approach has some broad
implications.  The historical preference for phrase-structure foundations  \mdash{} among
those who actually build Natural Language Processing code libraries, in disciplines
related to Artifical Intelligence \mdash{} arguably reflects how phrase structure more
cleanly models a theory of linguistic meaning and signification based on
\q{symbolic logic} \mdash{} a theory that the \i{meaning} of a complete and self-contained
linguistic expression is the logical state of affairs which it asserts or in other
ways connotes.  Correlated with this assumption is the idea that phrase structure
logically transforms its constituent parts; so from the word \q{students} we
can form the phrase \q{many students} to designate a kind of plurality \mdash{} a plural
set but also, more specifically, a set which is reasonably large relative to some context.
In the hierarchical model presented by these norms, phrases subsume the roles of
individual words and represent discrete semantic units with respect to still larger phrases.
}
\p{It is certainly true that one role for phrases is to satisfy a semantic niche \mdash{} often
a place occupied in other (or even the same) language with single words, or vice-versa.
The French \q{laisse tomber} translates the English \q{drop}, for example; and
\q{parliamentarian} is a more exotic version of \q{Member of Parliament}.  There is no
evident pattern for when a single concept is conveyed, in one language or another,
by a single word or a multi-word phrase.  Moreover, the meanings of phrases are influenced
by semantic conventions no less than are individual words, and they are not solely a product
of phrase constituents.  Semantics is guided by what people need to talk and write about often; when
events in a linguistic community call for some fairly rigid and repeatable designation for
an important concept, the resources of language adjust to provide that role, either through a
complete neologism, or a lexical variant \mdash{} a new usage; or the entrenchment of a phrase.
In current events, the expression \q{Syrian Refugees} recurs when discussing people
displaced by the Syrian civil war, and potentially other interrelated conflicts also;
convention seems to allow that nominal \q{Syrian} Refugees don't have to be Syrian nationals.
The meaning of the phrase is fixed by its niche in familiar discourse more than
by its literal form.  Phrases
exhibit conventionalization and usage pressures analogous to single words; which lends
credance to the notion that phrases subsume the role of single words, and that the semantic
contribution of words to sentences is determined through the phrases where they occur.
}
\p{On the other hand, it is well established that words' contributions are not \i{wholly}
subsumed by their surrounding phrase-structure.  The famous joke about the Holy Roman
Empire \mdash{} or its reprise in the current line that the Islamic State is neither Islamic
nor a State \mdash{} point to evidence that as language-users we still hear the individual
words outside their phrase context.  To subsume a word into a phrase is also to suggest a
particular semantic (and pragmatic, real world) interpretation, one which conversants
may challenge.\footnote{How literally to take phrases is a notorious source of political controversy:
recall debates about the relevance of Afghanistan for Iraq, in US policy, and
Rudy Giuliani saying \q{There \i{is} Al Qaeda in Iraq \mdash{} it's called, \sq{Al Qaeda in Iraq}}.
}
}
\p{Arguably, joking or titular cases like \q{Holy Roman Empire} can be
relegated to thematic margins, especially if we accept formal-logical construals of
what semantics is all about, with an \i{a priori} contrast between Semantics and
Pragmatics, the former rooted in \i{states of affairs} and only the latter addressing
rhetoric and usage.
Counter to this counter-argument, however, we can observe that different phrases imply
different degrees of \q{autonomy} to their constituents, and different degrees of
coherence or unification into a single idea.  Some phrases act as direct substitutes
for single concepts (like \q{Member of Parliament}) where it seems mostly historical
accident that a phrase rather than a word emerged as the most popular; but many other
phrases have more complex usage scanarios, including everyday expressions that don't have
special rhetorical or sociolinguistic conventions that would make them tangential to
semantic or syntactic analysis proper.  Moreover, many of these examples are similar to
those used by Cognitive Grammar to challenge the syntax/semantic distinction and argue for
\q{morphosyntactic} models as reciprocating cognitive formations, not abstract language-rules.
}
\p{For example, in Ronald Langacker's \i{Foundations of Cognitive Grammar}, the sentence
\begin{sentenceList}\sentenceItem{} Three times, students asked an interesting question
\end{sentenceList}
is used to demonstrate how
grammatical principles follow from cognitive \q{construals} of the relevant situations,
those which language seeks to describe or takes as presupposed context.\footnote{For example, \cite[pp. 119 and 128]{LangackerFoundations},
discussed by \cite[p. 189]{LineBrandt}, and \cite[p. 9]{EstherPascual}.
}
In particular, Langacker argues that \q{students} and \q{question} can both be either singular or
plural: syntax is open-ended here, with neither form more evidently correct.  Langacker uses this
example to make the Cognitive-Linguistic point that
we assess syntactic propriety relative to cognitive frames and conversational context.  In this
specific case, we are actually working with two different cognitive frames which are interlinked
\mdash{} on the one hand, we recognize distinct events consisting of a student asking a question, but
the speaker calls attention, too, to their recurrence, so the events can also be understood
as part of a single, larger pattern.  There are therefore two different cognitive foci, at two
different scales of time and attention, a \q{split focus} which makes both singular and plural
invocations of \q{student} and \q{question} acceptable.
}
\p{Supplementing this analysis, however, we can additionally focus attention directly on
grammatical relations.  The words \q{student} and \q{question} are clearly linked as the subject and
object of the verb \q{asked}; yet, contrary to any simple presentation of rules,
no agreement of singular or plural is required between them (they can be singular and/or plural in
any combination),  Moreover, this anomaly is only in force due to the context established
by an initial phrase like \q{Three times}; absent some such framing, the singular/plural
relation would be more rigid.  For example, \q{A student asked interesting questions} would
(in isolation) strongly imply \i{one} student asking \i{several} questions.  So the initial
\q{Three times} phrase alters how the subsequent phrase-structure is understood while remaining
structurally isolated from the rest of the sentence.  Semantically, it suggests a
\q{space builder} in the manner of Gilles Fauconnier or Per Aage Brandt
\cite{Fauconnier}; \cite{PerAageBrandt}, but we
need to append Mental Space analysis with theory of how these spaces
influence syntactic acceptability, which would seem to be logically prior to the stage where Mental Spaces
would come in play.  This complex interplay of phrase-structures
is hard to accommodate from the grammar-hierarchy perspective.  There seems to be
no way to break down this example sentence into a tree-like phrase hierarchy wherein each
phrase, considering the semantic concept which it is apparently tasked to put into words,
can be seen to function in isolation.  The mapping of the sentence to a logical
substratum would be more transparent with a sentence like \q{Three students asked
interesting questions}; that sentence is a more direct translation of the facts
which the original sentence conveys.  But this \q{more logical} sentence has different
connotations than the sentence Langacker cites; the original sentence places the emphasis
elsewhere, calling attention more to the idea of something temporally drawn-out,
of a recurrence of events and a sense of time-scale.  The \q{more logical} sentence
lacks this direct invocation of time scale and temporal progression.
}
\p{We can say that the \q{Three students} version is a more direct statement of fact, whereas
Langacker's version is more speaker-relative, in the sense that it elaborates more
on the speaker's own acknowledgment of belief.  The speaker retraces the steps of her
coming to appreciate the fact \mdash{} of coming to realize that the \q{interesting questions}
were a recurrent phenomenon and therefore worthy of mention.  By situating expression
relative to cognitive process rather than to the facts themselves, the sentence
takes on a structure which models the cognition rather than the states of affairs.
But this shift of semantic grounding from the factual to the cognitive also apparently
breaks down the logical orderliness of the phrase structure.  \q{Three times}, compared
to \q{three students}, leads to a morphosyntactic choice-space which is
\q{underdetermined} and leaves room for speakers' shades of emphasis.
}
\p{This is not an isolated example.  Many sentences can be provided with similar
phrase-structure complications, particularly with respect to singular/plural agreement.
\begin{sentenceList}\sentenceItem{} Time after time, tourists (a tourist) walk(s) by this building
with no idea of its history.
\sentenceItem{} The streets around here are confusing; often people (someone)
will ask me for directions.
\sentenceItem{} Student after student came with their (his/her)
paper to compain about my grade(s).
\sentenceItem{} Student after student \mdash{} and their (his/her) parents
\mdash{} complained about the tuition increase.
\end{sentenceList}
On a straightforward phrase-structure reading, \q{Student after student} reduces to an
elegant equivalent of \q{Many students}, with the rhetorical flourish abstracted away
to a logical form.  But our willingness to accept both singular and plural agreements
(his/her/their parents, grades, papers) shows that clearly we don't simply substitute
\q{Many students}; we recognize the plural as a logical gloss on the situation but
engage the sentence in a more cogntively complex way, recognizing connotions of temporal
unfolding and juxtapositions of cognitive frames.  The singular/plural underdeterminism
is actually a signification in its own right, a signal to the listener that the
sentence in question demands a layered cognitive attitude.  Here again, syntactic
structure (morphosyntactic, in that syntactic allowances are linked with
variations in the morphology of individual words, such as singular or plural form)
serves to corroborate conversants' cognitive frames rather than to model logical
form.
}
\p{This is not to say that phrase-structure paradigms are refuted by these examples.  Cases
like these can be accommodated by layering new structural rules, such as
allowing exceptions for singular/plural agreement in the presence of certain
\q{lead-in} phrases like \q{Three times}.  It is not even accepted that these
examples clearly favor inter-word relations (as language formalization, in preference over
phrase-structure trees) \mdash{} cases like \q{Student after student} have
also been used \i{against} Dependency Grammar on the argument
that there is not a clear \q{single} word, in that phrase,
which should be seen as linking with words elsewhere in the sentence
\cite[pp. 400-401]{MullerBook}, \cite[p. 2]{MullerPDF}.  It
seems arbitrary to select either \q{student}, or \q{after}, as \q{the}
representative of the phrase to link with \mdash{} for example \mdash{} the verb
\q{complained}; on that argument, the least arbitrary analysis is to
treat the phrase as a whole as a single unit for purposes of grammatic
linkage.  In short, both paradigms have potential problems with these example.
Considering \q{Student after student} as an encapsulated phrase leaves
the singular/plural flexibility in the continuation of the sentence unexplained
(\q{Many students compained about \qmarkdubious{}his grade} is clearly dubious, so
\q{Many students} is not a direct substitution). But bracketing the phrase
when decribing the sentences' \q{linkage} leads
to an apparently arbitrary choice when it comes time to notate the subject/verb
linkage for \q{complained}.  I will address this particular ambiguity later;
but for now I'll just point out that a simplistic reading of
both Dependency and Phrase-Structure ideas seems to run aground.
}
\subsection{Comparing paradigms}
\p{Since Computational-Linguistic paradigms find practical expression in code
libraries, there are some options to assess competing theories empirically
\mdash{} comparing libraries' speed, accuracy, ease of use, and how readily can they be
modified in light of new research.  Arguably, however, the
quality of a code library does not automatically reflect the accuracy of
its underlying linguistic paradigms (as opposed to the skill, foresight, and
resources of its programmers); not to mention that more complex analyses of
human language may be both more correct and also harder to express in code.
There is, in any case, no apparent consensus amongst linguists and programmers that one
or another language theory has proven computationally preferable.  Another approach to
theory-comparison involves considering the range of linguistic phenomena which different methods can explain, without resorting to ad-hoc compilations of exceptions and
special cases.  Arguably, here, Dependency Grammar provides more
straghtforward explanations.  For example, the internal structure of phrases seems to
lend specificity and nuance to their meaning in ways that get lost when
trying to replace phrases with logico-semantic equivalents.
\q{Student after student} is not losslessly substitutable with \q{Many students},
and the former phrase has a temporal and multi-tier cognitive implication which
the latter discards.  The second phrase is compatible with \q{Many students}
complaining \i{at one time}, as well as drawn out over time; the former phrase
appears to clarify that the second kind of situation is the intended meaning,
Of course, in context, the two phrases may be understood to have similar meanings;
but this is a product of how the linguistic structure relates to its presumptive
conversational context, not to an intrinsic semantic equivalence.  I will
now consider these and other examples to discuss the dependency/phrase-structure
contrast in a little more detail.
}
\p{The contrast between the phrases \q{Student after student} and \q{Many students}
cannot be based on \q{abstract} semantics alone \mdash{} how the evident temporal implications of the
first form, for example, are concretely understood, depends on conversants' mutual recognition
of a relevant time frame.  The dialog may concern a single day, a school year, many
years.  We assume that the speakers share a similar choice of time \q{scale}
(or can converge on one through subsequent conversation).  \i{Some} time-frame
is therefore presupposed in the discursive context, and the first phrase invokes
this presumed but unstated framing.  The semantics of the phrase are therefore somewhat open-ended:
the phrase \q{hooks into} shared understanding of a temporal cognitive framing without referring
to it directly.  By contrast, the second phrase is less open-ended: it is consistent with both
a more and less temporally protracted understanding of \q{many}, but leaves such details (whatever
they may be) unsignified.  The factual circumstance is designated with a level of abstraction that sets
temporal considerations outside the focus of concern.  The second phrase is therefore both
less open-ended and also less expressive: it carries less detail but accordingly also relies
less on speaker's contextual understanding to fill in detail.
}
\p{Clearly the two phrases are therefore semantically different; but notice also that the
semantic properties of the first phrase are due explicitly to its internal structure.
The temporality implicatures could be expressed in a more \q{purely} semantic
fasion with a choice of wording, like \q{a procession of students complained}.  This
would rely on the conventional meaning of \q{procession} (or \q{stream}, \q{sequence},
etc.) to provide the expressive \q{time} dimension.  But the \q{Student after student}
phraseology achieves this effect more economically and with more \q{oomph} because the
internal repetition in the phrase itself effectively models the recurrence it
seeks to feature semantically.  Here linguistic form actually does reproduce
factual structure, like a syntactic version of onomatopoeia.  This fact of internal structure
clearly can only be fully modeled by taking seriously the exact composition of the phrase, not
treating the phrase-structure as a convention fully subsumed by a semantic role.
}
\p{In addition, aside from the expressive detail which depends on the actual phrase
structure (which therefore cannot be summarized away), this inner structure also
governs morphosyntactic possibilities over all.  \q{A procession of students}
captures a similar temporal progression but also fully absorbs \q{student} in a plural
guise, and \q{A procession of students complained about \qmarkdubious{}his grade} is straightforwardly
ungrammatical.  In Langacker's \q{Three times} example, the inter-word
\q{linkage} captures the aforementioned complexities in a reasonably non-arbitrary
way, I believe.  \q{Student} is linked as subject-to-verb with \q{asked}, and as
subject-to-object with \q{question}.  It is true that these link-pairs seem to
violate agreement norms, but there is nothing in the Link Grammar paradigm \mdash{} which
practices Dependency Grammar with a rather detailed and intricate
inventory of inter-word relations, or \q{links} \mdash{} mandating that
\i{all} link-pairs exhibit forced agreement (like singular/plural).  Agreement, when it
applies, is a property \i{of} link pairs.  There is also an implicit
(cross-phrasal) link between \q{student} and \q{Three} \mdash{} clarifying that, considered
in its entirely, the sentence is about three students precisely \mdash{} and the presence of
this kind of link alters how the other links connecting to the word \q{student} are
assessed.  In particular, this latter link stipulates that the word \q{student} is being
simultaneously understood in both a plural and a singular sense, so it permits
singular \i{and} plural link forms which, more commonly, could only be
singular \i{or} plural.  So link grammar can offer an elegant analysis of
singular/plural \q{underdeterminism}, expressed in the
same underlying graph-context terminology as most other link-grammar theorizing.
It would be unfair to use this as a case against Phrase Structure grammars without a
detailed presentation of how these grammars would handle such a case in turn, but
I'd argue that link grammar accommodates this complex example with relatively little
departure from its underlying theoretical and notational or presentational commitments.
}
\p{While my previous examples contrasted Phrase Structure and Dependency
Grammars in terms of their resources for explaining sentences with
unusual semantic patterns but relatively clear meanings (in context),
another form of comparison can address actual ambiguity.  Consider
\begin{sentenceList}\sentenceItem{} The Maple Leafs failed to win in overtime for the
first time this year.
\sentenceItem{} The Maple Leafs failed for the first time this year
to win in overtime.
\end{sentenceList}
The first can mean either that the Leafs had won \i{all} or \i{none}
of their prior overtime games.  From a phrase-structure perspective,
we have to image that \q{to win in overtime} can \q{migrate} so we
hear it as in the second version of the sentence.  For more inter-word
grammars, the alternation is simpler: \q{for}, initiating the phrase
\q{for the first time}, can be linked with either \i{failed} or
\i{win} \mdash{} notationally, it amounts to the presence or absence of one
graph-edge, when the syntax is represented as a graph with
inter-word labels for link kinds.  This could be a distinction without
a real difference, since choosing which inter-word link to recognize
triggers linking in the rest of the phrase along with it.
But perhaps reflecting on how we process the ambiguity \mdash{} realizing
that there are two competing parses and deciding which is the one
intended \mdash{} we picture the alternatives more as \q{horizontal}
options for connecting threads across the sentence,
more so than a \q{vertical} organization where
we hear \q{for the first time} as \q{contained} in a larger phrase.
My own feeling is of exploring competing relational patterns more than
exploring different ways that the phrases can be nested inside each
other.
}
\p{That being said, how much of our sense of ambiguity (or clarity, for
that matter) is driven by meaning, not form?  The \q{double parse}
just examined does not always generalize to similar cases:
\begin{sentenceList}\sentenceItem{} The Maple Leafs failed to win two consecutive games
for the first time this year.
\end{sentenceList}
The reading as in \q{this is the first time they failed to win two
consecutive games} makes no sense \mdash{} unless you've won every game,
but perhaps the first, you've at some point lost after a win.  Is this
case anomalous, where a syntactic ambiguity idiosyncratically fails to
yield logically plausible readings?
The ambiguity is found in \i{failed to make the playoffs
for the first time since 2013}, and many \i{for the first time this
season} cases, like \i{beat the Habs}, \i{sell out the arena},
\i{score a goal in the first period}.  But \q{failed to score a goal}
is almost surely read that they \i{did} score in every prior game.
Do we hear the construction as intrinsically ambiguous, and reject one
reading only when it is clearly flawed pragmatically?
}
\p{If we believe that language understanding unfolds in a predictable
operational sequence, then we should assume that both parses are
deemed plausible, and semantic considerations only retroactively
eschew one reading (if they do so at all).  This would explain why in
many cases the ambiguity persists enough to cast the practically
intended meaning in doubt.  But that account does not consider the
temporality of language itself; the hearer does not know in advance
that a trailing phrase like \q{for the first time this season} is coming,
and starts to make sense of the sentence up to there; once then hearing or
reading the addendum, the audience instinctively has to interpret the
final phrase as deliberately inserted to modify an already-complete
idea.  On this analysis, the addendum is initially approached as a
performative detail, something said for a reason to be determined \mdash{}
it is not structurally necessary to make the sentence well-formed.
Perhaps we then try to fit the last phrase into the sentence both
syntactically and semantically, together, triggered by a pragmatic
phenomenon (the speaker's choice to add on to a seemingly complete
thought) which then becomes logically prior to both syntax and
semantics.  If this is plausible, it supports an inter-word relational
model because we are forming a picture of language structure
relationally, assimilating new words and phrases to those already
heard by linkings referring back in time, rather than waiting until we
are sure we have a complete sentence and then treating it as a static
structure to vertically reconstruct.
}
\p{The examples I have used so far may also imply that a choice of phrase structure is
always driven by semantic connotations of one structure or another;
but seemingly the reverse can happen as
well \mdash{} speakers choose a semantic variant because its grammatic realization lends a useful
organization to the larger expression.  There are many ways to say \q{many},
for example: \i{a lot of},
\i{quite a few}, not to mention \q{time after time} style constructions.  Whatever their
subtle semantic variations, these phrases also have different syntactic properties:
\i{Quite a few} is legitimate as standalone (like an answer to a question);
\i{A lot of} is not, and \i{A lot} on its own is awkward.  On the other hand the \q{of} in
\i{A lot of} can \q{float} to be replicated further on: \q{A lot of students, of citizens,
believe education must be our top priority} sounds more decorous than the equivalent sentence with the
second \q{of} replaced by \q{and}.  If the cadence of that sentence appeals to the speaker, then
such stylistic preference will influence taking \q{A lot of} as the \q{many} variant of choice.
So speakers have leeway in choosing grammatic forms that highlight one or another aspect of
situations; but they also have leeway in choosing rhetorical and stylistic pitch.  Both cognitive
framings and stylistic performance can be factored when reconstructing what compels the
choice of one sentence over alternatives.
}
\p{One consequence of these analyses, should they be accepted, is that grammar
needs to be approached holistically: the grammatic structure of phrases cannot,
except when deliberate oversimplification is warranted, be isolated from
surrounded sentences and still larger discourse units.  Semantic roles
of phrases have some effect on their syntax, but phrases are nonetheless chosen
from sets of options, whose variations reflect subtle semantic and syntactic
maneuvers manifest at super-phrasal scales.  The constituent words of phrases retain some
autonomy, and can enter into inter-word and phrasal structures with other words outside their
immediate phrase-context.  We can still apply formal models to phrase
structure \mdash{} for example, Cognitive and Applicative Grammar (\CAG{}) considers phrases
as \q{applications} of (something like) linguistic or cognitive \q{functions},
in the sense that (say) an adjective is like a \i{function} applied to a noun,
to yield a different noun (viz., something playing a noun's conceptual role)
\cite{Descles2010}.
I will consider related \q{functional} and (by extension) Type-Theoretic
approaches in the next section.  But we should not read these transformations
\mdash{} like \i{Syrian refugee} from \i{refugee} \mdash{}
too hastily as a purely semantic correlation within a space of denotable concepts
\mdash{} \i{such that} the new concept wholly replaces the
contained parts, which then cease to have further linguistic role and effect.
Instead, applicative structures represent shifts or evolutions in
mental construal, which proceed in stages as conversants form
cognitive models of each others' discourse.  Even if phrase structure
sets landmarks in this unfolding, phrases do not wholly subsume their
constituents; the parts within phrases do not \q{vanish} on the higher scale,
but remain latent and may be \q{hooked} by other, overlapping phrases.
This argument rests on a vantage point from semantics as well as syntax;
therefore, I will discuss it briefly at present (I return to
this analysis at greater length in the next section).
}
\p{Consider the effect of \q{Many students complained}.  Propositionally, this appears
to say essentially that \i{students} complained; but, on hermeneutic charity, the
speaker had \i{some} reason to say \q{many}.  The familiar analysis is that
\q{many} suggests relative size; but this
is only half the story.  If the speaker chose merely \i{students complained}, we would hear an assertion
that more than one student did, but we would also understand that there were several
occasions when complaints happened.  Adding \q{many} does not just
imply \q{more} students, but suggests a mental shift away from the particular episodes.
In the other direction, saying \i{a student complained} is not just
asserting how at least one student did so, but
apparently reports one specific occasion (which perhaps the speaker wishes to
elaborate on).  In other words, we cannot really capture the singular/plural semantics,
or different varieties of plural, just by looking at the relative size of implied
sets; we need to track how representations of singleness or multitude imply
temporal and event-situational details.  So \i{a student complained} focuses not
on the numeric count of one, but on a singular event (unlike
\q{\i{only one} student complained}); \i{students complained} focuses not on the
plural measure of students involved, but on the fact that a certain type
of event happened several times.  \i{Many students complained}
focuses not on sheer number (unlike \i{a large number of students complained}),
but rather on the implication that complaints were widespread enough to represent
a significant sample, perhaps a majority sentiment, among the student body.
The semantics of the former two forms seems to focus attention
on the \i{events} of complaining, while the \i{many students} construction seems
to focus more on their suggesting a prevailing attitude.  \q{Students complained}
seems to single out each event as distinct, even though there are several of
them; whereas \i{Many students complained} seems to construe the
events as each resembling the other, to the point where they partly
lose their individuality.  \q{Isolated events},
in the English idiom, are those which are atypical; as we cognitively
shift from the events as discrete to recurring patterns, they become
suggestive of a larger state of affairs.  By implication, if many students
complained, many other students may be unhappy; the extent of students'
unrest is no longer measurable by the multiplicity of the complaining-events.
}
\p{Against this backdrop, \i{Student after student complained} captures both dimensions,
implying both a widespread unrest among the student body and also
temporal recurrence of complainings.
Formal models of syntax and semantics often borrow notation from formal
language theory; for example, notations for Parts of Speech lifted
from functional programming languages.\footnote{A note on notation: I adopt the Haskell convention (referring to the Haskell
programming language and other functional languages) of using arrows both between
parameters and before output notation, but for visual cue I add one dot above the
arrow in the former case, and two dots in the latter: \argsToReturn{}.
I use \N{} for the broadest designation of nouns (the broadest
noun type, assuming we are using type-theoretic principles),
with extra markings for more specific types (in principle
similar notation could be adopted for verbs, propositions, and so on).
}
This notation can help us picture the \q{flow} of ideas building up
to a complete sentence, formally represented via type theory
(where sentences reveal a type hierarchy culminating in a self-contained idea,
that is, a proposition); more informally we can picture a similar
\q{conceptual} flow tracing how listeners come to make sense of the
language they encounter enunciated by speakers.  By way
of illustration, Figure ~\ref{fig:ESA} shows a
Depency-style descructuring, with implicit type annotations.  \begin{figure}
\caption{Dependency-style graph with argument repetition}	
\label{fig:ESA}
\vspace{1em}
\hspace{0.15\textwidth}	
\begin{minipage}{0.7\textwidth}
\begin{tikzpicture}

%\tikzset{snake it/.style={decorate, decoration=snake, segment length=5mm, amplitude=15mm}}

%\draw

%\node [s1] at (0,0) {Student};

\node (s1) at (1,1) {\textbf{Student}};
\node (after) [right=5mm of s1] {\textbf{after}};
\node (s2) [right=5mm of after] {\textbf{student}};
\node (complained) [right=5mm of s2] {\textbf{complained}};

\node (s1Rep) [double,draw=black,shape=circle,thick,fill=gray!50,inner sep=.5em,below=3.5cm of s1] {};
\node (afterRep) [double,draw=black,shape=circle,thick,fill=gray!50,inner sep=.5em,below=2cm of after] {};
\node (s2Rep) [double,draw=black,shape=circle,thick,fill=gray!50,inner sep=.5em,below=3.5cm of s2] {};
\node (complainedRep) [double,draw=black,shape=circle,thick,fill=gray!50,inner sep=.5em,below=2.5cm of complained] {};

\draw [ |-,-|, <->, line width = .8mm, draw=gray!70, 
 dashed, double equal sign distance, >= stealth, shorten <= .25cm, shorten >= .25cm ]
 (s1) to (s1Rep);

\draw [ |-,-|, <->, line width = .8mm, draw=gray!70, 
 dashed, double equal sign distance, >= stealth, shorten <= .25cm, shorten >= .25cm ]
(after) to (afterRep);
 
\draw [ |-,-|, <->, line width = .8mm, draw=gray!70,  
 dashed, double equal sign distance, >= stealth, shorten <= .25cm, shorten >= .25cm ]
(s2) to (s2Rep);

\draw [ |-,-|, <->, line width = .8mm, draw=gray!70, 
 dashed, double equal sign distance, >= stealth, shorten <= .25cm, shorten >= .25cm ]
(complained) to (complainedRep);
 
\draw [shorten <= .25cm, shorten >= .25cm ] 
(afterRep) to node [draw=black,shape = star,star points=4,thick,inner sep = 0mm, above] {1} (s1Rep);

\draw [shorten <= .25cm, shorten >= .25cm ] 
(afterRep) to node [draw=black,shape = star,star points=4,thick,inner sep = 0mm,above] {1} (s2Rep);

\node (s1E) [left=-2mm of s1Rep.east]{};
\node (s2E) [left=1mm of s2Rep.west]{};

\draw [shorten <= 2cm, shorten >= .3cm, double, % bend left = 5, 
decorate, decoration={snake, segment length=5mm, amplitude=.5mm}] %, amplitude=15mm]  
(s1E) to node [draw=black,shape = star,star points=4,thick,inner sep = 0mm, below] {2} (s2E);

\draw [shorten <= .5cm, shorten >= .5cm ] 
(afterRep) edge [bend left=20,looseness=1] node [draw=black,shape = star,star points=4,thick,inner sep = 0mm, 
 above, near end ] {3} (complainedRep);

\node (frameTopLeft) [below left = 1.5cm and -.65 cm of s1] {};
\node (frameBottomLeft) [below = 2.5cm of frameTopLeft] {};
\node (frameBottomRight) [right = 3.95cm of frameBottomLeft] {};
\node (frameTopRight) [above = 2.5cm of frameBottomRight] {};

\draw [shorten <= 0.15cm, shorten >= 0.15cm ] 
(frameTopLeft) edge [bend right=30,looseness=1] (frameBottomLeft);

\draw [shorten <= 0.15cm, shorten >= 0.15cm ] 
(frameBottomLeft) edge [bend right=30,looseness=1] (frameBottomRight);

\draw [shorten <= 0.15cm, shorten >= 0.15cm ] 
(frameBottomRight) edge [bend right=35,looseness=1] (frameTopRight);

\draw [shorten <= 0.15cm, shorten >= 0.45cm ] 
(frameTopRight) edge [bend right=30,looseness=1]
node [draw=black,shape = regular polygon,regular polygon sides=3,thick,inner sep = .2mm, 
above, near start, shape border rotate = 180] {4} (frameTopLeft);

%node [draw=black,shape = star,star points=4,thick,inner sep = 0mm, above, 
%bend left=100,looseness=3] {3}

%;


\end{tikzpicture}
\end{minipage}

\hspace{0.1\textwidth}
\begin{minipage}{0.8\textwidth}
			\renewcommand{\labelitemi}{$\blacklozenge$}
	
\begin{itemize}\setlength\itemsep{-.3em}
\item 1 \hspace{12pt}  Head/dependent relation
\item 2 \hspace{12pt}  Argument repetition ({}\BlankAfterBlank{} idiom)
\item 3 \hspace{12pt}  Propositional completion ({}\VisNtoS{})	
			\renewcommand{\labelitemi}{$\blacktriangledown$}
\item 4 \hspace{12pt}  Phrase (modeled as applicative structure), typed as {}\NPl{} 
\end{itemize}
\end{minipage}
\end{figure}
As this shows, the \q{Student after Student} idiom can be notated as, say,
\AfterNSingAndNSingToNPl{} (using \NSing{} and \NPl{} to mean singular
and count-plural nouns, respectively), but with the special case that the \q{argument} to
\i{after} is repeated in both positions, suggesting an unusual degree of repetition,
something frustratingly recurrent: \i{He went on and on}; \i{Car after car passed
us by}; \i{Time after time I got turned down}.
Although I have no problem
treating these constructions as idiomatic plurals, I also contend (on the
premise of phrase-overlap) that the dependent constituents in the \BlankAfterBlank{}
construction can be hooked to other phrases as well (which is why
\q{and [their/his/her] parents} can also be singular, in this case).  I dwell on
this example because it shows how type/functional accounts of phrase structure
can be useful even if we treat phrases more as frames which overlay linguistic
structure, not as rigid compositional isolates.  Each \q{students} variation uses
morphology to nudge cognitive attention in one direction or another, toward events or the
degree to which events are representative of some global property (here of
a student body), or both.  The \NSingToNPl{} transformation is not \i{the}
morphosyntactic meaning, but instead the skeleton on which the full meaning
(via cognitive schema) is designed, its hints solicited.
}
\p{If this analysis has merit, it suggests that a \CAG{}
approach to phrases like \i{many students} or \i{student after student}
(singular-to-plural or plural-to-plural mappings) should be understood not just
as functions among Part of Speech (\POS{}) types but as adding cognitive shading, foregrounding
or backgrounding cognitive elements like events or typicality in some context.
In other words, \i{many students} is type-theoretically \NtoN{} or \NpltoNpl{};
but, in more detail, it adds a kind of cognitive rider attached to the mapping which focuses
cognition in the subsequent discourse onto events (their recurrence and temporal distribution);
similarly \q{student after student} has a \q{rider} suggesting more of a temporal
unfolding.  The second form implies not only that many students complained, but that
the events of these complainings were spread out over some stretch of time.
Each such functional application (mappings between \POS{} understood as linguistic types)
produces not only a resulting \POS{} \q{type}, but also a reconfiguration of cognitive
attitudes toward the relevant situation and context.
Language users have many ways to craft a sentence with similar meanings, and arguably one
task for linguistic analysis is to model the space of choices which are available in a
given situation and represent what specific ideas and effects are invoked by one
choice over others.  It would be an argument in favor of Dependency Grammar if
Dependency-oriented representational models, like Link Grammar, prove to be
especially adept in this modeling.
}
\p{In this analysis, I am already switching to functional and type notions that will be
discussed in greater detail below; my current emphasis is on link grammar as a syntactic
conception, although I have also tried to argue that separating syntax from semantics
can be at most provisional.  Inter-word \q{link pairs} are vehicles for expressing
syntactic rules (like singular/plural agreement) but are also a ground level for
semantic analysis, since we can explain how semantic nuances are carried, in
specific sentences, by the actual link-pairs in evidence (violations to
agreement norms, for example).  These semantic nuances in turn can be given
cognitive interpretations, revealing the syntax-semantics-cognition pattern
which I am sketching here through specific perspectives like Link Grammar and
Type Theory.  Returning to the initial grammatic stage of analysis, however,
my tactics for contrasting overall Dependency and Phrase-Structure paradigms
rest on an implicit picture of how theories should be evaluated.
While such a picture is probably fairly consistent across perspectives, it is still worth
making a little more explicit.
}
\subsection{Explanation and Formality}
\p{Both Dependency and Phrase Structure grammars
presuppose that the fundamental exposition and achievement of their theory involves
formal transformation of linguistic givens, resulting in a more complex data structure
which, to the extent that the theory is correct and useful, models something
of the inner structure of language (\i{qua} abstract formal system and/or cognitive phenomenon).
The \q{data structure} might be a phrase-structure \q{tree} or a graph-like dependency
\q{linkage}, but while these representations have different form they share certain
criteria: they are formally describable systems which allow some structures but reject others;
they are rigorous enough to be given a mathematical (e.g., algebraic) definition;
and they can be expressed in computer code which builds these structures out of
Natural Language artifacts, can verify that an instance of the relevant data structures
satisfies the system rules, and can execute operations which modify the structures.
The phrase trees or word-link graphs are \q{formal substrata} which encapsulate Natural Language
patterns but also are rigidly mathematical and computational.  How thoroughly these substrata
capture linguistic meaning, is therefore directly relevant to questions of whether and
in what degree natural language itself, as social and cognitive, is also formal and computational.
}
\p{Translating NL content into (say) a linked-grammar graph does not make software capable
of \q{understanding} language.  If Dependency Grammar is a reasonable foundation
for linguistics in general, then properly parsing sentences into their
auxiliary graphs is, at most, a step in the direction toward \q{understanding}.
Even this may beg the question of what constitutes a \q{correct} parse: when writing
real-world code, language engineers appear to rely principally on their own intuitions, based
on their familiarity with the underlying theory, the idea they have of what a
\i{correct} \q{re-presentation} looks like (for link grammar, of the correct collection of
link types between the various words).  They then add code to ensure that this representation
is indeed identified by the software in specific examples, and try to do so in such a way
as to generalize to other examples.  This methodology can be gleaned from observing internet chat
sites and other informal research venues; one can witness developers painstakingly constructing
systems which \q{work right} in the sense of producing the interpretation for each sentence
which corresponds with what the human linguists perceive, even for sentences which the
software has never encountered before.  The code is considered reliable the more that
new sentences are \q{correctly} parsed.  Again, \q{correctly} here means, conformant to
linguists' own interpretations; insofar as these are subjective, such conformance is not
conclusive evidence that the transformational algorithms are \q{correct}.
}
\p{In order to assess linguistic \q{competence} (or whatever computational ability may simulate it),
it is needed to check specific \q{behavior} and compare it to some expectation.  The gold standard
for linguistic behavior is just participating in a linguistic community, judged by the community
at large as fully competent and included.  Unfortunately, however \mdash{} at least for those who
want to profit from Artificial Intelligence \mdash{} achieving true \q{language-like behavior} may be
impossible.  Scholarship therefore has to turn toward more limited notions of
competence, such as representational transformation of sentences \mdash{} but since
each theory has its own picture of what sentences should be transformed \i{into},
the justification of competence measures can be circular.  It is the theory which
dictates how the software should act, and the software is deemed \q{intelligent} if it
acts accordingly.  We can be skeptical of such non-theory-neutral conceptions of
\q{intelligence}.  Nonetheless it does count
in theories' favor if they both propose accounts of language structure which
are independently defensible and also can produce computing systems that
reliably and without external direction map language onto those structures.
Language-like behavior then involves producing a transformed representation of
language embodying a particular theoretical conception of
linguistic \q{deep structure}.
}
\p{It would serve Computational-Linguistic theories still further to create systems
that demonstrate behavior which is \q{language-like} on terms less wedded to their
own hypotheses.  More satisfying definitions of linguistic behavior would involve intuitions
of language users in general, not just language experts.  For example, document classifiers
\mdash{} which typically use statistical analysis to predict which topic will be deemed most
relevant for documents like news stories and technical articles \mdash{} again illustate a kind
of transformational representation, converting Natural Language to a formal data structure
(in this case a relatively simple one, naming one or multiple topics from a predefined list).
In this case however a broad user public can provide feedback on how well the system performs.
For another example, artificial translators map language onto formal structures but then
attempt an opposite map, translating the formalized representation into natural-seeming
expressions in a different language.  This case is different in that formal representation is
an intermediary rather than end point of the transformation, but like document classifiers it
is a kind of behavior whose effectiveness can be judged by a large community of speakers.
People who interact with text \q{chat} bots, or talking robots, and feel that the experience
is similar to talking with another person, are also providing evidence of more complete and
larger-scale language-like behavior.  Again, though, it is not now and may never be possible
to engineer intelligent behavior to this level of perfection.  Existing language \AI{} platforms
are flawed but useful, which suggests both that formal re-presentation is an important
step toward language understanding but also that attempts to use these formalisms as a springboard
to more holistic behavior \mdash{} like automated translation, but also extracting practical
information, or gleaning emotion and sentiment \mdash{} are missing something essential.
Doing useful things with or gleaning useful insight from the re-presentational target structures
appears to be a separate problem from that of generating them \mdash{} which calls into question the
degree to which the target structures sufficiently encapsulate linguistic meaning,
even if they reveal structures which are essential to linguistic meaning.
}
\p{This does not have to mean that Natural Language Processing is basically impossible, only
that more modest criteria of \q{correct} \NLP{} systems need to be adopted.  This is complicated
by the fact that artifical language behavior can be flawed but meaningful: \q{Urine shift one
step forward} is an awkward English sentence but its meaning seems clear enough
(this real example comes from a shopping center in New York's Flushing, Queens Chinatown).
We have an intuition that some expressions are \q{incorrect} but not so completely
off-base that they fail to signify anything at all \mdash{} but in this case we need
criteria for how a linguistic performance can be both incorrect \i{and} nonetheless coherent.
}
\p{These issues influence any theory which approaches linguistic competence from the
viewpoint of formal re-presentations, and therefore effectively all branches of
Computational Linguistics.  The reigning assumption appears to be that transformational
representation which converts language to theory-regulated data structures, for which
in many cases the transformation achieved by mechanical algorithms matches that
intuited as most accurate by human experts, serves as \i{prima facie} evidence of
something like computationally-engineered  \q{intelligent (language) behavior}.
This leaves room for language-like behavior to productively replicate dimensions
of language understanding while also being very incomplete: language-like relative
to experts' opinions on deep linguistic structure, not real-world communication.
Structures like link grammar graphs can be essential formal substrata that linguistic expression
relies on to achieve communication, without being the sole medium of this expression.
}
\p{My arguments so far have used Link Grammar as a representative example
of \q{transformational representation} where a computational system
can be judged to reveal some level of language competence, some
kind of \q{language like behavior}, insofar as it translates natural
language expressions to data structures conformant to Dependency
Grammar (and particularly Link Grammar) theory.
As I also just argued, performance \visavis{} structural transformation
may be only tangential to human language, so whatever theory is
built up needs a separate, more philosophical or metatheoretical analysis
to consider how the theory is purported to engage with its phenomena.
But I now take this as a starting point for pivoting the discussion from
grammar to semantics; and will defer until after that speculating
on philosophical implications of the theory thus extended.
}
