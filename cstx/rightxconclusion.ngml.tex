\spsubsectiontwoline{Distinguishing Computational Models From AI}
\p{I contend we need to tease apart the pursuit of valuable computational
tools and models from an (often reductionistic) paradigm of
seeking artifical, computationally engineered replicas of
human cognition.  \i{Computational} does not have to equal
\i{AI}.
}
\p{Holmqvist's and Selway's research that I have cited are good examples
of paradigm-overlap between cognitive and computational
linguistics.  I will cite other scholarship which
also finds philosophical inspiration in cognitive linguists
like Langacker, G\"ardensfors, George Lakoff, and
Eleaonr Rosch, but which also target cognitive-science
formalizations and \q{cognitiv architecture}:
\cite{AntonioLieto} etc.  A recurring pattern in this
scholarship is to \i{first} propose a structural intermediate
representation \mdash{} a model of intellectual structures which
plausibly embody the processing of language and
cognitive-perceptual content, partly abstracted from
surface-level sensory or signifying details \mdash{} and
\i{second} propose algorithmic or software
models of how our minds translate linguistic and perceptual
givens to abstract, or partly-abstract, schema.
}
\p{I have argued that we bring abstract situational prototypes
to bear on understanding all of the world and social situations
around us, and that language taps into these models so that
people can coordinate situation-appropriate activity.  Given
that there is an abstract and scehmatic dimension to
how we understand situations, we should expect a
partially abstract sheen to how we intellectually
engage objcts and concepts once they are situationally
\q{located}.  Having identified objects as
butter or carving knives, pitchers or glasses of
water, wine or beer bottles, corkscrews and bottle openers \mdash{}
identifications themselves mediated by situational
awareness, viz. if we are hosting or attending a dinner
party \mdash{} we no longer often attend actively to
sense-perceptual minutiae.  Our mental map of our
surroundings \mdash{} there's the corkscrew, there's
the carving knife \mdash{} pulls these referents
outside the register of sensate consciousness and
into the pragmatic hum of worldly activity.  Insofar
as they nestle in our intellectual faculties in that
semi-abstract state, it seems fair to capture the
schematic, structural appearance they have in
this intellectual register \mdash{} phenomena without the
full-cloth phenomenology.
}
\p{This in turn seems to invite us to imagine how the
structural essentials of such \q{pragmatic appearance}
may be captured by computers.  We do not need to endow
computers with human consciousness or emotions, because
our mental traffic with the corkscrew or carving
knife at some point evolves outside the sensate and
passionate fabric of momentary consciousness.  There
is a schematic and mechanical dimnsion of
human action, and we can imagine computers
simulating human intellligence at least on
\i{that} theatrical level.
}
\p{Or at least, such seems to be the intuition behind attemps
to model our human representations of objects and
concepts in terms of abstract structures.  But even a feasible
theory of these semi-abstract layers of cognitive processing
is only half the story.  Suppose we agree that there
are legitimate cognitive insights in Holmqvist's model of cognitive
frames, incorporating (but also extending, including in a more pragmatist
direction) Conceptual Space Theory \mdash{} employing
a generalized mereology that renders objects and concepts
as \i{parts} of situations (I have suggested a
more conceptual-role account for analogous phenomena).
Suppose also we find plausible cognitive-frame
models in Selway's intermediate representation
for natural language, via which
his proposed implementation can potentially
map natural language to formal specifications.
In these cases we hagve potentially
valuable Intermediate Representations which capture
cognition, in effect, mid-stream, or in-the-act:
neither conscious phenomenology nor neurophysical hardware.
}
\p{Howevere, Holmqvist's and Selways' work appears to
operate in an environment where these
Intermediate Representations are valued
primarily because and insofar as they allow
human cognition to be mechanically recapitulated.
This of course demands not only that
compuers \i{represent} IR models, but also \i{create}
them \mdash{} that is, when presented with an artifact of
natural language, or the visual data of a scene, that
computers should \i{automatically} map these givens
to the theorized IR models, as if retracing
the steps of human intelligence.
}
\p{But just because IR models can be given
computational form and representations, it
does not automatocically follow that automated
generation of IRs is possible or effective.
We can and should thereby distinguish the computational
\i{study} of cognitive Intermediate Representation
from the AI vision of programming computers
not just to \i{host} but to \i{derive}
Intermediate Representations.
I am sympathetic to the former methodology
but skeptical of the latter.
}
\p{I also believe that most research in, e.g., computational
linguistics, ends up conflating those two goals.  In that
case, IR models are judged based on whether
they facilitate automated, AI-driven generation
of IR, not on whether the IRs are insightful
suggestions of how human cognition itself
builds an intermediary cognitive register \mdash{}
paricularly if we accept Vakarelov's overall
picture of language as an interface between
speech-givens and prelinguistic cognitive
fsculties.  Interface theories and Intermediate
Representations tend to go together \mdash{} the IR is
the representation of some input during
intermediate processing yielding an
output; a structure between two other structures,
where the role of the interface is to bridge
the structures as well as to activate the correct
capabilities via the output.  This is the
architecture of an \q{interface theory}, in
science or computer programming; it carries
over to linguistics of we take Vakarelov's
ITM seriously.
}
\p{An equally intrinsic aspect of interface theories,
however, is that the processes operative at the intermediate
level aree theoretically distinct from the realms which the
interface bridges.  For example, the theory of programming-language
compilers and runtimes is distinct both from
the theory of programming-language parsers and
specifications, and from the theory of CPU
architecture and system-kernel development.  Runtime
engineers can work through the medium of IR
models, and compiler design itself is split between
parsing surface-level source code \i{to} IR and
mapping IR structures to their proper runtime
paths of execution.  It would be a breach of
design architecture to attempt to solve
source-to-IR problems within modules devoted to
IR-to-runtime engineering.
}
\p{Unfortunately, I get the sense that AI research does
not respect a comparably disciplined Separation Of
Concerns.  There are multiple parts to a
typical AI platform \mdash{} modules for representing information
(or knowledge/facts/beliefs, or the state of the system's
physical or digital environs, etc.); for populating
these reprsentations with data deliberately introduced
by human users or absorbed via some real-time engineering
from the outside world; for analyzing
reprsentations to glean insights or calculate a course
of action.  Individual parts of the overall architecture
can evince notreworthy engineering
achievements, separate from the goals of
the overall system.  In this sense the pursuit of
AI can yield positive contributions in other
branches of computer science and other disciplines,
without the stated rationale of AI realizing
(and monetizing) systems that exhibit
humanlike intelligence.
}
\p{So perhaps \q{AI} is
best understood as shorthand for a suite of
research agendas across several aspects of
computer science, not restricted to the
fields \mdash{} like Machine Learning, Robotics,
and Artificials Neural Networks \mdash{} that are
publicly associated with the term.  This is
not, however, how AI seeems to be represented by
companies and institutions (including in academia)
who have a vested interest in the products AI may
yield.  A benevolent reading would be that
institutions understand the diversity of research
that can be loosely aggregated under the AI umbrealla,
but use the particularly science-fictional facets of
this science to excite public support: visions
of humanoid robots and conversationalists provide a compact
story to that is more meaningful to non-experts
than technical outlines of the intermediate machinery
beneath the hopefully-intelligent surface.
However, a more cynical interpretation
is that AI is valued as a cash cow, and
residual disciplines which contribute to the
engineering infrastructure that AI requires
\mdash{} but are agnostic as to the AI vision
itself \mdash{} are appreciated only so much as
needed to keep the AI project moving forward.  On that
interpretation support for AI-agnostic
research becomes lukewarm and transactional, and
actual innovation in such areas may not be properly
celebrated.
}
\p{This situation is not irrelevant to either linguistics or to Cognitive
Phenomenology.  In Computational Linguistics,
for example, linguistic IR models seem
to be valued based on their utility in AI-driven
Natural Language Processing.  This presents a disciplinary
bifurcation, where potential computational models
are \i{either} connoted rather informally as part of
a thoretical investigation among linguists
(or philosophers of language, etc.), \i{or}
concretely implemented, in some kind of software package,
but then measured as components in a Natural Language Processing
system: assessed on the basis of how the system overall
approximates human language understanding via artifical means.
Another genre of formal models,
such as the type- or monadic theories I have alluded to here,
may also have potential software incarnations but
tend to be developed instead in a mathematical
style, effectively \q{programming} in the abstract
space of theorems and syllogisms rather than actual
compuers.  Each of these methodologies skirt
around the potential intermediary tie:
concrete computational systems that are
designed as exemplifications of semantic, grammmatical,
or pragmatic theories, presented as hands-on
software to anchor theoretical discussion but also intended
as tools to advance the human study of language, rather
than as steps toward synthetic avatars.
}
\p{At the same time, there is another side to the story: software
implementations offer a focal center for research, something
tangible that scholars can experiment and collaborate on.
The AI story provides a target goal; it helps developers understand
the local, technical code they are working on by
connecting it to a larger system.
Whatever philosophical objection one may have to AI
initiatives, we should recognize the value of
expanding academic and institutional practice beyond
just writing and reading research papers.  Insofar as
part of one's scholarly modus operandi can include
writing computer code, and studying code
repositories developed by others, we can benefit from a
hands-on, even trial-and-error kind of experimentation.
}
\p{In effect: software which can be given concrete tasks \mdash{} if it
does \i{this} properly (whatever \i{this} is),
then there is some larger thoretical point that
is demonstrated \mdash{} and then, incrementally, evolves
to realize those tasks, provides a distinct form
of intellectual engagement.  We get \i{that} to work, then
\i{that}, then fix \i{that}.  This kind of
\q{code-and-fix} cycle is quicker than conventional
research, especially in the humanities, where the
routines of authorship and publication and conferences can
feel like they are unfolding in slow motion.
}
\p{Perhaps for this reason, some of the most interesting
cognitive models hacve come from
computational and academic environemnts informed by
ambitious \q{Artificial General Intelligence}
programmes, like Carnegie Melon University's
OpenCog, and the \q{lmnTal} project at Waseda University
in Tokyo.  These projects both employ
formal-semantically expressive, hypergraph-oriented
data systems that embody both the structural
and procedural dimensions of computer systems \mdash{}
manifesting theories of both the execution
of computational processes and the representation of
formalized information.  These are important
models even outside the Artificial General Intelligence
ideology.
}
\p{In fact, these are models which in linguistics
and phenomenology may deserve more attention
than Artificial General Intelligence \i{qua}
philosophy.  But we should not discredit the
role that Artificial General Intelligence may
provide as a kind of intellectual compass helping
scholars and engineers reason through the
intrestitial machinery which may in fact be
more real than the philosophical vision, but
also less effective as theoretical \i{vie ferrate}.
Metaphors can triangulate research
whereas analogies guide transfers of theories
or methods between fields \mdash{} i.e., analogies are
more trustworthy landmarks than metaphors
for surveying the envisioned future of a science
\mdash{} but metaphors can still be intuitive guides;
maybe AI and Artificial General Intelligence can
stabilize into our overall science and
philosophy as a modest but suggestive metaphor.
}
\p{}
\p{}
