\spsubsectiontwoline{Phenomenology and the Limits of Logic-Compositionality}
\p{The perpetuation of 20th-century philosophical tropes 
\mdash{} Analytic vs. Continental Philosophy; humanities vs. 
logicao-mathematics \mdash{} perhaps diagnoss something incomplete in 
20th century thought that the 21st century needs to transcend, but 
it also bears witness to intellectual structures that directly 
abut phenomenology itself.  The idea, for example, that 
logico-mathematical analysis is distinct from cultural-humanities 
theory \mdash{} that works of art, for example, do not have logical 
or mathematical forms; that the structures evinced by cultural 
and artistic productions (and any system of signs and communication 
necessarily has \i{some} rigorous structure) does not itself 
have a logical and mathematical dimension, or a structural 
form that if specified in isolation from its signifying context 
would reveal logical and/or mathematical properties.  A parallel 
assumption is that logical and mathematical structures need to be 
expressed \mdash{}and indeed \i{located} \mdash{} via a few canonical 
theories, such as first-order predicate logic, first-order 
quantification, and (perhaps modally inflected) set theory.  
So 20th-century Analytic Philosophy unfolded in a 
kind of semi-explicit dialog with disciplines such as 
linguistics and computer science which promoted, from the 
full spectrum of mathematical domains, a handful of 
specific paradigms \mdash{} like Zermelo-Frankel Set Theory, 
Tarskian quantificational-predicate logic, and 
Hindley-Milner type theory \mdash{} that in turn became equally 
paradigmatic in fields like formal language analysis and 
\q{knowledge enginering}.  This is not to imply that 
linguists and computer scientists took their marching orders 
from philsophers \mdash{} it's probably equally true that Analytic 
Philosphers gravitated to these kinds of mathematical frameworks 
because they were proving empircally useful in other fields.  
But we can say that a group of disciplines aiming to 
technically examine subjects at the intersection between 
human thought and formal systems \mdash{} language, knowledge, 
valid inference, Artificial Intelligence \mdash{} collectively 
progressed toward a vision of what logicomathematical 
structure \i{is}, a paradigm (or network of paradigms) 
which undergirds any reception of Phenomenology insofar 
as a \q{Naturalizing Phenomenology} project seeks to 
initiate dialog and comparisons between how Phenomenology 
and Analytic Philosophy, respectively, treat logic, 
mathematics, or formal-systematic explanations in general.
}
\p{Interestingly, the logico-mathematical formulations that 
arguably have come to dominate methodology on the scholarly 
\i{frontier} between science and humanities \mdash{} lingustics, 
cognitive science, computer science \mdash{} are different from 
those that seem to dominate mathematics and physical sciences 
themselves.  Of course, some logical form is intrinsic to 
any formal method since formal systems cannot be blatantly 
self-contradictory.  But the vehicles by which criteria 
of logical consistency are expressed in, say, 
linguistics and philosophy \mdash{} presenting sentence-meaning via 
first-order logical glosses of their predicate 
(locutionary, assertorial) semantics, for instance, or 
equating meanings to the conditions in the world 
(or possible worlds) that would make linguistic expressions true 
\mdash{} these first-order and set-theoretic intuitions do not 
necessarily coincide with working mathematicians' visualization of 
the spaces where their own axioms and formulae are in force.  
Category theory, for example, recognizes that set and predicate logic 
are in a sense emergent properties of the different formal domains 
where mathematics is enacted \mdash{} sheaves, topoi, manifolds, sites, and 
so on \mdash{} which are not isomorphic, propagating to variation 
in how logic itself is understood.  Insofar as there is no single 
formalization to specify what a \i{predicate} is, or a \i{set}, 
we are on shakier foundation if we want to apply these notions to 
conceptual or linguistic investigations \mdash{} to consolidate the concept 
\q{cat}, say, as the set of all \i{felix catus}, or 
the meaning of \i{Hugo is sleeping on the sofa} as a particular state 
of affairs pertaining wherein a certain tabby cat sleeps on a 
certain piece of furniture.  The problem with this 
sort of logico-mathematical gloss is that it either employs sets and 
predicates in an informal manner \mdash{} which seems circular, since 
an honest assertion that Hugo is sleeping on the sofa obvious 
intends to report, modulo the imperfections of informal logic, 
that as a point of fact Hugo is sleeping on the sofa \mdash{} or 
it tries to elevate the logic from informal to formal, 
which legitimately then adds explanatory value to our technical 
grasp of the sentence, but then we get tangled in questions like 
what sets are.
}
\p{To put it differently, it certainly seems as if the 
sentence \q{Hugo is sleeping on the sofa}, as an intended and 
sensed artifact in a human world, in some fashion encapsulates and 
\i{designates} a proposition, and/or the state of affairs, that this 
tabby cat is in repose on that piece of furniture.  The sentence 
(deliberately produced sound patterns, perhaps) provides us 
with mental access or handle to that proposition, making it an 
unitary and isolated object of thought, as well as a thought that
can be shared between people via language.  There is a predicate 
there that seems to be \i{intended} in the Phenomenological sense 
by the sentence as a linguistic gestalt, just as Hugo is intended 
in our visual consciousness when we eplicitly see him on the couch, 
or at least analogous to that regime of intentionality 
insofst as Husserl finds both parallels and contrasts between 
\i{apophantic} and preceptual-integrative analysis.   
That is, the precicate or fact of Hugo's sleeping at that spot 
is the noema to the sentence's phenomena, just as Hugo is the 
noema to our seeing-Hugo.  In this sense it is non-circular to 
proceed with \q{Hugo is sleeping on the sofa} to mean the 
fact of Hugo's sleeping on the sofa, since we are exploiting the 
seemingly reundant repetition of the sentence to designate 
indirectly a predicate which we have no way of expressing except 
through the sentence.  So there is \i{something} there that seems to 
have a logical form we need (at least if our goal is theoretical 
completeness) to cast light on.
}
\p{But when we then pass from informal to formal logic 
we encounter the gap separating mathematics circa 2019 from  
mathematics circa 1919.  Hugo, for example, is a cat, 
which (in the precincts of informal logic, i.e. a logic 
solid enough to shape communication and coordinations of 
thought but backgrounded as only one of many structural 
parameters delimiting how thoughts and the world avolves) 
means that Hugo is a member of the set of all cats.  But 
what is a set?  Garfield (a fictional, comic-strip cat) 
is also a cat but he does not \q{exist}, so the set of all cats 
cannot be enumerated just by (even as a mental exercise) pointing 
to them one-by-one.  Moreover, what is referenced by the word 
or idea \q{Hugo} in the thought that Hugo belongs to the set of 
cats?  We can look at and point to Hugo to get a rough idea; but 
the fur Hugo just shed is not a member of the set of cats, even 
though it is (or was) a part of Hugo.  It is by no means 
obvious that we can, even as another mental exercise, define 
\q{Hugo} by an enumeration (metaphysically possible even 
if logistically infeasible) of hairs, or cells or molecules, that 
via mereological summation grounds the name in a 
logically well-established referential framework.
}
\p{On course, it is possible to logically restate a sentence like 
\q{Hugo, who is a cat, sleeps on the couch} in first-order logic, 
which seems to be arriving at a formal exposition of the predicate 
informally intended by the sentence \mdash{} the noema to the 
sentence's phenomena.  But formal machinery 
only papers over the logical incompleteness of concpts like 
\q{set} and \q{predicate}: a logical denotation of the sentence 
\i{seems} to refer to the state of affairs that the sentence 
tokens (and not just the sentence itself) via quantifier 
symbols and other logic artifacts.  I'd argue, however, that these 
elements are meta-discursive cues rather than theoretical posits: 
they invite us to entertain the thought of the predicate lurking 
beyond the sentence via a departure from discursive norms, 
the same way that simply repeating the sentence does.  In short, 
it may be non-circular to gloss \q{Hugo, who is a cat, 
sleeps on the couch} via formations like \ExistsX{} and the 
logical form gets us closer to apprehending the abstract predicate, 
but this is not because of logical symbols like \q{\Exists{}} 
in themselves, but because of their rhetorical effects as 
interrupting the flow of conventional philosophical writing.  
Symbols of quantification, set membership, or part/whole 
inclusion only carry the imprimateur of logical rigor 
insofar as we can synchronize what flavor of mathematics 
we are using with which to theorize quantification or parthood. 
}
\p{We can, for example, argue that Hugo is an extended 
three-dimensional thing (or maybe a four-dimensional 
perdurant, since he's still here day after day), and 
any reasonably compact and continuous \ndim{} manifold can 
potentially be designated in thought and language.  This is 
reasonable, but it positions our infrastructure for the semantics 
of proper names (and a large class of referring expressions) 
against a topological backdrop in which we can talk about sets 
of points, or perhaps regions spanned by extremum points 
(Hugo's fur as the convex hull of Hugo's solid mass) or just 
reasons as first-class posits in a point-free algebraic 
topology.  We can then present a case for the semantics of 
\q{Hugo} as a conventionalized sound adopted by a small group 
of humans to designate a particular four-dimensional manifold 
\mdash{} at least if we have a sufficient topological or Categorical 
foundation to give logical exactitude to concepts like 
\q{four-dimensional manifold}.  This still neglects 
merological problems, like Hugo losing some fur 
but still being himself; either the manifold Hugo 
is scattered wherever his fur (and, well, peepee and poopie) 
are or the Hugo manifold has fragmented into lots of disjoint 
other manifolds, only one of which is \q{Hugo}.  This latter 
option suggests that we need to identify the \q{essential parts} 
of Hugo, not the inessential parts like those in the litter box; 
then the Hugo manifold is the one that contains those parts 
\mdash{} i.e. we need a mereological formalization to embed in 
the topological one.
}
\p{It is certainly possible that by patching philosophical incompletions 
and importing mathematical vocabularies we can piece together a 
reasonably compelling theory of mereological/topological 
complexes, maybe with graded sets (some parts are more essential than 
others), to logically model the semantics of singular referents.  
By taking sets of such complexes, maybe 
with a modal machinery (for, say, fictional cats) or 
graded membership (some things are only partially 
subsumed by their concept: is a pet malawolf a dog?) we can 
similarly flesh out a set-theoretic semantics, and combining 
the singular and set analyses perhaps achieve a logically 
inflected semantics applicable to a broad class of linguistic events.  
But have we therefore truly surpassed informal logic, 
coiled within dialogic conventions, toward a rigorous 
theory of linguistic meanings via formal logic?  Do we need 
topology, mereology, graded sets, and modal logic because these 
are intrinsic to the epistemic domain where signification 
happens, or are we just grabbing an assortment of mathematical 
tools to plug the leaks in a philosophical attitude wherein 
there is a relatively straightforward formalization of 
quantifier-predicate logic that can polish the informal 
logic of language, but which perhaps failed to 
anticipate the mathematical complexities of formalizing what 
exactly are the \q{thises} that quantifiers can quantify over?  
}
\p{These questions are directly relevant to Phenomenology \mdash{} 
and to Analytic Philosophy's reception of and dialog with 
Phenomenology \mdash{} because a Phenomenological account of 
issues like manifold synthesis and mereological essentialism 
can easily bring forward formulations that overlap 
with a \q{logistic} paradigm in some details but not others, 
and profitably pursuing the dialog depends on identifying both 
similarities and differences.  The story of Hugo as a 
four-dimensional manifold might not be told in a Phenomenological 
context, but we might study our visual episodes involving Hugo 
in terms of perceptual synthesis: seeing Hugo from the front 
we anticipate that we \i{could} see Hugo's tail if we were 
standing behind him and it was not as much tucked beneath him; 
this anticipation is also a formula for how we integrate 
successive impressions as our gaze travels from his ears to 
his tail that we see partially.  There is a certain continuity and 
topograpgy of our accumulation of successive visuals tnat 
retraces a sort of manifold, even if now we're talking 
not about Hugo (or his body) himself but rather the 
interplay between the space where he's laying and my own 
perspctive and vantage.  I will examine the 
\q{vectors} of this protention and synthesis below; 
the point I want to emphasize here is that even 
on a Phenomenological basis we encounter fragments of analysis 
that point toward something like a mathematical treatment, 
but that doesn't mean we are necessarily committed to a 
quantitative-predicate logical theory of preception, or 
equally important, of the communicative acts that 
perception yields.   
}
\p{In pursuit of logical rigor, 20th-century Analytic Philosophy 
(plus, say, linguistics, cognitive science, and fields related 
to knowledge engineering and information technology, like  
AI-driven medical diagnosis) have highlighted 
specific logico-mathematical representation-regimes, 
building up a paradigm of logicality (as a philosophical 
quality, something one senses in analytic methods) 
and propositionality (as a corresponding analyzand).  
The various layers of this construction address different 
aspects of logic, or \q{logicality} as such, such as 
predication (that which can be qualified as true or 
false in the first place), quantification (establishing 
a scope for empirical truthness, so logic does not not decay 
into tautology \mdash{} that \i{one} thing or \i{some} 
members of set have a property where others don't, modeling 
contingency and a posteriority), modality 
(capturing the further contingency that most facts 
could have been otherwise), mereology and individuation, 
quantification scope, supervenience, and other logical 
themes that seem intrinsic to any intellectual territory 
where predication and true-falseness pertains.  
Codification of these facets of logic not only 
fine-tunes philosophical analysis; it also provides a 
semantics for natural language insofar as we accept that 
meaning is centered on locutionary truth-conditions (from 
which phenomena like illocutionary force and polarity 
\mdash{} our feelings and pragmatic intentions \visavis{} 
states of affairs at the predicate substrate of language \mdash{} 
can be reached in supplemental analysis) and a philosophical 
basis for the formal/computational representation of facts 
and knowledge (such as, in database design).  So 
insofar as a philosophical-computational-linguistic consensus 
emerged on the broad outlines of what logic entails, 
this paradigm spread into multiple disciplines, 
scientific as well as humanistic, and applied as well 
as purely speculative.  Insofar as Naturalizing Phenomenology 
involves dialog with Analytic Philosophy, it 
intersections with these different faces of logicality and 
the mathematical systems which prior generations believed best 
captured them.  
}
\p{But this does not mean Phenomenology needs to endorse the full 
stack of paradigms layered in Analytic Philosophy's 
excavation of logic; that we need to approach logical-mathematical 
eidetics via quantifier-predicate logic, for example, or to 
prefer the mathematical frameworks through which 
quantification and predication are usually modeled.  
Mathematics itself pulls in different directions: quantification, 
precication, and set-membership have different 
theories in different contexts, e.g. mereotopology compared to 
algebraic topology compared to Category theory.  These 
lines of development are largely tangential to Analytic 
Philosophy partly because the philosophical (and, say, 
cognitive-scientific) push toward formalized logic is not really 
a push toward \i{mathematics}, but rather a vision of 
bringing rigor to what is still essentially \i{informal} logic.  
That is, pursuing for instance a formal treatment of 
quantification to undergird natural language semantics 
does not mean building up to a rigorous, maybe 
Categorial theory of quantification in terms 
of (say) dependent coproducts, because what \q{logicality} 
does in language does not depend on that degree of technical 
detail.  But that means in effect that operationalizing 
the \q{logicalness} of language is not so much a problem 
in \i{formal} (mathematical) logic but a formalized 
\i{informal} logic, logico-linguistic or logico-mental 
(as in \q{folk theory of other minds}) more than 
logico-mathematical.  Fair enough; but this then places other 
analytic conveniences (like referents as 4-dimensional 
manifolds) on shakier ground, because we're no longer 
in an all-encompassing logico-\i{mathematical} paradigm.  
Topology and Category Theory, in effect, reveals the difference 
between formal-mathematical and formalized-informal logic.  
Trying to negotiate with mathematics as source of methodological 
rigor, mathematics proves to be an elusive partner.
}
\p{}
\p{}
\p{}
