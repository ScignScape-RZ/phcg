\section{Channel-Algebraic Grammar}
\p{This section will briefly introduce what I call 
\q{Channel Algebra} and how it can lead to a 
theory (and practice, in a sense) of formal 
and natural-language grammar.  Channel Algebra 
is discussed in greater detail in \cite{Neustein}.  
It's fairly divergent from other formalizations 
in computer science, though loosely decended from 
Process Algebras and from the \q{sigma calculus}, 
which is a formal model of Object-Oriented 
programming \cite{AbadiCardelli}.  Channel Algebra 
may also be seen as distantly related to Santanu 
Paul's \q{Source Code Algebra} \cite{SantanuPaul} 
and to a network of discussions \mdash{} not necessarily 
coalesced into technical publications 
\mdash{} about how to unify Object Oriented 
and Functional Programming.  There are many 
interesting analyses presented by scientists like 
Bartosz Milewski, on web forums such as 
Milewski's blog (the address is his full name as a 
dot-com domain).  In general, 
though, I am developing Channel Algebra 
in an \q{experimental} manner, using a concrete software 
implementation in lieu of a technical or 
mathematical axiomatic description.
}
\p{In the present context I want to focus on Channel Algebra as a 
potential theory in linguistic \mdash{} particularly 
Cognitive Grammar \mdash{} but initially I'll describe 
the underlying theory in a more computational manner.  
A lot of the Channel Algebra formalization carries 
between formal and natural languages.
}
\p{A key notion in Channel Algebra is \i{procedures}.  As in 
Part 1, we can think of procedures as either cognitive 
processes or as functions implemented in a software 
system, although for exposition the latter interpretation 
is simpler.  So, assume we have a computing environment 
where many functions are available to be 
called \mdash{} in effect, a bundle of software libraries 
each exposing some collection of function-implementations.  
For reasons I'll cover momentarily, I'll call thse 
\i{ambient procedures}.
}
\p{At one level, Channel Algebra is conceived as an alternative 
to data-sharing paradigms like the Semantic Web; so, one 
kind of analysis is concerned with cases where some body 
of information (which can be called a \i{data set}) 
needs to be transferred between two different 
computing environments.  Channel Algebra takes the 
view that data does not have intrinsic semantics outside 
the computational environments where it is used.  
As I argued in the context of Searle's \q{Chinese Room}, 
our identification that a software system represents facts 
\mdash{} like someone's full name (the example I used over several 
paragraphs in the earlier discussion) \mdash{} depends on the 
software possessing capabilities to display the 
information (usually visually).  In other words, 
among the total of all procedures that can be performed by 
the system, only a small set of procdures are involved in 
user-interactions where semantic intentions like 
\q{this piece of data represents someone's full name} are 
relevant.
}
\p{As a consequence, when sharing data that includes 
information like \i{full-name}, we should not assume 
that the raw data, in its semantic interpretation, 
actually \q{means} \i{full-name}, or some kind  
of propositional assertion about full names.  For 
example, a graph-edge in a Semantic Web resource 
intended to model the proposition \q{This person has full 
name Jane Doe} should not be seen as \q{meaning} anything about 
full names.  Instead, it represents some computational 
artifact which \i{becomes} an asserion of that fact 
when a procedure is eventually calld which converts 
the full name to a (usually visual) representation 
which a human user would recognize as a view 
on a full name (and hence on the proposition).
}
\p{In sum, the Semantic Web (or any data sharing 
network) only \i{has} a semantics because software 
connected by the network has requisite procedures to 
make data-views that people can understand.  Data does 
not have semantics (or at least not 
human-conceptual semantics); \i{views} do.  
This is consistent with an Interface Theory: most procedures 
manipulating Semantic Web data are part of an 
interface connecting networked data sources to the 
handful of procedures which create views for human users.  
Within the local structure of this interface, data does not 
have a \q{human} semantics; instead, it must be passed around 
between proceurs before eventually 
reaching human-interaction procedures where 
(what we would call) the \q{real} 
procedures come into play. 
}
\p{When data is shared btween localities, then, the procedures 
that will receive and manipulate this data are logicallly 
prior to the data itself and constrain when data-sharing is 
possible.  Without the proper network of procedures, 
the data can never be transformed into the views 
where non-local semantics are relevant.  
This motivates my choosing the term \q{\i{ambient} procedures}, 
because a certain collection of procedures must be in place 
on the receiver end of a data-sharing event.  This also implies 
that one important role of data modeling is to indicate 
which procures a potential receiver needs to have 
available \mdash{} i.e., needs to implement \mdash{} to qualify as a 
capable recipient of data conforming to the model.  
Data models should describe what procedural 
capabilities must be afforded by software libraries in 
order for the human-level, conceptual semantics of 
the data can actually emerge from humans' 
interactions with the system.  
}
\p{Analogous to Ontologies as data model specifications 
for the Semantic Web, I'll use the term \q{Ambient-Procedural 
Ontologies} to express the paradigm that implememting data-sharing 
protocols involves crafting software libraries around 
procedural requirements.  This has two implications for 
how we theorize software systems.  First, we need to characterize 
procedures in a manner that expresses the proceural 
capabilities that a system offers, or must have to satisfy a 
data-sharing protocol.  Second, we can assume that 
whenever data is sent, received, manipulated, or visualized, 
there is a collection of procedures available in the system 
which enact these computational processes.  
}
\p{On this basis, then, I will develop a Type Theory that operationalizes 
this intuition about \q{Ambient Procedural Ontologies}.  
The main outlines of this type theory are first that procedures 
have types; and second that procedures are \q{ambient} or 
logically prior to (or at least equiprimordial with) the 
type system itself.  This is not a mathematical type system 
where every underlying type (like Natural Numbers) and 
every operation (like arithmetic operators) have to be 
mathematically described in the theory.  
Instead, we can always take certain types and procedures as 
\q{primitive} or (at least in their inner workings) 
external to the type theory.  For any type system \tSys{}, 
whose structure is regulated by a type theory we intend 
to prsent or assess, we can say that \tSys{} is built 
around a \i{kernel} \kErn{} of \q{primitive} 
types and functions.
}
\p{In general, an assumption of Channel Algebra is that a significant 
portion of information, present in some structured system,  
can be extracted by identifying elements in 
the system with types in a suitably developd 
type system \tSys{}.  In the case of Natural Language 
\mdash{} specifically Cognitiv Grammar \mdash{} this means that 
many syntactic and semantic details for each word 
in a sentence can be provided by mapping words to types.  
As I have mentioned, linguists like Luo and Pustejovsky 
have given persuasive analyses of certain type systems 
for \i{semantics}, but I intend to apply 
type theory also to \i{syntax}.  
Later in this section I will demonstrate 
this in practice, but for now 
I will hust note that such analysis requires 
a sufficiently complex type system.  For example, 
semantic notions like dot-typs and dependent-product 
typs, which have proven to be effective in shedding 
light on common lexico-semantic phenomena, may need to be 
expressible in a \tSys{}, \i{along with} link- or cognitive-grammatic 
notions like \i{connectors} and \i{expectations}.  
}
\p{Similar comments apply to modeling and sharing 
scientific data.  Here, I have in mind projects 
like Conceptual Space Markup Language, and 
applications of Conceptual Spaces to study the nature 
and evolution of scientific theories, as reflected in 
research by (notably) Frank Zenker and Gregor Strle 
\cite{Zenker}, \cite{Strle}.  
Implicit in this research is the philosophy 
that scientific data models are not just electronic 
specifications for transmitting raw data, but embody 
scientific theories in terms of how data is structured and 
constrained.  CSML, for example, 
defines criteria on data parameters such as ranges, 
dimensions (called \q{units} in CSML), and structural 
protocols (including CSML \q{scales}) \cite[p. 6]{RaubalAdamsCSML}.  
As a concrete example, the biomedical concept \q{blood pressure} 
is usually understood as a pair of numbers whose dimnsions ar 
each in kilopascals (kPa) and whose first number 
(systolic pressure) is necessarily greater than 
the second (diastolic) \mdash{} technically, the pair 
is \i{monotone decreasing}.  In conventional Biomedical 
Ontology, cf. SNOMED-CT or the Vital 
Signs Ontology, these conditions might be stipulated by 
defining a \q{blood pressure measurement} concept subject 
to the relevant dimensional and range criteria (e.g. 
diastolic pressure must be greater than zero but 
less than systolic pressure), and/or by classifying 
systolic and diastolic pressure as subconcepts of 
blood pressure (see \cite{BarrySmithBlood} and 
\cite{AlbertGoldfain}).  In a kind of Ontology 
paradigm incoporating type theory, the same conceptual 
structure can be represented concisely by defining 
a type whose structure conforms to the 
semantic requirements on the \i{blood pressure} concept 
as it is scientifically understood: i.e., a 
monotone-decreasing integer pair whose 
dimensional units are labeled \q{kPa} or \q{kilopascals}.  
Note also that the \q{monotone decreasing} 
criterion is an example of the structure of 
dependent-product types (in this case because the valid range for 
the second number depends on the value of the first), 
a construction which elsewhere is used for Natural 
Language semanticcs, e.g. \cite[p. ?]{Luo}.
}
\p{Th point of this example is that many scientific concepts 
\mdash{} the semantic norms embedded in how scientific terms are defined 
and understood and how the concepts ar used in scintific theories 
and research \mdash{} can be rigorously specified with a suitably 
expressive type system.  Therefore, a sophisticated 
system \tSys{} is both a practical tool for scientific data 
sharing and scientific computing, but also an expository 
vehicle capturing theories' conceptual underpinnings.  
Developing scientifically useful type theories can 
then serve as a kind of formalized philosophy 
of science \mdash{} type theory as metascientific analysis.
}
\p{So in the past several paragraphs I have discussed the 
idea of using type attributions to represent semantic and 
syntactic details in natural language, and also 
metascientific concepts in scientific theories and data 
sharing.  A suitably developed type-system can, in light 
of these possibilities, act as a kind of multi-purpose 
tool capturing semantic principles in a broad array 
of formal and informal cotexts.  This is possible insofar as 
type theory is developed on a flexible basis so that 
type systems can expand in different directions for 
different intellectual environments \mdash{} dependent 
sum and product types for Natural Language semantics; 
\q{connectors} for Link Grammar; CSML-style 
units, ranges, and scales for scientific data; etc.  
The overarching goal of Channel Algebra is to 
enable these flexible, multi-purpose type systems.
}
\p{Another way of expressing this idea is that type systems 
should be \i{multi-paradigm}.  Suppose we are using a 
collection of types to model the semantics of some linguistic 
or scientific model.  We may realize that ther are some 
crucial semantic formulations that need to be recognized 
within the type system itself, if this 
type-oriented modeling is to be comprehensive.  For 
esample, in a link-grammar context, we may recognize 
that, in addition to assigning semantic and/or 
Part-of-Speech types to individual words, we need to 
represent the \q{potentialities} latent in words 
allowing them to link up (each word has a \i{connectors} 
such that a pair of 
\q{compatible} connectors can produce a \q{link}).  
Or, in a metascientific context, 
we may rcognize that we need to annotate type-attributions 
with CSML notions like range, scale, and units.  
The multi-paradigm criteria means that there would be 
a mchanism in place to enrich a type system 
\tSys{} so that such semantic details can be seen as
structural parts of types modeled via \tSys{}.\footnote{I refer to \q{semantic details} even in theories of 
Natural Language \i{syntax}, like Link Grammr, 
because many ideas about language \i{grammar} are 
relevant tobthe \i{semantics} of the linguistic 
\i{theory}, which is different than 
the Semantics of Natural Language which linguistic 
theories have as their subject matter.
} 
}
\p{One virtue of organizing a type system \tSys{} around 
\i{procedures} is that this \q{multi-paradigm} 
flexibility becomes easier to achieve when we can directly 
model requirements on procedures, and how procedurs 
interact with types (for instance, each type needs 
one or more procedures to construct elements 
of that type).  As I said earlier in this section, we 
can start from any \q{kernel} of types and procedures 
and build new types by describing procedures which 
create, and/or operate on, values of 
\kErn{}-types.  However, there ar many different ways that 
procedures can act on values and call other procedures.  
This means there is a lot of room for extending 
type systems to represent different kinds of inter-procedural 
relations.  For example, in Link Grammar a Part of 
Speech type is not just a \q{function} acting on 
other types \mdash{} e.g., adjectives act on nouns; 
\q{red apple} modifies the image we have when we 
conceptualize \q{apple}.  The adjective/noun pair also 
needs the right potentials to create a connector.  
A type-thoretic model of this idea can involve stipulating that, 
in some cases, a function-type \fFunT{} is not only 
defined by the type of its argument(s), but 
by some added \q{connector} structure which must math between 
\fFunT{}-instances and the instances of its argument types.
}
\p{Channel Algebra, as I will now argue, 
givs us a way to encode \q{extra information} 
along these lins in type descriptions.
}
\spsubsectiontwoline{Channels as Type-System Extensions}
\p{In concvtional type theory, every type system has \i{some} notion 
of functional types, but these are often treated in a 
simple form based on Typed Lambda Calculus.  
Canonically, a function \i{inputs} 
one or more values, and \i{outputs} on or more values 
(in many concrete type systems, only on output 
value).  We can (still rather informally) 
talk of these input and output collections as 
\q{channels}.  So, a procedure 
has an \i{input} channel \mdash{} that's where it vgets 
its arguments from \mdash{} and an \i{output}  
channel, where it provides a result.  The term 
\q{channel} is usually usd in the context of 
\q{process algebras} where procedurs can run concurrently, 
and channels may be two-way means of communication between 
concurrent procedures.  For this paper, I will only 
consider \i{sequential} procedurs \mdash{} where two 
procedurs follow each other in time; no 
parallelism \mdash{} and \i{one way} channels, which 
carry information from one procedure to another but have a fixed 
direction.  Intuitively, if \procOne{} sends a value 
to \procTwo{} via one channel, \procTwo{} cannot send a value 
back via the same channel (although it can \i{modify} the 
value in the channel).
}
\p{While type theory may conceive functions as (in effect) 
a combination of input and ouput channels, actually programming practice 
reveals multiple distinct \i{kinds} of input and output 
channels.  In Object-Oriented programming, Objects are 
passed to functions (i.e., \q{methods}) using a 
different protocol than ordinary parameters.  Also, many 
Object-Oriented languages support Exceptions, which are like 
non-standard return values that disrupt the 
normal program flow.  This suggests two different 
\i{input} channels \mdash{} I'll call them 
\i{lambda} and \i{sigma} for the calculii which 
can model their semantics \mdash{} and two different 
\i{output} channeos, which I'll call \i{result} and 
\i{error}, for normal and exceptional return values, 
respectively.  
}
\p{Furthermore, many languages support \q{lambda} or \q{inline} 
functions which can be \q{closures}, enabling a proceure 
to modify values in its \q{surrounding lexical scope}.  In 
other words, a \i{closure} is a procedure implemented 
inside the implementation of an enclosing procedure, and 
it has the ability to read and maybe modify data used by 
the enclosing procedure.  We can consider this 
\q{handing down} of values from outer to inner 
procedures to be a kind of input channel, which 
I will call a \q{capture} channel.  Some programming languages 
make this value-sharing explicit: in the case of 
modern-day \Cpp{} (since the 2011 standard) lambda 
functions have, as part of their definition, an 
explicit documentation of which symbols are 
\q{captured} from an enclosing lexical scope 
\mdash{} these symbols are listed in square brackets, just as 
regular arguments are listed in parentheses, which 
helps reinforce the idea that inputs via 
symboo-capture are similar to inputs via argument-passing 
(which I say in Channel Alegebr as \i{lambda} and 
\i{capture} being both input channels).
}
\p{Nested procedures can also be used to represent branching and 
control-flow, like \i{loop} and \i{if-then-else} 
formations.  For instance, \i{if-then-else} can be 
represented as a structure which involves two nested 
procedures: if some condition (in the outer, 
enclosing procedure) is true, then the first nested procedure 
is calld; if not, the second is called.  Similarly, a 
\i{loop} takes some nested procedure and calls it repeatedly.  
In the simplest case, the nested procedure is called again and 
again until the nested procedure returns in a manner 
that signals the loop should be brocken off (a common keyword 
for this condition, e.g. in \Cpp{}, is \q{break}).
}
\p{To demonstrate with a trivial concrete example, we 
might have pseudo-code like this:
\begin{sentenceList}\sentenceItem{} int $x$ = 0;
\sentenceItem{} loop \{
\sentenceItem{} \hspace{1em} if ($x$ > 10) break;
\sentenceItem{} \hspace{1em} ++$x$;
\sentenceItem{} \} // end the loop
\end{sentenceList}
If we want to analyze the inner code-block as a distinct 
(but nested) procedure, we would identify how 
the nested implementation \q{captures} the $x$ value.  
We also have to identify how \i{break} causes the nested 
procedure to terminate \mdash{} like \i{return} 
in an ordinary channel \mdash{} but does so as a signal 
for \i{loop} (in the outer procedure) to break off.  
In effect, a nested procedure used as a loop block 
has a spcial kind of output channel which can represent two 
possible states: \i{continue} the loop or 
\i{break} the loop.  I will call this hypothetical 
channel a \i{control} channel after Chung-Chieh Shan 
\cite{ShanControl} and Oleg Kiselyov \cite{KiselyovControl} 
(although they work in a different underlying context).  
Most programming language runtimes don't actually 
support the more \q{exotic} channels I discuss here, 
and without runtime implementations they can 
remain as just theortical descriptions of 
programming phenomena implemented or analyzed via 
some theortical framework quite different from 
what I am calling \q{Channel Algebra}.  
However, the data set accompanying this text 
demonstrates a runtine engine where the channel 
structures I describe here can be directly implemented.
}
\p{I'll also mention the Link Grammar example, say 
an adjective as a function modifying a noun.  
In a formal model, an adjective is therefore a kind of 
\q{procedure} which inputs a noun and outputs a noun.  
In my terminology, a bread-and-butter input 
channel is called \i{lambda}, and a bread-and-butter 
output channel is called \i{return}.  However, Link Grammar 
also recognizes various link kinds, each driven by connector-pairs.  
For example, consider a simple link-grammatic analysis 
of adjectives in the spirit of \cite[p. 16]{SleatorTamperley}.  
Note that most nouns take adjectives, but some words we 
might want to classify as nouns don't seem to:
\begin{sentenceList}\sentenceItem{} Today is Tuesday.
\sentenceItem{} The big departmental meeting is Tuesday.
\sentenceItem{} It is very windy. 
\sentenceItem{} The lousy weather is very windy.
\sentenceItem{} They are forecasting snow.
\sentenceItem{} The latest reports are forecasting snow.
\end{sentenceList}
We might want to consider \i{today}, \i{it}, 
and \i{they} as de-facto nouns, but notic the adjctival 
constructions do not carry over: we can't say 
\q{big today}, \q{lousy it}, or \q{latest they}. 
}
\p{In other words, some restriction on the \q{adjective} 
and \q{noun} types must be identified which 
blocks constructions like \q{big today} being parsed as valid 
exampls of \i{big} as an adjective type-instance.  
One option might be to define the adjective and noun 
\q{types} more narrowly, which is more in the spirit of 
type-theoretic semantics.  In that case, we 
don't take adjectives, say, as \q{functions} 
which input and output \i{any} noun, but rather 
model a suite of adjectival types operating 
on different noun types.  For example, the adjectives 
\i{salaried} and \i{elected} in 
\begin{sentenceList}\sentenceItem{} She is a salaried employee.
\sentenceItem{} He is an elected official. 
\end{sentenceList}
can only be attributed to persons, so qua functions 
these adjectives only \q{input} persons, as a subtype 
of nouns in gneral.  Via Link Grammmar, on the other hand, 
the basic theory involves adding extra conditions on 
the adjectives and nouns involved, which are required 
(along with the underlying type-compatibility) 
to permit the function (the adjective) to input 
the argument (the noun).   Then \i{today}, \i{it}, 
and \i{they} cannot take adjectivs because they do not have 
the proper \q{connectors}.  Of course, both idas 
can be combined, so we can define 
nouns and adjectives restricted to 
narrower subtypes (like person, living 
thing, physical object, etc.) and also marked with 
connectors, so adjective-noun pairings depend on 
compatibility both at the subtype level and the 
connector level.
}
\p{The important point for the current context is that 
connectors essentially extend any type system 
compatible with Link Grammar, so we need to imagine 
an extra kind of input and output channel representing 
\i{connectors} as orthogonal to underlying 
Part of Speech or lexical types.  In 
\i{big departmental meeting} we have to treat 
\i{departmental} as a kind of cognitive procedure 
modifying the concept \i{meeting}, and then \i{big} 
as a procedure modifying the \q{output} of the first 
procedure.  Then we \i{also} have to represent a 
\i{connector} on \i{meeting} establishing that 
this conecpt/lexeme can be modified by an adjective, 
and similarly a related connector inheres to 
the \i{output} of the \q{departmental} 
procedure.  So there is a kind of output-channel 
establishing which connectors ar eavailable on procedural 
outputs (connectors of the same varieties as 
would be available on individual words), and 
special input-channels which similarly model 
connectors which must be available on 
procedural inputs.  I'll call these input-connector 
and output-connector channels.  Type-thoretic semanticists 
like Luo and Pustejovsky might prefer to 
model connectors as aspects of types themselves, but we come 
closer to capturing the Link Grammar model if we 
reprsent the connector channels as distinct from 
the regular input and output channels, whose parameter-types 
are orthogonal to the language's connector-system.  
}
\p{I have, in any case, hereby prsented eight different kinds 
of channels applicable to different programming- or 
natural-language constructions: \i{lambda}, \i{sigma}, 
\i{capture}, and \i{input-connector} on the input 
side; \i{return}, \i{error}, \i{control}, 
and \i{output-connector} on the output side.  
Later I will introduce a few other channels.  
Notice though that the range of different channels, 
each with distinct semantics and theoretical 
roles, extends far beyond the basic intuition 
that every procesure has some inputs and some outputs.  
Instead, type theories can evolve to capture theoretical 
structures in diverse domains, because many theoretical 
concepts can be systematically represented by describing 
special kinds of channels applicable to certain procedures.  
One way to capture data models and theoretical 
commitments is to envision scientific models as 
organized \mdash{} implicitly or explicitly \mdash{} around 
some sort of \q{procedures} and then analyze the various 
protocols by which information is 
mapped into and out of procedures, with the \i{semantics} of 
these protocols described by stating requirements 
on specialized \q{channels}.
}
\p{The notion of \i{channel} is therefore closely tied to the notion 
of \i{procedure}, and \i{types} are similarly specified 
via both procedures and channels.  That is, most of 
th complex typs in a type system are functional 
types, which in turn are characterized in 
part by the kind of channels used by instantiating procedures.  
In principle, function-types are differentiated by the \i{kinds} 
of channels they use as well as the \i{types} of their 
arguments.  For example, in Object-Oriented 
programming, a \i{method} in a string class might 
take a string \i{object} (representing a string of 
textual characters) as the nethod \q{receiver}, 
aside from \i{arguments}: e.g., a \q{substring} 
method would take a pair of intgers.  This procedure 
would be considered to have a different type than an 
equivalent non-method function taking \i{three} arguments 
(one string and two integers) \mdash{} even if both versions had 
the same number and types of input parameters.  In \Cpp{}, 
\q{pointer-to-member-function} types are never equated to 
function-pointer types.\footnote{Note that \Cpp{} terminology is confusing in 
that pointer-to-member-functions can never point at 
\i{static} member functions (i.e., the fact that 
such functions are members has no bearing on their 
pointer-types), even though member functions themselves 
\i{can} be static.
}  The explanation for 
such distinctions in Channel Algebra is that a 
procedure which takes all inputs from a lambda channel 
has a different type than an (even if otherwise identical) 
procedure which takes one input from a \i{sigma} channel 
(equivalnt to the \Cpp{} \i{this} keyword).
}
\p{In Object-Oriented environments, non-static member-functions 
(aka \q{methods}) are distinguished from non-methods 
in part because there are different rules for resolving method-calls 
when a given function name refers to several different 
procedural implementations (such as a base-class function and 
a derived-class function which overrides it).  
This is one example of how \q{sigma} and \q{lambda} channels 
have different semantics: the types carried within 
sigma channels are more consequential for overload-resolution than 
those in lambda channels.  Conversely, sometimes channels are 
not semantically significant enough to decisively 
differentiate types.  In \Cpp{}, a function which 
\i{does} throw an exception cannot be overloaded 
with a function which \i{does not} throw 
an exception (assuming the rest of the functions' signatures 
are identical).  This means that \mdash{} if we model type systems like 
that of \Cpp{} via Channel Algebra \mdash{} procedurs 
with and without error channels can gbe the same type.  But it 
is always possible \i{in principle} to differentiate function-types 
based on the kinds of channels their associated procedures use; 
just that somethimes this results in fine-grained distinctions 
not recognized by a particular type system.
}
\p{The larger point is that function-types are specified 
via channels (\q{modulo} potential coarsening that equates types 
that could potentially be distinguished).  Therefore, the 
fundamental fabric of the type system is dependent on 
modeling channels, because this determines how 
function-types themselves are miodeled; and, as I will now 
explicate, function-types ar ethe logical core of 
almost any practical type system
}
\subsection{Channels and Carriers}
\p{Most type systems take it as self-evident that any type 
is associated with \i{values} of that type, and usually 
\i{sets} of values.  That is, intuitively, for any 
type (say, 16-bit signed integers) there is a 
set of values whichba are the \q{inhabitants} of 
that type.  One weakness of this picture is that 
many types vary in terms of \i{what} set of values is actually 
representable in a given computational context.  The 
case of 16-bit signed integers carries no such ambiguity; 
this type always has an extension equivalent to the 
interval -32768 to 32767 in $\mathbb{Z}$.  However, 
consider the type of \i{lists} of 16-bit signed integers; 
depending on a program's available memory, some relatively 
long lists which can be represented on one computer will 
exceed the capacity of a different computer.  Since computer 
code in general is not tied to a specific environment, 
we have to accept that for many types we 
do not know \i{a priori} what set of values 
conformant to that type can actually be used.
}
\p{In addition to this practical problem, it is also difficult to 
describe exatly what a value \i{is}.  Any instance of any type 
(at least in the software ecosystem) 
does have a numeric manifestation in computer memory \mdash{} 
essentially an encoding as a sequence of bytes, that is,  
8-bit unsigned integers \mdash{} but it is not obvious that such 
encodings (sometimes called a \q{bit pattern}) actually 
are \q{values} of types.  In reality, computer code almost never 
deals with \q{values} per se, but rather deals with types 
represented symbolically in function signatures: passing 
\q{values} around from procedure to procdure.  One exception to this 
rule is that some values are written directly into computer 
code \mdash{} the numeric literal \q{10} represents a specific 
value (usually in some intger type).  However, even here some 
procedure is needed to interpret the character strings in 
source code to the actual typed value.
}
\p{Given these considerations, while I will informally talk about 
\q{values} I try to minimize the use of \q{value} as a technical 
construct in Channel Algbra.  With this foundation we 
can consider a kind of \q{value free} type theory by 
rough analogy to \q{point-free} topology 
(and therefore distantly to mereotopology).  More central than 
the notion of \i{value} is the idea of a \i{carrier}.  
A carrier is a computational construct \mdash{} symbolically 
represented by a source-code token \mdash{} which represents an 
instance of a type.  Each carrier is asspociatd 
with on canonical type (though carriers may be indirectly 
associatd with supertypes of its canonical type).  
When we talk about procedures calling other 
procedures, we really mean that the value of symbols in one 
body of code (whre the calling procedure 
is implemented) is sychronized with tge value of symbols in 
anothr stretch of code (where the callee is implemented).  
In short, at any stage of program execution, we can form 
groups of source-code symbols across the code base unified by 
the guarantee that these symbols carry \q{the same} value.  
But rather than talking of values directly we can 
take this syncronization between symbols as the deeper notion: 
the notion \i{value} is itself defined as the correlation exhibited 
by synchronized \q{symbols}.  The notion of 
\q{carrir} is then a more rigorous extension of the 
notion of source-code symbols or tokens.
}
\p{The difference between a \i{carrier} and a \i{type} is that 
carrier have aditional states.  Some carriers are defind in 
function signatures, but others are introduced as lexical 
symbols in a procedure implementation.  In many programming 
languags, lexical symbols can be \i{declared} before being 
\i{defined}.  We can represent this scenario via a \i{carrier} 
which is in a particular \i{state} (I'll call it 
\i{preinitialized}).  When a carrier is initialized, it takes on 
a sftate of holding a specific typed value \mdash{} \i{values} are 
defined indirectly as characteristics of carriers in initialized 
states.  At some point (consider a pointer to deleted memory) carriers 
no long hold meaningful values, and they enter a 
state I'll call \i{retired}.  Introducing \i{preinitialized} 
and \i{retired} as carrier-states allows these to be separated from 
the type-system: we do not have to assume a \q{preinitialized} 
\i{value} which can be an \i{instance} of some types.  In some cases, 
type systems \i{will} reognize values which play 
similar semantic roles to these carrier-states: for example, 
Haskell's \q{bottom} value is an instane of ervery typ and represents a 
\q{null} or \q{missing} value.  However, using semenatics of carrier-states 
rather than type-instancs means that we do not have to introduce extra 
structure to type systems which we may want to model via Channel Algebra.
}
\p{Carrier which can be in preinitialized, retired, or initialized states 
are called \i{tropes} (there may be only one, or multiple, 
initialized states for tropes).  A different class of carriers ar e
called \i{emblems}, and represent abstract (\q{emblematic}) 
specfications on 
carriers rather than carriers which hold concrete values: 
in effect, emblems are carriers present in function 
signatures.  One or more carriers in an ordered list then 
form \i{channels} (though a channel can also be \i{empty}, 
with no carriers).  I distinguish an \i{empty} channel, 
which exists but has no carriers, from a 
\i{vacant} channel which does jot exist at all.  For instance, 
a function that \i{can} throw exceptions but, at some point, 
has returned a normal value instead, has an \i{empty} 
error channel; a function which can \i{never} throw 
an exception (e.g. the \Cpp{} \i{nothrow} keyword) has 
a \i{vacant} error channel.  Channels cannot include both 
tropes and emblems; those taking tropes 
are called \i{staged} channels, and those taking emblems are 
called \i{abstract} channels.
}
\p{Carriers and channels interoperate according to several 
operators, which give Channel Algebra its \q{algebraic} character.  
To present these oprators I'll also intoduce the notion 
of \i{stages}.  Briefly, any procedure is broken down 
into a sequence of stages, each of which involves 
constructing som aggregate of chanels.  The basic 
outline is as follows:
\begin{description}\item[Carrier Append]  Any staged channel can append 
a trope, and any abstract channel can append an 
emblem (represented as \ChaAppendCar{}).
\item[Channel Product]  Any collection of abstract 
channels can combine to a \q{product}, called a 
\i{channel complex}.  Similarly, any collection of staged  
channels can become a channel \i{package}.  
A channel complex or package is generically called a 
channel \i{product}.  A channel product is considered 
\q{complex} if any of its chanels are abstract.  
The basic channel-to-channel operator \ChaProductCha{} 
represents a channel product formed by combining 
\ChaOne{} and \ChaTwo{}.
\item[Carrier Handoff]  Given carriers \CarOne{} and 
\CarTwo{}, a \i{carrier handoff} \CarHandoffCar{} means 
that the value carried by \CarOne{} is (at last temporarily) 
carried to \CarTwo{}.  This means that there is some phase 
of program execution when \CarOne{} and 
\CarTwo{} are synchronized, or \q{aligned}; exhibiting the same state 
(or sufficiently related) states.  Alignment allows for 
imperfect handoffs, like coercing a floating-point 
number to an integer.
\item[Digamma Reduction]  A channel \i{package} 
can be \i{allied with} a channel \i{complex} 
(written \ChpAlliedChx{}) if carriers in \Chp{} and \Chx{} 
are \q{alignable}.  The actual description of such 
alignment is non-trivial; in \cite{Neustein} I address this 
in terms of hypergraph models of computer code.  But in 
general alignment means that handoffs are possible between 
tropes and emblems, and if there is enough alignment 
the chanel \i{package} \Chp{} can be interpreted as a call 
to a \i{procedure}, whose \i{signature} is modeled by the channel 
\i{complex} \Chx{}.  In this case we have a \q{Digamma Reduction} 
operator \DigammaR{}, meaning that the package \Chp{} is 
\q{evaluated} and then control passes to the stage 
labeled \stageLbl{}.\footnote{The motivation for the term \q{digamma} is first that the \q{sigma} in 
\sigmaCalc{} looksblike a smaller version of th Greek letter Digamma, 
and second that \q{gamma} is often used to represent graphs, and 
\q{digamma reduction} can be sen as a relationship between 
two different source-code graphs.
} 
\end{description}
}
\p{Each of these operators represent a step in a computational 
process whereby channel packages are constructed and then 
evaluated, which overbthe course of several stages 
(in general) provides implementation of procedures.  
A single operator is associated with a \i{microstage}, 
for example, appending one carrier to one channel.  
A sequence of microstages is than an \i{intermediate 
representation} used to translate high-level source code 
to data structures that can be executed direcly 
(or more or less directly, by interfacing to 
\q{ambient procedures}, e.g. via \Cpp{} reflection).  
This strategy is concretely put into practice 
in the sample code distributed with this paper.  In effect, 
high-level source code is \q{compiled} into an 
intermediate representation, and this IR is a more or less 
direct translation of Channel Algebra oprations.  
}
\p{One of the technical challenges of using this strategy for 
practical software developments involves mapping 
sophisticated Channl semantics to ordinary 
type systems.  Since a language like \Cpp{} does not 
support the full range of channels I have or will presented 
here, the more detailed channel-complees have to be mapped to 
function-call semantics which  \Cpp{} will actually 
recognize.  In practice, alot of this work involves 
manipulating function pointers and casting carriers 
to generic binary representations, like arrays of 
\voidP{}-pointers.  These techniques are not especially 
relevant to the philosophical issues I am focused on in 
this paper, so I'll leave them to the published 
data set (interested readers will find a relatively 
complete \Cpp{} development environment, intended to be 
used with the \Qt{} platform, that operationalizes the 
theory I am developing in this section).
}
\p{At this point however I \i{will} comment 
on the overall architecture of building and then evaluating 
channel packages.  Supporting \q{Digamma Reduction} 
in a runtime encironment requires reasonably complex 
\Cpp{} libraries (assuming \Cpp{} is the language 
of the runtime itself); but most of 
the Intermediate Representation is concerned with 
asserting carrier properties and then adding carriers 
to channels.  Channels themselves act as stack machines, 
in that they can be built up like machine 
stacks and then cleared in the passage from one 
stage to the next.  This is an example of a 
point I made much earlier, in the introduction, that 
computing environments ultimately 
involve stack machines at their most primitive level.  
}
\p{We are, in any case, operating here on several different 
semantic levels.  The Channel Algebra operators define 
one, intermediate level whose main theme is building channel 
packages.  At a \q{lower} level each package must be evaluated 
at runtim, for instance by converting it to a 
\Cpp{} function-call, so the semantic issues there are 
identifying the proper \Cpp{} function (or function-pointer) and 
marshaling the channels to mimic the \Cpp{} ABI (Application 
Binary Interface).  Conversely, at a higher level, 
source-code formations are interprted in terms of the channel 
products they imply: so in code like \i{x = y->f(z)} the 
implicit channels are likely populated as follows: $z$ goes 
into $f$'s \i{lambda}, channel, $y$ into its \i{sigma}, 
and $x$ becoms bound to its \i{return} (of course whether this 
is the actual meaning of the code depends on the 
high-level language's formal grammar).  This implies 
that grammars can be organized around the channel structures 
indicated by conformant code.  To the degree that 
grammar engines adopt this paradigm they can be 
called \i{Channel Algebraic Grammars}.
}
\p{This paper's data set includes one example of an 
implmented Channel Algebraic Grammar insofar as code 
written in a special high-level language is translated to 
a Channel Algebraic Intermediate Representation 
(and then evaluated).  The syntax and semantics of 
this spcial-purpose language is, of course, 
only distantly related to \i{natural} language; 
I won't address whether a comparable architecture could 
yield a workable Natural Language Processing engine.  
However, the design and role of this Channel Algebraic 
Grammar in the \i{Software Language Engineering} 
problem-space can perhaps serve as at least an 
analogy for how \i{natural} language 
orchstrates the \q{flow} of information, 
and the ordering of operations, across the cognitive 
proceures which underly linguistic understanding.  
I will return to this possibility at the end of the section. 
}
\subsection{Channels and Constructors}
\p{As I indicated, Channel Algebra downplays 
the notion of typed \i{values} except in a 
derivative sense (e.g. a \q{handoff} means an 
alignment between carriers which we can picture as a 
value being copied).  However, carriers do 
become initialized, and even if this initialization 
results from a handoff originating with 
another carrier, that carrier in turn had to be initialized.  
At some point these chains of initialization 
have to be grounded on some underlying data.  
This data may come from outside the software 
itself: for example, a program emulating a 
calculator relies on human users to type numbers 
on a kyboard or by pushing buttons on a User Interface.   
Other times values are literally written 
in computer code, or 8btained from files, databases, 
or over a network.  
}
\p{For sake of discussion, we can limit attention to 
values read literally from source code.  Even external 
data tends to depend on some numeric or string literal: 
reading data from a file requires specifying a file name, and 
obtaining data from a network location requires a 
URL.  So computer code itself provides the 
primordial values from which other values 
circulating through the software are derived.
}
\p{In a typical type system, then, at least some types 
should have procdures which construct values directly 
from source code literals.  I'll call these 
\i{literal-constructors}.\footnote{In \Cpp{}, similar constructor-functions are called \i{literal 
operators}.  This terminology only applies to \q{user-defined} 
constructors; the compiler itself handles initialization 
for built-in typs like integer and floats.  However th 
overarching term \q{literal-constructors} is helpful for 
theory-focused languag-agnostic discussion about types and 
constructors.
}  In addition to literal-constructors, 
there are some procedures to create type values from 
other values (or from no values at 
all); I will call these \i{co-constructors}.\footnote{This terminology has the benefit of distinguishing 
the property of functions I call \i{co-constructors} from 
what existing programming languages 
call \q{constructors}, which itself has different meanings for 
different languages.  For example, in \Cpp{} one 
cannot take the address of a constructor function; 
however, co-constructors are implemented such that 
you \i{can} have co-constructor pointers, which is essential 
to the idea of \q{preconstructors} that I will address below.
}
}
\p{There is no strict rule separating constructors for a type 
\tTyp{} from other procedures that may return values of 
\tTyp{}.  Intuitively, they play different roles: constructors 
are about creating new values, while other functions 
analyze or modify existing valus.  The canonical examples 
are so-called \i{trivial} constructors, which 
take no arguments \mdash{} for instance, the constructor for a 
list of numbers returns an empty list.  Such a 
function surely qualifies as a constructor.  
On the other hand, consider a function which takes a 
list of numbers and returns a new list 
with duplicate numbers removed \mdash{} such a procedure 
acts more as a utility function and probably 
would not be classified as a constructor. 
}
\p{The constructor/non-constructor distinction is anyhow 
typically left open by the programming language 
environment, so coders have to signal their 
intention to treat a given function as a constructor by 
some extra syntax.\footnote{E.g. in \Cpp{} a constructor is given the same name as the 
type it constructs for.  One consequence is that 
two different constructors cannot be declared without 
overload-resolution: each constructor has to have a different 
signature than any other constructor of the same type.  
However, the rationale for this restriction seems to be 
syntactic more than semantic.
}  In Channel Algebra, one option is to 
define a special \q{\i{construct}} channel for values 
output from constructors.\footnote{It is useful to pair this channel with a semantically 
similar channel I'll call \i{placement} (after 
\q{placment new} in \Cpp{}) for scenarios where the 
constructed result should be written to a buffer 
provided by the calling procedure rather than returned. 
}  Consider the case of appending values to 
a list of numbers: should a function which 
maps \LappendX{} to \Lprime{} \mdash{} where \Lprime{} is 
the same as \LpreX{} except it has $x$ at the end 
\mdash{} be considered a constructor?  
In most functional programming languages this 
is actually a classical case (along 
with trivial ones) of constructors, because they 
provide the basic mechanism wherein instances 
of list types are defined.  In other words, 
this \i{append} function seems logically anterior 
to other functions which may create a list 
\mdash{} for instance, by \i{removing} the last 
element from a (non-empty) list. 
}
\p{Such a notion of \q{anteriority} can sometimes be made 
rigorous.  If we build up a list of numbers by appending 
values to smaller lists, we can eventually construct 
any list whatsoever.  We can also run this process 
in reverse: for non-empty \Lprime{} there is only 
one way to construct \Lprime{} from a pair 
\LappendX{}.  In other words, an \i{append} 
constructor is \q{reversible}.  Moreover, we 
can repeatedly reverse constructors like 
thse to get shorter and shorter (and eventually 
empty) lists.  This means that algorithms can 
traverse a chain of constructors \q{backward} and 
will be guaranteed to terminate.  For example, 
to dtermine if a list of numbers contains 
the number $n$, it is easy to check if it \i{ends} 
with $n$.  If not, \q{reverse} the construction, and 
see if each smaller list ends with $n$.  Eventually we 
would reach the mpty list, meaning that the 
original list did \i{not} have $n$. 
}
\p{The possibility of \q{reversing} constructors is a familiar 
pattern in functional programming, where it is often called 
\q{pattern matching}.  It allows algorithms to be 
implemented in a functional style, with heavy use of recursive 
functions and sparing use of side-effects and mutable state.  
On the other hand, pattern matching relies on some 
simplifying assumptions that may not be consistent with 
all type systems.  For example, \Cpp{} is laxer about how 
valus are constructed; I can obtain a list of numbers by 
dereferencing a pointer to a list, with no information 
about the provenance of the referenced data.  Data types in 
\Cpp{} do not usually carry around details about their 
\q{history}, and it is not always easy to 
reconstruct that history the way we can 
\q{obviously} reverse the construction of a list.  
}
\p{In any case, these variations present some potentially 
informative characteristics about individual types.  
Given type \tTyp{}, we can ask questions like: 
\begin{enumerate}\item{} Which instances of \tTyp{}, if any, can be the result of 
a trivial constructor?
\item{} Does \tTyp{} have a \i{default value} whish is the result of 
a trivial constructor? 
\item{} Which instances of \tTyp{} can be the result of a 
literal constructor?
\item{} Which instances of \tTyp{}, if any, can be the result of 
a reversible constructor?
\item{} If we have values which \i{are} the result of 
a reversible constructor, is there an efficient way 
to \q{un-construct} the value to support pattern matching?
\item{} Does \tTyp{} have co-constructors (i.e., they 
are not literal) which also are neither trivial nor reversible?
\end{enumerate}  
Notice that a trivial constructor does not necessarily produce 
a defaukt value.  For example, a type meant to represent 
days of the week could default to whichever day is current 
when a constructor is called: if an application is run on 
Tuesday, the day-of-week trivial constructor would return 
the value for Tuesday.  So a type may have \i{more 
than one} trivial-constructed values.  But 
if a type has \i{exactly one} trivial-constructed value, 
this is \i{usually} a \q{default} value.
}
\p{However, a type can have a default value without a trivial 
constructor.  A defalt value plays the 
conceptusl role of a \q{fallback}: 0 is a reasonable 
default for most numeric types.  However, some numeric types 
don't have trivial constructors at all.  Meanwhile, types 
representing calendar dates sometimes default to a 
standardized \q{day zero}, like January 1, 1970.  
But a trivial constructor for such a type may instead 
return today's date (when the function is called).  When a type has 
a default value, a (co-)constructor which returns that value 
can be called a \i{default} constructor.\footnote{For an example of a non-trivial default constructor, consider a 
literal constructor for a ratio type that returns 
0/1 given a malformed character string.
}  As the Calendar Date example shows, 
types can have default and trivial constructors 
that return different values.
}
\p{Sometimes (co-)constructors are significant even if they are 
not actually used.  To demonstrate how this may occur, consider 
the problem of enforcing dependent-type constraints without 
using dependent types explicitly.  A canonical example is 
a function which must take a (maybe monotone) 
increasing or decreasing pair of numbes (I used the 
monotone-decreasing example for systolic and diastolic 
blood pressure).  Suppose we implement a procedure to 
highlight a selection of characters in a screen display,
whose inputs are the start and end character indices in 
some displayed text.  We want to indicate that the second number 
(call it $y$) must be greater than the first ($x$).  
Via unrestricted dependent types, we could just define 
the type of $y$ as \q{numbers greater than $x$}).  
The problem with this type-declaration is 
that we therefore do not know what type $y$ has until 
the function is called (and $x$ has a 
value).  This could violate the principle that 
every argument to a function must have a type which 
is known in advance.  Alternatively, we can 
say that $y$'s type \q{provisionally} is, say, 
a 32-bit integer the same as $x$, but it's \q{real} type 
once $x$ is known is a dependent type that depends on 
$x$'s value.  Not every type system however allows 
for this kind of distinction between \q{real} and 
\q{provisional} types.
}
\p{One way to assert the dependent-type restriction while avoiding 
these problems is to note that $x$ and $y$ must form a 
monotone-increasing pair.  That is, the procedure 
mandates the \i{possibility} to create the monotone-increasing pair 
\xy{} from $x$ and $y$, even though the procedure 
does not use this pair-value directly.  As a construct 
in Channel Algebra, we can introduce the idea of 
a \i{rider} channel which asserts the \i{possibility} 
of creating certain values without actually creating them  
The simplest case is suggested by this number-pair 
example.  Let's assume we have an implemented type 
modeling monotone-increasing pairs, with a co-constructor 
that creates such a pair from two numbers 
(verifying that the two numbers are indeed monotone-increasing).  
We can then form a pointer to this co-constructor 
function, or obtain some other unique identifier for it.  
I call a function-pointer or similar value a \i{preconstructor} 
if it references a co-constructor.  In addition to 
being an indirect way of calling the co-constructor, 
a preconstructor serves as a \i{certification} that the 
co-constructor \i{could} be called.  For example, one 
way to indicate that $y$ has been checked and 
is indeed greater that $x$ is to use the 
monotone-increasing pair preconstructor as a kind of 
signal value: test for $y > x$, but if so, 
instead of using boolean \i{true} for an 
affirmative, use the preconstructor.   
}
\p{The key point here is that many conceptual details or 
programming requirements that concern interrelationships 
between values can be modeled within a 
type system \mdash{} even without full-fledged 
dependent types \mdash{} via preconstructros used as 
signal values.  For a given requirement, such as 
\q{is $y$ greater than $x$?}, we can identify a type 
that \i{could} be constructed if (and only if) 
the requirement is satisfied.  For instance, 
\i{$y$ is greater than $x$} is true iff we can form an 
instance of a monotone-increasing pair type out 
of $x$ and $y$ \mdash{} and so, the preconstructor 
for this type becomes a convnient signaling value 
for the affirmation of $y > x$.  Conceptual requirements 
can be defined by modeling types whose values 
necessarily exhibit some significant property, and 
then using preconstructors to those types as 
certificates that the property is true (in some context).  
A \q{\i{rider}} channel can then be populated with 
one (or possibly several) preconstructors affirming 
facts about the values carried in other channels.  For instance, 
a \i{lambda} channel holding $x$ and $y$ can be paired with 
a \i{rider} channel holding a monotone-increasing-pair 
preconstructor, certifying that $y$ has been checked to be 
greater than $x$.
}
\p{The larger point of this whole subsection is that 
details about the nature and existence of types' constructors 
can provide useful conceptual information about types
themselves.  Often this information dovetails 
with the metadata pertinent to Conceptual Space Theory 
and CSML.  Consider a simple but representative example of 
an object exhibiting core Conceptual Space 
\q{quality dimensions}: a colored rectangle in a computer 
graphics environment.  This object could have some 13 dimensions: 
colors for the shape's interior and border (including 3-dimensional 
color vectors plus transparency factors); the top-left position,  
width, and height of the rectangle; and the width of th border.  
A software engineer has numerous constructor options for 
this type of object: should there be one \q{pod-tuple} constructor 
that takes all 13 values in a flat list?\footnote{\q{POD} is a 
common coder's parlance for \q{plain old data}, and it generally 
describes data structures conformant to \i{struct}s in the 
C languag, rather than \Cpp{} classes.  A typical feature 
of \i{struct}s is that they are initialized by listing all of 
their fields in one tuple.
}  Should there be distinct data types for 3-vector 
colors or 4-vector colors (including transparency)?  Should the 
top-left point be merged into one 2-vector point type?  
The rectangle-constructor could potentially have its 13 fields 
folded into 5: two color 4-vectors, one top-left point, 
width, height, and border-width.  Or border width and color can 
be merged into one \q{border} type.  On top of that there is the 
question of default values: should pure black, or maybe pure 
white, be a default color for the interior?  Should the 
border default to width-one black, or width-zero (i.e., 
no border), or something else?  However the types are 
designed, there is a mutual dependency between simpler types 
like colors and points and the constructors for complex objects 
like rectangles: since complex objects are built from simpler 
ones, a type interface should be designed with consideration of 
how to assemble the whole from the parts, or retrieve the 
parts from the whole.
}
\p{Some compound objects are conceptually analogous to basic pairs or 
tuples of values: 2D graphics points are 2-vectors, say, and 
solid colors are RGB 3-vectors.  One way to signal that a 
compound type is like a tuple \q{product} of its 
component fields is to allow instancs of that type 
to be constructed just by listing each field separately.  
In \Cpp{}, this is often implemented by defining a 
constructor which takes an \q{initializer list}.  
Analogously, in Channel Algebra we can define a \i{pod-tuple} 
channel that serves as the \i{lambda} channel 
for pod-tuple co-constructors.  A \i{pod-tuple} channel 
indicates that a given type behaves essentially like a tuple, 
with a sequence of values that, in most cases, vary 
independently of one another.  These kinds of types 
hew closely to the Conceptual Space idea of 
multi-dimensional quality spacs.  However, 
other kinds of types have more complex, interdependent 
internal structuration.  I think Channel Algebra is a 
way to make Conceptual Space theory relevant for these more 
complex types also, using different Channel Semantics 
and the modeling roles of different forms of 
types' constructors; how these rooes reveal types' conceptual 
foundations.   
}
\p{Having argued for Channel Algebra as a formalization 
potentially relevant to Conceptual Space models, I 
want to conclude this section by considering 
how Channel Algebra msy be intuitively applicable to 
Cognitive Grammar.  
}
\spsubsectiontwoline{Rider Channels and Deferred Coercions}
\p{Earlier in this section I described how Channel Algebra can 
represent dependent types and coercions, both of which 
are essential ingredients in type-theoretic semantics.  
I am not proposing Channel Algebra as a \i{literal} 
theory for natural language, particularly 
in its implemented form as a runtime and parsing 
environment for programming languages.  However, some 
of its formal details for processing programming languages 
could be at last suggestive of procedural patterns factoring 
in to natural-language understanding.
}
\p{Before going into detail, I think it's worth 
distinguish different aspects of natural language 
where type theory may play a role.  
On one level, very general Part of Speech classifications 
can be approached type-theoretically.  This helps 
reinforce the intuitive idea that syntactic categories like 
verbs, adjectives, and adverbs corresponds to 
conceptual, cognitive processes which effectuat changes
in beliefs, conceptualizations, or situational 
records.  A noun by itself is an abstract concept; 
during the course of a sentence, it is 
subject to concretization and preicatization, 
ending up in an idea that has some degree of 
concrete propositional content.
}
\p{For example, we might go from \i{dog} to \i{dogs} to 
\i{those dogs} to the sentence \i{Those 
dogs are barking}.  Logically we can see this as 
building up a propositional structure from sub-propositional 
parts.  But cognitively the process is more like 
migrating from a cognitive register dealing in conceptual 
abstractions (like \q{dog}) to a register for propositional 
attitudes and situational understanding.  Logical 
constituents are more like stages in a 
concept-to-belief evolution than mereological units.  
That is, the relation between the signifying whole and 
its semantic or morphosyntactic parts is more 
the relationship between a process's ending 
to its intermediaries than a mereological hierarhy.  
If this picture seems compelling it can be 
conveyed by figuring the majority 
of Part of Spech types as function-like: 
the nature of verbs, for example, is to 
functionally transform (in our 
cognitive frames) nouns to 
propositions.  Analogously, adjectives 
transform nouns to other nouns \mdash{} in 
the sense that \q{red apples}, say, can 
substitute as a noun-concept in most places 
that \q{apples} alone can; \i{red} being like a 
procedure that produces a modified \i{apples} concept 
which substitute for just \i{apples} in 
subsequent processes.  Likewise, adverbs produce 
new verbs from other verbs; subordinators like 
\i{that} transform propositions back to 
nouns (as I argued in the last section), and 
so forth.  So Parts of Speech can be usefully analyzed 
as \q{function-like} types (I will discuss 
specific language examples arguing for this prespective 
in the next section).  Thse are, however, very general 
types, which would be the \q{uppermost} types 
in a natural language semantcs; I will call 
these \i{macrotypes}.
}
\p{On the other hand, at a much finer level, individual 
lexical \q{cliques}\footnote{Borrowing this term from \cite{GaumeVenantVictorri} though I'm 
not using it exactly the same way
} \mdash{} the concept or extension or intensional 
property-set associatd with particular word-meanings 
\mdash{} often seem to behave as \i{types} as well.  In \q{Formal 
Concept Analysis}, \i{concepts} are (statistically) 
defined as a combination of extensional and intensional 
criteria: neither extension nor attributes alone 
fully characterizes a concept, either cognitively 
or extra-mentally.  In particular, practical fluency 
in a concept involves knowing canonical examples 
(like seing many of the medium-size, traditionally 
architected dwellings that would be prototypical 
\i{houses}) and also the prototypical features 
that characterize the concept (houses are 
three-dimensional, enclosed, divided into rooms, function 
as a place of residence, etc.).  A combination of 
exemplars and attribute-prototypes define 
the \q{core} or \q{center} of a concept: insofar as 
some example is \q{peripheral}, an outlier which 
is somehow not representative of the concept (but 
still can be classified to it), it must have some 
\i{featural} difference from examplars; but we can 
also compare it as an individual to more 
exemplary individuals.  That is, we understand outliers 
both intensionally \mdash{} we can articulate the features 
which make both a log cabin and a gated mansion atypical 
as houses \mdash{} and extensionally: we can mentally 
juxtapose cabins and mansions with ordinary houses.
}
\p{If this gloss on the nature of concepts is accurate, 
it helps explain why the lexical entrenchment of 
concepts often has a type-theortic feel.  In the 
more logical aspects of semantics, lexicalized 
concepts do present operationally as types: 
\begin{sentenceList}\sentenceItem{} \label{itm:penguin} A penguin is a flightless bird.
\sentenceItem{} \label{itm:rhino} Rhinos in that park are threatened by poachers.
\sentenceItem{} \label{itm:elephant} Baby elephants don't have tusks.
\end{sentenceList}
The point of these examples is that their signifying structures 
involve operations that can be explaind type-theoretically.  
In (\ref{itm:penguin}), a concept is 
charactrized via a subtype/supertype relation.  In 
(\ref{itm:rhino}), a concept is being 
\q{extensionally filtered}: we start with an 
expression that seems to designate the extension of some concept 
(\i{rhinos}) and then narrow it based on 
extensional criteria (\i{in that park}).  This suggests 
a type/set interface similar to my analysis earlier of 
\q{students polled} (\ref{itm:maj}).  And (\ref{itm:elephant}) 
suggests an \i{intensional} filter, narrowing a type 
for contxtual purposes into a subtype that is 
conceptually meaningful but not ntrenched:
there is no common word (peer to \q{kitten} 
or \q{puppy}) for \q{baby elephant}.  
}
\p{All of these operations \mdash{} fine-tuning 
concepts to the specific ideas salient in a 
specific signifying act \mdash{} have plausible 
representations based on a concepts-as-typs 
analogy, and suggest another level where 
type theory could be an effective formalizing 
tool.  But in this case, we are discussing 
the almost minimal groupings of 
a linguistic hierarchy, individual word-senses 
\mdash{} I'll call these \i{microtypes}. 
}
\p{Meanwhile, in between coarse Part of Speech types and 
fine lexical \q{cliques} are the kinds of 
Ontologically-characterized categories or 
\q{lexical sorts} \cite{MeryRetore} associated with 
linguists working in a formal type-thoretic 
vein, like Zhaohui Luo and Bruno Mery.  I 
will call types in this vein \i{mesotypes}, 
being intermediate in generality between individual 
lexemes and syntactic categoires (Parts of 
Speech and their refinements, like plural nouns).  
The broad philosophical implications of mesotypes 
is that they give some formal traction to 
what happens cognitive-procedurally 
when we negotiate between different 
word senses and/or use language in seemingly 
metaphoric (but not unrestricted) ways.
}
\p{Consider these exmples:
\begin{sentenceList}\sentenceItem{} \label{itm:koreanstore} My favorite Korean restaurant is adjacent to the bookstore.
\sentenceItem{} \label{itm:koreanclosed} My favorite Korean restaurant is closed today.
\sentenceItem{} \label{itm:koreanposter} My favorite Korean restaurant is decorated with posters from the 2002 World Cup.
\sentenceItem{} \label{itm:koreanstarted} My favorite Korean restaurant started out in a food court.
\sentenceItem{} \label{itm:koreanloyal} My favorite Korean restaurant said I'm their most loyal customer.
\sentenceItem{} \label{itm:bark} She's constantly barking at her employees.
\sentenceItem{} \label{itm:back} He's not playing well because his back is barking at him.
\sentenceItem{} \label{itm:actup} He's not playing well because his sore back is acting up.
\sentenceItem{} \label{itm:kitten} The kitten barked at the dog.
\end{sentenceList}
The first three profile (using Langacker's term) 
a restaurant as a location, a 
building (with commercial and architectural properties), or 
an institution/organization.  Note that we can distinguish 
the commercial/architectural sense from the institutional sense: 
to \i{close} can be temporary (as in \label{itm:koreanclosed}), 
which we hear in conjunction with the former sense, or 
more permanent as in \q{cease operations}, i.e., close 
\q{qua institution}.  The former sense seems to mix 
the institutional and location sense: while location 
is not explicit as in \label{itm:koreanstore}, an asserion 
that some \i{chain} restaurant is closed today would usually be 
heard as referring to one location, not the 
entire chain.  In (\ref{itm:bark}) through (\ref{itm:kitten}), 
a common idiom applies \q{bark} to things that are not 
dogs; but this is not just a matter of 
metaphor, blending different \q{spaces} with 
no conventionalizing pressues.  The (\ref{itm:bark}) and 
(\ref{itm:back}) senses are rather entrenched, using 
\i{bark} to suggest something a little obnoxious or 
mean.  But (\ref{itm:kitten}) does not sound right, even though 
it is no more of a strecth to imagine a kitten's 
hissing as argry bark-like than a boss's orders.  
Perhaps the \i{lotral} similarity btween cats and 
dogs maks the figurative usage harder to accept 
as a purported abstraction from literal situations.
}
\p{The conventional type-semantic picture is that 
language has a collection of (what I am calling) \q{mesotypes} 
and we modulate word-uses by switching between 
this medium-grain types, sometimes in unexpected ways.  
An effective way to develop such an analysis 
is to identify one or two senses as the most 
prototypical for some concept/lexeme, and read 
other senses as transforms involving other types.  
So the commercial/architectural mesotype 
may be canonical for \i{restaurant}, and 
then we branch out to a more geo-spatial 
sense (\i{where} the building 
is located), or institutional (the food-court 
incarnation may have been in a different 
place with diffrent staff and business classification).
Hence a \i{commercial building} type is 
\q{cast} to a geosptial point or to a social institution.   
Thse \q{casts} are not really metaphors; 
they have conventions, limitations, and rules.
}
\p{The mainstream theory is simplest when analyzing 
casts between mesotypes of similar 
gnerality, like plant-to-sentient in 
\q{flowers like water} or cerebral-to-physical 
in \q{heavy thoughts}.  But not all casts follow 
this recipe: (\ref{itm:bark}) and (\ref{itm:back}) 
seems to cast from a \i{microtype} to a \i{mesotype}.  
The construction in (\ref{itm:actup}) 
is a common \q{personifacation} of medical/anatomical 
concepts \mdash{} a similarly scalar idiom is 
cases like \i{my arthritis is acting up}, or 
\i{flaring up} to render figure ailment as physical 
rather than personalistic \mdash{} but here we 
care operating between typs of comparable Ontological 
granularity (anatomical part/medical ailment to 
person/physical thing)  But semantically and 
syntacticallly (\ref{itm:actup}) seems a lot 
like (\label{itm:bark}), even though the ploy 
in that cast is to compare things to a lexical 
\q{clique} (dogs).  And in (\ref{itm:koreanloyal}) 
we infer some \i{person} complimnted the speaker, but 
this is more of a synecdoche than a cross-type 
cast like (\ref{itm:koreanstarted})-(\ref{itm:koreanstarted}): 
the point is not that a restaurant has a personhood 
\q{aspect} that can be highlighted 
(the way it \i{does} have an architectural or geo-spatial 
aspect) but that reference to the restaurant can be proxy 
for people closely associated with it.
}
\p{Another complication for type-case theories is potential 
ambiguity between word-senses and coercing 
\q{usages}, as in:
\begin{sentenceList}\sentenceItem{} \label{itm:teaduck} Tea-smoked duck is a Sichuan delicacy.
\sentenceItem{} \label{itm:rubberduck} The baby's favorite toy is a rubber duck.
\sentenceItem{} \label{itm:weddingduck} Wedding ducks are traditional Korean gifts.
\end{sentenceList}
If the prototypical duck sense is an animal, these 
cases disrupt the basic type hierarchy: 
Wedding ducks are made of wood; 
tea-smoked duck is a kind of food; 
rubber ducks are toys (is the phrase descriptive 
\mdash{} ducks which happen to be rubber \mdash{} or 
a distinct noun-concept, like \q{decoy duck}?).  This translation 
from one branch of sub/supertype relations to another is 
what we would expect of type-cases, but 
we can also see these cases as distinct lexemes, or 
word-senses, or entrenched phrases with de facto lxical 
status.  The overall picture which seems to 
emerge is that type-coercions remain relatively 
metaphorical or metonymic in cases like 
(\ref{itm:bark}) or (\ref{itm:koreanloyal}); 
become idiomatically entrenched in cases like 
(\ref{itm:koreanstore}) and (\ref{itm:koreanstarted})
\mdash{} especially where there is a similar sccale of 
granularity btween mesotypes in a coercion \mdash{} and can 
become \i{lexically} entrenched in cases like 
(\ref{itm:rubberduck})-(\ref{itm:weddingduck}).
}
\p{Let's assume in any case we can give a thorough 
analysis of type-coercions as semantic devices.  
This is still working on the level of generalized 
lexical use-patterns; we are not arefully studying the 
\i{cognitive} steps involved in actuallly 
making whatever imagistic, situational, or 
conceptual modifications are directed by a type-case.  
In other words, there is a cognitive process of 
thinking of a restaurant (say) and 
then revising this framing to accommoate senses like 
spatial points or social institutions.  We can 
weave different senses and coercions into one 
sentence: 
\begin{sentenceList}\sentenceItem{} This book, which costs \$40, has some crazy ideas inside.
\sentenceItem{} This book, which the library classifies as young adult 
nonfiction, has some dude's phone number scrawled inside.
\sentenceItem{} My boss barks louder than his dog does.
\end{sentenceList}
In short, we can't assume that type-coercions 
simply replace an image forged according to one micro- or 
mesotype with one shaped by an alternative type.  
Instead, signifying elements seem to carry 
a package of type attributions around, and different 
types are \q{activated} to different extents, and in different 
ways, at different stages of linguistic processing.
}
\p{Another complication is that types of different scales 
seem to coexist and yet in other respects macrotypes, 
mesotypes, and microtypes seem to be distinct hierarchies.  
A microtype like \i{dog} acts sometimes as a subtype 
for a mesotype like \i{animal}, but also 
mesotype concepts like \i{animal} and \i{person} are 
sometimes microtypes themselves: after all, 
these are lexical entries as well as Ontological 
categories.  So on the one hand we may say that words 
have three different types \mdash{} even excluding 
cross-mesotype or cross-microtype 
(\q{bosses barking} cases) casts \mdash{} and that 
macro-, meso-, and microtypes are distinct thoretical 
posits with different analytic mthodology; 
but sometimes the boundaries between micro- and 
meso, and thir analyses, seem to plur.  Analogous comments 
can apply to the meso/macro boundary as 
evinced by issues like classifying forms 
of pluarality, as in \cite{MeryMootRetore}'s 
analysis of noun plurals. 
}
\p{This metathoretic hemming and hawing may be 
acceptable \mdash{} even desirable \mdash{} at the philosophical 
level.  But it causes problems if wecwant to se 
typ thory as a formalizing (albit simplifying) 
window onto cognitive-linguistic processes.  
In a computational context values have one canonical 
type; they can be cast to other types but 
do not maintain a kind of superposition 
\q{history} over multiple types.  
These kinds of phenomena can be technically modeld 
in various ways (e.g. Luo's \q{dot-product} types), 
but in that case we want a formalization that seems 
appropriate for how the corresponding cognitive procedures 
might unfold.
}
\p{In the case of Channel Algebra, I mentioned the 
idea of \q{rider} channels which supplement 
the type information contained in other channels, adding 
more information without actually introducing new 
parameters into a proceure.  So a rider asserting that 
$y$ can be made into an \xy{} pair is semantically analogous 
to giving $y$ the type \i{integer greater than x}, 
but $y$ itself is not presented as having a dependent 
type.  In short, \i{greater than x} is not presented 
as an \i{alternative} type for $y$, nor is $y > x$ presented 
as an alternative \i{value} as if we are \q{casting} 
$y$ from a wider to narrower type.  Howevercwe are 
asserting properties of $y$ by certifying that such 
casts \i{would} be possible.  
}
\p{I think this may be a useful analogy to how 
type-cases work in practice in Natural Languages, 
especially in complex cases where multipl word-senss are 
involved.  Not every type associated with a word 
makes sense in every conceptualization.  When for instance we 
attend to the conceptual properties of a book as an
intellectual object there may be physical details which 
are conceptually incompatibl with this attitude.  It would be 
simpler if there were a neat partition among senses, and 
some overarching cognitive order \mdash{} Ok, now we're figuring 
the book as physical; now it's an intellectual 
artifact; now it's a commodity \mdash{} supervises the 
coercions back and forth.  Neither semantic nor syntactic 
evidence warrants this simpler picture: it seems more as 
if conceptualization across types is a matter of 
nerworked cognitive procedures taking turns operating 
on one conceptual package, where latent type atttributions 
are available to each procedure whether or not they are 
\q{usable} in the sense of involving type structures 
that logically fit each procedure's purpose.
}
\p{I will give concrete cases of word-senses that I think 
substantiate this picture in the next section.  
Here, though, I'l conclude as follows: 
Channel Algebra can perhaps model what is going 
on insofar as procedures can take \q{rider} channels 
that can add detail to procedural inputs.  In a 
cognitive-linguistic setting, we can imagine these riders as 
multifaceted and complex.  Because riders are not part 
of procedures' actual inputs, they do not necessarily need 
to use types that the procedure recognizes or knows about; 
the riders may only come into plays as values are 
passed among procedures.  The analogous computational 
case would be some extra data associated with a value that 
is only relevant for certain scurity-oriented 
validations; i.e., permissions to modify a file.  Functions 
can pass around such values without considering the security-related 
data, except for a few procdures where scurity-sensitive 
operations are attempted.  So the security details are 
\q{part} of the data carried by a value, but only 
become semantically salient parts in certain procedural contexts.  
Rider channels are a way to represent this kind of 
extra information within a type system directly.  
I think they are a plausible analogy for 
cognitive re-inscriptions thatboccur 
in the evolving significations in a sentence, 
according to analyses like I will entertain 
in the next section.
}
\p{}
