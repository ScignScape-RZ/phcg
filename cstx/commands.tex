
%\usepackage{sectsty}
\usepackage{titlesec}
\usepackage{url}

\usepackage{eso-pic}

\usepackage{setspace}
%\doublespacing
%\onehalfspacing


% line
\usepackage[object=vectorian]{pgfornament} %%  http://altermundus.com/pages/tkz/ornament/index.html

\definecolor{darkRed}{rgb}{.2,.0,.1}

\newcommand{\decoline}{%

\vspace{-2em}
{\color{darkRed!60!cyan}\noindent\hfil{\EnglischeLinie}\hfil}
%{\color{darkRed!60!cyan}\noindent\hfil\rule{0.5\textwidth}{.8pt}\hfil}
\vspace{-2.25em}}


\newcommand{\whdecoline}{%

\vspace{-2em}
{\color{white}\noindent\hfil{\whEnglischeLinie}\hfil}
%{\color{darkRed!60!cyan}\noindent\hfil\rule{0.5\textwidth}{.8pt}\hfil}
\vspace{-2.25em}}

\newcommand{\sectionline}[1]{%
  \noindent
  \begin{center}
  {\color{#1}
    \resizebox{0.5\linewidth}{1ex}
    {{%
    {\begin{tikzpicture}
    \node  (C) at (0,0) {};
    \node (D) at (9,0) {};
    \path (C) to [ornament=84] (D);
    \end{tikzpicture}}}}}%
    \end{center}
  }
  
\newcommand{\EnglischeLinie}{
\sectionline{darkRed!60!cyan}
}

\newcommand{\whEnglischeLinie}{
\sectionline{white}
}
% ///

\usepackage[flushmargin]{footmisc}

\usepackage[letterpaper, left=.45in,right=.45in,top=1in,bottom=1in]{geometry}

\colorlet{codegr}{black!80!blue}

\setlength{\columnsep}{7mm}

%% rightx
\newcommand{\htwoo}{H$_2$O}
\newcommand{\dtwoo}{D$_2$O}
\newcommand{\POne}{$P_1$}
\newcommand{\PTwo}{$P_2$}
\newcommand{\Pone}{$P_1$}
\newcommand{\Ptwo}{$P_2$}
\newcommand{\Pprop}{$P$}

\newcommand{\Stwo}{$S_2$}
\newcommand{\Sone}{$S_1$}

\newcommand{\facondeparler}{\textit{fa{\c{c}}on de parler}}

\newcounter{sentenceCounter}{}

\newcommand{\writeIncSentenceCounter}{\refstepcounter{sentenceCounter}(\arabic{sentenceCounter})}{}


\newcommand{\listItemMark}{\rotatebox{30}{\raisebox{2pt}{\color{green!30!yellow!40!black}{\begin{tiny}$\blacktriangleright$\end{tiny}}}}}
\newenvironment{sentenceList}{\begin{list}{\listItemMark}{\setlength{\leftmargin}{.5em}\setlength{\itemsep}{-.1em}\setlength{\topsep}{.85em}}\begin{small}}{\end{small}\end{list}}

\newcommand{\sentenceItem}{\item \writeIncSentenceCounter}{}
% ///

\usepackage{etoolbox}

\AtBeginEnvironment{thebibliography}{\linespread{1}\selectfont}

%?\usepackage{mathptmx}


%?\titleformat*{\subsection}{\small\bfseries}

%?\usepackage{eufrak}
\usepackage{wasysym}
\usepackage{textcomp}
\usepackage{amssymb}

\usepackage{microtype}



\DeclareMathAlphabet{\mathcal}{OMS}{cmsy}{m}{n}
%\usepackage{euler}

\let\OldI\i


\newcommand{\mdash}{---}
\newcommand{\q}[1]{``#1"}
\newcommand{\sq}[1]{`#1'}
\renewcommand{\i}[1]{\textit{#1}}

\newcommand{\T}[1]{\raisebox{-2pt}{\ensuremath{\mathcal{T}}}\textit{\tiny #1}}
%\newcommand{\T}{\ensuremath{\mathscr{T2}}}

\newcommand{\TSupT}{\ensuremath{{\T2}\makebox[4pt][r]{\raisebox{5pt}{{\scalebox{.6}{\T1}}}}}}

\let\OldFootnoteSize\footnotesize
\renewcommand{\footnotesize}{\scriptsize}
	
\newcommand{\Tnoindex}{\raisebox{-2pt}{\ensuremath{\mathcal{T}}}}

\newcommand{\typeAbove}{%
\raisebox{-1pt}{\rotatebox{90}{\begin{tiny}$\diagdown$\makebox[1pt][c]{$\diagup$}\end{tiny}}}}

%\newcommand{\typeT}{\ensuremath{type\raisebox{.5pt}{\makebox[3pt][c]{-}}T}}
\newcommand{\typeT}{\ensuremath{\mathcal{T}}}

\newcommand{\TValues}{\ensuremath{T}-values} 

\newcommand{\emigres}{\'emigr\'es}

\newcommand{\Retore}{Retor\'e}
\newcommand{\Aurelie}{Aur\'elie}
\newcommand{\Descles}{D\'escles}

\newcommand{\ala}{\`a la}

\newcommand{\picalculus}{\ensuremath{\pi}-calculus}

\newcommand{\qmarkdubious}{\raisebox{.2cm}{{\footnotesize ?}}}

\newcommand{\TypeCat}{\ensuremath{\mathcal{T}}}


\newcommand{\ADJplusNPeqNP}{\ensuremath{ADJ + NP = NP}}

%\newcommand{\outarrow}{\ensuremath{\overset{..}{\rightarrow}}}

\newcommand{\outarrow}{\makebox[-2pt][l]{%
\raisebox{4pt}{..}}\ensuremath{\rightarrow}}

\newcommand{\argarrow}{\hspace{1pt} \makebox[-2pt][l]{%
		\raisebox{4pt}{.}}\ensuremath{\rightarrow} \hspace{1pt}}


\newcommand{\smoutarrow}{\makebox[-1pt][l]{%
		\raisebox{3pt}{..}}\ensuremath{\rightarrow}}

\newcommand{\smargarrow}{\hspace{1pt} \makebox[-2pt][l]{%
		\raisebox{3pt}{.}}\ensuremath{\rightarrow} \hspace{1pt}}



%\newcommand{\NounToNoun}{N \outarrow N}
%\newcommand{\VisNtoS}{V :: N \outarrow N}

\newcommand{\mmbox}[1]{#1}

\newcommand{\NtoN}{\mmbox{\ensuremath{N} \outarrow{} \ensuremath{N}}}
\newcommand{\NounToNoun}{\mmbox{\ensuremath{N} \outarrow{} \ensuremath{N}}}
\newcommand{\VisNtoS}{\mmbox{\ensuremath{V :: N} \outarrow{} \Prop}}
\newcommand{\AdjisNtoN}{\mmbox{\hspace{2pt}\ensuremath{\mathcal{A}
\scalebox{.7}{\ensuremath{\mathcal{DJ}}} :: N} \outarrow{} \Prop}}


\newcommand{\thatPhrase}{\AcronymText{that}-phrase}
\newcommand{\thatPhrases}{\AcronymText{that}-phrases}

\usepackage{graphicx}

\newcommand{\rzauthor}[3]{{
{\vspace{-4em}}{\fontfamily{gar}\fontseries{b}\selectfont% 
{\begin{center}\textls*[103]{#1}\\ \vspace{-2em}%
{\scalebox{.7}{\fontfamily{phv}\fontshape{it}\selectfont\footnotesize #3}}#2\end{center}}

{\vspace{-2.25em}}
{\hfill\small\today}

{\vspace{.25em}}
}}}			 


\newcommand{\rzauth}[3]{{
		{\vspace{-1em}}{\fontfamily{gar}\fontseries{b}\selectfont% 
			{\begin{center}\textls*[103]{#1}\\ \vspace{1em}%
					{\scalebox{.7}{\fontfamily{phv}\fontshape{it}\selectfont\footnotesize #3}}#2\end{center}}
			
			{\vspace{-6em}}
			{\hfill\small {\raisebox{1em}{\today}}}
			
			{\vspace{0em}}
		}}}			 




\newcommand{\rztitle}[1]{\begin{center}\fontfamily{phv}\fontseries{b}\selectfont #1\end{center}}


\newcommand{\Jorgen}{J{\o}rgen}

\newcommand{\Prop}{\ensuremath{\mathcal{P}rop}}

\newcommand{\NN}{\ensuremath{N}\hspace{-1pt}\argarrow\hspace{-1pt}\ensuremath{N}}
\newcommand{\Npl}{%
\ensuremath{N}\raisebox{8pt}{\hspace{-1pt}\ensuremath{\rotatebox{180}{{\begin{footnotesize}$\dotplus$\end{footnotesize}}}}}}
\newcommand{\NPl}{\Npl}
\newcommand{\NtoNpl}{%
\mmbox{\ensuremath{N} \outarrow{} \Npl}}
\newcommand{\NpltoNpl}{\mmbox{\Npl{} \outarrow{} \Npl}}
\newcommand{\PropToN}{\mmbox{\Prop \hspace{1pt} \outarrow{} \N}}

\newcommand{\VisNNtoProp}{\mmbox{\ensuremath{V :: N} \argarrow \ensuremath{N} \outarrow{} \Prop}}
\newcommand{\VisNProptoProp}{\mmbox{\ensuremath{V :: N} \argarrow  \Prop{} \hspace{2pt} \outarrow{} \Prop}}


\newcommand{\NSing}{\ensuremath{N}\raisebox{5pt}{\scalebox{0.65}{\ensuremath{%
{\odot}}}}}


\newcommand{\Z}{\ensuremath{\mathbb{Z}}}

\newcommand{\VPpNPeqS}{\ensuremath{VP + NP = S}}
\newcommand{\NSingToNPl}{\mmbox{\NSing{} \outarrow{} \Npl}}

\newcommand{\NNtoS}{\mmbox{\NN{} \outarrow{} \Prop}}
\newcommand{\NNtoN}{\mmbox{\NN{} \outarrow{} \ensuremath{N}}}
\newcommand{\N}{\mmbox{\ensuremath{N}}}
\newcommand{\NtoProp}{\mmbox{\N{} \outarrow{} \Prop}}
\newcommand{\NNtoProp}{\mmbox{\NN{} \outarrow{} \Prop{}}}
\newcommand{\ProptoN}{\mmbox{\Prop{} \outarrow{} \N}}


\newcommand{\PropToNYieldsN}{\PropToN{} \raisebox{1pt}{\ensuremath{\textgreater\hspace{-5pt}\raisebox{.15pt}{\ensuremath{\Rightarrow}}}}{} \N}

\newcommand{\SeqNPplVP}{\mmbox{{\small\texttt{S = NP + VP}}}}

\newcommand{\Arg}{\ensuremath{A}\tiny{rg}}
\newcommand{\argsToReturn}{\mbox{\ensuremath{{\Arg}_0} \hspace{-4pt} \smargarrow 
		\hspace{-3pt}
\ensuremath{{\Arg}_1} 
\hspace{-4pt} \smargarrow \hspace{-2pt} \ensuremath{\cdot\cdot\cdot} \hspace{0pt} \smoutarrow  \hspace{1pt} \textit{Result}}}

\newcommand{\lowBlank}{\raisebox{-1pt}{\textthreequartersemdash\textthreequartersemdash\textthreequartersemdash}}

\newcommand{\BlankAfterBlank}{\lowBlank\texttt{after}\lowBlank}
\newcommand{\AfterNSingAndNSingToNPl}{\mbox{\texttt{after} :: %
\NSing \argarrow \NSing \hspace{1pt} \outarrow \hspace{1pt} \Npl}}

\newcommand{\archiveDate}[1]{{\footnotesize #1}}

\newcommand{\sentenceexample}[1]{
\begin{noindent}
{{\raisebox{2pt}{{\scriptsize {\color{green!30!yellow!40!black}  
\rotatebox{20}{\RIGHTarrow}}}}} \hspace{10pt} #1
	
}
\end{noindent}
}

\newcommand{\sentenceexamples}[1]{
	
\vspace{1em}		
{\setstretch{1.0}	
#1
}
\vspace{.7em}		

\noindent\hspace{-4pt}}



\newcommand{\FeqMtimesA}{\hspace{-2pt}{\small \ensuremath{F}=\ensuremath{M}{\texttimes}\ensuremath{A}}}
\newcommand{\piCalculus}{\ensuremath{\pi}-calculus}


\newif\iffootnote
\let\Footnote\footnote
\renewcommand\footnote[1]{\begingroup\footnotetrue\Footnote{#1}\endgroup}

\newcommand{\AcronymText}[1]{{\iffootnote\begin{footnotesize}{\textsc{#1}}\end{footnotesize}%
\else\begin{OldFootnoteSize}{\textsc{#1}}\end{OldFootnoteSize}\fi}}


\newcommand{\NLP}{\AcronymText{NLP}}
\newcommand{\POS}{\AcronymText{POS}}

\newcommand{\Gardenfors}{G\"ardenfors}


\newcommand{\Haskell}{\AcronymText{Haskell}}
\newcommand{\C}{\AcronymText{C}}
\newcommand{\NP}{\AcronymText{NP}}
\newcommand{\Cpp}{\AcronymText{C++}}
\newcommand{\RDF}{\AcronymText{RDF}}
\newcommand{\Java}{\AcronymText{Java}}
\newcommand{\IT}{\AcronymText{IT}}
\newcommand{\IBMinc}{\AcronymText{IBM}}
\newcommand{\XDG}{\AcronymText{XDG}}
\newcommand{\AI}{\AcronymText{AI}}

\newcommand{\HPSG}{\AcronymText{HPSG}}
\newcommand{\CAG}{\AcronymText{CAG}}
\newcommand{\Lisp}{\AcronymText{Lisp}}

\newcommand{\TS}{\AcronymText{TS}}

\newcommand{\ThreeD}{\AcronymText{3D}}

%?\usepackage[inline]{enumitem}
\usepackage{enumitem}




\usepackage[font=small,labelfont=bf]{caption}

\usepackage{xcolor}

\definecolor{Bkg}{RGB}{250,245,252}
\newcommand{\leader}[2]{\hspace{#1}\colorbox{Bkg}{#2}}


\newcommand{\saying}[1]{\vspace{2ex}\noindent{%%
				\leader{2em}{\begin{minipage}{.38\textwidth}{\footnotesize #1}\end{minipage}}\vspace{2ex}}}

\newcommand{\sayingsrc}[1]{\vspace{0ex}\\\hspace{2pt} --- {\footnotesize #1}}

\renewcommand{\labelitemi}{$\blacklozenge$}

\newcommand{\itemmark}{\raisebox{-4pt}{\rotatebox{90}{{\Large $\bracevert$}}}}

\usepackage{tikz}
\usetikzlibrary{positioning}
\usetikzlibrary{shapes,snakes}

\newcommand{\visavis}{vis-\`a-vis}

%\sectionfont{\fontsize{12}{8}\selectfont}


\newcommand{\underlines}{{\normalsize \lowBlank\hspace{.1em}}}
%\newcommand{\underlines}{{\normalsize \lowBlank\lowBlank\lowBlank\hspace{.5em}}}

\newcommand{\XAcrossY}{\mbox{\ensuremath{x} \texttt{across} \ensuremath{y}}}

\newcommand{\Rel}{\raisebox{.5pt}{\texttt{{\OldFootnoteSize r}}}}

\newcommand{\XRelY}{\mbox{\ensuremath{x} \Rel{} \ensuremath{y}}}

\newcommand{\tinyurl}[1]{{\raisebox{2pt}{{\scriptsize \url{#1}}}}} 

\usepackage[colorlinks=true]{hyperref}

\colorlet{urlclr}{red!40!magenta!50!orange}

\hypersetup{
 urlcolor = urlclr,
 urlbordercolor = cyan!60!black,
 linkcolor = red!30!black,
 citecolor = orange!30!black,
 citebordercolor = yellow!30!black,
} 


\makeatletter
\Hy@AtBeginDocument{%
	\def\@pdfborder{0 0 0}% Overrides border definition set with colorlinks=true
	\def\@pdfborderstyle{/S/U/W .25}% Overrides border style set with colorlinks=true
	% Hyperlink border style will be underline of width 1pt
}
\makeatother


%\titlespacing{\section}{.25\textwidth}{*2.8}{*-.4}[5pc]


\newcommand{\p}[1]{
	
	\vspace{.575em}
	#1	
}

\renewenvironment{abstract}
{\small
	\begin{center}
		\bfseries \abstractname\vspace{-.5em}\vspace{0pt}
	\end{center}
	\list{}{%
		\setlength{\leftmargin}{.43in}% <---------- CHANGE HERE
		\setlength{\rightmargin}{\leftmargin}%
	}%
	\item\relax}
{\endlist}


\let\OldSection\section
\renewcommand\section[1]{
	\vspace{5pt}
	\OldSection{#1}
	\vspace{-4pt}
}

%?\let\OldSubsection\subsection

%?\renewcommand\subsection[1]{
%?	\vspace{-5pt}
%?	\OldSubsection{#1}
%?	\vspace{-16pt}
%?}

\usepackage{transparent}

\definecolor{logoCyan}{RGB}{66, 206, 244}
\definecolor{logoBlue}{RGB}{4, 2, 25}

\titleformat*{\subsection}{\Large\bfseries}

\let\OldSubsection\subsection
\renewcommand\subsection[1]{

	\vspace{12pt}
	
    %{\LARGE
	%\colorbox{logoCyan}{%
	%\begin{minipage}{\linewidth}
		\OldSubsection{% 	
			\hspace{-2.75em}
			%\begin{minipage}{}
			\protect\raisebox{-5pt}{%
			\colorbox{logoCyan!50}{\hspace{2.1em}}}%
			\hspace{-5pt}{\protect\transparent{0.3}{\colorbox{logoBlue!80}{\protect\transparent{1}{%
						   \protect\raisebox{1pt}{\textit{{\large #1}}} }}}}
			%\end{minipage}
		}
	%\end{minipage}
    %}
    %}
	\vspace{-6pt}
}


\newsavebox{\twolinebox}

\newcommand{\stwoline}[1]{%
\sbox{\twolinebox}{\raisebox{-3pt}%
{\parbox{7.4cm}{\linespread{1.25}\selectfont\raggedleft{\textit{\textbf{{\large #1}}}}}}}}

\newcommand{\twoline}{\usebox{\twolinebox}}

\newcommand{\subsectiontwoline}[1]{\stwoline{#1}\subsection{\twoline}}

\newcommand{\spsubsection}[1]{%
\vspace{-.25em}
\subsection{#1}
\vspace{.3em}
}
\newcommand{\spsubsectiontwoline}[1]{%
\subsectiontwoline{#1}
\vspace{.3em}
}




\let\OLDthebibliography\thebibliography
\renewcommand\thebibliography[1]{
\let\section\OldSection
\setlength{\leftmargin}{-4pt}
\vspace{.1em}
\OLDthebibliography{#1}
\vspace{.7em}
\OldFootnoteSize 
\setlength{\parskip}{0pt}
\setlength{\itemsep}{2.3pt plus 0ex}
\raggedright
}

\makeatletter
%\def\@biblabel#1{\hspace{-6pt}[#1]}
\def\@biblabel#1{\hspace{-6pt}#1}
\makeatother

\newcommand{\bibtitle}[1]{{\small \textit{#1}}}
\newcommand{\intitle}[1]{{\hspace{3pt}\textls*[-80]{\texttt{\textit{#1}}}}\hspace{-1pt}}

\renewcommand{\i}[1]{\textit{#1}}




\newcommand{\itemtitle}[1]{{\color{green!10!red!40!black} \textls*[-80]{\texttt{#1}}}}
\newcommand{\MIT}{\AcronymText{MIT}}

\newcommand{\nulltt}{\AcronymText{\texttt{null}}}
\newcommand{\Maybe}{\AcronymText{\texttt{Maybe}}}
\newcommand{\bind}{\AcronymText{\texttt{bind}}}
\newcommand{\return}{\AcronymText{\texttt{return}}}

\newcommand{\XML}{\AcronymText{XML}}
\newcommand{\XQuery}{\AcronymText{XQuery}}
\newcommand{\XQueries}{\AcronymText{XQueries}}
\newcommand{\API}{\AcronymText{API}}
\newcommand{\IR}{\AcronymText{IR}}
\newcommand{\Clang}{\AcronymText{Clang}}
\newcommand{\DAG}{\AcronymText{DAG}}
\newcommand{\DAGs}{\AcronymText{DAG}s}

\newcommand{\Qt}{\AcronymText{Qt}}
\newcommand{\NL}{\AcronymText{NL}}
\newcommand{\langC}{\AcronymText{C}}
\newcommand{\SDK}{\AcronymText{SDK}}

\newcommand{\SOV}{\AcronymText{SOV}}
\newcommand{\SVO}{\AcronymText{SVO}}

\newcommand{\YtwoK}{\AcronymText{Y2K}}

\newcommand{\qi}[1]{\q{\textit{#1}}}

\newcommand{\CSharp}{\AcronymText{C\#}}

\newcommand{\Tesniere}{Tesni\`ere}

\newcommand{\sentenceExample}[1]{``#1"}


\newcommand{\expose}{expos\'e}
\newcommand{\ExistsX}{\ensuremath{{\exists}X:Hugo(X){\wedge}Cat(X)%
{\wedge}Sleeps(X,on-the-couch)}}
\newcommand{\Exists}{\ensuremath{\exists}}
\newcommand{\ndim}{\ensuremath{n}-dimensional}

\usetikzlibrary{decorations.pathmorphing}


\newcommand{\kErn}{\ensuremath{\mathcal{K}}
\newcommand{\tSys}{\ensuremath{\mathbb{T}}
\newcommand{\procOne}{\ensuremath{\mathcal{P}_1}}
\newcommand{\procTwo}{\ensuremath{\mathcal{P}_2}}
\newcommand{\fFunT}{\ensuremath{\mathcal{F}_t}}
\newcommand{\ChaOne}{\ensuremath{\mathcal{C}_1}}
\newcommand{\ChaTwo}{\ensuremath{\mathcal{C}_2}}
\newcommand{\CarOne}{\ensuremath{\mathfrak{c}_1}}
\newcommand{\CarTwo}{\ensuremath{\mathfrak{c}_2}}
\newcommand{\CarHandoffCar}{\ensuremath{\mathfrak{c}_1{\looparrowright}\mathfrak{c}_2}}
\newcommand{\ChpAlliedChx}{\ensuremath{\mathcal{ChpAlliedChx}}}
\newcommand{\Chp}{\ensuremath{\mathcal{Chp}}}
\newcommand{\Chx}{\ensuremath{\mathcal{Chx}}}
\newcommand{\ChaProductCha}{\ensuremath{\mathcal{ChaProductCha}}}
\newcommand{\DigammaR}{\ensuremath{\mathcal{DigammaR}}}
\newcommand{\stageLbl}{\ensuremath{\mathcal{stageLbl}}}
\newcommand{\sigmaCalc}{\ensuremath{\varsigma} Calculus}
\newcommand{\voidP}{void*}
\newcommand{\ChaAppendCar}{\ensuremath{\mathcal{ChaAppendCar}}}
\newcommand{\LappendX}{\ensuremath{\mathcal{LappendX}}}
\newcommand{\Lprime}{\ensuremath{\mathcal{L}'}}
\newcommand{\LpreX}{\ensuremath{\mathcal{LpreX}}}
\newcommand{\tTyp}{\ensuremath{\mathfrak{t}}}
\newcommand{\xy}{\ensuremath{\mathcal{xy}}}


