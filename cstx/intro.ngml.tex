\p{A sign demands an individuation \mdash{} a criteriology, for anyone addressed and solicited by each sign, 
to recognize and isolate it as such.  Signs, and their referents, 
need an isolating, from the world around (and one another), 
or they are not signs.  In extreme cases a sign may 
stand alone, like the smoke fire telling a hiker's location; but by norm, 
embedded in discourse and performance, signs require (and carry with them, internally) 
understood inter-boundaries.  Both the recognition and the interpretation of signs therefore 
implicates cognition-logics of part/whole (mereology), and (dis/)continuity, 
each sign/referent disconnected in some ways, and continuous in others, 
with larger wholes inside which they are semi-autonomous parts.
}  
\p{Ad-hoc signs can blur the boundaries 
between conventionalized languages (verbal or not) and more impromptu social interactions 
\mdash{} the smoke fire is in part a conventional distress signal and in part a tool, an 
engineered natural phenomenon designed with intended effect, like causing rescuers to see it.  
The distress fire is also in a sense its own referent: its function as a sign is to call attention 
to itself, and so its location.  Or, choosing to write on one's own body 
\mdash{} like Hamas commander Mahmoud Ishtiwi, betrayed and killed by his own movement, carving 
into his leg the word \q{zulum} (\q{wronged}) \mdash{} is expression spreading beyond the  
conventions of words; the signs are made to signify the conditions of their execution.
These rather dramatic examples are more the provenance of semiotics than 
linguistics.  But people in ordinary verbal communication equally rely on a mixture of 
linguistic and other signs \mdash{} we point, make gestures, use \q{body language} and tone of voice.  
When conversation turns to some topic, like \q{that building over there}, such cues help 
speakers synchronize their attentions.  Tone and gestures clarify sentiment (honest, joking, sarcastic, anger) 
that may be ambiguous in spoken words alone, taken \q{out of context}.  
The weight of linguistic meaning is 
borne by semantics and pragmatics in fusion.  
Semantics has both formal and informal dimensions 
\mdash{} linking first to cognitive schema, or (as I will argue) prototypes of how 
schema are triggered; and second to pragmatics and contexts.
Conventionalized in semantic norms, schema, part abstract and part cognitive, 
help prime language users to manipulate formal structures 
in language, relative to the situational aether.  (Dis/)continuity in the plane of reference 
brings consciousness to a mereo-logic \cite{KitFine}, \cite{BarrySmith} that language-cognition 
can then reshape into syntax and semantics \cite{BittnerSmithDonnelly}.  
Semantic layers are abstract tools, but 
they offer a tableau of forms and combinations which users adapt, concretely, to each context.  
The deep potential of language, I believe, comes from the perpetual combination of the formal/abstract 
and the concrete/phenomenological.  
}
\p{For signs, the largest whole is a 
\q{plane} of articulation; for referents, it is an overall phenomenological surround; or, 
for more abstract signifieds, a space of concepts.  Like a footprint, whose very existence 
depends on both material continuity and visual break, for each sign there must be a blend of 
continuity and discontinuity, both around 
the sign and its referent.  Attending to a mereologically ordered world, 
we need innate theories warranting criteria for seeing things as both individuals 
and as causally/behaviorally constrained by and from a whole.  These criteria 
include structural consideration of the whole, and it is often in structural terms that 
the blend of autonomy and linkage for each part is realized.  Attunement to structured organization 
therefore warrants the perceptual and mental isolation of particular foci of attention 
\cite{WiegandMereology}, \cite{WiegandGestalts}. 
}
\p{As this plays out on planes of articulation alongside general situational awareness, the 
structures of discourse \mdash{} its division into distinct signs and their structural 
interrelationships \mdash{} and that of patterns we identify in our surroundings, that 
provide a context of discourse, play off one another.  Grammar does not iconify 
interrelationships among referents \mdash{} unlike  diagrams, maps, scientific simulations, 
or scale models \mdash{} but it ensures communicators may create structures among words that 
suggest, in each others' minds, concordant patterns in the environing world/situation.  
Even where there is direct sensory and perceptual evidence 
for objects' individuation and intercontinuity, visually and experientially present 
(which of course is only one kind of talk and reference), our preparedness to focus 
attention here or there depends on mental models of situations which are 
more abstract and schematic, and receptive to functional and interpersonal 
details.  The objects around us are not just blobs of matter, but usually 
have a constructed purpose, socially sanctioned 
meaning, nostalgic weight, and other significance that cannot be grasped by perception alone.
}
\p{A central theme in Cognitive Linguistics is that language meaning depends on situational 
understanding, and by extension on mental schema of spatial, temporal, and functional 
organization \mdash{} not only how environments are arranged, but how they are causally 
and physically determined \cite{Pinker}, \cite{Talmy}.  The difference between \i{pour water} and \i{spill water}, 
for example, is the person's deliberate intentions in relation to natural forces and 
tendencies (such as that of water to fall downward).  To \q{apply paint} to something, 
compared with to \q{cover} with paint, suggests different spatial configurations; to 
\i{fill a glass with water}, versus \i{pour water into} the glass, suggests both 
different spatial details and maybe rationale as well.  These are differences in 
emphasis, not necessarily in actuality: those pairs of alternatives could describe 
identical state of affairs. \hspace{-2pt} But they direct conversants' attention in different ways, they 
choose one or another part of a scene as a reference frame, and suggest 
different \q{takes}.  These are driven by \i{semantic} variations 
\mdash{} the choice of verbs like \i{pour} versus \i{fill}, \i{pour} versus \i{spill}, 
or \i{apply} versus \i{cover}.  But semantic and syntactic rules work in federation, 
relative to context: for example, different verbs take different prepositions in different 
situations.  Pour \i{into} vs. fill \i{with}.  To join \q{pour} 
with \i{with} places emphasis elsewhere \mdash{} onto the device which enables the 
pourer to do the pouring.  So the grammatic and semantic norms of a language 
jointly offer a terrain of options from which speakers assemble combinations 
invoking those aspects of situations that they wish to emphasize. 
}
\p{In short, grammar is language's substitute for \i{visual} or \i{physical} resemblance-to-structure.  
Take Ronald Langacker's \q{landmark}/\q{trajector} model as an example: one (very general) 
manner of spatial gestalt, subject to either intuitive, reflective analysis or to formalization 
(\cite{BergenChang}, \cite{VisettiCadiot}).
\q{That boat crossing the lake}: \i{boat} (trajector) 
perceived against \i{lake} (landmark), which provides context; together they 
produce a mental model; a figured spatial relationship.  This is 
communicated, not by visual or kinaesthetic effect,\footnote{At least in prose \mdash{} poetry, which can bring back a semiotics of raw visual layout and auditory effect, is an 
exception that proves the rule.
} but by the more abstract effects of intentions signaled, 
via both exact words (\q{crossing} paints a different picture than would \i{across}, \i{on}, \i{by}; 
still more so, \i{at the bottom of}), and morphosyntactic tropes 
(like the form \XRelY{}; where \q{\Rel{}} here means one from many spatial relations, taking 
\i{trajector} to the left and \i{landmark} to the right).  \i{Landmark} and \i{trajector} are anchors around which 
both syntactic and semantic selections are organized.
}
\p{Language needs both abstract laws and cognitively-mediated construals of ambient situations.  
The abstract laws are shaped by the situations, not directly \mdash{} it is that extra 
indirection which cleaves language from other sign systems \mdash{} 
but derivatively: language rules are optimized for conversants to 
mold linguistic possibilities into selection-spaces, which then become 
raw materials for representations of 
situational context.  Each choice of word and form adds a piece to 
a representational complex, and the sum of those pieces \mdash{} be this a sentence, 
a conversation turn, or an entire discourse \mdash{} is a language act that hews 
to the structure of a situation, as the speaker wants to emphasize it.  
Here I take this perspective as pregiven: not as a perfected or homogenous theory, 
but as a working hypothesis on the origins of linguistic structuration as 
such.  The scope of this paper is then to analyze its ramifications for our understanding 
of grammar and formal semantics.
}
\decoline{}
\p{This essay adresses topics in linguistics and the philosophy 
of language, though (by conventional measures of expertise) 
I am more of a Phenomenologist and a Computer 
Programmer than a linguist.  I confess this not as biography, 
but to introduce my metatheoretical anchor points, from 
which derive intuitions that others might find 
unconventional.  I am, in particular, sensitive 
to the experiential nuances of human cognition and 
skeptical that mechanical systems can emulate human 
minds except for narrowly defined tasks.  At the same time, 
I think computational systems have interesting aspects 
that can enrich our understanding of cognition, even if 
we do not philosophically buy a \q{cognition is computation} 
metaphor.  
}
\p{To be precise, I am skeptical about \q{AI}; and I am also 
skeptical about a kind of logical reductionism that 
I believe exerts a definitive influence on several interrelated 
filds, including philosophy, linguistics, and computer 
science.  As the paradigm seemingly goes, if we accept some 
form of \q{mind as computer} analogy, then we 
intrinsically accept \i{first} the idea that \q{mind} 
encompasses as some important part a logically articulated 
subsystem, which can be scientifically studied via formal 
logic; and \i{second} that as a consequence of this 
scientifically tractable logicality, AI is a good 
model or proxy for the study of mind.  The unconscious 
deduction here seems to be that \i{mind as computer} 
has as a consequence that mind is (to some salient degree) 
a logical system, following a premise that computers are 
logical systms.  But this premise is more false 
than it is true; so for me the whole paradigm is on shaky 
grounds.  I will explain later why computers are 
not as logical as non-programmers seemingly believe.  For 
now I'll just say this: there are rigorous accounts 
of computation that, I contend, are not grounded on 
formal logic in any techical or reductive sense.  As a 
result, someone's non-logical-reductive views 
on language and consciousness do not \i{a priori} 
preclude computational models having some intuitive, 
explanatory, or structural-analogy place in their 
analyses of cognition.  
}
\p{Meanwhile, on the Phenomenological menu I am a committed 
\q{realist}.  What I mean is that, in a nutshell, we 
should renew our commitment not to read Husserl too 
psychologically; for instance, not to read 
\i{intentionality} as a psychological phenomenon.  If I 
see a red sofa, we should go ahead and accept that 
what I see is a red sofa \mdash{} that very object.  I do 
not see a mental image of a red sofa or a phenomenal 
appearance of a red sofa or a token of 
red-sofa-appearance-ness.  We should not be led stray by the 
sofa being a few feet away from me, so it is not 
\q{in my brain}.  Yes, my brain is over here, not 
over there \mdash{} If I am suddenly distracted by something, 
look away, and forget about the sofa, my sofa-impression 
(but not the sofa) goes away, which seems to 
suggest that there my sofa-impression is not 
the same kind of thing as a sofa \mdash{} which in turn 
invites us to qustion whther what I am really 
seeing is that sofa-impression, not that sofa.
}
\p{But, without disputing that in 
\i{some} sense the impression is not ontologically 
identical to the thing itself, I still maintain 
that the best gloss on the situation still starts 
from the givenness that I do see the sofa (and 
not the sofa-impression or any other psychologistic 
posit).  I will have mor about to say about this 
realism, also.  For now I'll say 
this: the case for \q{impressions} over 
\q{things themselves} seems stronger when talking 
about vision (which works at a distance) rather than 
touch \mdash{} if I actually sit down on the sofa and 
physically contact it, we may feel more comfortable 
saying that my experience is directly encountering 
that physical object (though someone could still say 
that tactile sense-impressions are still not identical 
to objects; for one thing, the contact point 
between my hands/torso and the sofa \mdash{} the locus 
of those haptic nerve cells \mdash{} is not in my brain 
either, ergo a spatial gap still exists between brain and the 
sensed object).  If we accept that our nervous 
system is in some sense a functionally organized complex, 
then an encounter between some external body and \i{part} of 
that system, with suitably holistic functional 
response, can plausibly be treated as \q{my brain} 
(or nervous system or mind) contacting the sofa 
\mdash{} we don't need to rule out this gloss because my 
\i{central} nervous system remains physically 
isolated, any more than we would dispute that a knife 
has punctured a sealed carton when in fact only the 
knife-tip did so.  In short, sense-causing physical 
contact as part of my embodied propensity to register 
tactile contact experientially, through the medium of 
functionally-organized processing that eventually 
includes the brain, is \mdash{} I would 
say \mdash{} a sensate manifestation of my contact with the 
sofa (not with a tactile-sense-impression or 
haptic-phenomenon of the sofa).
}
\p{And if we accept 
this line of reasoning for touch, we should do so for 
vision also \mdash{} partly because an intrinsic 
feature of \i{seing} something is that we \i{could} with 
proper movement touch it, and apprehension of 
visual from includes anticipation of how 
surfaces will respond when we kinaesthetically interact 
with them (we might presume, for instance, that 
we can run our hand over the wall to the right but not 
to the left, if there's another wall there: this 
visual disclosure is also in a sense proto-kinetic).
}
\p{So, before making claims about language, I have hereby 
asserted two main intuitive feints guiding my subsequent 
discussion: computer programs as useful but not 
logically reductive analogs for cognitive processes, and 
the virtues of a \q{realist} Phenomenology 
which accepts language to the effect that we experience 
\q{things themselves}: that touching and seeing 
(etc.) are expriential encounters with real, external, 
non-psychological entities.  This does not have to be a 
blunt realism \mdash{} I don't dispute that we experience 
appearances in some sense \mdash{} but we need to articulate 
the thing/apearance distinction in a way that does not 
disallow common-sense intuitions like \q{seeing the sofa} 
meaning that I do see some real, external sofa-thing.  
That would make for an analysis in 
pure Phenomenology if I just framed my arguments with 
reference to, say, Husserl's own treatment of the 
noemata/phenomena distinction.  Here, howevere, 
I am going in a different direction and package 
a loose theory of \q{realism} about intended 
\q{external} objects within a treatment of 
\q{externalism} (and \i{internalism}) in 
the philosophy of language. 
}
\p{I am not ignoring that \q{external} in the sense 
of \q{wide scope} mind-world relations as a 
Semantics hypothesis is only tangentially related 
to \q{external} in a phenomenological sense of 
experienced external objects (as opposed 
to experienced internal, e.g. somatic, states).  
But I \i{will} present a theory that connects 
these two senses of \q{external} (and likewise 
two senses of \q{internal}).  
}
\p{All told, my goal here is to sketch a theory of 
cognitive linguistics which can resonate soundly 
with Phenomenology (while not being especially 
phenomenological on its own).  This theory 
will be incomplete \mdash{} deliberately, strategically 
incomplete.  Indeed, every theory should be incomplete: 
an essential quality of modern science is 
our recognition that scientific explanation covers a 
vast breadth of scales and kinds of phenomena, and 
\q{science} as a singular human institution only exists 
insofar as there are many sciences, each with some measure of 
theoretical autonomy but also areas of overlap, so 
scientific explanations can bridge across scales.  Bioligists 
take it for granted that the basic intellectual structures of 
their disciplines can be justified by appeal to 
chemistry (as a causative or emergent base of bioloigical 
phenomena); and the presence of \i{parts} of biology 
where this connection is explicit (like organic chemistry) 
is important for our overall sense of biology as 
something grounded in a general scientific method.  But 
these \q{reductive} links are not typically 
operationalized in biology as a whole \mdash{} a biologist 
is not \q{doing} chemistry, biological properties 
are not necessarily chemical properties, biological 
laws are not necessarily chemical laws, and 
biological terms are not semantically (or even 
arguably referentially) reducible to chemical terms.
}
\p{We can consider whether biological concepts are 
\q{in some sense} reducible to (or extensionally 
equivalent to or \q{the same stuff as}) chemical 
concepts, but framing this discussion as a nuanced 
debate implies that biology is not 
\i{trivially} reducible to chemistry, and we may accept 
such a reduction as a plausible option only insofar as 
some of us may hold philosophical commitments, which 
\q{we} collectively do not want to dismiss out of hand, 
that higher-scale sciences are necssarily reducible 
to lower-scale ones that are their causal or 
physical-constitutive base.  But even if there is a 
sense of \q{reduction} and of \q{biology} and \q{chemistry} 
that makes biology reducible to chemistry, this 
dos not make biological \i{science} reducible to chemical 
\i{science} \mdash{} that is, a well-constructed 
and discursively evaluable biological theory should not be 
expected to consider in any detail its own reductive 
interprtations, or express its concepts in chemical 
(rather than biological) terms, or attempt to \i{explain} 
rather than just \i{presuppose} chemical laws 
(preservation of quantities in chemical reactions, 
acid/base qualities, solvents and solubility, 
molecular interactions, etc.).  Ditto for chemistry 
in relation to molecular physics, molecular physics 
in relation to quantum physics, neurology in 
relation to biology, and so forth.  In short, 
whatever our philosophical intuitions about emergent 
phenomena and the ontological duality (or monism) between 
emergent and base scales, these philosophical points 
are only tangentially related to the equally 
important philosophical question about what 
makes a good theory in a science.  
}
\p{This bears reiterating: when considering a science 
(I'll include social sciences and humanities here) 
philosophically, there are two different sorts of questions 
that can arise.  On the one hand, what is the ontological 
status of the entities, laws, and quantitative models 
postulated by the science and its currently influential 
theories?  Should we understand terms to be 
proposed natural kinds (like \q{protons}), structural 
features that don't necessarily align with straightforward 
patterns of reference (like \q{dark matter}), referring 
expressions into complex systems whose parts have somehow 
fuzzy or underdetermined boundaries or criteria of 
individuation (like \q{storms} in the context of climate science), 
or quasi-references which have the form of 
concrete designations but are really 
just shorthand for elaborate paradigms (like 
\q{natural selection})?  These are various options in the 
semantics of scientific jargon, which are clues to 
the proper ontological status of scinces' theoretical 
posits (so much applies to linguistics also, 
with its theortical vocabulary of concepts, lexemes, 
syntax rules, generative semantic rules, and so forth 
\mdash{} are these mental subsystms?  Innate cognitive 
faculties?  Clusters of nerve cells?  Neural pathways 
reinforced during language acquisition?).  But, on 
the other hand, there is a different order of question 
philosophers can ask with regard to a particular 
science: what qualifies as a well-constructed theory for 
that science?  What sorts of formal models hold explanatory 
merit as, seemingly, capturing the casuative factors 
determining the behavior of the systems that science 
investigates: continuum-based numerical models?  
Models in discrete mathematics?  Systems of logic?  
State machines?  And interconnected 
with that question is the proper scope of the science: 
a well-constructed theory needs to honor bounaries between 
and autonomy of different sciences.  Having a clear 
picture of what beliefs in \i{other} sciences to take 
as explanatory primitives in \i{this} science is an 
essential criteria of theoretical soundness 
\mdash{} no less than the urge to pursue explanatory 
closure within the proper bounds of each science.
}
\p{One of my objections to \q{logical reductive} paradigms 
in linguistics (and computer science) is their failure to 
distinguish these two aspects of a philosophy of science, 
by my lights.  When discussing chemistry or biology, we can 
make a clear distinction between metaphysical 
commitments according to which higher-scale systems 
reduce (via physical composition and the propagation of 
causality across levels of organization) to lower ones 
\mdash{} biology to chemistry to physics \mdash{} as a genre 
of reduction obviously diffrent from reducing sciences 
as collective intellectual exercise.  We do not reduce the 
community of biologists to the community of chemists, 
or the kinds of expertise and fluency in certain 
mental gymanastics, or the criteria of what makes 
good biological theories, to the concordant 
community, conceptualizations, gymnastics, and 
theory-criteria of chemists.  This is for me part of 
what makes biology a successful science \mdash{} it 
is incomplete in an ontologically necessary, 
intellectual fertile way.  But if this is a 
reasonable criterion, what can we say about 
linguistics as a science?  Is it incomplete 
in an ontologically necessary and intellectually  
fertile way?  In fact, I intend to argue here 
that some popular linguistic theories are \i{not 
incomplete enough}.  They are 
(or would be, if successful) too complete 
\mdash{} while also, I will claim, incomplete 
in the wrong ways, leaving too many \i{relevant} 
phenomena, issues that \i{are} in the scientific 
wheelhouse, incompletely explained.  
}
\p{I will make these arguments as a prelude to 
describing the (incomplete) 
linguistic theory I \i{am} prepared to defend.  
Specifically, the first two sections here will 
weigh in on Conceptual Role Semantics and 
Truth-Theoretic Semantics and explain why I 
believe some popular paradigms in the philosophy 
of language are problemmatic.  While the details 
will vary, the main thurst of my points will be 
that philosophers of language fail to appreciate the 
importance of sciences internalizing a map of 
the division of labor between science \mdash{} a science is 
constituted in part by how it touches but remains 
autonomouus from other (both higher- and lowe-scale) 
sciences.  So biology is constituted in part by its 
status as a potential reductive base for (e.g.) neuroscience, 
medicine, genetics, and paleontology, while having 
its own reductive base in chemistry and 
physics.  Part of what it means to be biology is 
to be the explanatory bridge between, say, 
medicine and physics.
}
\p{Analogously, I believe, 
part of what it means to be linguistics is to be 
the explanatory bridge between, say, sociology, 
anthropology, and ethnolinguistics, with cognitive 
science (or Cognitive Phenomenology).  Language 
can be intrinsically characterized as the cognitive bridge 
between our everyday world \mdash{} of social situations 
and kinaesthetic/pragmatic enaction and anticipation, 
planning and memory \mdash{} with the neurophysical 
substratum (whatver it is) of our mental faculties.  
Language, that is, is an important tool for our 
negotiating the duality of our higher-scale 
social/situational world with our lower-scale 
neurophysical existence.  Analogously, 
a phenomenon in language \mdash{} say, a sentence \mdash{} 
should be analyzed as a kind of transition-system 
between a social/situational layer of reality and a 
cognitive/neurological layer.  Linguistics 
is accordingly suspended between these layers \mdash{} or, 
better, I claim that linguistics should be the 
\i{theory of being suspended} between social/situational 
and cognitive/neurological strata.  A linguistic 
analysis starts with entities shooting in 
from the first stratum (sentences we hear uttered, 
canonicaly), and it ends with some restructured 
representation or consummation of that sentence (parsed, 
lexified, etc.) understood as inputs to some 
neurophysical process belonging (ontologically, 
and as a matter of scientific jurisdiction) to the second 
stratum.  Such an analysis is \i{correctly} incomplete 
because it recognizes that a basic criterion of well-formdness 
for linguistic theories is that they \i{refrain from} 
direct analysis of either societal/interpersonal or 
cognitive/neurophysical processes.  Linguistic 
analysis is incomplete because a theortical machinery 
fine-tuned for analyzing processes of linguistic understanding 
at the intermediate level between social/situational and 
neurological strata cannot be the same as a theortical machinery 
for analyzing either (in one explanatory direction) 
sociological or (in the other) neurophysical laws in turn \mdash{} 
by analogy, the exeprimental (and theoretical) machinery for 
detecting Earthlike exoplanets cannot be the machinery for 
detecting Higgs bosons (and vice-versa). 
}
\p{Here I find an analogy to computer software useful: 
programs don't run themselves, so application developers 
realize that they do not control, or have access 
to much information about, when applications 
are launched (or when users will perform actions 
that require response from the software, like clicking a 
mouse button or pressing a key).  Nor do programmers 
control input/output commands like emitting colors 
to the screen: they only influence electronic 
devices (like displays and networking capabilities) 
indirectly, via preimplemented system calls.  In 
other words, the essential structure of a computer 
application is to be poised to react to various events 
(a mouse click, a key click, plus of course program 
startup initially) by eventually requesting certain 
operations (like changing the state of the screen) 
whose exact functioning remains outside the 
programmer's theoretical arsenal.  Application 
developers have only a vague idea of how values and 
types in code are marshalled to an from electrical signals 
physically affecting (or reporting state from) devices 
like monitors, mouses, and keyboards.  This is by design: 
if you're too closely attuned to low-level cyberphysical 
details, like how source code function calls map to 
digital signals, you're no longer doing computer 
programming (maybe you're doing chip design).  
To the degree that programming has a theory, it's a theory 
of how to \i{bridge} users' desired interactions with 
the software you are building \i{to} the digital 
structures encoded at the level of microprocessors and 
machine language.  It is not a theory of microprocessors 
themselves.  Theory well-formedness in the realm of 
programming \mdash{} the field sonetimes called 
Software Language Enginering \mdash{} reflects the 
transforms bridging \q{Human Computer Interaction} 
with machine language; it is not a theory of 
HCI or of machine languages themselves.  Indeed, HCI 
methodology is subjective and statistical; and 
the methods of physically realizing machine 
language in microprocessors depend on physical and 
nanochemical properties. Well-formed Software Language 
Enginering theories \i{have} to leave both HCI 
and microprocessors out of the frame, since 
software programming languages are not statistical 
or subjective, nor physical or nanochemical.   
}
\p{The hierarchical nature of computer architecture 
complicates any \q{mind as computer} metaphor: computers 
have many subsystems, with significantly different 
structures and properties.  Using computers as 
case-studies of artifacts that are in some sense 
\q{intelligent} can take us in different directions 
for different answers as to what scale of computers' 
organization we propose to inform, say, 
cognitive-linguistic research: microprocessors?  Machine 
Language?  Programming languages?  Software systems?  
The internet?  Many instances of 
\q{mind as computer} analogic reasoning are not explicit 
research paradigms being proposed forth but are more 
like reports of intuitions: a community of linguists 
feeling that there's something going on in how computers 
work that usefully models or resembles how human 
rasoning or language-processing works.  But cashing 
these intuitions into systematic models can prove 
challenging: even insofar as a computer may exhibit 
intelligent behavior, it does so only 
in an emergent manner, the whole \q{intelligence} 
being possible only through specific kinds of 
interaction between subsystems, in particular 
a tightly determined transition between high-level 
systems (like applocation source code) and 
low-level systems (like machine code).  
}
\p{Of course, many researchers probably believe that this 
emergent dimension is precisely why computers are a 
plausible cognitive analogy: they suggest that 
intelligence can be realized in structural systems whose 
lowest-level operations are not particularly complex.  
No-one would argue that in and of itslf a simple 
Van Numann machine is particularly \q{intelligent}; 
but software evincing intlligent bhavior can be implemented 
as emergent phenomena for which Van Numann machines are 
their reductive base.  This may sem like a 
useful analogy to consciousness, realized in 
neurons and synapses even though neurons and synapses 
are not themselves conscious.  That's an 
acceptable intuition, but it also leads to a 
kind of philosophical bait-and-switch: what 
starts as a \q{mind as computer} intuition ends up 
as a different kind of analogy, something more 
like comparing minds to functional systems 
\i{implemented} on a computer.  There is a 
difference between being a system 
realized on a computer and actually 
being a computer. 
}
\p{In the case of language comprehension, someone may 
find a useful analogy in database-like 
constraint-solving applications, like Prolog: 
language users maintain an internal store of 
beliefs \mdash{} about both language and the world \mdash{} and a 
record of prior steps in the current conversation.  
This \q{database} gets updated as we hear new sentences, 
and we are equipped to make or reject inferences based 
on inference rules and constraints, respectively: 
from \q{John is my younger step-brother} we can conclude 
both that the speaker's parents are divorced and that John 
is not female.  Of course, real-world complications 
sometimes intrude on the kinds of tidy frames 
linguists build around words: a brother can actually 
be a transgender woman, and divorc\'es can remarry each 
other.  We can debate whether these are semantic or 
pragmatic issues (I think they're the former, but let's 
say they are the latter for sake of argument).  So 
let's say language has enough logical order that 
conversations can be modeled rather like Prolog 
programs.  These leads to a maybe-interesting 
mind-as-Prolog analogy, but \mdash{} here's the crux 
\mdash{} mind-as-Prolog analogies are \i{not} 
mind-as-computer analogies.  Computers 
\i{run} Prolog programs; it's not that 
they \i{are} Prolog sessions.
}
\p{Indeed, I think many \q{mind-as-computer} analogies are 
actually more like \q{mind-as-Prolog} analogies, 
or substitute some other technology for Prolog.  
For instance, mind-as-artifical-neural-network 
analogies are not mind-as-computer analogies, because 
computers are not ANNs (though they may implement them).  
Indeed, ANNs are designed to make computers more 
humanlike: to transcend the mechanistic limitations 
of Van Neumann architecture by realizing, at 
some virtual level, a more connectionist manifestation 
of computation.  A mind-as-ANN analogy is therefore 
really mind figured as a computer programmed 
to operate like a mind (so, the analogy is 
basically circular).  Mind-as-symbol-processing analogies 
have similar issues: computers are 
not symbol-processors, though they can implement 
symbol processing systems.  As I'll defend below, 
I think computers are basically stack machines, and 
stack machines do not \q{process} symbols \mdash{} what 
they do is process stacks, and jump around to different 
subroutines.  When people think about 
\q{computers} in mind-as-computer analogies \mdash{} or 
write in ways suggestive of such an analogy \mdash{} I often 
get the impression that what people are really 
thinking about is not \q{computers} but some sort of 
mathematically formalizable, functionally specified 
system that can be \i{realized} on a computer.
}
\p{These other analogies are not \i{a priori} bad 
\mdash{} it's reassuring if we have accounts of 
intelligence that traffic through functional 
organizations that can be realistically embodied 
in mundane physical artifacts, rather than needing 
some magical mind-gunk.  But if our 
\q{mind-as-computer} analogies are nothing more than a 
desire to find logico-functional systems that can credibly 
undergird cognitive behavior, computers are 
basically irrelevant: the computational realizability of 
such logico-functional systems is a nice reminder 
that we're not asking non-philosophers to believe 
in magic, but the structure of these systems are 
sufficiently remote from how computers internally operate 
that computer realizability should have no 
\i{theoretical} role.  In other words, 
mind-as-computer analogies are usually basically 
mind-as-logico-functional-system analogies.  
}
\p{We're entitled to find these latter analogies 
intuitive.  My problem is only with 
mind-as-logico-functional-system analogies that 
get defended \i{by appeal to} mind-as-computer analogies.   
There seems to be a kind of metatheoretical pattern that 
goes something like this: mind is 
metaphorically a logico-functional system \i{because} 
mind is metaphorically a computer, and 
logico-functional systems (when not just abstract 
mathematical territories) are realized via 
implementation in computer architecture.  
But mind-as-computer is not logically related 
to mind-as-logico-functional-system \mdash{} any apparent 
link between these analogies is a biproduct of 
intellectually bakgrounding the 
distinction between \i{being} and \i{implementing}.  
We may or may not like a mind-as-computer 
analogy, but even if we \i{do} accept such a 
perspective, even if provisionally, this does 
not then legitimize or entail that we are accepting 
a mind-as-logico-functional-system analogy.  Of course, we 
can judge the latter analogy on its own merits, but 
the former analogy in no way retroactively 
justifies the latter. 
}
\p{So, my strategy for the remaining sections of this 
paper is as follows: I will review some 
language-philosophy controveries and argue 
that disentangling mind-as-logico-functional-system 
from mind-as-computer analogies should change 
our estimation of theoretical claims apparently 
motivated by mind-as-logico-functional-system 
paradigms.  I will then present a different 
basic account of language processing which, 
I believe, does not work in any logico-functional-system 
orientation, though I will recognize some 
fashion of functional orientation.  I will in 
places appeal to computer architecture, though 
the framework I propose will only absorb 
\q{mind as computer} analogies to a limited, targeted 
extent.  A central theme will be that 
\i{logic} is not terribly relevant for either 
language or computers: functionally-organized 
systems do not have to be \i{logico-functional} systems.  
The ambient philosophy guiding these arguments 
is that \q{logic} in any formal, symbolic 
sense is not the proper vehicle for understanding 
the structured transitions and causative propagation 
endemic to multiscale, emergent systems.
}
\p{Biology, for example, is not a \i{logical} 
intermediary between medicine and physics.  We can 
consider how best to describe its \q{intermediariness}: 
as a theory-construction maxim (in the sense  
that intermediariness is expressed in which 
laws/obervables are thematized and which are 
deferred to other sciences), as a causal network 
grounded in cross-scale physical constitutions and 
mereologies (e.g., tissues are both physically 
composed of cells and are wholes where cells are parts), 
as an emeregnt system which both \i{is} 
(\visavis{} other sciences) and \i{has} a reductive 
base.  Sciences are like computer programs in 
that they have \i{inputs} (observables from 
other sciences) and \q{outputs} (laws from 
other sciences).  I use thse terms because they 
track analytic trajectories: medical 
\i{observables} (e.g. that many people 
who are exposed to a toxin develop neurological damage) 
are linked \i{via biological analysis} to 
causal/material explanations (how the toxin chemically 
damages nerve cells).  Biology neither statistically 
models the medical observations nor physically explains 
the causative mechanism, but it provides a theoretical 
machinery for rigorously modeling 
and analyzing the transition between them.  It writes  
the second act of the explanatory play, so to speak.  
Pictured computationaly, the analysis is like a 
computer program whose inputs are higher-scale 
observables (saay, medical data) and whose \q{outputs}  
are numerical models whose formulae or justifications 
are solved by other sciences \mdash{} by analogy 
to programmers calling system-kernel functions.  
In this analogy, medicine is like the end-user, 
biology is the application developer, and physics 
is the system kernel.  
}
\p{These are analogies informed by computers, of course, 
but I am trying to focus in on the \q{intermediariness} 
they evoke.  How computer software 
bridges users and bare-metal is a useful metaphor 
for how scientific theories bridge observations 
and causal/mathematical microphysical models.  
And, correlatively, how sciences bridge between 
other sciences \mdash{} higher sciences yielding 
obsevrations that are \q{inputs} to intermediary 
explanations, lower sciences defining 
formats for \q{outputs}.  Biology, for example, can 
defer to chemical or physical explanation if it can 
provide data in structures adequate to 
chemical or physical formulae: the reference frames, 
quantitative measures, dimensional systems.  
Biology does not need to solve such equations, 
just marshal data into their form.  So a good 
biological analysis will take observation 
data (e.g. from physics) and transform it 
to equational data in some sense, wherein chemistry 
and physics take over.  By analogy, correct computer 
software takes obsevrations (data in computer files 
and user actions) and translates these observations 
to the proper system-kernal calls, whrein the 
Operating System takes over.  
}
\p{Sometimes logical constraints come to bear on these 
transitions, but the importance of logic per se is 
overshadowed by the overarching phenomenon of 
\q{intermediariness}: how the technical and ontological 
status of computer applications is defined by 
their intermdiary position between input data/user actions 
and low-level system calls.  Analogously, sciences 
are characterized by their posits' ontological status 
as \mdash{} and their theories' structural criteria 
regulated by \mdash{} intermediariness between observational data 
from one per science and causal/mathematical formula from 
another.  As a philosophical gestalt, such 
\q{intermediariness}, I believe, should 
take the place of \q{logic} in our intuitions.
}
\p{In the specific contexts of Conceptual-Role and Truth-Thoretic 
Semantics, I will now show what for me this means in practice.
}
