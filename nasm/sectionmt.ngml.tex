\section{Mereotopology and Functional Organization}
\p{When we conceptualize something \mdash{} or some class of \q{things} \mdash{} 
we inevitably interject cognitive and perceptual shema, 
mental summarials of various kind which create cognitive routes 
away from the explicit presentation of things themselves in 
momentary consciousness.  Some sensory and perceptual details  
have (at least for situations at hand) comparatively 
minor individual importance even if they cumulatively 
synthesize into experiential wholes which are important.  These 
are details we tend to forget, or not talk about, or notice 
only pre-consciously.  The detailing committed to memory, 
focal attention, and language, tends to exhibit some recurrent 
schema and often to operate on a higher scale of integration 
\mdash{} a traveller may recall a memorable meal or train trip 
more than the name of her server, or the number on her sleeping car.  
The nature of these schema, and their patterns of selectivity, 
is of obvious importance to cognitive scientists, but also 
to computer scientists.  Computer databases, Natural 
Language Processing engines, and well-designed interactive 
software, need to understand the schema through which 
we cognize everyday objects insofar as these objects, 
and our patterns of interaction with them, are modeled 
as computer data or metaphorically incorporated into 
software functionality.  
}
\p{While analyses of cognitive schema need to integrate 
multiple perspectives, two facets of schema deserve 
special attention: the mereological (concerned with part/whole 
relations), and topological (concerned with spatial and functional 
relations of contact, connectedness, boundaries, and 
non-geometric properties of objects' shape, such as the 
presence or absence of holes, the number of dimensions 
\mdash{} contrast a point, line, surface, solid \mdash{} and relations 
like a surface covering a solid, as in a table cloth, 
or a solid inside an open area bounded by a surface, 
like an object in a bag).  In computer science and 
the philosophy of mathematics, these two areas are sometimes 
grouped together as \q{Mereotopology}, which \mdash{} in a more 
philosophical vein \mdash{} generally concerns alternative 
constructions for mathematical topology which avoid the 
arguably problematic notion of dimensionless points, and 
\mdash{} in a more computational vein \mdash{} concerns models of 
spatial reasoning and representation where individual 
points, infinitely divisible space, and real-valued  
geometry are cognitively or technologically implausible or 
impractical.  Both human and Artificial Intelligent 
conceptions of space seem to often rely on schemas of 
coarser-grained relations, like overall direction or overlap 
or relative distance, rather than precise numerical, 
geometric recognition of objects' locations within a 
real-valued grid.  From this basic orientation, I also 
believe the notion of Mereotopology can be extended 
to concern schemas where mereological and topological 
relations or aspects are interrelated, in a variety 
of cognitive, linguistic, and computational contexts.  
Therefore I assume a rather general understanding 
of \q{Mereotopology} as any theorizing involving 
mereological and topological analyses in consort, while 
not neglecting to orient such theories against the 
\q{core} notion of Mereotopology as \q{region-based 
theories of space}.
}
\p{In this paper I am particularly concerned with 
extending mereotopological concepts to accounts 
of \i{functional organization}, especially  
as relevant to computational modeling of everyday 
things and situations.  Most computer models 
do not capture the rich, experiential nuance 
of perceptual object-interaction.  Certainly a 
realistic rendering of such interaction is important 
for fields like 3D Graphics and Virtual Reality, 
which seek to recreate some of the lived, real-time 
phenomenology of consciousness, or to allow 
such consciousness to be experienced within engineered, 
imaginary worlds.  Improving such technology is 
important in several domains, such as medical or combat 
training (or other high-stress contexts), Computer Aided 
Design (concerning, for example, architecture and urban 
planning), scientific visualization (like helping chemists 
reason about protein folding, or researchers \q{see} patterns 
in data), and digital art.  However, digital applications of 
topology and geometry can be extended outside the realm of 
computer graphics, insofar as the spatial form of 
objects contribute to their functional properties, and 
therefore to the kinds of data which should be modeled by 
general-purpose software and databases, in domains like 
e-commerce, digital archiving, and NLP.  As I will argue 
here, this suggests a role for Mereotopology as well.   
}
\p{I will explore this role on two fronts: first, from the 
perspective of how mereotopological properties of 
familiar physical objects suggest accounts of their 
functional properties which then carries over to 
data models; and, from a more conceptual angle, 
how spatial and functional properties help demarcate 
the sense and extent of concepts.
}
\p{The act of focusing attention on one object, or 
some interrelated cluster of objects, simulataneously 
marks their collective separation from surroundings 
and highlights their internal connectedness.  Often 
there is a finer-grained scale where internal 
separation, instead, is more clearly perceived; 
attention tends to move across scales as readily as 
across spatial location.  At least as an intuitive 
picture \mdash{} and perhaps, in some contexts, as a formal 
model \mdash{} we can consider this balance of aggregate 
connection and surrounding separation in topological 
terms.  That is, attention tends to introduce (or 
magnify) discontinuity between some focal entity and 
its surroundings, and to minimize discontinuities within it.  
Each particular focus of attention is one manner in 
which a complex object or system may be revealed.  
The continuities and discontinuities thereby involved 
in a particular attention-act are therefore facets of 
the object/system intended in experience, which can 
perhaps be modeled via topology.  In this 
phenomenological thought milieu, objects do not 
have \q{a} topology; but, often, many different 
topologies depending on whether certain discontinuities 
are or are not attended to.
}
\p{Suppose I look at a Canadian flag. 
Insofar as I identify the Maple Leaf pattern and the outer red stripes, 
I perceptually divide it into distinct, visual pieces: I identify 
discontinuities in the flag as an ingegrated union of material and color.  
On the other hand, if I observe its blowing to and fro in the wind, 
or physically manipulate it (folding it ritualistically, for example), 
I cease to attend specifically to the color patterns.  The flag therefore 
lends itself to (at least) two different viewer comportments, 
either as a continuous material object or as a setting for visual 
patterns and icons, and (depending on which viewpoint we take) thereby 
exhibits two different \q{topologies}.  This duality is indeed intrinsic 
to its semiotic functioning: a flag typically relies on visual forms 
for the construction of symbolic patterns, representing characteristic 
or celebrated aspects of a community's culture and history \mdash{} the Maple Leaf
endemic to the \q{true North strong and free}; the thirteen stripes for 
the American colonies; the superposition of the symbols of England, 
Scotland, Ireland, and Wales; the tricoleur for the three Estates 
or later, in an act of re-inscription or re-symbolization, 
for Liberte, Egalite, Fraternite.  Typically these symbols are 
inscribed with a deliberate lack of artistic flourish but with 
a simplicity and repeatability of form, making the flag not art 
but a visual representation of community whose communicational 
properties are an interesting contrast to visual art.  At the same 
time, a flag in its semiotic function must be a physical entity 
that can be perched over certain places and buildings, in plain 
air, with specific material properties.  A reproduction of the 
visual appearance of a flag, for example in an encyclopedia or 
as a metonymic icon for a particular language, is not itself a 
flag (contrast this to the reasons why a picture of a stop sign 
in a driver's instruction manual is not a stop sign, or conversely 
why a graphical reproduction of a copyrights or trademark symbol 
may indeed be a binding example of such a symbol).  As \i{sign}, 
in short, a flag exists simultaneously as material cloth and 
visual tableau, and has different morphological properties depending 
on which (or both) of these guises dominate how it is disclosed to 
us on each occasion.   
}
\p{Our collectively surrouding environs, insofar as this forms the 
horizon and selection-space for our signfications, embody multiple 
potential topologies, multiple parameters of continuity and discontinuity.  
Instead of a binary notion of (dis)continuity, we therefore need an 
image of \i{partial} discontinuity, in which separation in one 
parameter is joined with connectedness in another.  The Canadian flag, 
for example, has regions separated by color but connected by 
material substrate.  Such a combination is intrinsic to certain 
concepts: for example, a \i{footprint} is not a foreign object in some 
medium (like mud or sand), but involves a physical continuity crossed 
with a distinct shape and impression, typically outlined by a separation 
in one direction (down into the ground).  Partial discontinuity has 
an intrinsic mereological aspect, insofar as its mixture of continuity 
and separateness implies both a distinct unit and a larger whole of 
which it is one part.  Therefore mereotopology can be adopted as a 
study of topological properties in these cognitive and semiotic 
contexts, where multiple topologies overlap.  
}
\p{Mereology in this kind of analysis can be more general than \i{partonomy}, or 
splitting a whole into crisp parts.  Our experience of part/whole 
\q{articulation} can depend on the regularity or symmetries 
in some pattern, irregardless of whether we experience crisply delineated 
parts.  For example, our experience of a chessboard depends on symmetries by 
which a color space is mapped onto a spatial extent, as 
much as on the visual discontinuity between black and white.  Imagine 
an artistic chessboard on which the color lines are replaced with 
smooth gradient transitions \mdash{} the underlying color 
space becoming a \q{loop} instead of a simple 
two-valued collection.  Despite being topologically more complex, 
this color-loop is nonetheless a distinct subset of all possible 
color values, and the distribution of color values projected onto 
the checkerboard surface can exhibit the same translational 
symmetries of the two-valued, black/white board, creating the impression 
of similar mereological articulation even in the absence of 
identifiable distinct parts.  Furthermore, the chessboard's mereology has 
functional as well as purely geometric properties.  Each square represents 
a distinct site where, in a game, at most one piece may be placed.  
It is certainly physically possible to violate this norm, 
but such an arrangement would not model a real game situation, or 
physically instantiate the abstract specifications of a chess match.  
The pattern of lighter and darker squares also helps to visually 
specify the rows, columns, and diagonals, so that players can 
clearly see possible moves for the different pieces.  As such, a \q{square} 
is more one site in a functionally organized system than just a spatial 
shape; it may function just as well even if it is not actually clearly 
delineated as a shape (as in the hypothetical \q{artistic} board).  
By analogy, geospatial regions like islands or political units 
function as delineated wholes even if their precise spatial borders 
are impractical to specify with arbitrary granularity (considering 
how changing tides and water levels minutely alter the outline 
of islands, or how border conflicts can obscure the actual 
geospatial extent of nations).
}
\p{For a mathematical model of the chess board or Canadian flag, we can consider 
\q{product spaces}, in these examples a simple, compact two-dimensional space 
crossed with a discrete color dimension (red/white or black/white).  In the 
statistical classification of dimensions as nominal, ordinal, interval, or ratio, 
these color dimensions would be nominal, so there is no concept of relative distance 
between colors.  This changes in the \q{artistic board} example, but symmetries 
of the color/board product space preserve the mereological morphology.  This suggests 
that a simple product-space model, such as one generated by an arbitrarily fine-grained 
mapping (one color value for each dimensionless point), does not fully capture the 
topological properties of the integrated space as a \i{functional} totality.  
In Peter Gardenf-">ors Conceptual Space theory, color is a canonical \q{geometrized} 
concept space, in which different (possibly overlapping) color-concepts can be 
associated with well-defined quantifiable regions in a \q{color space} suitable 
to several mathematical dimensionalizations (the \RGB{}, \CMYK{}, or \q{color wheel} of 
computer graphics).  The coloration of a 2D surface is then a simple spatial-point 
to color-point map.  Seen from a more functional perspective, however, color space 
is less crisply quantified; for example, a chess board functions pretty well even 
if fading and smudging corrupt the idealized black/white pattern.  We have a 
tendency to \q{see} the pattern even if actual perceived colors merely approximate it, 
which suggests that our disposition to find such patterns is driven by organizational 
cues (like symmetries) as much as by crisp forms.  So color may have several different 
functional roles.  It may be purely \q{contrastive} \mdash{} consider how many tournament 
chess boards replace black with green, to equal effect; or decorative, in which case 
precise hues become important; or also symbolic, like the colors in most flags, 
which often carry patriotic and/or geographic metaphors 
(blue for sky or ocean, brown for soil, etc.).\footnote{Flags of African nations often 
carry tones suggestive of the local landscapes (browns, greens, yellows), whereas 
flags of European nations feature colors which are more \q{abstract} (like red and 
white), or harder to match to geographic artifacts, except perhaps for blue, 
which in an admittedly rather speculative analysis might signify that the most 
concrete earthbound feature in those nations' historical collective consciousness 
was the oceans \mdash{} the routes to expansion, colonization, increasing national influence.
}    
}
\p{The previous examples considered mereology in terms of color, a dimension which has 
functional as well as visual aspects, but similar analyses can involve parameters 
with more purely functional interpretations.  Consider 
the contrast between a dumbell fasioned from a single metal part and 
those separated into a central shaft and detached outer rings.  While the 
topology of these shapes is different, they have comparable functional 
form: a good dumbell must be easy to grasp but relatively hard to 
lift, so it needs parameters of both bulk and graspability.  In both 
types, a central shaft provides graspability while outer bulges 
provide heft; therefore these two parameters are distributed around 
the shape of the dumbell, by analogy to the red/while of the 
Canadian flag.  Relative to the \q{functional space} defined 
by these parameters \mdash{} a two-valued {graspable/bulk} contrast \mdash{} the central area 
is isolated by virtue of its graspability, analogous to the 
central Maple Leaf pattern isolated by color, even if in some dumbells 
there is no comparable discontinuity in the physical 
extent considered alone.  Functional parameters can introduce 
partonomies into an otherwise connected substratum, 
or, conversely, topologically unify disparate pieces.  
Geospatial semantics provide clear-cut examples of both phenomena.  
For most of its extent, the border between the New York City boroughs 
of Brooklyn and Queens injects discontinuity into a terrestrial continuum.   
Conversely, a jogger whose itinerary crosses East Harlem and 
Randall's Island is making a closed loop in the borough of 
Manhattan but not the island of Manhattan; the borough integrates 
several other islands and also the small Marble Hill neighborhood on 
the mainland, attached to the Bronx.  The tendency of geospatial 
designations to either partition or bridge together geospatial 
extent, sometimes with no apparent relation to geographical 
features, is a canonical example of Barry Smith's distinction of 
\i{fiat} and \i{bona fide} boundaries as constituting objects 
of reference (and therefore of domain-ontological record).  
The mere presence of a naming scheme induces \q{product spaces} 
where geospaptial regions are crossed with names selected from 
an nominal axis, like {M, B, Q, S, T} for the New York boroughs 
(\q{T} for \q{The Bronx}).
}
\p{Whether or not the outlines of named regions correspond with natural 
features, the mere presence of fiat boundaries alters the functional 
organization of geospatial extents seen as sites of human activity.  
A walk along Myrtle Avenue, for example, might cross between Brooklyn 
and Queens.  This discontinuity may have no \q{natural} meaning, but 
could still be functionally significant; for example, if that path 
is actually a proposed route for a parade, or central strip of a 
business development zone, where regulations from both boroughs need 
to be accommodated.  Fiat boundaries can also exercise causative 
influence on the natural land itself: the border between 
nations with different environmental laws might evolve into a 
line between lands with different ecological properties.  Moreover, 
fiat boundaries tend to be drawn on functional considerations: fiat lines 
tend to be straight, presumably to facilitate surveying and resolving 
land-claims; alternatively, borders may group together 
speakers of a common language, or citizens who have 
voted to form a union.  Geospatial regions, if they have political 
and legal significance, are not just spatial forms, but 
functional collections of people and places.  So \q{fiat} 
boundaries are not really arbitrary.  Evolutionary pressures 
act against truly random or arbitrary geospatial forms.  For example, 
a border so misshapen that it causes inconvience for citizens 
or government might end up being modified by popular demand.  A 
dramatic example is the ill-conceived partition of India, which 
resulted in the topologically disjoint marphology of Pakistan, and 
finally a further war as East Pakistan seceded to become Bangladesh.  
As a more peaceful example, on the Brooklyn/Queens border, the Ridgewood 
neighborhood voted in 1950 to switch from being part of Brooklyn to part 
of Queens.  There were several socioeconomic motives for this demand, 
but the areas' architecture and urban space does seem 
more reminiscent of Queens, and, probably 
more important, the pedestrian and transportation routes seem more 
conventient heading toward Queens business and government districts 
than those of Brooklyn.  The point of this example is that the notion 
of \q{fiat} boundary, while not entirely unmotivated, should not 
be understood too simplistically.  Borders which seem arbitrary from a 
purely geometric or geographic perspective may make sense 
sociologically, or at least have their own social histories.  
The connection of geospatial regions to socially recognized 
names, and functional units (post codes, electoral districts, etc.), 
both encapsulates and helps to create a complex system in which 
human activity plays out against a geographic background.    
}
\p{A \q{topological} theory of functional organization should therefore 
explore how notions of contact, connection, continuity, and 
discontinuity are to be defined in functional as well as 
purely spatial terms.  To say that two spaces are \q{in contact} can 
mean different things, and purely physical contact may be replaced 
with contact via travel, shipping, digital media, causal correlation, 
etc.  A convey of cars can be modeled as a single unit, as if there 
were actually physical forces connecting them, even though the 
actual synergy between their movements depends instead on their 
collective desire to stay together.  Similarly, a group of hikers 
may adopt a spacing on a trail so that each hiker is visible to 
one behind, but the group covers a relatively large area (consider a 
search-and-rescue mission).  Here \q{contact} most 
usefully means \q{visual contact}.  Of course, 
different notions of contact can functionally interrelate, or 
fail to function properly in consort.  Political union can 
forge contacts between disjoint regions, but (as in the 
Pakistan/Bangladesh example) such non-physical connectedness 
is not logistically equivalent to geospatial continuity.   
Or, suppose that I remove a key from a keychain, 
attach a small metal ring, and reattach that key to the new ring.  
During this exercise the keychain has undergone several topological 
modifications, but it has retained a certain cognitive unity as a 
functional object \mdash{} even when dissambled into keys and rings, 
I can with some justice refer to the ensemble as \q{my keychain}. 
I can even carry around the disjoint lot in my pocket with the 
intention of completing it later, while still thinking of it 
in the singular.  But the loose key is be more likely to be 
lost in that scenario: there are (at least) two different 
topologies which are functionally relevant, one based 
primarily on direct physical morphology and one based on 
my intentions.  Functional parameters merge disparate 
parts and introduce new topologies than yielded 
directly from three-dimensional geometry, usually without however 
rendering the original topology functionally irrelevant.
}
\p{On the other hand, functionalities can also introduce new criteria of 
separation.  A zipper is distinguished from a jacket, partly perhaps by 
a difference in color, but more importantly because it has 
dynamic affordances, patterns of movement, distinct from those 
of the jacket itself.  Similarly, a black knob on a white 
stovetop stands out not only by color contrast, but by 
the actions it permits.  The zipper and the knob introduce 
patterns of continuous but delimeted change which give rise 
to new concepts \mdash{} on or off, zipped or unzipped, or 
hitching the zipper just enough to keep the jacket from 
flying open, or zipping \q{all the way to the top} on a cold 
day.  These dynamic patterns represent the emergence 
of geometric dimensions within an underlying topology; I 
can visually represent, and to some degree capture by 
concepts, the degree to which the zipper is zipped along 
an up/down axis, or the knob turned between \q{off} and 
\q{max}.\footnote{Perhaps it is not by accident that knobs and 
faucets, in North America, given our left-to-right orthography 
and our metaphor of \q{clockwise} as \q{forward in time}, 
that turning to the right tends to produce \q{on} or 
\q{more} (heat, light, volume).
}  A \q{functional mereology} does not only isolate 
parts by virtue of functional parameters (like the dumbell's 
{graspable/bulk}); it also reveals parthood isolated 
by the presence of functional dynamics concentrated within 
one area of a larger whole (the up/down of the zipper, or 
left/right of the knob, which have no correspondants across 
the general space of the jacket and stovetop).  Moreover, 
and importantly for the current paper, these patterns 
tend to elicit schema which abstract from specific spatial 
and perceptual forms.  The idea of the zipper or the knob 
as a zone of special dynamic affordances, relative to some 
larger whole \mdash{} both the mereological fact of a parthood 
thus established, and the (at least partially) geometric 
fact of the smaller zone, when attended to, revealing a 
bounded but continuous dynamics \mdash{} this pattern 
is instantiated by a given jacket or stovetop but 
is expected and replicated across almost all examples of 
these concept.  The position of the zipper or the knob 
provides a specification of the jacket or stovetop's \q{state}, 
and moreover a dynamics of how such state may be changed.         
}
\p{These examples demonstrate a correlation between \i{state} and 
\i{mereology}: a particular parameter for the state of a whole 
is often the state of one part, and a part can be isolated 
by virtue of its bearing this state, or being \q{in} one 
of a related set of possible states.  The fact of some 
detail being part of a whole's state therefore tends to 
suggest a mereological corrolary: there may well be 
some part which embodies the state, and so the state 
is a state of the whole because of (or by causal or some other 
correlation with) being a state of the part.  Moreover, 
the dynamics of the part-state reveal various degrees of 
automony relative to other parts or to the whole.  If a stovetop 
has several burners, each knob can be on or off independent of 
each other.  For the stovetop to be considered \q{on}, however,  
(e.g., to be a fire hazard if unattended), it is enough for any 
one knob to be on.  This network of automony and correlation 
governs both the partonomy and the functional organization of the 
whole: the knobs (and also the burners they control) are different 
parts because they are mutually autonomous, at least 
to some degree \mdash{} they remain if nothing else physically 
connected, part of one single appliance \mdash{} but they are 
all parts of one whole, whose state they can all influence.  
This kind of partial autonomy can pose design challenges 
for computater models, because the knobs are neither wholly 
separate entities, but by virtue of having autonomous 
inner state it may be effective to model them as distinct 
computational units.  These kind of scenarios helped 
motivate the idea of \q{Object Oriented} modeling: a given 
\q{object} in this sense can be a complex units whose 
part are separate objects, where this \q{separateness} does 
not mean ontological or empirical disjointness, but 
rather some kind of (partial) functional autonomy.  An \i{object} 
can be generically and roughly defined as a unit of computation 
with semi-autonomous state.  But the relation between objects 
so modeled, and real-world entities, can be imprecise, 
insofar as an ontological partition of the \q{world} into distinct 
things does not necessarily correspond with a partition in terms 
of functional autonomy.  I will suggest, toward the end of this 
paper, that this rather abstract problem has been a kind of 
hidden impediment to an effective integration between software 
and the Semantic Web. 
}
\p{What I have hoped to emphasize in the discussion so far, however, 
is that mereotopological structures within real-world 
phenomena sometimes take on functional patterns, which 
engender schema that abstract from explicit experiential engagement.  
These schema then become rules of interacting with classes of 
objects, or metaphors which can be adoted in artifical-interactive 
settings, like the use of knobs or scrollbars in a Graphical User 
Interface, or scrolling though pages in an electronic document 
with a motion that simulates \q{flipping pages}.  Philosophically, 
our study of these metaphors and these electronic environments 
involves crossing from a phenomenology of the experienced world, 
in which at some level of consciousness we register all sensory and 
perceptual details, to a phenomenology of special \q{operational} 
worlds \mdash{} like the world of a computer application, or the 
World Wide Web \mdash{} in which consciousness is directed toward 
norms and purposes of engagement, and less engrossed with fine-grained 
sensory detail.  This is still phenomenology \mdash{} in other words, 
we need to understand Phenomenology more broadly than an 
Analytic Philosophy of Mind sense preoccupied with qualia 
and \q{first-person experience}.  Qualia, in all their 
experiential preciseness, are indeed intrinsic to the phenomenology 
of the perceptual life-world, but the \i{life-world} is also the 
world of all our actions and purposes, which often have schematic 
forms that abstract from sensory detail, and which in our modern 
age often engage us with constructed environments wherein consciousness 
adapts to hyletically imperfect conditions.  Such ecosystems 
\mdash{} watching television, videos online, using software \mdash{} lack in full 
experiential \q{presence} as this is qualified 
in phenomenologies of Virtual Reality Environments.  Yet as we become 
engrossed in these constructed \q{worlds}, the structures 
of mental intentionality adapt to noemata individuated less by 
sensory distinction and more by how artifacts in these worlds 
implement learnt or familiar functional patterns, so that these 
patterns permit the formation of \q{noemata}, or foci of 
directed attention, no less than \q{sense data}.  For practiced users, 
the rule that clicking a hyperlink opens a new page becomes a kind of 
\q{operational quale}, the left mouse button a physical instrument 
barely experienced as extra-corporeal, so that our momentary 
interest in the link manifests in the action of following it 
with almost the same immediacy as seeing an object manifests in the 
cognitive-perceptual acknoweldgement of its general type (e.g., 
we do not see a collage of colors in form of an automobile and 
deduce \q{car}; we see the colors \i{as} car, as the visible 
disclosure of that type as tokened; the concept, object, and coloration 
are equiprimordial as presences emanating from the lifeworld).  Good 
computer software creates specialized ecosystems where recurring, predictable 
functional design replaces sensory detail as the vector which guides 
mental attention to things of potential interest.  
}
\p{So far I have explored the transition from perceptual expanse to functional 
schema by reference to hypothetical concrete objects, and places \mdash{} flags, 
chess boards, jackets, stove tops, and New York.  Having asserted my claim 
that this transition represents a phenomenological shift from 
sense-consciousness to operational, constructed special \q{worlds}, 
I now want to retrace this transition from the more general level 
of concepts, rather than objects \mdash{} in other words, to study 
functional parameters revealed through the intensions and extensions 
of real-world concepts.  This is the next two sections.  The final 
and concluding section will then tie together the functional 
articulation of individual objects and general concepts, looking 
for patterns that can be a practical guide for computer modeling paradigms. 
}
\subsection{The Functions of Concepts}
\p{In Formal Concept Analysis, a concept is defined by the combination of 
its instances and its indicators.  Two distinct concepts can share 
the same extension: the set of black US presidents 
equals the set of Hawaian born presidents.  Within a collection of objects and 
properties, a \q{formal concept} is a set with both objects and 
properties such that every object bears every property.  In this 
formal setting, a concept is a statistical artifact, 
which may or may not correspond with concepts of thought and 
language (despite its well-defined extent, in the singular person of Barack 
Obama, \i{black Hawaian president} probably does not express a concept 
with cognitive value beyond its constituent properties).  Such analysis 
however suggests a more general truth, that concepts depend on 
both intension and extension; to be acquainted with a 
concept is to understand to some degree both a set of instances and 
the reasons or properties for why they belong.  Moreover, 
within real-world concepts \mdash{} typically imprecise and dynamically 
evolving, insofar as they are cognitive and communicative tools 
\mdash{} intension and extension evolve in consort.  Borderling cases 
need to be either excluded from or included into a concept's reach, 
and this choice forces an evolution in intension.  Suppose we 
start with a simple, provisional account for a familiar concept 
\mdash{} a \i{house} is a place of residence.  We can then consider 
places which we may or may not consider houses \mdash{} an apartment, 
apartment building, hotel, cabin, tent, tree house, the White House.  
Based on how we classify these examples, our posited \q{house 
indicators} evolve; e.g., a house is a place of permanent, year-round 
residence for at most a few families, where habitation (as opposed 
to, say, government or commerce) is its primary purpose.  Such  
thought experiments do not \i{define} concepts, but perhaps they retrace 
their history as linguistic and cognitive phenomena within a relevant community.
}
\p{Although I will consider how conceptual attribution suggests patterns 
of \i{functional organization} within objects thereby identified, we 
can also consider the \q{function} of concepts as serving thought:  
what \i{roles} do typical concepts serve within mental life overall?  
Given the intension/extension co-evolution, we recognize at least 
two distinct roles; a concept both characterizes an \i{extension}, 
a set of instances, and also an \i{intension}, a set of properties 
or indicators which provide a rationale, and a suggestion of 
further characterization, for identifying an instance with a concept.  
To assert that something tokenizes a concept is not only to 
provide a very rough suggestion of what it is like, but to imply 
a strategy or \q{script} for providing more detail.  Once I 
identify something as a \i{house}, I imply a kind of 
organization \mdash{} a yard or some sort of surrounding property; 
an inside and outside; an external style and architecture; 
a set of several internal separated but interconnected rooms.  
I also imply (in contrast to a motor home) that the house 
is built in one fixed place and has a permanent location, 
perhaps a postal address, that visiting someone at their 
home means returning to the same location on each occasion.   
But implying this general schema also \q{initiates} a script 
for further specification; identifying something with a 
concept is more often the start of a conceptual and linguistic 
process, rather than a conclusion.  I can learn the address 
of your new house, directions to get there, you can describe 
its various features, its yard and interior design and so forth.
}
\p{Concepts play different roles, and these can be \q{activated} in 
turn by different situations, in particular by different 
linguistic and grammatic formations.  Similar meanings can be 
achieved by subtle variation in which conceptual roles are 
suggested, providing a case study for how grammar connotes role.  
Consider Ronald Langacker's comparison of sentences like 
\begin{sentenceList}\sentenceItem{Three times, students asked an interesting question.}
\sentenceItem{Three times, a student asked an interesting question.}
\end{sentenceList}
In (1), the plural \i{students} reflects how a type of noteworthy 
situation occurred multiple times; whereas the singular in (2) reflects 
how, on each occasion, one student was involved.  The \q{student} in 
(2) does not designate a particular person, but is rather a generic 
token of the concept \i{student} \q{conjured}, as Langacker says, to 
provide a kind of unspecified cognition of a conceptual tokening without 
implying reference to some specific token.  Along these lines, consider 
\begin{sentenceList}\sentenceItem{Giraffes are mammals.}
\sentenceItem{A giraffe is a mammal.}
\sentenceItem{The giraffe is a mammal.}
\end{sentenceList}
(3) refers explicitly to the set of all giraffes; each of them is a 
mammal.  (4) second \q{conjures} a \q{generic} giraffe, from an 
abstract \q{plane}, again using Langacker's terminology, to make the point that 
any conceivable giraffe is a mammal; that is, mammalness is a generic feature 
of the entire species, it is not a quality which depends on the nature of 
any given instance.  In (5), the signification by contrast selects from the 
\q{plane} of all species types; so \i{giraffe} here applies in 
the guise of naming a discrete element in categorizing thought.  These examples 
illustrate roles of \i{selecting an extension}, \i{expressing intension} 
\mdash{} in the specific sense of concept-intension; that is, envisioning a 
representative abstract case which explifies the conjunction of properties 
and indicators that are a concepts' signature \mdash{} and \i{naming a type}, 
referring into a space of \i{kinds} insofar as these are present in thought 
as discrete units.  Which of these roles is most clearly \q{activated}, 
given semantic and grammatic cues, then shapes the surrounding discourse: 
\begin{sentenceList}\sentenceItem{The giraffe is related to the horse.}
\sentenceItem{Young giraffes are threatened by poachers.}
\sentenceItem{Giraffes in that park are threatened by poachers.}
\end{sentenceList}
 
Here (6) forefronts giraffes as a natural kind, invoking a familiar 
relation between species; (7), I would argue, invokes a more 
intensional sense of the concept insofar as it adds a further 
specifier to suggest a narrower concept (a young giraffe has the 
properties associated with giraffes in general, and then further those 
associated with young animals, such as being exceptionally vulnerable and 
being cared for by parents); while (8) seems to construct a narrower 
concept by adding specifiers to \i{extent}, not \i{intent}.  There 
is no suggestion that giraffes \q{in that park} have any further 
resemblance aside from their being there; as a result, (8) comes 
to designate some set of animals by invoking the \i{set} of 
giraffes and then adding semantics to focus on some select portion of that set.
In these examples the concept \i{giraffe} plays different semantic 
roles \mdash{} guise of a species, a bundle of typical properties and indicators, 
and a set of jointly classified individually \mdash{} corresponding to 
different cognitive roles.
}
\p{This multiplicity of roles precludes simplistic theorizations of 
concepts as \i{just} instance-sets, or property-bundles, or taxonomic entries.  
Rather than a single (meta-) definition, a general theory of concepts should 
begin by classifying different ways that concepts are used.  An initial 
distinction is that a concept can \q{profile} (another Langacker term) 
\i{either} an individual \i{or} a collection of individuals, 
including (but not necessarily) a set of all (actual or possible) instances.  
In the latter guise a concept (alone or in consort with other semantic 
elements) demarcates a collection from some more general or expansive collection.  
In the former case a concept \i{singles out} an individual; as I discussed earlier, 
this implies some mixture of continuity and discontinuity, insofar as the 
individual is (to some degree) posited as separable (in an act of cognition 
and/or perception) from its surroundings and also as internally connected or 
integrated.  There is some \q{theory} of the \i{internal coherence} 
of the individual, of how it is appropriate in some context to consider it 
a discrete unit \mdash{} acting as as singular whole, causally integrated, or 
in some other fashion disposed to function unitarily.
}
\p{Certainly the \q{individual coherence} of a totality may be provisional.  Some 
collections are bound together only by people's desire to group them, e.g., 
the books in a library.  Nevertheless, even this \q{external} connectedness 
is caussally efficacious, with non-neglibable effects of cohering the integral 
whole more than a random baseline: books in a library are much more likely to 
remain spatially proximate than a random pair of books which happen to be in 
some one building at the same time.  On the other hand, the degree to which 
parts \i{do} cohere into a whole is an essential conceptual detail, and, 
once a whole is characterized as an individual tokenizing a concept, part of that 
concept's role is to suggest the degree and nature of this coherence.  
For example, the United Nations is a more diffuse collection of political 
units than the United States.  Nevertheless, there is a well-established 
semantics where the UN functions as a singular entity which can, say, 
pass a resolution.  NATO is an alliance of quite independent nations, 
but its Cold War antagonism to the Warsaw Pact helped consolidate 
both groups as functional unities; there is a thread in Cold War 
history constituted by how NATO and the Warsaw Pact interacted.  
While conceptual schema characterize wholes as units, they also 
capture variations in how and how tightly this unity binds its parts, 
and ground further semantics in which some diffuseness or autonomy 
of parts is understood.  It is a matter of political fact, which 
then structures the semantics of a name like \q{United Nations}, that 
UN resolutions are not as binding as national laws or even 
binational treaties; or that NATO's popular conceptualization as a 
distinct unit has diminished with the dissolution of its former 
adversaries (the Warsaw Pact, Soviet Union, \q{Iron Curtain}).    
On the other hand, insofar as a totality has a strong degree of 
internal coherence, this may result from a complex causal or 
processual integration between many parts, or conversely by a relative lack of 
internal complexity or \i{internal structuration}.  A torpedo and a 
school of fish may both follow a quantifiable trajectory through 
water, but the directedness of the former's motion depends on 
complex biosocial synchronization much different from the 
latter's straightforward physics.  For each conceptualized 
totality, then, there is a measure of Individual Coherence 
and of Internal Structuration \mdash{} which can be quite independent 
from another \mdash{} that specifies \i{in what sense} the whole 
is a (single, unitary) individual.  The concept used to 
designate the whole, along with (in a language setting) 
surrounding grammar and context, suggests a particular 
account of the degree and schematic nature of this individuation.
}
\p{Grammatic choices, like singular/plural and defininte/indefinite article, 
play a role here: 
\begin{sentenceList}\sentenceItem{A giraffe is by the lake.}
\sentenceItem{The giraffe is by the lake.}
\sentenceItem{Giraffes are by the lake.}
\sentenceItem{The giraffes are by the lake.}
\end{sentenceList}
These grammatic variations induce subtle cues to construct or 
indicate a \q{plane} from which individuation selects.  In 
(9), the hearer's attention is newly directed to something 
which, the speaker suggests, either the hearer or both parties 
had not previously observed.  In other words, (9) wants to 
shift the topic of conversation (even if only temporarily, 
so as to register the asserted fact as something collectivly 
realized) and so functions both to individuate the giraffe, 
to propose it as a new object collectively recognized as something 
in their mutual surroundings, and to shift attention in its direction.  
By contrast, the giraffe in (10) has a presumptive 
prior acknowledgement in the current discourse; the point of 
the definite article is to select the giraffe from among the 
set of all entities which have in some fashion been 
talked about already.  These different \q{planes of selection} 
carry over to the plural examples, but the designation there 
selects a group of animals, not one single individual.  
Nevertheless, (11) implies that the group has some 
sort of internal connectedness, perhaps moving roughly 
as a unit.  By referring back, as does (9), to an earlier phase 
in the discourse, (11) implies that the group represents (more or less) 
\q{the same} giraffes as observed earlier, an effect of comparison 
which helps call attention to the giraffes in their totality.  
Without the definite article, (12) is 
less specific in asserting such a totality; it may, 
but need not, suggest that the giraffes by the lake are situated 
in some spatial or interacting formation so that the speakers 
should be disposed to consider them collectively.  If the speaker 
wanted to nudge the emphasis more or less toward this grouping interpretation, 
she would have to select some further semantics: e.g., 
\i{A school of giraffes}, or conversely \i{Some giraffes}, are by the lake.  
}
\p{In these examples, the concept \i{giraffe} does not, on its own, guide one 
interpretation or another (singular/plural; more itegral or 
more scatterrred); it is rather a resource which is deployed 
in a linguistic and communicative setting along with further 
details, and the effect of conceptual resonance \i{in context} bears these 
further connotations.  Whether a concept designates a group or an individual, 
it is the speaker (or a person using a concept as a tool in thought) 
who uses ideas activated by the concept so as to evoke patterns 
by which the selected individual or collection is internally 
connected and externally separated.  Any conceptualization presents 
schema where continuities and discontinuities are mixed, but 
in different ways.  Suppose a game warden says:
\begin{sentenceList}\sentenceItem{Those footprints were left by giraffes.}
\sentenceItem{That trail of footprints was left by giraffes.}
\sentenceItem{Those tracks were left by giraffes.}
\end{sentenceList}
I argued above that the concept \q{footprint} internally suggests both 
continuity (of medium) and separation (of shape or contour).  Analogously, 
a collection of footprints as in (13) has a presumptive internal 
separation into discreet individuals, but enough spatial proximity to be 
plausibly grasped as a perceptual group, relative to a reasonably 
typical vantage point.  The choice of words in (14) implies a further notion 
of directedness, and a somewhat different relation of the observer's spatial position  
\visavis{} the observed.  While the two sentences might describe the 
same situation, (14) carries an additional implication that the 
footprints suggest some line of direction which extends beyond the speakers' 
immediate line of sight, and can perhaps be followed.  This directionality 
is also implied in (15), but in that case the internal continuity which 
supports this sense of direction is further emphasized.  In this context 
\q{tracks} can mean not just footprints but also, say, bent grass or hairs 
or broken twigs, so that each particular artifact which cumulatively 
constitutes the \q{tracks} may have less precise distinctness than 
a particular footprint.  Moreover, the word \q{tracks} tends overall to 
refer to physically continuous structures, like train tracks, whereas 
\q{trail} can be used in more metaphorical senses and with greater 
sense of internal pauses and separations (a dective \i{on the trail} of a 
fugitive; a scientist \i{on the trail} of a discovery).  Being 
\q{on track} connotes making steady progress toward some goal.  
Scientists at CERN were \i{on the trail} of the Higgs Boson; it would 
have been quite optimistic to say they were \i{on track} to find it.  
Finally it was revealed in the \i{tracks} of secondary particles whose 
movements were recorded in the LHC (Large Hadron Collider) bubble chamber.  
Those were literally streaky lines on a photograph, thought to 
show the actual path which particles took, to \i{track} the particles, 
whereas the \i{trail} of an airplane or comet is a less clearly localized 
after-the-fact indication of their path across the sky.  
}
\p{So (13)-(15) show progressively less attention paid to internal 
constituents and more to the nature and direction of a whole, 
even if some scenarios tolerate all three sentences. 
Each usage would convey slightly different connotations to the 
scene at hand.  The point of this comparison is that individuation 
through concepts depends on properly connoting the patterns of 
continuity and discontinuity which define individuals' coherence 
and separateness (as well as that of sets of individuals, when 
concepts are used to profile collections).  A concept implies a 
schema both for individuals' internal structuration, and for 
its suitability as being conceptually and perceptually isolated 
from its surroundings.  The function of concepts is, typically, 
to \i{set the stage} for further characterization of their tokens, 
first by individuating them as points of attention, and then 
by implying the presence of \q{scripts} or \q{schema} leading to 
more precise descriptions.  Both of these roles present challenges 
for \i{formal} semantics, if this means defining rigorous of 
quasi-mathematical theories of how concepts select a \i{set} of 
instances, or of how each instance \i{instantiates} a concept.  
The act of linking a concept to an instance is a complex 
cognitive/semantic event, one which occurs against the backdrop of a 
mental and often dialogic context, and which should be theorized 
as one step in an evolving process.  A concept is an opening 
onto a more detailed (process of) characterization.  To say that 
something is an instance of some concept is not a simple 
act of classification, then, analogous to declaring a variable 
in a programming language to be of some data type.  
}
\p{Of particular importance, I believe, is the notion that conceptualization 
(if we define this as concept-to-token identification) is only one 
stage in an extended process.  This means that no one single 
concept provides a definitive account of an identified instance, 
even a provisional or imprecise account.  Obviously concepts 
operate at some level of coarse-graining and can be combined 
together for greater precision.  However, if we fail to 
theorize conceptualization as \i{process}, we can end up with a 
view of concepts as imprecise but, \i{at some degree of detail}, 
self-contained mental \q{pictures} which cover some range of 
possible cases.  For each concept, there would correspond 
(on this account) a \q{rough} sketch of how a token of the 
concept is, insofar as it does bear the concept.  The degree 
of (im)precision in these \q{sketches} would depend on how 
general or specific is the concept itself: scientific terms, 
for example, are more specific than everyday words.  
But each concept would connote a family of instances whose 
features are more clearly or more vaguely aligned with a 
set of indicators.  Each house is vaguely aligned with a 
\q{sketch} of some locale with a yard, a front door, interior 
rooms, and so forth; each restaurant is aligned with a 
general notion of a place to sit at a table and eat prepared food, 
and so forth.  The concept leaves particulars imprecise 
(what kind of food, what kind of yard) but presents them at a 
level of vagueness in which they match all examples.  A 
concept, in other words, finds a suitable mix of vagueness 
and resemblance \mdash{} it abstracts from particular details 
enough that all tokens can be considered to resemble each other.
}
\p{While my analysis has likewise argued for \i{characterizing features}, 
where I differ from this kind of theory is in emphasizing conceptual 
imprecision as a matter of \i{process} and not of \i{vagueness}, 
or concepts as \i{scripts} rather than \i{sketches}.  It is misleading, 
I believe, to think of a concept as \i{imprecise} in that it seeks 
to achieve a degree of nonspecifity which allows many somewhat mutually 
resembling things to be grouped together.  This notions seems to both 
minimize the potential for using particular concepts to initiate 
more precise further characterization, and also to overstate the 
degree to which concept-tokens resemble each other.  As mental 
tools concepts are thereby weakened on two fronts: they are neither 
precise enough to truly describe their instances, nor general enough 
to group together these instance outside of some \q{resemblance} between 
them.  I would argue, however, that real-world concepts both trend 
toward a degree of generality which is hard to account for in terms 
of token-resemblance, \i{and also} serve within cognitive and linguistic 
episodes in which general concepts provide a framework for precisely 
describing individuals, up to a degree of resolution suited for a 
given thought or dialog.  The concept may not internally \i{imply} 
these details, but it implies a \i{framework} for accumulating them.
}
\p{If we consider the concept of \i{fly}, for example, we note that it 
covers many different cases: the flight of birds, planes, coments, commercial 
passengers, leaves, kites, debris, shrapnel, types of aircraft 
(\q{that model flew during World War II}), carriers (\q{we fly 
Korean Air whenever possible}), their fleet (\q{that company flies the 
youngest planes of any discount carrier}).  All of these somehow 
involve travel through air, so we might say that the role of \q{fly} is to 
invoke a highly stylized or fuzzy sketch of \q{movement above the ground}; 
in other words, to find the point of resemblance among all of these cases.  
The different senses of \i{fly} might then be considered \i{subconcepts}, 
each with a more precise picture, which finds more detailed resemblances 
between its tokens.  By this account, the concept \q{fly} is really a 
loose aggregate of more precise concepts, which manifest as different 
senses of the polysemous English word.  However, such an analysis 
seems to too neatly group these various senses into a conceptual 
hierarchy, as if word-senses can assemble into a taxonomy akin to the 
classification of living species, each sense finding a degree of 
precision and inter-token resemblance relative to its hierarchical position.  
I would argue that different \q{senses} of words or concepts do not 
generally reveal such straightforward taxonomic patterns, at least 
unless (as with biological names for species, genera, etc.) they are 
deliberately constructed to do so.  In normal usage new senses evolve 
gradually and can emerge ad-hoc for some specific language community 
(or \q{micro} community).  For example, the sense of \q{fly} as 
in \q{Commercial Flight} \mdash{} with its typical associations of 
buying a ticket, going to an airport, checking baggage, passing 
security, and so on \mdash{} emerged only gradually from the more general 
sense of flying in an aircraft.  Moreover, we can imagine specific 
situations where new senses emerge \mdash{} for example, executives 
of a multinational company who use the phrase \q{Flying to London} 
to mean a trip to specific offices, thereby mixing (in this specific 
context) the destination with the purpose of the trip. 
}
\p{The fecundity with which these senses arises suggests to me that we should 
not consider each \q{sense} as its own (sub)concept; instead, I would 
consider \i{fly} to be \i{one} concept, not a loose assortment of 
vaguely related concepts.  This concept coverts many different cases, 
but the cases can be compared and characterized according to certain 
overall criteria, such as the \i{cause}, \i{reason}, \i{agent}, and 
\i{destination} of the flight.  Flying \i{debris}, \i{leaves}, and \i{kites}   
all rely on the power of wind, but the former implies a relatively strong gust 
to dislodge solid objects naturally resting on the ground, whereas the latter 
implies a human agent who orients the kite to best gather the wind.  
Flying \i{birds} and \i{planes} are both self-powered, but the latter relies 
on human agency to provide fueal.  The various senses of human flight are 
comparable based on the kind of craft used, the relation between passengers, 
pilots, and whoever owns the craft (contrast hobbyist who owns and flies 
planes, an air force pilot, a passenger service where flying is a commercial 
transaction).  Each of these distinctions provide parameters for comparing 
different examples of flight.  No parameters, except perhaps for the 
most general notion of airborn motion, apply to all cases and kinds of 
flights; but each case involves some mixture of some relevant parameters, 
and each parameter offers a ground of comparison.  For example, only 
commercial flight involves a ticket, but once this parameter is 
\q{activated} we can compare the price of different flights.
}
\p{It is hard to identify isolated \q{subconcepts} within general 
flight because different parameters overlap in different 
ways.  Passenger flight has unique aspects of cost and 
value (and whatever legal and social considerations 
come along with commercial transactions), but it shares 
other parameters with other (but not all) senses: 
relatively specific times and destinations of travel, 
like the flight of birds but unlike that of leaves or debris; 
the distinction between the subject and the agent of flight, 
a \i{person} ia flying but by virtue of being on an aircraft, 
so the subject/agent relation becomes a conceptual parameter, 
which is also present in some other contexts, like cargo.  
For an analogous example, we might think to subdivide the 
concept of \q{restautant} into more specific cases \mdash{} after 
all, merely expressing the desire to visit a restaurant
in general, or searching for one, can cover a wide spectrum 
much of which may not actually reflect the speaker's/searcher's 
interest.  Someone may want a formal dining experience, a casual 
spot for family dinner, a healthy and inexpensive lunch, a 
quick meal, or a place to have a snack and go online.  Certain 
lexemes cover some part of these more specific spectra 
\mdash{} bistro, steak house, diner, coffee shop, cafeteria \mdash{} yet none 
of these compete with \q{restaurant}'s dominance of the relevant semantic 
territory: one will often use the more general word even when one 
of the more specific ones apply.  I believe this can be explained, 
in part, because the more general concept can be narrowed in 
several ways at once, thereby providing a more effective point 
of origin for a specified description; providing more 
\q{avenues} for elaboration.  For example, a \q{Five Dishes and Soup} 
spot in a Chinatown might be considered a Chinese restaurant and a 
cafeteria, but there does not appear to be an established 
usage \q{Chinese Cafeteria}.  Even if \q{cafeteria} might be 
plausibly defined as any restaurant without table service, it 
seems to connote further details about the kind and pricing of the food.  
Instead of specifiers offering a clear-cut segmentation of the 
\q{restaurant} spectrum, English speakers seem to prefer the more 
general usage and to provide further specification by an unfolding, 
dialogic, and open-ended process, relying on communicative cues to 
converge on an image of the \q{kind of place} some restaurant is, 
rather than trusting some narrower lexeme (like \i{cafeteria} or 
\i{diner}) to impart these cues via convention.  Arguably, 
in some specific circumstances, a narrower lexeme has indeed 
achieved something like canonical status (\i{pizzeria}, \i{coffee bar}); 
so a speaker using those terms, amongst a particular language community, 
shows confidence that the allusions embedded in them convey a 
sufficiently precise picture of the \q{kind of place} being mentioned.  
But this specificity stands in semantic relation to the more general term, 
even if the latter is replaced on some occasions: the ubiquity of the 
word \q{restaurant} adds an extra dimension to a speaker's choice on 
some occasion to instead say \i{pizzeria} or \i{coffee bar}.  The concept 
\q{flight}, I would argue, has a similar semantic ubiquity; it 
encompasses a spectrum of cases so broadly that the occasional 
choice of a different word (\q{I jetted to London}; \q{The birds migrated 
south}) carries the semantics not only of the word used but the dominant 
one \i{not} used. 
}
\p{What this analysis suggests is that real-world language does not necessarily 
show a tendency to semantic specificity, to a preference for more specific 
usages and a relegation of more general terms to cases of deliberate 
imprecision or abstraction.  If a language community had an instinctive 
drive toward semantic precision, then dominant but imprecise words like 
\q{restaurant} would tend to gradually be replaced in many cases by 
narrower concepts, and to be retained mostly when speakers deliberately 
intend to avoid any narrower connotations.  However, this does not 
appear to be the way language in general evolves.  I believe that, 
to an important degree, this fact can be explained by appeal to 
conceptual roles: the point of a concept is not to condence as 
precise a picture of its tokens into as concise a semantic unit as 
possible, but rather to initiate a further descriptive process 
as efficiently as possible.  If I had one word to describe a 
\q{Five Dishes and Soup}, I would probably prefer \i{cafeteria} to 
\i{restaurant}; but in normal dialog, that choice would probably 
inhibit rather than facilitate further elaboration (a qualifier 
along the lines of \q{it's \i{like a} cafeteria} would immediately 
walk back from the concept-attribution, opening a space for me to 
refine the description).  Insofar as concept-attribution is only 
one part of an evolving process, computational models of 
real-world concepts (like restaurants or air carriers or houses) 
should not design associated data-types with the intention that 
their corresponding tokens have some \q{resemblance} to or 
\q{substitute} for their real-world counterparts.  A data-model 
which simulates a real-world object or system does not 
\i{iconify} its referent, analogous to a photograph or a 3D model.  
Instead, it initiates a \i{descriptive process} analogous to 
how concept-attribution in language stimulation a subsequent discourse.    
}
