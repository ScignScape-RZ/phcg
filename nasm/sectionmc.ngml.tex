\section{Coherence and Concepts}
\p{In Formal Concept Analysis, a concept is defined by the combination of 
its instances and its indicators.  Two distinct concepts can share 
the same extension: the set of black US presidents 
equals the set of Hawaian born presidents.  Within a collection of objects and 
properties, a \q{formal concept} is a set with both objects and 
properties such that every object bears every property.  In this 
formal setting, a concept is a statistical artifact, 
which may or may not correspond with concepts of thought and 
language (despite its well-defined extent, in the singular person of Barack 
Obama, \i{black Hawaian president} probably does not express a concept 
with cognitive value beyond its constituent properties).  Such analysis 
however suggests a more general truth, that concepts depend on 
both intension and extension; to be acquainted with a 
concept is to understand to some degree both a set of instances and 
the reasons or properties for why they belong.  Moreover, 
within real-world concepts \mdash{} typically imprecise and dynamically 
evolving, insofar as they are cognitive and communicative tools 
\mdash{} intension and extension evolve in consort.  Borderling cases 
need to be either excluded from or included into a concept's reach, 
and this choice forces an evolution in intension.  Suppose we 
start with a simple, provisional account for a familiar concept 
\mdash{} a \i{house} is a place of residence.  We can then consider 
places which we may or may not consider houses \mdash{} an apartment, 
apartment building, hotel, cabin, tent, tree house, the White House.  
Based on how we classify these examples, our posited \q{house 
indicators} evolve; e.g., a house is a place of permanent, year-round 
residence for at most a few families, where habitation (as opposed 
to, say, government or commerce) is its primary purpose.  Such  
thought experiments do not \i{define} concepts, but perhaps they retrace 
their history as linguistic and cognitive phenomena within a relevant community.
}
\p{Although I will consider how conceptual attribution suggests patterns 
of \i{functional organization} within objects thereby identified, we 
can also consider the \q{function} of concepts as serving thought:  
what \i{roles} do typical concepts serve within mental life overall?  
Given the intension/extension co-evolution, we recognize at least 
two distinct roles; a concept both characterizes an \i{extension}, 
a set of instances, and also an \i{intension}, a set of properties 
or indicators which provide a rationale, and a suggestion of 
further characterization, for identifying an instance with a concept.  
To assert that something tokenizes a concept is not only to 
provide a very rough suggestion of what it is like, but to imply 
a strategy or \q{script} for providing more detail.  Once I 
identify something as a \i{house}, I imply a kind of 
organization \mdash{} a yard or some sort of surrounding property; 
an inside and outside; an external style and architecture; 
a set of several internal separated but interconnected rooms.  
I also imply (in contrast to a motor home) that the house 
is built in one fixed place and has a permanent location, 
perhaps a postal address, that visiting someone at their 
home means returning to the same location on each occasion.   
But implying this general schema also \q{initiates} a script 
for further specification; identifying something with a 
concept is more often the start of a conceptual and linguistic 
process, rather than a conclusion.  I can learn the address 
of your new house, directions to get there, you can describe 
its various features, its yard and interior design and so forth.
}
\p{Concepts play different roles, and these can be \q{activated} in 
turn by different situations, in particular by different 
linguistic and grammatic formations.  Similar meanings can be 
achieved by subtle variation in which conceptual roles are 
suggested, providing a case study for how grammar connotes role.  
Consider Ronald Langacker's comparison of sentences like 
\begin{sentenceList}\sentenceItem{Three times, students asked an interesting question.}
\sentenceItem{Three times, a student asked an interesting question.}
\end{sentenceList}
In (1), the plural \i{students} reflects how a type of noteworthy 
situation occurred multiple times; whereas the singular in (2) reflects 
how, on each occasion, one student was involved.  The \q{student} in 
(2) does not designate a particular person, but is rather a generic 
token of the concept \i{student} \q{conjured}, as Langacker says, to 
provide a kind of unspecified cognition of a conceptual tokening without 
implying reference to some specific token.  Along these lines, consider 
\begin{sentenceList}\sentenceItem{Giraffes are mammals.}
\sentenceItem{A giraffe is a mammal.}
\sentenceItem{The giraffe is a mammal.}
\end{sentenceList}
(3) refers explicitly to the set of all giraffes; each of them is a 
mammal.  (4) second \q{conjures} a \q{generic} giraffe, from an 
abstract \q{plane}, again using Langacker's terminology, to make the point that 
any conceivable giraffe is a mammal; that is, mammalness is a generic feature 
of the entire species, it is not a quality which depends on the nature of 
any given instance.  In (5), the signification by contrast selects from the 
\q{plane} of all species types; so \i{giraffe} here applies in 
the guise of naming a discrete element in categorizing thought.  These examples 
illustrate roles of \i{selecting an extension}, \i{expressing intension} 
\mdash{} in the specific sense of concept-intension; that is, envisioning a 
representative abstract case which explifies the conjunction of properties 
and indicators that are a concepts' signature \mdash{} and \i{naming a type}, 
referring into a space of \i{kinds} insofar as these are present in thought 
as discrete units.  Which of these roles is most clearly \q{activated}, 
given semantic and grammatic cues, then shapes the surrounding discourse: 
\begin{sentenceList}\sentenceItem{The giraffe is related to the horse.}
\sentenceItem{Young giraffes are threatened by poachers.}
\sentenceItem{Giraffes in that park are threatened by poachers.}
\end{sentenceList}
 
Here (6) forefronts giraffes as a natural kind, invoking a familiar 
relation between species; (7), I would argue, invokes a more 
intensional sense of the concept insofar as it adds a further 
specifier to suggest a narrower concept (a young giraffe has the 
properties associated with giraffes in general, and then further those 
associated with young animals, such as being exceptionally vulnerable and 
being cared for by parents); while (8) seems to construct a narrower 
concept by adding specifiers to \i{extent}, not \i{intent}.  There 
is no suggestion that giraffes \q{in that park} have any further 
resemblance aside from their being there; as a result, (8) comes 
to designate some set of animals by invoking the \i{set} of 
giraffes and then adding semantics to focus on some select portion of that set.
In these examples the concept \i{giraffe} plays different semantic 
roles \mdash{} guise of a species, a bundle of typical properties and indicators, 
and a set of jointly classified individually \mdash{} corresponding to 
different cognitive roles.
}
\p{This multiplicity of roles precludes simplistic theorizations of 
concepts as \i{just} instance-sets, or property-bundles, or taxonomic entries.  
Rather than a single (meta-) definition, a general theory of concepts should 
begin by classifying different ways that concepts are used.  An initial 
distinction is that a concept can \q{profile} (another Langacker term) 
\i{either} an individual \i{or} a collection of individuals, 
including (but not necessarily) a set of all (actual or possible) instances.  
In the latter guise a concept (alone or in consort with other semantic 
elements) demarcates a collection from some more general or expansive collection.  
In the former case a concept \i{singles out} an individual; as I discussed earlier, 
this implies some mixture of continuity and discontinuity, insofar as the 
individual is (to some degree) posited as separable (in an act of cognition 
and/or perception) from its surroundings and also as internally connected or 
integrated.  There is some \q{theory} of the \i{internal coherence} 
of the individual, of how it is appropriate in some context to consider it 
a discrete unit \mdash{} acting as as singular whole, causally integrated, or 
in some other fashion disposed to function unitarily.
}
\p{Certainly the \q{individual coherence} of a totality may be provisional.  Some 
collections are bound together only by people's desire to group them, e.g., 
the books in a library.  Nevertheless, even this \q{external} connectedness 
is caussally efficacious, with non-neglibable effects of cohering the integral 
whole more than a random baseline: books in a library are much more likely to 
remain spatially proximate than a random pair of books which happen to be in 
some one building at the same time.  On the other hand, the degree to which 
parts \i{do} cohere into a whole is an essential conceptual detail, and, 
once a whole is characterized as an individual tokenizing a concept, part of that 
concept's role is to suggest the degree and nature of this coherence.  
For example, the United Nations is a more diffuse collection of political 
units than the United States.  Nevertheless, there is a well-established 
semantics where the UN functions as a singular entity which can, say, 
pass a resolution.  NATO is an alliance of quite independent nations, 
but its Cold War antagonism to the Warsaw Pact helped consolidate 
both groups as functional unities; there is a thread in Cold War 
history constituted by how NATO and the Warsaw Pact interacted.  
While conceptual schema characterize wholes as units, they also 
capture variations in how and how tightly this unity binds its parts, 
and ground further semantics in which some diffuseness or autonomy 
of parts is understood.  It is a matter of political fact, which 
then structures the semantics of a name like \q{United Nations}, that 
UN resolutions are not as binding as national laws or even 
binational treaties; or that NATO's popular conceptualization as a 
distinct unit has diminished with the dissolution of its former 
adversaries (the Warsaw Pact, Soviet Union, \q{Iron Curtain}).    
On the other hand, insofar as a totality has a strong degree of 
internal coherence, this may result from a complex causal or 
processual integration between many parts, or conversely by a relative lack of 
internal complexity or \i{internal structuration}.  A torpedo and a 
school of fish may both follow a quantifiable trajectory through 
water, but the directedness of the former's motion depends on 
complex biosocial synchronization much different from the 
latter's straightforward physics.  For each conceptualized 
totality, then, there is a measure of Individual Coherence 
and of Internal Structuration \mdash{} which can be quite independent 
from another \mdash{} that specifies \i{in what sense} the whole 
is a (single, unitary) individual.  The concept used to 
designate the whole, along with (in a language setting) 
surrounding grammar and context, suggests a particular 
account of the degree and schematic nature of this individuation.
}
\p{Grammatic choices, like singular/plural and defininte/indefinite article, 
play a role here: 
\begin{sentenceList}\sentenceItem{A giraffe is by the lake.}
\sentenceItem{The giraffe is by the lake.}
\sentenceItem{Giraffes are by the lake.}
\sentenceItem{The giraffes are by the lake.}
\end{sentenceList}
These grammatic variations induce subtle cues to construct or 
indicate a \q{plane} from which individuation selects.  In 
(9), the hearer's attention is newly directed to something 
which, the speaker suggests, either the hearer or both parties 
had not previously observed.  In other words, (9) wants to 
shift the topic of conversation (even if only temporarily, 
so as to register the asserted fact as something collectivly 
realized) and so functions both to individuate the giraffe, 
to propose it as a new object collectively recognized as something 
in their mutual surroundings, and to shift attention in its direction.  
By contrast, the giraffe in (10) has a presumptive 
prior acknowledgement in the current discourse; the point of 
the definite article is to select the giraffe from among the 
set of all entities which have in some fashion been 
talked about already.  These different \q{planes of selection} 
carry over to the plural examples, but the designation there 
selects a group of animals, not one single individual.  
Nevertheless, (11) implies that the group has some 
sort of internal connectedness, perhaps moving roughly 
as a unit.  By referring back, as does (9), to an earlier phase 
in the discourse, (11) implies that the group represents (more or less) 
\q{the same} giraffes as observed earlier, an effect of comparison 
which helps call attention to the giraffes in their totality.  
Without the definite article, (12) is 
less specific in asserting such a totality; it may, 
but need not, suggest that the giraffes by the lake are situated 
in some spatial or interacting formation so that the speakers 
should be disposed to consider them collectively.  If the speaker 
wanted to nudge the emphasis more or less toward this grouping interpretation, 
she would have to select some further semantics: e.g., 
\i{A school of giraffes}, or conversely \i{Some giraffes}, are by the lake.  
}
\p{In these examples, the concept \i{giraffe} does not, on its own, guide one 
interpretation or another (singular/plural; more itegral or 
more scatterrred); it is rather a resource which is deployed 
in a linguistic and communicative setting along with further 
details, and the effect of conceptual resonance \i{in context} bears these 
further connotations.  Whether a concept designates a group or an individual, 
it is the speaker (or a person using a concept as a tool in thought) 
who uses ideas activated by the concept so as to evoke patterns 
by which the selected individual or collection is internally 
connected and externally separated.  Any conceptualization presents 
schema where continuities and discontinuities are mixed, but 
in different ways.  Suppose a game warden says:
\begin{sentenceList}\sentenceItem{Those footprints were left by giraffes.}
\sentenceItem{That trail of footprints was left by giraffes.}
\sentenceItem{Those tracks were left by giraffes.}
\end{sentenceList}
I argued above that the concept \q{footprint} internally suggests both 
continuity (of medium) and separation (of shape or contour).  Analogously, 
a collection of footprints as in (13) has a presumptive internal 
separation into discreet individuals, but enough spatial proximity to be 
plausibly grasped as a perceptual group, relative to a reasonably 
typical vantage point.  The choice of words in (14) implies a further notion 
of directedness, and a somewhat different relation of the observer's spatial position  
\visavis{} the observed.  While the two sentences might describe the 
same situation, (14) carries an additional implication that the 
footprints suggest some line of direction which extends beyond the speakers' 
immediate line of sight, and can perhaps be followed.  This directionality 
is also implied in (15), but in that case the internal continuity which 
supports this sense of direction is further emphasized.  In this context 
\q{tracks} can mean not just footprints but also, say, bent grass or hairs 
or broken twigs, so that each particular artifact which cumulatively 
constitutes the \q{tracks} may have less precise distinctness than 
a particular footprint.  Moreover, the word \q{tracks} tends overall to 
refer to physically continuous structures, like train tracks, whereas 
\q{trail} can be used in more metaphorical senses and with greater 
sense of internal pauses and separations (a dective \i{on the trail} of a 
fugitive; a scientist \i{on the trail} of a discovery).  Being 
\q{on track} connotes making steady progress toward some goal.  
Scientists at CERN were \i{on the trail} of the Higgs Boson; it would 
have been quite optimistic to say they were \i{on track} to find it.  
Finally it was revealed in the \i{tracks} of secondary particles whose 
movements were recorded in the LHC (Large Hadron Collider) bubble chamber.  
Those were literally streaky lines on a photograph, thought to 
show the actual path which particles took, to \i{track} the particles, 
whereas the \i{trail} of an airplane or comet is a less clearly localized 
after-the-fact indication of their path across the sky.  
}
\p{So (13)-(15) show progressively less attention paid to internal 
constituents and more to the nature and direction of a whole, 
even if some scenarios tolerate all three sentences. 
Each usage would convey slightly different connotations to the 
scene at hand.  The point of this comparison is that individuation 
through concepts depends on properly connoting the patterns of 
continuity and discontinuity which define individuals' coherence 
and separateness (as well as that of sets of individuals, when 
concepts are used to profile collections).  A concept implies a 
schema both for individuals' internal structuration, and for 
its suitability as being conceptually and perceptually isolated 
from its surroundings.  The function of concepts is, typically, 
to \i{set the stage} for further characterization of their tokens, 
first by individuating them as points of attention, and then 
by implying the presence of \q{scripts} or \q{schema} leading to 
more precise descriptions.  Both of these roles present challenges 
for \i{formal} semantics, if this means defining rigorous of 
quasi-mathematical theories of how concepts select a \i{set} of 
instances, or of how each instance \i{instantiates} a concept.  
The act of linking a concept to an instance is a complex 
cognitive/semantic event, one which occurs against the backdrop of a 
mental and often dialogic context, and which should be theorized 
as one step in an evolving process.  A concept is an opening 
onto a more detailed (process of) characterization.  To say that 
something is an instance of some concept is not a simple 
act of classification, then, analogous to declaring a variable 
in a programming language to be of some data type.  
}
\p{Of particular importance, I believe, is the notion that conceptualization 
(if we define this as concept-to-token identification) is only one 
stage in an extended process.  This means that no one single 
concept provides a definitive account of an identified instance, 
even a provisional or imprecise account.  Obviously concepts 
operate at some level of coarse-graining and can be combined 
together for greater precision.  However, if we fail to 
theorize conceptualization as \i{process}, we can end up with a 
view of concepts as imprecise but, \i{at some degree of detail}, 
self-contained mental \q{pictures} which cover some range of 
possible cases.  For each concept, there would correspond 
(on this account) a \q{rough} sketch of how a token of the 
concept is, insofar as it does bear the concept.  The degree 
of (im)precision in these \q{sketches} would depend on how 
general or specific is the concept itself: scientific terms, 
for example, are more specific than everyday words.  
But each concept would connote a family of instances whose 
features are more clearly or more vaguely aligned with a 
set of indicators.  Each house is vaguely aligned with a 
\q{sketch} of some locale with a yard, a front door, interior 
rooms, and so forth; each restaurant is aligned with a 
general notion of a place to sit at a table and eat prepared food, 
and so forth.  The concept leaves particulars imprecise 
(what kind of food, what kind of yard) but presents them at a 
level of vagueness in which they match all examples.  A 
concept, in other words, finds a suitable mix of vagueness 
and resemblance \mdash{} it abstracts from particular details 
enough that all tokens can be considered to resemble each other.
}
\p{While my analysis has likewise argued for \i{characterizing features}, 
where I differ from this kind of theory is in emphasizing conceptual 
imprecision as a matter of \i{process} and not of \i{vagueness}, 
or concepts as \i{scripts} rather than \i{sketches}.  It is misleading, 
I believe, to think of a concept as \i{imprecise} in that it seeks 
to achieve a degree of nonspecifity which allows many somewhat mutually 
resembling things to be grouped together.  This notions seems to both 
minimize the potential for using particular concepts to initiate 
more precise further characterization, and also to overstate the 
degree to which concept-tokens resemble each other.  As mental 
tools concepts are thereby weakened on two fronts: they are neither 
precise enough to truly describe their instances, nor general enough 
to group together these instance outside of some \q{resemblance} between 
them.  I would argue, however, that real-world concepts both trend 
toward a degree of generality which is hard to account for in terms 
of token-resemblance, \i{and also} serve within cognitive and linguistic 
episodes in which general concepts provide a framework for precisely 
describing individuals, up to a degree of resolution suited for a 
given thought or dialog.  The concept may not internally \i{imply} 
these details, but it implies a \i{framework} for accumulating them.
}
\p{If we consider the concept of \i{fly}, for example, we note that it 
covers many different cases: the flight of birds, planes, coments, commercial 
passengers, leaves, kites, debris, shrapnel, types of aircraft 
(\q{that model flew during World War II}), carriers (\q{we fly 
Korean Air whenever possible}), their fleet (\q{that company flies the 
youngest planes of any discount carrier}).  All of these somehow 
involve travel through air, so we might say that the role of \q{fly} is to 
invoke a highly stylized or fuzzy sketch of \q{movement above the ground}; 
in other words, to find the point of resemblance among all of these cases.  
The different senses of \i{fly} might then be considered \i{subconcepts}, 
each with a more precise picture, which finds more detailed resemblances 
between its tokens.  By this account, the concept \q{fly} is really a 
loose aggregate of more precise concepts, which manifest as different 
senses of the polysemous English word.  However, such an analysis 
seems to too neatly group these various senses into a conceptual 
hierarchy, as if word-senses can assemble into a taxonomy akin to the 
classification of living species, each sense finding a degree of 
precision and inter-token resemblance relative to its hierarchical position.  
I would argue that different \q{senses} of words or concepts do not 
generally reveal such straightforward taxonomic patterns, at least 
unless (as with biological names for species, genera, etc.) they are 
deliberately constructed to do so.  In normal usage new senses evolve 
gradually and can emerge ad-hoc for some specific language community 
(or \q{micro} community).  For example, the sense of \q{fly} as 
in \q{Commercial Flight} \mdash{} with its typical associations of 
buying a ticket, going to an airport, checking baggage, passing 
security, and so on \mdash{} emerged only gradually from the more general 
sense of flying in an aircraft.  Moreover, we can imagine specific 
situations where new senses emerge \mdash{} for example, executives 
of a multinational company who use the phrase \q{Flying to London} 
to mean a trip to specific offices, thereby mixing (in this specific 
context) the destination with the purpose of the trip. 
}
\p{The fecundity with which these senses arises suggests to me that we should 
not consider each \q{sense} as its own (sub)concept; instead, I would 
consider \i{fly} to be \i{one} concept, not a loose assortment of 
vaguely related concepts.  This concept coverts many different cases, 
but the cases can be compared and characterized according to certain 
overall criteria, such as the \i{cause}, \i{reason}, \i{agent}, and 
\i{destination} of the flight.  Flying \i{debris}, \i{leaves}, and \i{kites}   
all rely on the power of wind, but the former implies a relatively strong gust 
to dislodge solid objects naturally resting on the ground, whereas the latter 
implies a human agent who orients the kite to best gather the wind.  
Flying \i{birds} and \i{planes} are both self-powered, but the latter relies 
on human agency to provide fueal.  The various senses of human flight are 
comparable based on the kind of craft used, the relation between passengers, 
pilots, and whoever owns the craft (contrast hobbyist who owns and flies 
planes, an air force pilot, a passenger service where flying is a commercial 
transaction).  Each of these distinctions provide parameters for comparing 
different examples of flight.  No parameters, except perhaps for the 
most general notion of airborn motion, apply to all cases and kinds of 
flights; but each case involves some mixture of some relevant parameters, 
and each parameter offers a ground of comparison.  For example, only 
commercial flight involves a ticket, but once this parameter is 
\q{activated} we can compare the price of different flights.
}
\p{It is hard to identify isolated \q{subconcepts} within general 
flight because different parameters overlap in different 
ways.  Passenger flight has unique aspects of cost and 
value (and whatever legal and social considerations 
come along with commercial transactions), but it shares 
other parameters with other (but not all) senses: 
relatively specific times and destinations of travel, 
like the flight of birds but unlike that of leaves or debris; 
the distinction between the subject and the agent of flight, 
a \i{person} ia flying but by virtue of being on an aircraft, 
so the subject/agent relation becomes a conceptual parameter, 
which is also present in some other contexts, like cargo.  
For an analogous example, we might think to subdivide the 
concept of \q{restautant} into more specific cases \mdash{} after 
all, merely expressing the desire to visit a restaurant
in general, or searching for one, can cover a wide spectrum 
much of which may not actually reflect the speaker's/searcher's 
interest.  Someone may want a formal dining experience, a casual 
spot for family dinner, a healthy and inexpensive lunch, a 
quick meal, or a place to have a snack and go online.  Certain 
lexemes cover some part of these more specific spectra 
\mdash{} bistro, steak house, diner, coffee shop, cafeteria \mdash{} yet none 
of these compete with \q{restaurant}'s dominance of the relevant semantic 
territory: one will often use the more general word even when one 
of the more specific ones apply.  I believe this can be explained, 
in part, because the more general concept can be narrowed in 
several ways at once, thereby providing a more effective point 
of origin for a specified description; providing more 
\q{avenues} for elaboration.  For example, a \q{Five Dishes and Soup} 
spot in a Chinatown might be considered a Chinese restaurant and a 
cafeteria, but there does not appear to be an established 
usage \q{Chinese Cafeteria}.  Even if \q{cafeteria} might be 
plausibly defined as any restaurant without table service, it 
seems to connote further details about the kind and pricing of the food.  
Instead of specifiers offering a clear-cut segmentation of the 
\q{restaurant} spectrum, English speakers seem to prefer the more 
general usage and to provide further specification by an unfolding, 
dialogic, and open-ended process, relying on communicative cues to 
converge on an image of the \q{kind of place} some restaurant is, 
rather than trusting some narrower lexeme (like \i{cafeteria} or 
\i{diner}) to impart these cues via convention.  Arguably, 
in some specific circumstances, a narrower lexeme has indeed 
achieved something like canonical status (\i{pizzeria}, \i{coffee bar}); 
so a speaker using those terms, amongst a particular language community, 
shows confidence that the allusions embedded in them convey a 
sufficiently precise picture of the \q{kind of place} being mentioned.  
But this specificity stands in semantic relation to the more general term, 
even if the latter is replaced on some occasions: the ubiquity of the 
word \q{restaurant} adds an extra dimension to a speaker's choice on 
some occasion to instead say \i{pizzeria} or \i{coffee bar}.  The concept 
\q{flight}, I would argue, has a similar semantic ubiquity; it 
encompasses a spectrum of cases so broadly that the occasional 
choice of a different word (\q{I jetted to London}; \q{The birds migrated 
south}) carries the semantics not only of the word used but the dominant 
one \i{not} used. 
}
\p{What this analysis suggests is that real-world language does not necessarily 
show a tendency to semantic specificity, to a preference for more specific 
usages and a relegation of more general terms to cases of deliberate 
imprecision or abstraction.  If a language community had an instinctive 
drive toward semantic precision, then dominant but imprecise words like 
\q{restaurant} would tend to gradually be replaced in many cases by 
narrower concepts, and to be retained mostly when speakers deliberately 
intend to avoid any narrower connotations.  However, this does not 
appear to be the way language in general evolves.  I believe that, 
to an important degree, this fact can be explained by appeal to 
conceptual roles: the point of a concept is not to condence as 
precise a picture of its tokens into as concise a semantic unit as 
possible, but rather to initiate a further descriptive process 
as efficiently as possible.  If I had one word to describe a 
\q{Five Dishes and Soup}, I would probably prefer \i{cafeteria} to 
\i{restaurant}; but in normal dialog, that choice would probably 
inhibit rather than facilitate further elaboration (a qualifier 
along the lines of \q{it's \i{like a} cafeteria} would immediately 
walk back from the concept-attribution, opening a space for me to 
refine the description).  Insofar as concept-attribution is only 
one part of an evolving process, computational models of 
real-world concepts (like restaurants or air carriers or houses) 
should not design associated data-types with the intention that 
their corresponding tokens have some \q{resemblance} to or 
\q{substitute} for their real-world counterparts.  A data-model 
which simulates a real-world object or system does not 
\i{iconify} its referent, analogous to a photograph or a 3D model.  
Instead, it initiates a \i{descriptive process} analogous to 
how concept-attribution in language stimulation a subsequent discourse.    
}
