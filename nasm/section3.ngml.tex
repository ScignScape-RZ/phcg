\section{Multiscientific Objects}
\p{A statue and the small piece that has fallen off of 
it is at best a \q{trivial} mereological sum because 
it is not a \q{meaningful} whole \mdash{} it is not an 
anarchaeological object, say, while the statue 
now missing the piece \i{is} an 
anarchaeological whol.  Similarly, $Me$ after 
a nail-clip is a biological object, 
while $Triv$ ($Me$ plues the piece of nail) 
is not a biological whole. 
}
\p{Also, the (current) statue is additionally 
a minerological whole in the sense that clay minerology 
affects its properties (e.g., its fragility).  
The sum of the statue and the piece is probably 
not an analogous minerological whole (it is 
not a single clay object).  Parallelwise, 
$Triv$ is certainly a \i{psychological} object 
as $Me$ is; I don't feel sensations from 
the ssnipped-nail part, see it as myself, 
and so forth.
}
\p{A clear case for mereological integrity or 
nontriviality seems to be the convergence of 
multiple sciences: the sculpture is 
\i{archaeological} and \i{minerological}; 
$Me$ is biological and psychological. 
}
\p{}
