\section{Concepts' Descriptive Spaces and Functional Schema}
\p{A concept-attribution, I have argued, does not primarily serve to 
suggest the nature of some token by aligning it with a vague 
schema, a fuzzy picture it \q{resembles}.  Instead, saying that some 
instance bears some concept, like \i{restaurant}, intiates a 
process or \q{script} in which further characterizations 
are available.  The attribution does not just imply a 
\q{bundle of properties} which can be enumerated, like in a 
relational database or \q{facts at a glance} \mdash{} here is a 
restaurant; here is its address, kind of food, hours, 
coded numbers for price and quality, etc.  The tabular 
form of a database relation can be considered one very 
specific kind of \q{script}, in which description 
proceeds just by listing a set of parameters and  
values.  Typically, however, characterization is more 
unfolding and \q{branching}: does a restaurant have 
table service?  If so, how polite/professional/knowledgable 
are the wait staff?  Does it have a wine list?  If so, 
how thoughtful/inexpensive/well-matched are its selections 
relative to the menu?  And so forth.
}
\p{Taking departure from Formal Concept 
Analysis, we can say that a single concept does not have a 
single \q{frame}; a single set of properties (or 
even parameters, property-dimensions which take different 
values for different instances, like the averge price 
of a restaurant entr\'ee) which collectively characterize 
all tokens.  Instead, concepts associate with \i{description spaces} in 
which different parameters are \q{packaged} to be activated collectively, or 
not \mdash{} a table-service or wine-list package is appropriate 
to some, but not all, restaurants.  In a characterizational 
process, these packages are sequentially activated and 
provide particular details, but the precise collection of 
details will vary from case to case.  In other words, for a 
general concept like \i{restaurant}, there is no one 
set of properties or parameters which fully characterizes 
all tokens.  There may be general parameters which apply 
everywhere (like a restaurant's address or opening hours), 
but these tend to offer only the beginnings of a useful description. 
}
\p{Aside from the semantic subtlety of how a single concept 
gives rise to a spectrum of different \q{description 
packages}, we also need to honor the ontological 
complexity of how the properties attributed to concept-instances 
vary across these packages.  Real world concept-tokens tend 
to be ontologically \q{complex}, which is to say 
that there is no single fashion in which they 
\q{exist}.  A restaurant, for example, is a physical building and 
location, a collection of people (social institution), and a 
collection of dishes served.  These various facets of a restaurant's 
existence belong to different ontological \q{registers}.  
A restaurant which moves to a new location but 
keeps more or less the same staff, name, and menu 
might be considered \q{the same} restaurant.  So too might a 
new operation which opens in the prior location \mdash{} 
it is permissable to say \q{that restaurant is Chinese now}.  
We can say that a restaurant is \i{unionized}, \i{vegetarian}, 
and \i{remodeled}, asserting properties respectively of 
its staff, menu, and building, and by 
extension, metonymmically, of the restaurant itself 
\mdash{} like calling a book with yellow cover (and white pages) 
\mdash{} \q{yellow}.   
}
\p{Certainly the staff, building, and menu are \q{parts} of a restaurant.   
Yet \q{sum of parts} here is not simplistic set-theoretical union; 
it is a more complex functional interrelationship.  A restaurant 
joins a set of people (staff) with a set of places and rooms 
(dining areas, kitchen areas, etc.) according to schema involving 
how the physical space enables the staff to carry out typical 
operations, and how the combination of physical and operational 
design (the procedures which employees understand and consent to enact) 
allow the restaurant to carry out its expected roles 
(taking orders, preparing food, bringing them to patrons).  
We expect the restaurant to be a \q{site} where certain kinds 
of things can happen, so that certain kinds of activities are 
afforded to us, and it is the nature of these affordances 
(insofar as they have gustatory, interpersonal, economic, 
and spatial aspects) 
\mdash{} to cross ontological \q{registers}.  
}
\subsection{Mereology and Referential Vagueness}
\p{The ontological complexity of \i{objects} (like 
\i{restaurants}) then complicates semantics and 
reference (like \i{a restaurant} as a designated 
signified).  Which \q{facet} of the 
restaurant is a referential target in different 
discursive contexts?  These questions complicate 
attemps to map units of language to things 
\q{in the world}.
}
\p{Granted, we should assess any \q{formal} semantics against 
a presumptive distinction of \i{semantics} and \i{pragmatics}, 
such that linguistic nuances which seem to violate referential 
rigidity are artifacts of actual \i{usage}, as separate from formal 
\q{meaning}.  For example, the apparent ambiguity of \i{London} 
as either the larger city of the financial district, and 
\i{Manhattan} as either an island or a borough, is a product of 
how the respective language-communities reuse a single name for 
multiple, slightly different but overlapping purposes.
}
\p{To be precise, the \q{City of London} is 
technically a small district 
in what people informally call \q{London}; and the \i{borough 
of Manhattan} includes Manhattan island but also several 
other islands and the small Marble Hill neighborhood on the mainland.  
If we accept a \q{model-theoretic} paradigm 
where formal semantics takes a collection of symbols (names, lexemes) and 
maps them to a collection of empirical things (like geospatial regions), 
we then ask, which regions are designated by the respective names?
}
\p{Conventional usage in the two cities suggests that this question is 
not settled; New Yorkers for example will use \q{Manhattan} for 
either the borough or the island (or both) depending on context.  
There seems to be no nonarbitrary reason to name one or the other 
as the \q{real} (referent of) Manhattan.\footnote{Maps on the westbound 
$\textcircled{F}$ train show Roosevelt Island as the first stop in 
Manhattan (borough), but riders would not necessarily consider 
this stop (across the river from Midtown) to be \q{in} Manhattan, 
and Real Estate listings might be considered misleading if they 
classified Roosevelt Island addresses as in Manhattan.
}
}
\p{Distinguishing \i{semantics} from \i{pragmatics}, we might 
argue that there is nonetheless an underlying semantic 
system here where the two locations actually are differentiated, 
so the name \q{Manhattan} actually has two senses.  Speakers 
understand from context which sense is intended \mdash{} this 
is the pragmatic layer \mdash{} but \i{each} sense, if 
we can treat it as a distinct lexical 
unit, may still function in accord with (say) 
model-theoretic semantics.  On this account, 
the name \q{Manhattan} reflects a pragmatic, communicational 
speech-gesture which singles out one or another formal-semantic 
meanings depending on context; underlying the surface pragmatics 
of everyday usage there is a formal semantic framework, where the 
two \q{Manhattan}s are distinguished.  
}
\p{The problem for this account is that most pragmatic usage does not 
make this distinction \mdash{} and the reason is 
not because one sense or another can be inferred 
from context, but because it is not relevant.  
This can be called an example of \i{vagueness} rather than 
\i{ambiguity}; not that references are ambiguous between 
Manhattan-the-borough and the island, 
but rather they are vague about even positing the distinction.
For example \mdash{} to borrow an 
example due to Mark Heller \mdash{} the assertion \i{Syracuse is north of 
Manhattan} holds despite fine-grained ambiguities in construing the 
designatum of \q{Manhattan} (or of Syracuse, for that matter).  Heller 
does not mention the island/borough confusion, but points out that 
geospatial regions are in some sense \q{fuzzy} as sets of geospatial 
points \mdash{} the precise borders of an island, for example, seem to depend 
on everchanging variations in water levels.  Heller develops a theory 
according to which, considering this imprecision, the referents of 
proper names should be considered \q{conventional objects} which 
overlap to large extent with empirical objects or regions.  For 
another example, he runs a Sculpture/Clay discussion: 
how a sculpture may begin as a 
block of clay, be completed and then exhibited as a work of art, 
and (let's imagine) at some point destroyed; the block of clay 
is there all along, but only in the middle phase (when on display) 
do we consider it a statue.  Therefore (for Heller) the statue is a 
\q{conventional object} distinct from the block but 
overlapping with it for much of its history.
}
\p{In \cite{JubienOntology} Michael Jubien counters this analysis 
with a reference theory based on \i{properties}; specifically, 
corresponding to conventionalized concepts (like Manhattan or 
a sculpture) there is a property of \i{being} that thing 
(e.g., \i{being} Manhattan), and linguistic expressions 
(like proper names) semantically designate those properties.  
\q{There is no contingent
identity.... There are no conventional objects.
There are just things and their properties} 
\cite[p. 38]{JubienOntology}.
Depending on contexts and speakers' particular cognitive orientations, 
different specific entities can be conceived as 
instantiating the respective properties \mdash{} adopting this notion to 
my examples, both the island and the borough can plausibly be 
supposed as \i{being Manhattan} for the sake of interpreting a 
particular speech-act.  Better, since either island or borough 
may plausible be construed as \i{being Manhattan}, most 
uses need not commit to one interpretation or another.  
In other words, examples like the Syracuse/Manhattan directions 
would, on Jubien's account, be interpreted roughly as asserting 
that there is some geospatial territory (its precise 
borders not necessarily characterized) with the property of 
\i{being Syracuse}, another with the property of \i{being Manhattan}, 
and the former is north of the latter.  
}
\p{Suppose someone 
actually asserts, or thinks to themselves about how, \q{Syracuse is north of 
Manhattan}.  They may be driving down from Syracuse University to 
join a protest at Columbia University (to imagine one possible scenario).  
Neither of the two geospatial names are involved, in these thoughts or statements, 
in such a way that their precise geospatial extents are relevant.  It is entirely 
possible that by \q{Syracuse} the speaker does not specifically mean that city per se 
but rather its immediate metro area \mdash{} one could certainly describe a drive from 
Skaneateles to Columbia as 
\q{Syracuse to Manhattan}.  The place names are understood \q{functionally}; 
that is, we conceive places not as precise sets of points on the ground, or on 
maps, but as functional ecosystems, \i{places} constituted conceptually 
by dynamic activities they afford: entering into and out of 
(which means, for \i{geo}spatial regions, going to or leaving them 
by car, train, etc.), targeting a specific destination within them,
associating specific such places with addresses, etc.  In this functional 
sense the distinction between \i{Manhattan island} and 
\i{borough of Manhattan} is not often relevant, because many 
functional conceptualizations \mdash{} planning a trip, anticipating 
taking the train to Penn Station and catching the right subway, and so forth 
\mdash{} unfold mentally and operationally with no regard to the distinction.  
Depending on context, it is entirely possible that in someone's mind 
the locale with the \i{property of being Syracuse} is roughly the 
metro area, and the locale which instantiates the property 
\i{being Manhattan} is the general stretch of land extending outward 
from Penn Station or some other familiar landmark or neighborhood.  
Only in specific circumstances \mdash{} like visiting a friend's apartment 
on Roosevelt Island, or voting for the proper elected officials 
while living in Marble Hill \mdash{} does the island/borough distinction 
become relevant.  As a result, this distinction is only 
\i{semantically} relevant for the meaning of the lexeme 
\q{Manhattan} insofar as it is \i{functionally} relevant 
for the lifeworld activity relative to which the name is used.
}
\p{Conventional \q{model-theoretic} or 
\q{truth-theoretic} semantics has to assume that, 
separate and apart from functional associations 
between word-usesand environments, 
there is a purely formal function definable to map 
specific lexemes to specific entities 
of a corresponding kind \mdash{} place names 
to geospatial regions, personal names to human 
bodies, species names to sets of living things, etc.  
Such an assumption \mdash{} that the world around us 
provides a \q{domain of reference} for a formal 
semantic, a \q{set} of \q{things} which can be 
singled out semantically or referentially \mdash{} 
this perspective is questionable on ontological 
grounds, let alone invoking subtleties 
of language.  Our macroscopic, 
processual lifeworld does not readily partition into 
a neat agglomeration of self-identical things, like 
molecules into atoms.
}
\p{It is almost impossible 
to take any real-world concept or referent and 
formulate a precise definition (or \q{nomination}) 
of it as a wholly self-contained object or place.  
A restaurant is not just a building; 
a nation is not just a landmass; 
even a building is not just a mereological sum of 
architectural features.  
The \q{identity} of almost all concept-instances 
is dependent on how they function within a larger 
totality, whose activities include the people 
having occasion to speak about and conceive 
them through the structures of language.  
}
\p{In \i{Ontology, Modality, and the Fallacy of 
Refererence} Jubien's analysis centers on the 
apparent mereological ambiguity of referenctial 
semantics: how mereological composition seems functionally 
orthogonal to referring acts.  Signifying designations 
of \i{Manhattan} seem unaffected by the fuzziness 
surrounding Manhattan's borders (both because islands have 
imprecise extent and because there are several distinct but overlapping 
senses of Manhattan).  Or, as Jubien analyzes in the 
context of modal logic, referring to 
\i{this statue} does not seem to be 
actually referring to a \i{lump of clay}.  As Cotnoir 
points out, modal contrasts 
(like \i{the clay could be flattened to a disk}) 
have inspired consideration of 
Statue/Clay pairs as \q{mutual proper parts}.  
Jubien's theory, on the other hand, proceeds 
more on an underlying distinction between 
\q{the \sq{is} of predication rather than the \sq{is} 
of identity} (p. 34).  It is not that the 
sculpture \i{is} the lump of clay; rather, the 
lump of clay is a physical mass bearing 
the property \i{being the statue}.
}
\p{As Jubien puts it in \i{Possibility}:
\begin{quote}While I believe that typical DR [Direct Reference] 
theorists would agree at the outset that the bearer is a 
physical object, I also believe they have not noticed the 
present sort of difficulty.  A typical theorist ... would initially think that loading Daffy 
\mdash{} a statue \mdash{} into a proposition was nothing more nor less than 
loading a certain physical object into the proposition.  
But the present difficulty might lead such a theorist to think anew and 
seize upon the idea that although it's a statue that gets loaded in, 
it is somehow not just a physical object.  This might be thought to provide 
a way to avoid the great-divide problem of the object's not being a 
statue essentially, since one would think that the statue \mdash{} 
whatever exactly it is \mdash{} is surely essentially a statue.  
Thus a DR theorist might seek a different way of understanding 
the proposition that Daffy might have been made of different 
matter, one that avoids the absurd conclusion that 
I claimed derails DR. But the options are not inviting. 
They would be views according to which a 
statue \mdash{} the loaded entity \mdash{} (1) is not a physical object 
at all, (2) is only partly physical, or (3) is a physical 
object of a kind somehow different from a simple 
mereological sum.... \cite[p. 121]{JubienPossibility}.
\end{quote}  The \q{present sort of difficulty} is the Sculpture/Clay 
\q{overlap} that inspires things like \q{conventional 
objects} or mutual-proper-parthood.  Two years 
later Jubien would suggest his \q{property instantiation} 
model in \cite{JubientOntology}, wherein the 
clay becomes an instantiator of \i{being the statue}, 
so the relevant relation is not parthood but 
property-substrate.  Mereological sums are, on Jubien's later 
theory, material aggregates that participate in 
conceptual or complex/functional organization 
as \i{substrata} instantiating \i{properties}, 
which in turn are the actual referents of 
conceptual intentions \cite[p. 12]{JubienOntology}. 
}
\p{The consequence for conventional mereological analysis 
lies in how mereology is supposed to 
underlie semantics: in particular, the idea that 
meanings can be unpacked by mapping words 
to objcts (or, say, geospatial regions) and 
higher-scale expressions to propositions.  
So \i{this sculpture is Greek} would assert a 
proposition whose intention can be recovered by 
substituting some object for \q{this sculpture}, 
and which should be deemed true if the 
corresponding worldly object did indeed 
originate in Greece.  And \i{Syracuse is 
south of Manhattan} has concrete meaning 
by substituting geographical regions for 
the corresponding words.  The underlying 
dynamic in both cases is that things and 
extents \q{in the world} are to be 
substituted for units \q{in language}, 
implying that linguistic complexes are composed 
of sites for eventual substitutions, 
laying out (at the sentence level) sentences' 
\q{truth conditions}.
}
\p{The mereological 
complexities here then arise from the indeterminacy 
of a mereological whole which 
can be the target of reference distinguished 
from other wholes.  Since Manhattan is \q{fuzzy}, 
\i{which} whole should substitute for \i{Manhattan} 
when stating truth conditions for sentences 
about Manhattan?  Which whole is 
\i{this statue} when the object may erode over 
time, losing its boundary precision?  
}
\p{I will return to these issues in a later section.  At this point, 
however, I want to point out 
that these mereological controversies lie 
not within cognitive scheman but in the, 
perhaps it can be called, \q{cognitive/extramental} 
boundary or intrface.  The problem 
of \q{selecting} one whole from among 
mereologically similar candidates is not an 
analysis of the parts of our \i{conceptualization} 
of Manhattan or the statue, but rather trying to 
relate the conceptualized thing \i{as cognized} from 
the extramental object whose properties \mdash{} at least 
on many theories \mdash{} ground the accuracy and 
fallibility of cognitive attitudes.
}
