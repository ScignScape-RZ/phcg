\section{Mereology and Models}
\p{The upshot of my argument last section was 
that the realm of mereologuical theory should 
be divided into \i{cognitive} and \i{multiscientific}.  
Criteria of mereological non-triviality are 
derived in the first case from \i{cognitive frames}, 
and in the second case from an \q{$R$-patchwork}: 
a patchwork of scientific theories delimiting the 
extent of different scientific registers and the 
dynamics and autopoesis sustaining the 
integrtity of (in various domains) 
non-trivial wholes.  Moreover, there is a structural 
resonance and interpenetration between 
cognitive frames and the \q{$R$-patchwork}.
}
\p{Whether from the perspective of cognitive frames or the 
\q{extramental} perspective of the $R$-patchwork, 
mereological theories can be reasonably elaborated 
in parallel to theories of the cognitive/ontological 
status of wholes participating in \xhppy{} style 
parthood assertions.  Any formalization of 
parthood relations can assume that all wholes have 
some integral unity that can be described on cognitive 
or scientific grounds (or really multiscientific, because 
any macroscopic object has properties at the 
intersection of different sciencs, like physics 
and biology and/or chemistry).  
}
\p{Of course, we can investigate mereologies as abstract, 
axiomatic logical systems.  I contend however that 
such essentially mathematical theories have only 
limited applicability to philosophy.  
In particular, what would be \i{models} of these 
theories in the logical sense?  Imagine a collection 
of \q{voxels} \mdash{} three-dimensional indivisible 
cubes with two states (\i{empty} or \i{full}).  
Two (orthogonally) adjacent voxels can be said 
to form a \q{block}, and in general a set of 
connected voxels to form a \q{connected block}, say.  
The system of connected blocks then forms a 
classical mereological model where proper parthood 
is voxel set inclusion.  Let's call this a 
\q{Voxel Mereology}.
}
\p{Some picture like this may feel like a natural bridge 
between logical-axiomatic mereological theories and 
philosophical analysis.  The rason is that 
a Voxel Mereology undeniably instantiates Classical 
Mereology and also fits our intuitive picture 
of physical reality.  In our folk phsyics, we experience 
objects as resisting force and sustaining shape as if they 
were composed of hard little bricks.  And we can match this 
intuitive picture to actual science by imagining voxels 
sized to the small units of quantifying possible, 
e.g., the Planck scale.  Science suggests that there is indeed 
a scale of reality finer than which there is no 
possible scientific measurement, information, or 
discrimination.
}
\p{Such quantum analogies however are of limited value in 
scientifically validating the \q{voxel} picture.  
Space itself at the Planck scale does not fit the ambient 
Euclidean topology that voxels presuppose; nor is there 
a precise quantum distinction between \q{matter} and \q{emptiness}.
So we can't just assume our voxel picture plugs 
in to quantum physics.  Instead, Voxel Mereology 
is at best a brdige \mdash{} it encapsulates how we \i{experience} 
physical reality while also offering a rough 
conceptual figuration of physical principles, like the 
idea of a smallest measurable unit of space.  Rather than as 
an analogy to the quantum realm, Voxel Mereology is 
perhaps most realistic as describing in 
approximate detail the Emergent Properties of ordinary macroscopic 
physical objects.  Compact solids do tend to behave 
as if composed of hard \q{bricks} on a scale fine enough to 
mold to their convex geometry.
}
\p{Having said that, \i{parthood} within macroscopic 
physical objects does not really work like little 
lego pieces being lifted apart.  Usually 
parthood reflects some functional or 
manufactured assemblage; the knob on a drawer, 
the cap on a bottle; the twig on a branch, etc.  
Parts can of course break off with no apparent 
functional integrity: if the bottom of a table leg 
spliters away, or a small tree branch is blown 
apart from the tree, the surface where the detachment 
occurs will presumably evince a jagged, apparently 
random edge.  Setting the \q{voxel} scale small enough can 
still model the geometry of the parts, but that seems 
besides the point.  In reality, patterns of breaks are not 
completely random; they reflect stress points and accumulated 
forces acting upon the larger object, and 
the geometry of the detached part reflects the structures 
of thse forces.  Like the patterns formed by water spilling 
over a surface, complexity and semi-randomness still 
has some mathematically tractable background.
}
\p{In short, a \q{voxel} model has limited usefulness even as 
an approximate picture of solid wholes.  While connected voxel-blocks 
are not unrealistic discrete approximations of three-dimnsional 
shaops, voxels are less appropriate for representing 
part-whole relationships: a model of parthood (in the 
context of solid physical objects) is better organized around 
functional or mechanical constructs.  There are parts induced by 
the accumulative processes which form wholes, like 
twigs on a branch.  There are also parts deriving from 
stressors, fault-tolerance, and other material qualities 
of solid substance in the sense of materials science, 
where lines of force and/or vulerability can dispose pieces 
of a whole to become semi-detached, giving them 
some definitional autonomy relative to the whole.  But 
properly modeling these material 
tendencies presumably involves mathematical 
formations like vector fields, where the voxel 
dicretization is at best tangential.  Even if connected 
voxel-blocks can model induced parts as well as wholes, it is not 
clear what physical reality such a model 
would capture; it is more like a predefined 
conceptual scheme imposed on a mostly unrelated physical 
situation. 
}
\p{So \q{Voxel Mereology}, as a straightforward model 
of an axiomatic Classical Mereology, has at least dubious 
merit as a model of physical states of affairs.  
We could try to refine the model in a more scientifically 
rigorous way \mdash{} in the direction of mereotopology, 
for example, allowing $n$-dimensional generalizations 
of voxels and/or infinite subdivision.  So we could 
have a system whose elements have varying dimensions 
(perhaps to a maximum 
of four, representing temporal change) and smaller-dimensional 
objects can be part of larger-dimensional ones.  
Moreover, any objects of non-zero dimension can 
potentially have proper parts; there are no \q{atoms}.  
This picture may be more appropriate for 
topological or differential-geometric 
representations of surfaces and manifolds as they 
would apply to science.  To have a useful notion 
of parthood we may want to restrict partonomic 
assertions to parts which are not too \q{bizarre} 
\mdash{} perhaps to connected open or closed manifolds, 
excluding \q{fractal} shapes of fractional dimension, 
or infinitely scattered subsets.  By way of 
naming a juxtaposition with \q{Voxel Mereology}, 
I'll call this theory \q{Manifold Merology}.
}
\p{Certainly topological spaces can be a foundation for 
Classical Mereology models.  As with voxels, though, 
we need to consider how applicable these models 
are to empirical phenomena.  The topology of 
manifolds may be a better (e.g., 
non-discrete) approximation of material objects' nature; 
for instance solids' outer bounday is reasonably seen 
as a 2-manifold immersed in 3-space.  There still is 
not a perfect isomorphism between material parts and 
submanifolds, however.  Granted, to a reasonable 
approximation every recognizable part of an 
object has a corresponding submanifold.  For sake of 
discussion, suppose we equate all wholes and parts to 
their 2-manifold surface (and assume 
parts are three-dimensional submanifolds of 
three-dimensional wholes).  Assume the 
surface of a part either overlaps with the 
surface of a whole or else is entirely inside 
the whole (that is, assume we are using a 
version of the Region Connected Calculus).  
The system of parts in relation to wholes is 
then an example of a system of submanifolds 
in relation to manifolds.   However, the system of 
manifolds has some properties that 
may not be desired in a mereological thory.  
For instance, for a submanifold properly contained in a 
larger manifold, a sufficiently small deformation of 
the smaller manifold yields a new manifold 
also contained in the larger.
}
\p{The mereological equivalent to this property is 
that if \xhppy{}, there is an infinite space of 
topological deformations of $y$ yielding $y'$s 
that, without some extra criteria, will also be in $x$.  
We can of course stipulate that only some submanifolds 
correspond to actual parts, but then the formalization of 
submanifolds is only one dimension in the 
description of parthood.  Whereas Voxel Mereology was a 
flawed model of (solid, material) objects because its discrete 
architecture was at best a rather arbitrary 
structure superimposed on physical forms, 
Manifold Mereology has limits of applicability 
for essentially the opposite reason: mathematical 
continuity allows for infinitessimal modifications 
of geometric forms in ways that have no physical meaning.
}
\p{Both the Voxel and the Manifold pictures are only 
indirectly representations of \q{solid matter}; 
they are actually models of geometric extension, 
figured either discretely or continuously.  
The \q{voxel} as a binary empty-or-filled spatial region 
is an idealization; but physical surfaces as 2-manifolds 
is an idealization also.  Physical solids do not have a 
crisp boundary where their interior ends and surroundings 
space begins, such that exterior surfaces can be mathematically 
embodied in 2-manifolds.  This may be a reasonable 
approximation for some solids, just as the voxel 
\q{little brick} picture is somtimes mostly 
appropriate.  But neither are physically oriented 
theories of how matter actually extends in and relates to 
surrounding space.  Instead, they are 
mathematical formulations of extension 
per se, as geometric primitive.  This may yield 
valuably mathematics \mdash{} consider differential 
geometry as a theory of physical procsses 
on objects' surface, or even tools like 
NURBS (Non-Uniform Rational B-Spline, a feature of 
Computer-Aided Design and Computer Graphics) 
for computarionally modeling solid geometry.  But 
mathematical formalizations always have 
at best an indirect relation to physical phenomena. 
}
\p{This may suggest an intrinsic limit to any 
axiomatic mereology invoked as part of a philosophical 
treatment of parts and wholes: any \i{model} of 
such a system has to be interpreted as a model 
(to some approximation) of physical wholes.  
But the most natural mereological models are likely 
to be mathematical systems, like 
discrete geometries or topologies \mdash{} perhaps 
with certain mereologically-inspired 
restrictions (e.g. distinguishing 
connected submanifolds from submanifolds in 
general) \mdash{} or perhaps algebraic models, like 
lattices.  In any case we then need a further 
account of the relation between the 
mathematical idealization and 
physical reality.  So the mereological 
system is removed at two steps from 
physical phenomena, which means the 
philosophical analysis has to preovide at 
least two \q{bridge} theories alongside 
the formal statement of the mereological 
system, to explain why the 
proposed axiomatization is a 
philosophical subject-matter.
}
\p{The tenor of this line of analysis is not 
rstricted to mereology; I would 
make similar arguments about the theory of 
concepts, say, or \q{truth-theoretic} semantics.  
Given an axiomatic system like Formal Concept Analysis, 
we need to explain how models of that theory actually 
relate to conceptualization as a phenomenon of 
human intellect.  Formal Concept Analysis engages a 
structural morphology of concepts 
not very different from the building 
blocks of a mereology; intensional 
\q{features} and extensional \q{examples} jointly 
characterize concepts, which can then 
be merged together or split apart.  For example, 
merging the feature-set of two concepts, and then 
selecting all examples which have all (or a sufficient 
number of) the combined features, represents an 
\q{addition} between concepts (a variation is to 
consider examples with \i{common} features of 
two concepts: so a potential 
milk-plus-almond-milk concept would have as examples any 
liquid exhibiting the features milk and 
almond milk have in common).
}
\p{Formal Concept Analysis may provide an interesting systematization 
of discriminitive or evolutionary factors shaping 
how humans intuit and modify concepts.  There are other 
formal concept models as well, such as prototype-based 
models, or the idea of mapping concepts to rgions in 
qualitative spaces, featured in Conceptual Space Theory.  
These various theories can be combined; for example, 
Formal Concept Analysis can be extended with prototype criteria 
such that features and examples are weighted in terms 
of how essential they are to the relevant 
concept (as features) or how characteristic they are 
(as examples).  The most representative prototypes 
of a concept are (on such a model) ones 
that strongly reflect the features most 
indicative of the concept.  These criteria 
of weights and prototypes can be adjoined to 
the feature/example matrix of Formal Concept Analysis, 
yielding an axiomatic system which may model, to some 
approximation, human conceptual activity.
}
\p{When we start to compare these formal models with actual 
conceptual patterns, however \mdash{} for instance 
in Cognitive Semantics or Cognitive Grammar \mdash{} 
the models start to seem unnaturally simplistic 
and superimposed.  It's not that structural 
analysis of cognitive representations is impossible; 
but rather that each structure tends to have its 
own \i{sui generis} rationale, and trying to 
trace cognitive gestalts to an 
axiomatic foundation starts to feel, at best, 
reductionistic.  
}
\p{Analogously, a \q{truth-theoretic} model of linguistic 
meaning \mdash{} mapping sentences to logical formulae 
or to truth conditions \mdash{} have intuitive appeal but 
also many apparent counter-examples.  A full discussion 
is beyond the scope of this paper, but I'll give some 
of my favorite examples demonstrating why truth-theoretic 
readings are, in my opinion, at best incomplete: 
\begin{sentenceList}\sentenceItem{} \label{itm:sangalong} Everyone sang along to two songs.
\sentenceItem{} \label{itm:performed} Everyone performed two songs.
\sentenceItem{} \label{itm:boroughs} New Yorkers live in one of five boroughs.
\sentenceItem{} \label{itm:commute} New Yorkers often gripe about long commutes.
\sentenceItem{} \label{itm:democratic} New Yorkers reliably vote Democratic.
\sentenceItem{} \label{itm:after} Student after student came out against the tuition hikes.
\sentenceItem{} \label{itm:mass} A critical mass of students came out against the tuition hikes.
\sentenceItem{} \label{itm:tipping} Students' anger about tuition hikes may have reached a tipping point.
\end{sentenceList}
If a truth-theoretic model is accurate, it should be possible 
to capture the approximate signifying content of each of these sentencs 
by expressing them in logical form; or at least show a 
systematic translation of surface linguistic structure 
to truth-making conditions.  I want to argue that every of these 
sentences fail the test.
}
\p{Start with (\ref{itm:sangalong}).  The most natural interpretation 
to my ear \mdash{} i.e., I imagine the kind of scenario 
where (\ref{itm:sangalong}) is most likely to be expressed 
\mdash{} involves some sort of concert which included two 
particularly popular songs, that most people in the audience 
could sing along with.  On that reading, there 
were \i{exactly two songs} and the sentence does not 
commmit to \i{literally} everyone singing them.  
Conversely, (\ref{itm:performed}) strikes me as talking 
about musicians onstage rather than the audience, and 
suggesting that each performer had a two-song 
set; i.e., each performed two 
\i{different} songs.
}
\p{So, I hear a wide-scope reading for \q{two songs} in 
(\ref{itm:sangalong}) but a narrow scope in 
(\ref{itm:performed}).  What I want to emphasize is \i{why} 
we would make that judgment: there is nothing 
in the sentence itself that points toward 
a wider or narrower scope.  Instead this follows from a 
pragmatic construal of the verbs involved: 
to \i{perform} seems to bind the verb to its subject 
more tightly than \i{sing along}.  The situational 
model most appropriate for \i{perform} leans toward 
narrow scope, while the situational 
model for \i{sing along} leans toward wide scope.
}
\p{In (\ref{itm:boroughs}), a reasonable interpretation is that, 
according to the sentence, \i{all} New Yorkers live in 
one of five boroughs.  Indeed, the territory of 
New York is precisely partitioned into five boroughs. 
On the other hand, (\ref{itm:commute}) does not 
appear to commit to referencing literally \i{all} 
New Yorkers; meanwhile it also appears to use the 
phrase \q{New Yorkers} differently than (\ref{itm:boroughs}), 
to mean generically people who work in New York or 
live in New York's metro area.  In the case of 
(\ref{itm:democratic}) we hear \q{New Yorkers} narrowly 
as in (\ref{itm:boroughs}), because voting relates 
a person to their actual place of residence (not 
where they work or their metro area).  However, 
the implicit quantifier in (\ref{itm:democratic}) 
is \i{most}, not \i{all}; it does not fit our 
conceptual picture of \q{voting} to imagine 
one party winning \i{all} votes.  
Again, these contrasts are not produced by sentence 
or phrasalform; they are instead driven by 
lexical peculiarities of words 
like \i{borough}, \i{commute}, and \i{vote}.
}
\p{In the last three sentencs, we can similarly hear an 
implied quantifier to the effect that \i{many} 
students are complaining or unhappy about some tuition 
increases.  But each sentence adds a shading on the generic 
form like \i{many students complained}: in 
(\ref{itm:after}) the speaker tries to connote the 
pervasiveness of students' anger by giving it a temporal 
figure.  She suggests how a temporal recurrence of 
some phenomenon reinforces our sense of its extent; 
not only is it aserted that many students are unhappy, but that 
this fact has come to hear awareness multiple times.  
The verbiage used to describe unhappiness, also 
\mdash{} \i{come out against} \mdash{} carries an extra spatial or 
narrative dimension than a plainer alternative like \q{complained}.   
To \i{come out against} implies a public, maybe even activist 
display of anger.  In (\ref{itm:mass}), a similar figuration 
of \i{many} imposes an interpretive attitude on the 
sentence: to refer to \i{many} as a \i{critical mass} 
implies not just magnitude, but \i{enough} magnitude 
to effectuate something: (\ref{itm:mass}) might be said of 
a case where student protests forced a school 
to cancel planned hikes.  And (\ref{itm:tipping}) is 
similarly implying that some threshold may be 
crossed, without indicating (at least within the 
sentence) what the speaker thinks might happen then.
}
\p{These readings do not dispute that each sentence in 
(\ref{itm:sangalong})-(\ref{itm:tipping}) have a logical 
substance that could be modeled as quantified assertions, 
using the appropriate quantifiers in each 
case (\i{every} performer/audience member; 
\i{all} New Yorkers; \i{most} New Yorkers; 
\i{many} students).  Lexical variation 
(like \q{New Yorker} meaning both a resident of the 
city and the metro area) has to be accounted for, 
as does scope variation (like an \i{everyone ... two songs} 
case, where \q{everyone} as \q{space builder} 
introduces both a global and a local space such that 
scopes must be resolved for the rest of the sentence).  
I grant, however, that an underlying predicate logic can 
be enriched to represent these added scope and 
lexical details.  So I believe that, at least among 
the sentences I've analyzed, we \i{can} construct a 
logical representation which captures 
the intended meaning of the sentencs.
}
\p{However, this by itself does not strike me as legitimating 
a truth-thoretic semantics.  One reason is that the actual 
words chosen provide shades of meaning \mdash{} narrative or figurative 
construals, interpretive connotations, visual imagery 
\mdash{} which add communicative content 
that cannot be readily modeled within logical 
details themselves.  In the \q{many students} examples, 
different formations taking the place of 
\i{many} present different interpretations 
or figurations of the described states of affairs.  
These added details, however, can still be said to 
have a logical structure: for instance 
the \i{critical mass} language implies the speaker thinks 
the sheer number of students complaining \i{caused} something 
to happen.  As such, the interpretive or 
figurative implications of the word-choices can be 
seen as a compact or rhetorically effective way to denote an 
underlying logical structure.  We can, even in these cases, find 
a propositional content; except, the language 
itself uses figurative or connotative devices to actually refer 
us toward that content.
}
\p{My broader claim is that there is no effectively logical 
transformation \i{from} the linguistic 
content as presented \i{to} the constituent elements 
of the relevant, signified propositional content.  
What \i{rules} are we following when we hear 
\q{sing along} as having wide scope and 
\q{perform} as having narrow scope?  Or 
hear the implicit quantifier in (\ref{itm:democratic}) 
and (\ref{itm:commute}) as \i{most}, but in 
(\ref{itm:boroughs}) as \i{all}?
}
\p{I am prepared to admit that a truth-thoretic \i{semantics} 
need not be a truth-theoretic \i{grammar}; that is, 
we can say that sentences have a propositional content 
that semantics should isolate, even if there 
is only an indirect mapping from the surface-level 
linguistic performance to the predicate structure 
which, in many cases, we could see as sentences' 
\i{meaning}.  Linguistic expressions do 
not, in general, structurally recapitulate the form 
of propositional assertions laid out mechanistically.  
Truth-theoretic models could still have semantic validity 
if we say that situational awareness and sensitivity 
to rhetorical nuance allows us to proceed, perhaps 
circuitously and interpretively, from 
language acts to predicate structures.  The predicate logic 
of language is not so baldly apparent that we can 
program computers to understand language, but 
everything that seems not-quite-logical about 
language \mdash{} its reliance on connotation and 
figuration \mdash{} could, we might speculate, 
be cordoned off as an economy of \i{grammar}.  
We speak figuratively because it is grammatically 
more efficient \mdash{} and, partly for this reason, 
rhetorically more persuasive \mdash{} to designate 
propositional content indirectly, rather than speaking 
in surface forms mimicking predicate algebra.     
}
\p{However, my analysis suggests that the 
indirection of the expression-to-proposition 
path is \i{not} just grammatical.  As I argued, 
our disposition to read logical 
formations as one way and not another 
\mdash{} wide scope or narrow, \i{most} or \i{all} 
\mdash{} depends in part on the semantics lexically 
embeded in words like (in these cases) 
\i{borough}, \i{commute}, \i{vote}, or 
phrases like \i{critical mass} and \i{tipping point}.  
Our belief that parsing a sentence includes 
mapping down to a propositional intention 
does not help us trace the semantic effects which different 
word-senses contribute to this process.  In short, 
the cognitive structures which must receive language 
\mdash{} the situational awareness and 
interpretive empathy that tunes us toward 
communicative intent \mdash{} is not only manifest 
in how linguistic form translates to propositional 
signification.  It is also embedded in the lexical 
specificity of word-senses themselves.  Even when the 
semantics of \i{sentences} can be expressd propositionally, 
the semantics of thir constituent words does not 
fit mechanically into the semantics of the whole.
}
\p{Taking these analyses together, I am trying 
to describe cases where philosophers have turned to 
some logical, axiomatizable system to 
compose a model judged for its value as a 
philosophical sketch of some kind: a model 
of parthood as a precis of part/whole relations 
among physical phenomena; a feature/example statistic 
as a Formal Concept space; a prototyped and dimensionalized 
feature-space as a Conceptual Space theory; or a 
construal of semantcs (abstracted from pragmatics and syntax) 
equating the meaning of sentencs to 
their propositional content.  
In each of these cases we have to consider how a model 
for the logical \i{system} could also 
be a \i{philosophical} model; an explanatory tool 
or simplifying idealization that casts philosophical 
light on cognitive or physical phenomena.
}
\p{Philosophical models do not need to be models of 
physical objects or systems direcly; they could encapsulate 
or productively simplify how we perceive or conceptualize 
external reality.  In that case we want to question the 
cognituve place for the \i{logical} model  
\mdash{} the model-of-the-logical-system in 
the model-thoretic sense.  For instance, is actual 
human conceptual process a model of Formal 
Concept Analysis in its guise as an axiomatic 
system whose formmulae can b realized in 
axiomatic models?  Do concepts as cognitive 
entities model those axiomatics?
}
\p{Or if we look for extramental models, to what extent can 
physical objects and systems realize axiomatic structures?  
Certainly sciences present models (often useful and, on 
that basis, probably relatively accurate) of 
physical phenomena, and often do so via mathematical 
structures that are logical modles of an 
axiomatic system.  So the relevant 
scientifically studied phenomena become indirectly 
associated with the stipulated axiomatic.  But 
as I have argued, the logic is two stps removed 
from the phenomena; at best the siting of 
logical structures in empirical reality 
\mdash{} the idea that material systems ralize structures that can 
be abstractly specified \mdash{} depends on complex 
scientific investigation.  It requires detailed analysis 
to show how a terrain of scientific givens \mdash{} the populations 
of species in an ecosystem, say \mdash{} play out 
as exemples of a logically hypotheized (e.g. in 
this case Darwinian or genetic) theory.
}
\p{Scientific models are therefore thoretical utilizations 
of axiomatic structures (usually via a mathematical intermediary), and 
provable consequencs of the axiomatic system can factor in 
to scientific reasoning.  But scientific models are 
not \q{logical} models in the sense that science gives 
us sets that axiomatically realize logical systems by 
grounding quantification domains that allow 
positions in logical formulae to be inhabited.  
The logical model-theoretic picture \mdash{} that axiomatic systems 
get instantiated in sets of entities into which 
logical symbols quantify \mdash{} is at best only 
abstractly applicable to scientific models, 
which do not yield a physical reality composed of 
\q{sets} whose members can be unproblemmatically labeled 
with logical symbols.
}
\p{To the degree that logical systems have even indirect 
axiomatic realization in philosophically salient 
contexts, then, they have to be \q{routed} either 
through cognition or through \q{science}.  Either an 
axiomatic schema represents (perhaps with some 
deliberate simpification) how 
we cognize some region of phenomena, 
or else it models how the phenomena behave 
irrespective our cognitive inclinations 
(of course, it can do both in mutual 
influence).  But either we have to see axiomatic 
systems embedded in structures 
sutured in cognitive structure, or we have to see 
axiomatic systems embedded in scientific 
theories of phenomenal composition and behavior.  
In any case axiomatic models have to \q{latch on to} 
cognitive and/or scientific models to have a 
philosophical resonance.
}
\p{This effectively metaphilosophical ideology, I think, 
includes mereology as an (important) special 
case.  Introducing mereology via an axiomatic 
system is quite common, even in philosophy.  But 
I'd argue that the philosophical value of 
an axiomatized mereological system is dependent 
on how and whether it can latch on to 
the requisite cognitive and/or scientific 
models to make its own, axiomatic model-building 
something more concrete than afforded by set-based 
model theory in pure logic.
}
\p{This brings me to my final case for NAM: 
non-antisymmetric mereologies have 
more or better \i{concrete models} than 
Classical mreologies.  I'll call this 
the Model-Theoretic Argument: 
when we look at \i{concrete} models in 
cognitive or scientifically-described systems 
\mdash{} not just imagined models like 
\q{Voxel Mereology} which are more about 
notating philosophical intuitions than 
philosophical expos\'es of thought or reality \mdash{} 
non-antisymmertry, not antisymmetry, predominates.
}
\spsubsectiontwoline{Concreteness and Computation}
\p{I would be remiss to draw a contrast between 
\i{abstract} and \i{concrete} models without mntioning 
the emergent field of \q{Ontology Engineering} and the role 
of mereology therein.  Arguably, mereology 
in these contexts is neither abstract like logical-axiomatic 
models, nor concrete in the sense of \q{routing} through 
cognitive or scientific models.  Ontologies 
\mdash{} as this term applies in computer science and 
Information Systems \mdash{} are computational artifacts, effectively 
documents in a special-purpose language, enabling 
programmers and scientists to define and 
annotate data structures so as to describe logical 
patterns obtaining within data.  Ontologies allow 
data to be correctly shared and reproduced, and 
also sometimes to be logically analyzed, identifying 
structures within a data set that can be 
extracted via logical rules.  Embedding logical axioms, 
including those of mereology, 
in data modeling frameworks therefoe becomes 
part of an overall data nanagement technology.  Logical 
analysis becomes a part of data analytics in general. 
}
\p{Ontologies are abstract prototypes; they play a determinative 
modeling role only in conjunction with actual data sets.   
Also, although computer programs can perform 
some \q{reasoning} on modeled data sets, the 
explicit correlation between Ontologies and 
empirical data depends on people consciously 
aligning data structures to Ontologies' types and 
relations.  Data sets annotated via formal Ontologies 
therefore reprsent deliberate human attempts to 
relate conceptual and empirical 
tsructures \mdash{} they are both cognitively 
and scientifically concrete.
}
\p{Consider, for sake of demonstration, a derivation of 
the fact that an ingrown toenail should 
qualify as a malady of the foot.  An Ontology 
might represent ingrown toenails afflicting 
toes, and also that toes are part of the foot, 
without identifying an ingrown toenail as a 
malady of the foot per se.  Someone searching 
a database for foot afflictions, however, 
might well expect cases of ingrown toenails to be  
among the search results.  Thus mereological 
rules can extend the semantic reach of an information system.
}
\p{But that semantics depends first on conscious design 
of Ontologies to represent things like 
disease classifications and part/whole relations, and 
also on human c;assification activity.  Someone has 
to specifically notate that a patient's condition 
exemplifies an ingrown toenail.  The Information 
System is designed around human practice, according 
to different operational needs: entering a disagnostic code 
in a medical record, for example, and then 
querying a database to gather a set of results 
such as cases of diagnoses filtereed by some 
condition (say, conditions that might be 
treated by a podiatrist).  An Information 
System \mdash{} its technology as well as its 
actual data \mdash{} can evolve in response to 
operational preferences.  In short, 
deductive reasoning like \i{ingrown 
toenail is a kind of foot malady} is desired not 
out of some abstract concern for logical completeness, 
but it fits operational needs to 
pair a data entry about a patient with an 
ingrown toenail matching a potential database 
query about podiatric diagnoses.
}
\p{Mereology is, accordingly, part of an overall logical 
structure designed to facilitate the operational 
requirements of adding and filtering data 
from information spaces.  This adds a further dimension to 
mereological analysis, because the 
mereology (along with other logical 
relations) becomes part of an engineered system, 
where we have prior anticipation of \i{how} 
we want the system to perform.  We identify specific 
kinds of inferences which should be engineered 
into the system based on anticipated 
preferences of human users.
}
\p{To clarify, we can recognize in an Information 
System at least two different classes of 
\q{behaviors}: \i{input} and \i{output}.  
Input behaviors are the system's implementations, 
and the interactions it affords, for people 
entering new data (say a doctor entering a 
new ingrown-toenail diagnosis); output behaviors are 
implementations and afforances for getting information 
out of the systm, such as by queries (perhaps, a 
list of patints with podiatric diagnoses).  
We can add \i{internal} behaviors involving 
internal processing to link inputs and outputs 
in (according to users' expectation) correct 
ways.  The system-outputs are products of 
the cumulative history of its inputs up to the 
point where output behavior is requisite.  
For example, the prior input of an ingrown-toenail 
diagnosis constitutes \i{relevant} input history 
for a query (demanding output 
behavior) for podiatric diagnoses.
}
\p{So mereology and other logical relations play a role when it 
comes to identifying \i{input history} that is \i{relevant} 
for a certain output rquest.  Logical reasoning 
expands the space of input history which will be 
considered relevant in a given output-context.  
This points to an interpretation of mereology in 
terms of \i{relevance}; properties or 
assertions involving a part (e.g., diagnoses) 
are relevant to corresponding whole.  
}
\p{Notionally, then, mereology serves as a kind of 
semantic enrichment of a system whose essential 
nature is a mapping from \i{input history} 
to \i{output behavior}.  This core 
operational motive \mdash{} \i{input history 
determines output behavior} \mdash{} should be reiterated 
in the context of the Semantic Web in particular, 
to help establish an underlying conceptual 
model to properly analyze (and implement) 
Semantic Web technology.
}
\p{Conceptually, the Semantic Web is one 
example of a data-sharing platform; 
and the keyingredient of data sharing is 
the replication of information spacs, 
in whole or in part, between different 
places.  Insofar as \q{input history 
determines output behavior} is the 
constituent feature of information spaces, 
replication means that the points 
where the data is shared will replicate 
analogous input-to-output patterns.  
Data sharing does not necessarily mean that 
all points process data identically; when data is 
exported between environments the new location 
may serve different operational purposes, 
so that analogous queries or interactions 
which receieve output data may beevaluated different 
at different points in a data-sharing network.  
But assuming the \i{receiver} of shared data 
does differ in its output structures from 
the \i{senders}, these differences should be 
systematically accounted for \visavis{} how 
the respective systems are engineered. 
}
\p{Conventional Semantic Web data srructures are 
organized around the principle of \i{labeled graphs}, 
where the building blocks of data models are 
so-called \q{triples}: \i{Subject/Predicate/Object} 
graph units where \i{Subject} and \i{Object} 
are graph-nodes and \i{Predicates} are labels 
applied to graph-edges, asserting that the \i{Object} 
bears to the \i{Subject} some 
identified relationship.  Ontologies then, 
first and foremost, provide lists 
of identifier for both nodes and edge-labels: 
this allows nodes to be associated with a kind or 
type (a \i{person}, say, or a \i{diagnostic code}) 
while edges are labelled according to a controlled 
vocabulary (so edges represent concrete 
relations, for example, that a person has a diagnosis).  
Ontologies then superimopose logical axioms 
pertaining to restrictions or guarantees among relations and/or 
nodes: that a relation is transitive, 
say, or that a given relation is only reasonable 
for certain \i{Subject} and \i{Object} types 
(e.g., a diagnostic relation is only 
coherent as a relation between a person or 
patient and a diagnosis or diagnostic code).
}
\p{Although a fair range of empirical semantics can be 
captured by individual \i{Subject/Predicate/Object} 
triples, in many cases a realistic semantics 
demands a model of how multiple relations aggregate 
to form meaningful units.  For example, instead of 
a diagnosis being modeled as a single relation between 
a patient and a diagnostic code, a systm may 
stipulate a more detailed representation where a single 
\q{diagnosis} has several associated piecs 
of information, such as a diagnostic code, a date, a 
doctor's name, and potential 
reference to laboratory or radiological findings that 
substantiate the diagnosis.  This relflects how, in 
most computational settings, the basic units of 
evaluationb are not single data points but internally 
structured data aggrgates.
}
\p{Thus far, no standard model has emerged to represent 
larger-scale data structures in the Semantic Web 
context.
}
\p{}
