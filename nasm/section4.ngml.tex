\section{Conclusion}
\p{The upshot of my argument lat section was 
that the realm of mereologuical theory should 
be divided into \i{cognitive} and \i{multiscientific}.  
Criteria of mereological non-trivialoty are 
derived in the first case from \i{cognitive frames}, 
and in the second case from an \q{$R$-patchwork}: 
a patchwork of scientific theories delimiting the 
extent of different scientific registers and the 
dynamics and autopoesis sustaining the 
integrtity of (in various domains) 
non-trivial wholes.  Moreover, there is a structural 
resonance and interpenetration between 
cognitive frames and the \q{$R$-patchwork}.
}
\p{Whether from the perspective of cognitive frames or the 
\q{extramental} perspective of the $R$-patchwork, 
mreological theories can be rasonably elaborated 
in parallel to theories of the cognitive/ontological 
status of wholes participating in \xhppy{} style 
parthood assertions.  Any formalization of 
parthood relations can assume that 
}
\p{}
